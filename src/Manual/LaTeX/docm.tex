% =====================================================================
% Macros for typesetting hol reference manual entries
% =====================================================================

% ---------------------------------------------------------------------
% boolean flag for verbose printing of reference manual typesetting
% ---------------------------------------------------------------------

\newif\ifverboseref
\verbosereffalse			  % don't be verbose

% ---------------------------------------------------------------------
% Macro for generating right-hand page running titles.
% ---------------------------------------------------------------------

\makeatletter

\def\mkhead{\futurelet\@t\chsize}
\def\chsize#1.{\ifx a\@t \markright{{\protect\bf #1}}\else
   	       \ifx b\@t \markright{{\protect\bf #1}}\else
	       \ifx c\@t \markright{{\protect\bf #1}}\else
	       \ifx d\@t \markright{{\protect\bf #1}}\else
	       \ifx e\@t \markright{{\protect\bf #1}}\else
	       \ifx f\@t \markright{{\protect\bf #1}}\else
	       \ifx g\@t \markright{{\protect\bf #1}}\else
	       \ifx h\@t \markright{{\protect\bf #1}}\else
	       \ifx i\@t \markright{{\protect\bf #1}}\else
	       \ifx j\@t \markright{{\protect\bf #1}}\else
	       \ifx k\@t \markright{{\protect\bf #1}}\else
	       \ifx l\@t \markright{{\protect\bf #1}}\else
	       \ifx m\@t \markright{{\protect\bf #1}}\else
	       \ifx n\@t \markright{{\protect\bf #1}}\else
	       \ifx o\@t \markright{{\protect\bf #1}}\else
	       \ifx p\@t \markright{{\protect\bf #1}}\else
	       \ifx q\@t \markright{{\protect\bf #1}}\else
	       \ifx r\@t \markright{{\protect\bf #1}}\else
	       \ifx s\@t \markright{{\protect\bf #1}}\else
	       \ifx t\@t \markright{{\protect\bf #1}}\else
	       \ifx u\@t \markright{{\protect\bf #1}}\else
	       \ifx v\@t \markright{{\protect\bf #1}}\else
	       \ifx w\@t \markright{{\protect\bf #1}}\else
	       \ifx z\@t \markright{{\protect\bf #1}}\else
	       \ifx y\@t \markright{{\protect\bf #1}}\else
	       \ifx z\@t \markright{{\protect\bf #1}}\else
               \markright{{\protect\small\bf #1}}\fi
	       \fi\fi\fi\fi\fi\fi\fi\fi\fi\fi\fi\fi\fi\fi\fi
	       \fi\fi\fi\fi\fi\fi\fi\fi\fi\fi}

\makeatother

% ---------------------------------------------------------------------
% \DOC{<object>}  : start a manual entry for <object>.
% ---------------------------------------------------------------------

\newcommand{\DOC}[1]%
{\bigskip
 {\ifverboseref{\def\_{\string_}\typeout{Typesetting: #1}}\fi}
 \mkhead #1.
 \autoindex{#1@{\tt #1}}
 \begin{flushleft}
 \begin{tabular}{|c|}\hline
 \begin{minipage}{\minipagewidth}
 \bigskip
 {\LARGE\tt #1}
 \bigskip
 \end{minipage}\\ \hline
 \end{tabular}
 \end{flushleft}
 \bigskip
}

% ---------------------------------------------------------------------
% Commands for parts of a \DOC:
%    \SYNOPSIS 
%    \DESCRIBE
%    \FAILURE
%    \EXAMPLE
%    \USES
%    \SEEALSO
% ---------------------------------------------------------------------

\newcommand{\SYNOPSIS}%
{\bigskip{\noindent\large\bf Synopsis}\newline\mbox{}}

\newcommand{\DESCRIBE}%
{\bigskip{\noindent\large\bf Description}\newline\mbox{}}

\newcommand{\FAILURE}%
{\bigskip{\noindent\large\bf Failure}\newline\mbox{}}

\newcommand{\EXAMPLE}%
{\bigskip{\noindent\large\bf Example}\newline\mbox{}}

\newcommand{\USES}%
{\bigskip{\noindent\large\bf Uses}\newline\mbox{}}

\newcommand{\SEEALSO}%
{\bigskip{\noindent\large\bf See also}}

% ---------------------------------------------------------------------
% \ENDDOC = do nothing
% ---------------------------------------------------------------------

\newcommand{\ENDDOC}{}

% =====================================================================
% Commands for typesetting theorems
% =====================================================================

\makeatletter

% ---------------------------------------------------------------------
% define \@xboxverb<thing>\ENDTHEOREM to mean <thing>\ENDTHEOREM
% ---------------------------------------------------------------------

\begingroup \catcode `|=0 \catcode `[= 1
\catcode`]=2 \catcode `\{=12 \catcode `\}=12
\catcode`\\=12 |gdef|@xboxverb#1\ENDTHEOREM[#1|ENDTHEOREM]
|endgroup

% ---------------------------------------------------------------------
% \bboxverb<thing> = <thing> in a verbatim box 5mm from left margin
% ---------------------------------------------------------------------

\def\@boxverb{\bgroup\leftskip=5mm\parindent\z@
\parfillskip=\@flushglue\parskip\z@
\obeylines\tt \catcode``=13 \@noligs \let\do\@makeother \dospecials}

\def\boxverb{\@boxverb \frenchspacing\@vobeyspaces \@xboxverb}

% ---------------------------------------------------------------------
% \ENDTHEOREM just finishes off the group (and kick page if necessary)
% ---------------------------------------------------------------------

\def\ENDTHEOREM{\egroup\filbreak}

% ---------------------------------------------------------------------
% \THEOREM <name> <thy> ... \ENDTHEOREM = typeset a theorem
% ---------------------------------------------------------------------

\def\THEOREM #1 #2 {
 \autoindex{#1@{\tt #1}}
   \vspace{4mm plus2mm minus1mm}
   \noindent {\tt #1}\quad ({\tt #2}) \par \boxverb 
}

\makeatother

% ---------------------------------------------------------------------
% The theory name \none = italic "none"
% ---------------------------------------------------------------------

\def\none{{\it none}}
