% =====================================================================
% HOL Manual LaTeX Source: tutorial
% =====================================================================

\documentstyle[12pt,fleqn,../LaTeX/alltt,../LaTeX/layout]{book}

% ---------------------------------------------------------------------
% Input defined macros and commands
% ---------------------------------------------------------------------
% =====================================================================
%
% Macros for typesetting the HOL system manual
%
% =====================================================================

% ---------------------------------------------------------------------
% Abbreviations for words and phrases
% ---------------------------------------------------------------------
\newcommand\VERSION{{\small\tt 2.0}}
\newcommand\TUTORIAL{{\footnotesize\sl TUTORIAL}}
\newcommand\DESCRIPTION{{\footnotesize\sl DESCRIPTION}}
\newcommand\REFERENCE{{\footnotesize\sl REFERENCE}}
\newcommand\LIBRARIES{{\footnotesize\sl LIBRARIES}}

\def\HOL{{\small HOL}}
\def\LCF{{\small LCF}}
\def\LCFLSM{{\small LCF{\kern-.2em}{\normalsize\_}{\kern0.1em}LSM}}
\def\PPL{{\small PP}{\kern-.095em}$\lambda$}
\def\PPLAMBDA{{\small PPLAMBDA}}
\def\ML{{\small ML}}

\newcommand\ie{\mbox{i{.}e{.}}}
\newcommand\eg{\mbox{e{.}g{.}}}
\newcommand\viz{\mbox{viz{.}}}
\newcommand\adhoc{\mbox{\it ad hoc}}
\newcommand\etal{{\it et al.\/}}
\newcommand\etc{\mbox{etc{.}}}
\def\see#1#2{{\em see\/} #1}

% ---------------------------------------------------------------------
% Simple abbreviations and macros for mathematical typesetting
% ---------------------------------------------------------------------

\newcommand\fun{{\to}}
\newcommand\prd{{\times}}

\newcommand\conj{\ \wedge\ }
\newcommand\disj{\ \vee\ }
\newcommand\imp{ \Rightarrow }
\newcommand\eqv{\ \equiv\ }
\newcommand\cond{\rightarrow}
\newcommand\vbar{\mid}
\newcommand\turn{\ \vdash\ }
\newcommand\hilbert{\varepsilon}
\newcommand\eqdef{\ \equiv\ }

\newcommand\natnums{\mbox{${\sf N}\!\!\!\!{\sf N}$}}
\newcommand\bools{\mbox{${\sf T}\!\!\!\!{\sf T}$}}

\newcommand\p{$\prime$}
\newcommand\f{$\forall$\ }
\newcommand\e{$\exists$\ }

\newcommand\orr{$\vee$\ }
\newcommand\negg{$\neg$\ }

\newcommand\arrr{$\rightarrow$}
\newcommand\hex{$\sharp $}

\newcommand{\uquant}[1]{\forall #1.\ }
\newcommand{\equant}[1]{\exists #1.\ }
\newcommand{\hquant}[1]{\hilbert #1.\ }
\newcommand{\iquant}[1]{\exists ! #1.\ }
\newcommand{\lquant}[1]{\lambda #1.\ }

\newcommand{\leave}[1]{\\[#1]\noindent}
\newcommand\entails{\mbox{\rule{.3mm}{4mm}\rule[2mm]{.2in}{.3mm}}}

% --------------------------------------------------------------------- 
% Font-changing commands
% ---------------------------------------------------------------------

\newcommand{\theory}[1]{\hbox{{\small\tt #1}}}

\newcommand{\con}[1]{{\sf #1}}
\newcommand{\rul}[1]{{\tt #1}}
\newcommand{\ty}[1]{{\sl #1}}

\newcommand{\ml}[1]{\mbox{{\def\_{\char'137}\small\tt #1}}}
\newcommand\ms{\tt}
\newcommand{\s}[1]{{\small #1}}

\newcommand{\pin}[1]{{\bf #1}}
\def\m#1{\mbox{\normalsize$#1$}}

% ---------------------------------------------------------------------
% Abbreviations for particular mathematical constants etc.
% ---------------------------------------------------------------------

\newcommand\T{\con{T}}
\newcommand\F{\con{F}}
\newcommand\OneOne{\con{One\_One}}
\newcommand\OntoSubset{\con{Onto\_Subset}}
\newcommand\Onto{\con{Onto}}
\newcommand\TyDef{\con{Type\_Definition}}
\newcommand\Inv{\con{Inv}}
\newcommand\com{\con{o}}
\newcommand\Id{\con{I}}
\newcommand\MkPair{\con{Mk\_Pair}}
\newcommand\IsPair{\con{Is\_Pair}}
\newcommand\Fst{\con{Fst}}
\newcommand\Snd{\con{Snd}}
\newcommand\Suc{\con{Suc}}
\newcommand\Nil{\con{Nil}}
\newcommand\Cons{\con{Cons}}
\newcommand\Hd{\con{Hd}}
\newcommand\Tl{\con{Tl}}
\newcommand\Null{\con{Null}}
\newcommand\ListPrimRec{\con{List\_Prim\_Rec}}


\newcommand\SimpRec{\con{Simp\_Rec}}
\newcommand\SimpRecRel{\con{Simp\_Rec\_Rel}}
\newcommand\SimpRecFun{\con{Simp\_Rec\_Fun}}
\newcommand\PrimRec{\con{Prim\_Rec}}
\newcommand\PrimRecRel{\con{Prim\_Rec\_Rel}}
\newcommand\PrimRecFun{\con{Prim\_Rec\_Fun}}

\newcommand\bool{\ty{bool}}
\newcommand\num{\ty{num}}
\newcommand\ind{\ty{ind}}
\newcommand\lst{\ty{list}}

% ---------------------------------------------------------------------
% \minipagewidth = \textwidth minus 1.02 em
% ---------------------------------------------------------------------

\newlength{\minipagewidth}
\setlength{\minipagewidth}{\textwidth}
\addtolength{\minipagewidth}{-1.02em}

% ---------------------------------------------------------------------
% Environment for the items on the title page of a case study
% ---------------------------------------------------------------------

\newenvironment{inset}[1]{\noindent{\large\bf #1}\begin{list}%
{}{\setlength{\leftmargin}{\parindent}%
\setlength{\topsep}{-.1in}}\item }{\end{list}\vskip .4in}

% ---------------------------------------------------------------------
% Macros for little HOL sessions displayed in boxes.
%
% Usage: (1) \setcounter{sessioncount}{1} resets the session counter
%
%	 (2) \begin{session}\begin{verbatim}
%	      .
%	       < lines from hol session >
%	      .
%	     \end{verbatim}\end{session}   
%
%            typesets the session in a numbered box.
% ---------------------------------------------------------------------

\newlength{\hsbw}
\setlength{\hsbw}{\textwidth}
\addtolength{\hsbw}{-\arrayrulewidth}
\addtolength{\hsbw}{-\tabcolsep}
\newcommand\HOLSpacing{13pt}

\newcounter{sessioncount}
\setcounter{sessioncount}{1}

\newenvironment{session}{\begin{flushleft}
 \begin{tabular}{@{}|c@{}|@{}}\hline 
 \begin{minipage}[b]{\hsbw}
 \vspace*{-.5pt}
 \begin{flushright}
 \rule{0.01in}{.15in}\rule{0.3in}{0.01in}\hspace{-0.35in}
 \raisebox{0.04in}{\makebox[0.3in][c]{\footnotesize\sl \thesessioncount}}
 \end{flushright}
 \vspace*{-.55in}
 \begingroup\small\baselineskip\HOLSpacing}{\endgroup\end{minipage}\\ \hline 
 \end{tabular}
 \end{flushleft}
 \stepcounter{sessioncount}}

% ---------------------------------------------------------------------
% Macro for boxed ML functions, etc.
%
% Usage: (1) \begin{boxed}\begin{verbatim}
%	        .
%	        < lines giving names and types of mk functions >
%	        .
%	     \end{verbatim}\end{boxed}   
%
%            typesets the given lines in a box.  
%
%            Conventions: lines are left-aligned under the "g" of begin, 
%	     and used to highlight primary reference for the ml function(s)
%	     that appear in the box.
% ---------------------------------------------------------------------

\newenvironment{boxed}{\begin{flushleft}
 \begin{tabular}{@{}|c@{}|@{}}\hline 
 \begin{minipage}[b]{\hsbw}
% \vspace*{-.55in}
 \vspace*{.06in}
 \begingroup\small\baselineskip\HOLSpacing}{\endgroup\end{minipage}\\ \hline 
 \end{tabular}
 \end{flushleft}}

% ---------------------------------------------------------------------
% Macro for unboxed ML functions, etc.
%
% Usage: (1) \begin{hol}\begin{verbatim}
%	        .
%	        < lines giving names and types of mk functions >
%	        .
%	     \end{verbatim}\end{hol}   
%
%            typesets the given lines exactly like {boxed}, except there's
%	     no box.  
%
%            Conventions: lines are left-aligned under the "g" of begin, 
%	     and used to display ML code in verbatim, left aligned.
% ---------------------------------------------------------------------

\newenvironment{hol}{\begin{flushleft}
 \begin{tabular}{c@{}@{}}
 \begin{minipage}[b]{\hsbw}
% \vspace*{-.55in}
 \vspace*{.06in}
 \begingroup\small\baselineskip\HOLSpacing}{\endgroup\end{minipage}\\
 \end{tabular}
 \end{flushleft}}

% ---------------------------------------------------------------------
% Emphatic brackets
% ---------------------------------------------------------------------

\newcommand\leb{\lbrack\!\lbrack}
\newcommand\reb{\rbrack\!\rbrack}


%These macros were included by jac1: they are used in two of the index entries

\def\per{\ml{\%}}
\def\pes{\ml{\%<}}
\def\pee{\ml{>\%}}


%These macros were included by ap; they are used in Chapters 9 and 10
%of the HOL DESCRIPTION

\newcommand{\inds}%standard infinite set
 {\mbox{\rm I}}

\newcommand{\ch}%standard choice function
 {\mbox{\rm ch}}

\newcommand{\den}[1]%denotational brackets
 {[\![#1]\!]}

\newcommand{\two}%standard 2-element set
 {\mbox{\rm 2}}













%macros for pictures in latex

\def\puthrule(#1,#2)#3{\put(#1,#2){\line(1,0){#3}}}
\def\putvrule(#1,#2)#3{\put(#1,#2){\line(0,1){#3}}}
\def\putdot(#1){\put(#1){\circle*{0.2}}}
\def\ignore#1{}
\def\putgrid(#1,#2)(#3,#4){\multiput(#1,#2)(1,0){#3}{\circle*{0.2}}
\multiput(#1,#2)(0,1){#4}{\circle*{0.2}}}

\def\putdevice(#1,#2)#3{\put(#1,#2){\framebox(4,2){\small{\tt #3}}}}
\def\putport(#1,#2)#3{\put(#1,#2){\makebox(4,1){\small{\tt #3}}}}


% =====================================================================
% Macros for typesetting hol reference manual entries
% =====================================================================

% ---------------------------------------------------------------------
% boolean flag for verbose printing of reference manual typesetting
% ---------------------------------------------------------------------

\newif\ifverboseref
\verbosereffalse			  % don't be verbose

% ---------------------------------------------------------------------
% \DOC{<object>}  : start a manual entry for <object> (to be used when
%	            <object> is an UPPER-CASE ML identifier.
% ---------------------------------------------------------------------
\newcommand{\DOC}[1]%
{\bigskip
 {\ifverboseref{\def\_{\string_}\typeout{Typesetting: #1}}\fi}
 \markright{{\protect\small\bf #1}}
 \autoindex{#1@{\tt #1}}
 \begin{flushleft}
 \begin{tabular}{|c|}\hline
 \begin{minipage}{\minipagewidth}
 \bigskip
 {\LARGE\tt #1}
 \bigskip
 \end{minipage}\\ \hline
 \end{tabular}
 \end{flushleft}
 \bigskip
}

% ---------------------------------------------------------------------
% \LDOC{<object>}  : start a manual entry for <object> (to be used when
%		     <object> is a lower-case ML identifier.
% ---------------------------------------------------------------------

\newcommand{\LDOC}[1]%
{\bigskip
 {\ifverboseref{\def\_{\string_}\typeout{Typesetting: #1}}\fi}
 \markright{\bf #1}
 \autoindex{#1@{\tt #1}}
 \begin{flushleft}
 \begin{tabular}{|c|}\hline
 \begin{minipage}{\minipagewidth}
 \bigskip
 {\LARGE\tt #1}
 \bigskip
 \end{minipage}\\ \hline
 \end{tabular}
 \end{flushleft}
 \bigskip
}

% ---------------------------------------------------------------------
% Commands for parts of a \DOC:
%    \SYNOPSIS 
%    \DESCRIBE
%    \FAILURE
%    \EXAMPLE
%    \USES
%    \SEEALSO
% ---------------------------------------------------------------------

\newcommand{\SYNOPSIS}%
{\bigskip{\noindent\large\bf Synopsis}\newline\mbox{}}

\newcommand{\CATEGORIES}%
{\bigskip{\noindent\large\bf Categories}\newline\mbox{}}

\newcommand{\DESCRIBE}%
{\bigskip{\noindent\large\bf Description}\newline\mbox{}}

\newcommand{\FAILURE}%
{\bigskip{\noindent\large\bf Failure}\newline\mbox{}}

\newcommand{\EXAMPLE}%
{\bigskip{\noindent\large\bf Example}\newline\mbox{}}

\newcommand{\USES}%
{\bigskip{\noindent\large\bf Uses}\newline\mbox{}}

\newcommand{\SEEALSO}%
{\bigskip{\noindent\large\bf See also}}

% ---------------------------------------------------------------------
% \ENDDOC = do nothing
% ---------------------------------------------------------------------

\newcommand{\ENDDOC}{}

\makeatletter

\begingroup \catcode `|=0 \catcode `[= 1
\catcode`]=2 \catcode `\{=12 \catcode `\}=12              
\catcode`\\=12 |gdef|@xboxverb#1\ENDTHEOREM[#1|ENDTHEOREM]
|endgroup                                                 

\def\@boxverb{\bgroup\leftskip=5mm\parindent\z@
\parfillskip=\@flushglue\parskip\z@
\obeylines\tt \catcode``=13 \@noligs \let\do\@makeother \dospecials}

\def\boxverb{\@boxverb \frenchspacing\@vobeyspaces \@xboxverb}

\def\ENDTHEOREM{\egroup\filbreak}

\def\THEOREM #1 #2 {
 \autoindex{#1@{\tt #1}}
   \vspace{4mm plus2mm minus1mm}
   \noindent {\tt #1}\quad ({\tt #2}) \par \boxverb 
}

\makeatother

\def\none{{\it none}}



% Counter Peano used in logic.tex 
\newcounter{Peano} 
\setcounter{Peano}{1}

%\includeonly{title,contents,preface,../LaTeX/ack,intro,ml,logic,
% proof,references,Studies/preface,Studies/parity/parity}
%             Studies/microprocessor/all,
%\includeonly{Studies/microprocessor/all}
%\includeonly{Studies/int_mod/mod_arith_study/tutorial}
%\includeonly{binomial}
%\includeonly{parity}


\begin{document}

   \setlength{\unitlength}{1mm}		  % unit of length = 1mm
   \setlength{\baselineskip}{16pt}        % line spacing = 16pt

   % ---------------------------------------------------------------------
   % prelims
   % ---------------------------------------------------------------------

   \pagenumbering{roman}	          % roman page numbers for prelims
   \setcounter{page}{1}		          % start at page 1

   % ===================================================================== %
% Standard titlepage for wellorder library                              %
% ===================================================================== %

\begin{titlepage}

\setcounter{page}{1}                      % titlepage IS page 1 !

% --------------------------------------------------------------------- %
% Name of the library.                                                  %
% --------------------------------------------------------------------- %

\mbox{}
\vskip20mm
\begin{center}
{\Huge\bf The HOL wellorder Library}
\end{center}

% --------------------------------------------------------------------- %
% Name of the author                                                    %
% --------------------------------------------------------------------- %

\vskip15mm
\begin{center}
\large\bf J.\ R.\ Harrison
\end{center}

% --------------------------------------------------------------------- %
% Address of the author                                                 %
% --------------------------------------------------------------------- %

\vfill
\begin{center}
\bf
University of Cambridge Computer Laboratory\\
New Museums Site\\
Pembroke Street\\
Cambridge, {\small\bf CB}2 3{\small\bf QG}\\
England.
\end{center}

% --------------------------------------------------------------------- %
% Date.                                                                 %
% --------------------------------------------------------------------- %

\vskip5mm
\begin{center}
\bf 30th May 1992
\end{center}

\end{titlepage}

% --------------------------------------------------------------------- %
% To kick a blank page with no header (back of title page is blank).    %
% --------------------------------------------------------------------- %
\thispagestyle{empty}
\mbox{}

% --------------------------------------------------------------------- %
% Copyright notice (if desired).                                        %
% --------------------------------------------------------------------- %
\vfill
\begin{center}
\copyright\ J.\ R.\ Harrison 1992
\end{center}
\newpage
			  % tutorial title page
   \chapter*{Preface}\markboth{Preface}{Preface}

This volume provides documentation for the user-contributed libraries
distributed with the \HOL\ system.  It should be read in conjunction with the
\HOL\ system documentation, which consists of three volumes:

\begin{myenumerate}
\item \TUTORIAL: a tutorial introduction to \HOL, with case studies.
\item \DESCRIPTION: a description of higher order logic,
the \ML\ programming language, and theorem proving methods in the \HOL\ system;
\item \REFERENCE: the reference documentation of the tools available in \HOL.
\end{myenumerate}

\noindent These documents are be referred to by the short names (in small
slanted capitals) given above.

The present volume, \LIBRARIES, contains a collection independent sections, one
for each documented library distributed with the system. This documentation,
which is written by the contributors of the corresponding \HOL\ system
libraries, is not, strictly speaking, part of the \HOL\ manual set, which
consists of the three volumes listed above.  The \LIBRARIES\ documentation is,
however, typeset and distributed with the system in a form consistent with the
\HOL\ system documentation.


		          % preface to entire tutorial
   \chapter*{Acknowledgements}\markboth{Acknowledgements}{Acknowledgements}

The three volumes \TUTORIAL, \DESCRIPTION\ and \REFERENCE\ were
produced at the Cambridge Research Center of SRI International with
the support of DSTO Australia.

The \HOL\ documentation project was managed by Mike Gordon, who also
wrote parts of \DESCRIPTION\ and \TUTORIAL\ using material based on an
early paper describing the \HOL\ system\footnote{M.J.C.\ Gordon, `HOL:
a Proof Generating System for Higher Order Logic', in: {\it VLSI
Specification, Verification and Synthesis\/}, edited by G.\ Birtwistle
and P.A.\ Subrahmanyam, (Kluwer Academic Publishers, 1988), pp.\
73--128.} and {\sl The ML Handbook\/}\footnote{{\sl The ML Handbook},
unpublished report from Inria by Guy Cousineau, Mike Gordon, G\'erard
Huet, Robin Milner, Larry Paulson and Chris Wadsworth.}.  Other
contributers to \DESCRIPTION\ incude Avra Cohn, who contributed
material on theorems, rules, conversions and tactics, and also
composed the index (which was typeset by Juanito Camilleri); Tom
Melham, who wrote the sections describing type definitions, the
concrete type package and the `resolution' tactics; and  Andy Pitts, who
devised the set-theoretic semantics of the \HOL\ logic and wrote the
material describing it.

The first edition of \TUTORIAL\ contained case studies on
microprocessor systems (by Jeff Joyce), protocol verification (by
Rachel Cardell-Oliver) and modular arithmetic based on group theory
(by Elsa Gunter).  These are now separate documents in the \HOL\
distribution directory {\small\verb%hol/Training/studies%}.  The
chapter in \TUTORIAL\ on the proof of the binomial theorem in \HOL{}
was written by Andy Gordon.

The second edition of \REFERENCE\ was a joint effort by the Cambridge
\HOL\ group.

The original document design used \LaTeX\ macros supplied by Elsa
Gunter, Tom Melham and Larry Paulson.  The typesetting of all three
volumes was managed by Tom Melham.  The conversion of the {\tt troff}
sources of {\sl The ML Handbook\/} to \LaTeX\ was done by Inder
Dhingra and John Van Tassel.  The cover design is by Arnold Smith, who
used a photograph of a `snow watching lantern' taken by Avra Cohn (in
whose garden the original object resides).  John Van Tassel composed
the \LaTeX\ picture of the lantern.


Many people other than those listed above have contributed to the \HOL\
documentation effort, either by providing material, or by sending lists of
errors in the first edition.  Thanks to everyone who helped, and thanks to DSTO
and SRI for their generous support.

		  % global acknowledgements
   \tableofcontents
		          % table of contents   

   \pagenumbering{arabic}		 % arabic page numbers
   \setcounter{page}{1}		         % start at page 1

   
\chapter{Introduction\label{intro}}

This document describes the facilities provided by the \ml{prettyp} library
for the \HOL\ system~\cite{description}. The library is a pretty-printer based
on the Pretty-Printing Meta-Language (\PPML) for the \CENTAUR\
system~\cite{PPML}. It is intended as a tool for embedding languages within
the \HOL\ logic. To be truly useful it should be used along with a special
parser for the embedded language. Although such applications only require
\HOL\ terms to be pretty-printed, the system described here can be used to
pretty-print any tree structure (after undergoing translation).

The pretty-printing program converts a tree represented as a particular \ML\
datatype to text, using a set of rules. The user must provide these rules. A
parser for a special-purpose language is provided to facilitate this. The
parser generates a file which can be read into the \HOL\ system. The file
contains declarations of \ML\ values. These values are the rules used by the
pretty-printer.

To pretty-print a tree structure, the tree must be converted (usually by an
\ML\ function) to the particular datatype used by the pretty-printer. Functions
are provided with the system for converting \HOL\ types and terms.

Chapter~\ref{language} describes the pretty-printing language in detail.
Chapter~\ref{mldatatypes} describes the techniques required to convert an
arbitrary tree structure to the datatype used by the pretty-printer.
Linking specialised pretty-printers into the standard \HOL\ pretty-printer is
discussed in Chapter~\ref{linking}, including the functions to convert \HOL\
types and terms to trees which the pretty-printer can use.
Chapter~\ref{examples} gives examples of defining rules for a variety of
languages. 

The remainder of this chapter illustrates the process of building a new
pretty-printer. Some notation of set theory is added to the standard \HOL\
pretty-printer. First, though, the loading of the library is described.


\section{Loading the library}

The \ml{prettyp} library can be loaded into a \HOL\ session using the
function \ml{load\_library}\index{load\_library@{\ptt load\_library}} (see the
\HOL\ manual for a general description of library loading). The first action
in the load sequence initiated by \ml{load\_library} is to update the \HOL\
help\index{help!updating search path} search path. The help search path is
updated with pathnames to online help files for the \ML\ functions in the
library. After updating the help search path, the \ML\ functions in the
library are loaded into \HOL.

There are three code files in the library of importance to the user. The first
is called {\small\verb%PP_printer.ml%}. This file must be loaded in order to do
anything with the pretty-printer. It is the main pretty-printing program.

The file {\small\verb%PP_parser.ml%} can be loaded after
{\small\verb%PP_printer.ml%}. It is the compiler for the pretty-printing
language. It also contains a pretty-printer for the pretty-printing language!

The file {\small\verb%PP_hol.ml%} can also be loaded after
{\small\verb%PP_printer.ml%}. It contains functions for converting \HOL\
types, terms and theorems into parse-trees. It also contains a complete
pretty-printer for the \HOL\ logic. When loaded, the standard \HOL\
pretty-printer is replaced by these new printers. {\small\verb%PP_hol.ml%} is
required for any extension to the pretty-printing of \HOL\ types, terms or
theorems.

Note that {\small\verb%PP_parser.ml%} and {\small\verb%PP_hol.ml%} do not
require each other to be resident to work. They can however be resident
together.

Use of \ml{load\_library} loads all three of the files. The following session
shows how the entire \ml{prettyp} library can be loaded:

\setcounter{sessioncount}{1}
\begin{session}\begin{verbatim}
#load_library `prettyp`;;
Loading library `prettyp` ...
Updating help search path
.............................................................................
.............................................................................
.............................................................................
.............................................................
Library `prettyp` loaded.
() : void

#
\end{verbatim}\end{session}

If the user wants to load only one or two of the three files, they can be
loaded separately. As an example of this, {\small\verb%PP_printer.ml%} can be
loaded using one of the following \ML\ function calls:

\begin{small}\begin{verbatim}
   loadf (library_pathname() ^ `/prettyp/PP_printer`);;
   loadt (library_pathname() ^ `/prettyp/PP_printer`);;
\end{verbatim}\end{small}

\noindent
where the former loads `quietly' and the latter displays details of the
declarations made within the file.


\section{Example: a pretty-printer for set theory in HOL}

\setcounter{sessioncount}{1}
\newwindow{{\small\tt sets.pp}}

This section illustrates the development process for an extension to the \HOL\
pretty-printer. Throughout the example we assume the user has two windows.
A \HOL\ session is running within the first window, which is represented by a
box of the following form:

\begin{session}\begin{verbatim}
...
\end{verbatim}\end{session}

\noindent
The other window is an editor in which a file named {\small\verb%sets.pp%} is
being edited. The editor is represented by:

\begin{window}\begin{verbatim}
...
\end{verbatim}\end{window}

\setcounter{sessioncount}{1}

\noindent
We begin by running \HOL\ and loading three files from the library
\ml{prettyp}.

\begin{session}\begin{verbatim}

          _  _    __    _      __    __
   |___   |__|   |  |   |     |__|  |__|
   |      |  |   |__|   |__   |__|  |__|

          Version 2

#loadf (library_pathname() ^ `/prettyp/PP_printer`);;
Updating help search path
.............................................................................
.......................................() : void

#loadf (library_pathname() ^ `/prettyp/PP_parser`);;
Updating help search path
.............................................................................
......................................................................
() : void

#loadf (library_pathname() ^ `/prettyp/PP_hol`);;
Updating help search path
.................................() : void
\end{verbatim}\end{session}

\noindent
The first file is the main pretty-printing program. It must always be loaded
when the pretty-printer is being used. The second file is a parser for the
pretty-printing language. The first file must always be loaded before the
second. The parser generates a file of \ML\ declarations. The third file is
a replacement for the standard \HOL\ pretty-printer. It has been written using
the pretty-printer described here. This allows it to be extended with the
special-purpose syntax.

The next thing to do is to load the library whose syntax we wish to extend:

\begin{session}\begin{verbatim}
#load_library `sets`;;
Loading library `sets` ...
Updating search path
.Theory sets loaded
.....................
Library `sets` loaded.
() : void
\end{verbatim}\end{session}

\noindent
The constant {\small\verb%EMPTY%} is now defined within the \HOL\ system. It represents
an empty set. Observe that no special syntax is attached to the constant.

\begin{session}\begin{verbatim}
#"EMPTY:(*)set";;
"EMPTY" : term
\end{verbatim}\end{session}

\noindent
Now we enter a small pretty-printer specification into the editor window.

\begin{window}\begin{verbatim}
prettyprinter sets =

rules
   'term'::CONST(EMPTY(),**) -> [<h 0> "{}"];

end rules


end prettyprinter
\end{verbatim}\end{window}

\noindent
The name of the pretty-printer is specified as {\small\verb%sets%}. There is
one rule. The rule instructs \HOL\ to print {\small\verb%{}%} whenever it
encounters the constant {\small\verb%EMPTY%}.

There are two parts to the rule: a {\it pattern\/} and a {\it format}. These
are separated by {\small\verb%->%}. When printing, the system compares the
pattern to the term which is to be printed. In the example, the pattern
matches the term only if the current {\it context\/} is {\small\verb%'term'%}.
The context is a string of characters which is specified when the
pretty-printer is called. It may also be modified by a rule during the
printing process.

The rest of the pattern represents the tree structure of a \HOL\ term. So, for
the pattern to match a term, the term must represent the constant
{\small\verb%EMPTY%}. The {\small\verb%**%} in the pattern is used to match
optional type information. We shall not concern ourselves with this notation
at the moment.

The format consists of a {\it box}, the components of which are to be composed
horizontally with no space between them. In the example, the box has only one
component, so the composition information is not required. The format instructs
the printer to display {\small\verb%{}%}. The double quotation-marks are used
to delimit a string which is to be displayed verbatim.

So, whenever the pattern matches, the format is used to determine what to
display. Let's see this in action. First the file must be saved. Then we
instruct \HOL\ to convert the pretty-printer specification into a file of \ML\
declarations.

\begin{session}\begin{verbatim}
#PP_to_ML false `sets` ``;;
() : void
\end{verbatim}\end{session}

\noindent
There should now be a file called {\small\verb%sets_pp.ml%}. This contains two
\ML\ declarations. The first declares \ml{sets\_rules} to be a list of
pretty-printing rules as understood by the pretty-printing program. The
second declares \ml{sets\_rules\_fun} to be a function which embodies the
properties of the rules. The names of the identifiers are derived from the
name of the pretty-printer specification given in the file.

The function \ml{PP\_to\_ML} invokes the parser. Its first argument indicates
whether the output is to be appended to the specified file. In the example
the output is not appended, i.e.~if the destination file existed previously it
will be overwritten. The second argument is the name of the source file. The
name of the source file must end in {`}{\small\verb%.pp%}{'}. The
{`}{\small\verb%.pp%}{'} may be omitted from the name given as the second
argument. The third argument is the name of the destination file. This should
either be given in full, or if, as in the example, a null string is given, the
parser will replace the {`}{\small\verb%.pp%}{'} of the source file name with
{`}{\small\verb%_pp.ml%}{'}.

We can now load the file of \ML\ declarations, and instruct \HOL\ to add them
to its existing pretty-printing rules.

\begin{session}\begin{verbatim}
#loadt `sets_pp`;;

sets_rules = 
[((`term`,
   (Const_name(`CONST`,
               [Patt_child(Const_name(`EMPTY`, [])); Wild_children])),
   -),
  [],
  PF(H_box[(0, PO_constant `{}`)]))]
: print_rule list

sets_rules_fun = - : print_rule_function


File sets_pp loaded
() : void

#top_print (\t. pp (sets_rules_fun then_try
#                   hol_term_rules_fun then_try
#                   hol_type_rules_fun) `term` [] (pp_convert_term t));;
- : (term -> void)
\end{verbatim}\end{session}

\noindent
\ml{top\_print} is an \ML\ directive which given a function of type
{\small\verb%(%}{\it type\/} {\small\verb%->%} {\small\verb%void)%} installs
that function as a printer for any object of type {\it type}. \ml{pp} is an
\ML\ function which pretty-prints in a way (almost) compatible with the
standard \HOL\ pretty-printer. That is, when used with \ml{top\_print}, the
text it produces merges properly with the surrounding text produced by other
means. The first argument to \ml{pp} is a `rule function'. In the example this
is made by composing three `rule functions' together using \ml{then\_try}. The
rules of \ml{sets\_rules\_fun} are tried first. If none of these match, the
standard \HOL\ rules are tried, first those for terms, then those for
types\footnote{If no rules match, default rules will be used which print the
object as a tree structure.}. The second argument is the {\it context\/}
mentioned above. The third is a list of parameters, which is empty in the
example. The fourth argument is an object of a type defined within the
pretty-printer. The type represents a parse-tree. In the example, the term to
be pretty-printed is converted into a parse-tree using the function
\ml{pp\_convert\_term}. This function is defined within the pretty-printer,
specifically the part of it concerned with printing \HOL\ terms.

{\small\verb%EMPTY%} is now printed as {\small\verb%{}%}.

\begin{session}\begin{verbatim}
#"EMPTY:(*)set";;
"{}" : term
\end{verbatim}\end{session}

\noindent
We have not yet attached special syntax to non-empty sets.

\begin{session}\begin{verbatim}
#"INSERT 1 (EMPTY:(num)set)";;
"1 INSERT {}" : term
\end{verbatim}\end{session}

\noindent
The constant {\small\verb%INSERT%} is an infix. It is used to form a new set
from a set and the element to be added. We can add a rule to pretty-print this.

\begin{window}\begin{verbatim}
prettyprinter sets =

rules
   'term'::CONST(EMPTY(),**) -> [<h 0> "{}"];

   'term'::COMB(COMB(CONST(INSERT(),**),*elem),CONST(EMPTY(),**)) ->
           [<h 0> "{" *elem "}"];

end rules


end prettyprinter
\end{verbatim}\end{window}

\noindent
The new rule matches something of the form:

\begin{small}\begin{verbatim}
   (INSERT *elem) EMPTY
\end{verbatim}\end{small}

\noindent
The {\it metavariable\/} {\small\verb%*elem%} matches any tree, and becomes
bound to that tree. When {\small\verb%*elem%} is used within the format, the
pretty-printer is called recursively on the tree it is bound to. In the
example, if the new rule matches the tree to be printed, the sub-tree bound to
{\small\verb%*elem%} is printed enclosed within braces.

To print the sub-tree, the system tries to match rules to it, beginning from
the first rule, {\em not\/} the rule following the one just used. If neither
of our new rules match the sub-tree, the rules for standard \HOL\ will be
tried.

So, let's save the file, recompile it, load the generated code and link the
new rules into the pretty-printer.

\begin{session}\begin{verbatim}
#PP_to_ML false `sets` ``;;
() : void

#loadf `sets_pp`;;
..() : void

#top_print (\t. pp (sets_rules_fun then_try
#                   hol_term_rules_fun then_try
#                   hol_type_rules_fun) `term` [] (pp_convert_term t));;
- : (term -> void)
\end{verbatim}\end{session}

\noindent
Now we try the example again.

\begin{session}\begin{verbatim}
#"INSERT 1 (EMPTY:(num)set)";;
"{1}" : term
\end{verbatim}\end{session}

\noindent
Unfortunately our rules do not work for sets of two or more elements.

\begin{session}\begin{verbatim}
#"INSERT 1 (INSERT 2 (EMPTY:(num)set))";;
"1 INSERT {2}" : term

#"INSERT 1 (INSERT 2 (INSERT 3 (EMPTY:(num)set)))";;
"1 INSERT (2 INSERT {3})" : term
\end{verbatim}\end{session}

\noindent
The problem is that the second rule only matches when the set into which the
new element is being `inserted' is the empty set. We can make the pattern more
general by replacing the part of it which matches {\small\verb%EMPTY%} with a
metavariable.

\begin{window}\begin{verbatim}
prettyprinter sets =

rules
   'term'::CONST(EMPTY(),**) -> [<h 0> "{}"];

   'term'::COMB(COMB(CONST(INSERT(),**),*elem),*elems) ->
           [<h 0> "{" *elem "," *elems "}"];

end rules


end prettyprinter
\end{verbatim}\end{window}

\noindent
We process the file again.

\begin{session}\begin{verbatim}
#PP_to_ML false `sets` ``;;
() : void

#loadf `sets_pp`;;
..() : void

#top_print (\t. pp (sets_rules_fun then_try
#                   hol_term_rules_fun then_try
#                   hol_type_rules_fun) `term` [] (pp_convert_term t));;
- : (term -> void)
\end{verbatim}\end{session}

\noindent
Try the examples.

\begin{session}\begin{verbatim}
#"INSERT 1 (EMPTY:(num)set)";;
"{1,{}}" : term

#"INSERT 1 (INSERT 2 (EMPTY:(num)set))";;
"{1,{2,{}}}" : term
\end{verbatim}\end{session}

\noindent
Not quite what we wanted. Once we have matched the second rule, and sent out
the braces, we want to treat an {\small\verb%INSERT%} in a different way. We
can do this by adding an extra rule which matches in a different context to
the others.

\begin{window}\begin{verbatim}
prettyprinter sets =

rules
   'term'::CONST(EMPTY(),**) -> [<h 0> "{}"];

   'term_set'::COMB(COMB(CONST(INSERT(),**),*elem),*elems) ->
               [<h 0> 'term'::*elem "," *elems];

   'term'::COMB(COMB(CONST(INSERT(),**),*elem),*elems) ->
           [<h 0> "{" *elem "," 'term_set'::*elems "}"];

end rules


end prettyprinter
\end{verbatim}\end{window}

\noindent
We also change the last rule so that the recursive call it makes to process
the remainder of the set is made in the context {\small\verb%'term_set'%}.

\begin{session}\begin{verbatim}
#PP_to_ML false `sets` ``;;
() : void

#loadf `sets_pp`;;
..() : void

#top_print (\t. pp (sets_rules_fun then_try
#                   hol_term_rules_fun then_try
#                   hol_type_rules_fun) `term` [] (pp_convert_term t));;
- : (term -> void)
\end{verbatim}\end{session}

\begin{session}\begin{verbatim}
#"INSERT 1 (EMPTY:(num)set)";;
"{1,CONST(EMPTY)}" : term

#"INSERT 1 (INSERT 2 (EMPTY:(num)set))";;
"{1,2,CONST(EMPTY)}" : term
\end{verbatim}\end{session}

\noindent
We now have no rule to match {\small\verb%EMPTY%} when it appears as an
argument to {\small\verb%INSERT%}. Since we have also changed context, the
\HOL\ rules no longer apply either. So, {\small\verb%EMPTY%} is displayed as
its tree representation.

We could easily add a rule to match {\small\verb%EMPTY%}, so that the
{\small\verb%EMPTY%} is just thrown away. However, observe that we would still
have a trailing comma before the right-hand brace. Instead, we can add a rule
to deal with the last element of the set in a special way. Note that the new
rule must come before the other rule which applies in the context
{\small\verb%'term_set'%}, so that it takes priority over that rule.

\begin{window}\begin{verbatim}
prettyprinter sets =

rules
   'term'::CONST(EMPTY(),**) -> [<h 0> "{}"];

   'term_set'::COMB(COMB(CONST(INSERT(),**),*elem),CONST(EMPTY(),**)) ->
               [<h 0> 'term'::*elem];

   'term_set'::COMB(COMB(CONST(INSERT(),**),*elem),*elems) ->
               [<h 0> 'term'::*elem "," *elems];

   'term'::COMB(COMB(CONST(INSERT(),**),*elem),*elems) ->
           [<h 0> "{" *elem "," 'term_set'::*elems "}"];

end rules


end prettyprinter
\end{verbatim}\end{window}

\begin{session}\begin{verbatim}
#PP_to_ML false `sets` ``;;
() : void

#loadf `sets_pp`;;
..() : void

#top_print (\t. pp (sets_rules_fun then_try
#                   hol_term_rules_fun then_try
#                   hol_type_rules_fun) `term` [] (pp_convert_term t));;
- : (term -> void)
\end{verbatim}\end{session}

\begin{session}\begin{verbatim}
#"INSERT 1 (EMPTY:(num)set)";;
"{1,CONST(EMPTY)}" : term

#"INSERT 1 (INSERT 2 (EMPTY:(num)set))";;
"{1,2}" : term
\end{verbatim}\end{session}

\noindent
Our rules now work for sets of two or more elements, but not for sets of only
one element. This is because the last rule consumes the first
{\small\verb%INSERT%}, leaving just {\small\verb%EMPTY%} for a one element
set, and there is no rule to match {\small\verb%EMPTY%} in the context
{\small\verb%'term_set'%}. We need to change the last rule so that it matches
in the same situations, and displays the braces, but the tree it passes on in
the changed context is the tree it was given, not some sub-tree of it. We do
this by labelling a node of the tree with a metavariable. This is denoted by
{\small\verb%|*elems|%}. The sub-trees that were being bound to metavariables
no longer need to be. We can therefore use {\small\verb%*%} without a name to
mean `match any sub-tree'.

\begin{window}\begin{verbatim}
prettyprinter sets =

rules
   'term'::CONST(EMPTY(),**) -> [<h 0> "{}"];

   'term_set'::COMB(COMB(CONST(INSERT(),**),*elem),CONST(EMPTY(),**)) ->
               [<h 0> 'term'::*elem];

   'term_set'::COMB(COMB(CONST(INSERT(),**),*elem),*elems) ->
               [<h 0> 'term'::*elem "," *elems];

   'term'::|*elems|COMB(COMB(CONST(INSERT(),**),*),*) ->
           [<h 0> "{" 'term_set'::*elems "}"];

end rules


end prettyprinter
\end{verbatim}\end{window}

\begin{session}\begin{verbatim}
#PP_to_ML false `sets` ``;;
() : void

#loadf `sets_pp`;;
..() : void

#top_print (\t. pp (sets_rules_fun then_try
#                   hol_term_rules_fun then_try
#                   hol_type_rules_fun) `term` [] (pp_convert_term t));;
- : (term -> void)
\end{verbatim}\end{session}

\begin{session}\begin{verbatim}
#"INSERT 1 (EMPTY:(num)set)";;
"{1}" : term

#"INSERT 1 (INSERT 2 (EMPTY:(num)set))";;
"{1,2}" : term
\end{verbatim}\end{session}

\noindent
Having worked hard to get here, our rules are still not quite right. In all the
formats the objects displayed are composed horizontally. This means that all
the text must appear on the same line. If the textual representation of the
set is longer than the length of one line it will overflow. We need to specify
where the set can be broken between lines.

The obvious place to break the set is after a comma. So if the line length was
very small, we might get output of the form:

\begin{small}\begin{verbatim}
   {1,2,3,4,
    5,6}
\end{verbatim}\end{small}

\noindent
We can achieve this form of {\it inconsistent\/} breaking by some simple
changes to our rules.

\begin{window}\begin{verbatim}
prettyprinter sets =

rules
   'term'::CONST(EMPTY(),**) -> [<h 0> "{}"];

   'term_set'::COMB(COMB(CONST(INSERT(),**),*elem),CONST(EMPTY(),**)) ->
               [<h 0> 'term'::*elem];

   'term_set'::COMB(COMB(CONST(INSERT(),**),*elem),*elems) ->
               [<hv 0,0,0> [<h 0> 'term'::*elem ","] *elems];

   'term'::|*elems|COMB(COMB(CONST(INSERT(),**),*),*) ->
           [<h 0> "{" 'term_set'::*elems "}"];

end rules


end prettyprinter
\end{verbatim}\end{window}

\noindent
A box labelled with {\small\verb%<hv%}~{\it dx,di,dh\/}{\small\verb%>%} in a
format means that the components of the box should appear on the same line
separated by {\it dx\/} spaces, but if this is not possible, the components
which will not fit on the line can go on a new line separated from the
previous line by {\it dh\/} blank lines. The text of the new line begins
{\it di\/} spaces to the right of the beginning of the first component of the
box.

\vfill

In our example the box of this type has two components. The first is itself a
box which instructs the printer to display the element of the set followed by
a comma {\em which must go on the same line}. The second component is the
remainder of the set.

\vfill

Let's try out the modified rules.

\begin{session}\begin{verbatim}
#PP_to_ML false `sets` ``;;
() : void

#loadf `sets_pp`;;
..() : void

#top_print (\t. pp (sets_rules_fun then_try
#                   hol_term_rules_fun then_try
#                   hol_type_rules_fun) `term` [] (pp_convert_term t));;
- : (term -> void)
\end{verbatim}\end{session}

\begin{session}\begin{verbatim}
#let test = "INSERT 1 (INSERT 2 (INSERT 3 (INSERT 4 (INSERT 5 (INSERT 6
#(EMPTY:(num)set))))))";;
test = "{1,2,3,4,5,6}" : term

#set_margin 15;;
72 : int

#test;;
"{1,2,3,4,5,6}"
: term

#set_margin 14;;
15 : int

#test;;
"{1,
  2,3,4,5,6}"
: term

#set_margin 12;;
14 : int

#test;;
"{1,
  2,
  3,4,5,6}"
: term

#set_margin 72;;
12 : int
\end{verbatim}\end{session}

\vfill

\noindent
The rules are not doing what we want. This is because instead of having all
the elements of the set appear at the same level of a single box, they occur
at different levels in a chain of nested boxes\footnote{The nesting is not
explicit in the rules, but occurs by way of the recursive calls to the
printer.}. To be able to express a relationship between {\em all\/} the
elements of the set, we need to be able to grab them all in one call to the
printer, so that we may place them all at the same box level. There is a
special pattern which allows us to do this.

The {\it looping\/} construct consists of two patterns. The first is enclosed
within square brackets. It is followed by the second pattern. The combined
pattern tries to match the first pattern zero or more times, and when the first
no longer matches it tries to match the second exactly once. This probably
requires further explanation. We begin by looking at the rule for our example.

\begin{window}\begin{verbatim}
prettyprinter sets =

rules
   'term'::CONST(EMPTY(),**) -> [<h 0> "{}"];

   'term'::[COMB(COMB(CONST(INSERT(),**),*elems),<>COMB(**))]
           COMB(COMB(CONST(INSERT(),**),*elem),CONST(EMPTY(),**)) ->
           [<h 0> "{" [<hv 0,0,0> **[<h 0> *elems ","] *elem] "}"];

end rules


end prettyprinter
\end{verbatim}\end{window}

\vfill

\noindent
The {\small\verb%<>%} within the first part of the looping pattern is used to
label the sub-tree which will be used on the next match attempt (the next time
round the loop). This will typically appear without any pattern following it.
This would indicate that no restriction is being placed on the sub-tree to be
used on the next match attempt. However in the example, {\small\verb%<>%} is
followed by {\small\verb%COMB(**)%}. This specifies that the sub-tree must
have a {\small\verb%COMB%} as its root node.

The looping part of the pattern matches a chain of {\small\verb%INSERT%}s. The
representation of a set is such a chain. However, the last
{\small\verb%INSERT%} in the chain is not matched by the looping part of the
pattern, because the sub-tree to be used on the next match attempt does not
have {\small\verb%COMB%} as its root (This is assuming that the chain of
{\small\verb%INSERT%}s is terminated by an {\small\verb%EMPTY%}). For those
{\small\verb%INSERT%}s which are matched during the loop, the elements being
`inserted' are bound as a list to the metavariable {\small\verb%*elems%}.

When the looping terminates, we are left with something of the form:

\begin{small}\begin{verbatim}
   (INSERT *elem) EMPTY
\end{verbatim}\end{small}

\noindent
which as we have seen before is matched by the remainder of the pattern.

We bind the last element separately because it needs to be treated differently
in the format. (The last element is not followed by a comma).

The {\small\verb%**[<h 0> *elems ","]%} in the format expands to a sequence of
boxes, one for each element bound to {\small\verb%*elems%}, in which the
element is followed by a comma (on the same line).

There is a lot more to be said about these looping patterns and expanding
boxes, but we shall not go into it here. Instead let's see if the new rules
really do do what we want.

\begin{session}\begin{verbatim}
#PP_to_ML false `sets` ``;;
() : void

#loadf `sets_pp`;;
..() : void

#top_print (\t. pp (sets_rules_fun then_try
#                   hol_term_rules_fun then_try
#                   hol_type_rules_fun) `term` [] (pp_convert_term t));;
- : (term -> void)
\end{verbatim}\end{session}

\vfill

\begin{session}\begin{verbatim}
#test;;
"{1,2,3,4,5,6}" : term

#set_margin 14;;
72 : int

#test;;
"{1,2,3,4,5,
  6}"
: term

#set_margin 12;;
14 : int

#test;;
"{1,2,3,4,
  5,6}"
: term

#set_margin 72;;
12 : int
\end{verbatim}\end{session}

\vfill

\noindent
There is one more thing to say before leaving the example. The pretty-printer
for \HOL\ terms uses a parameter called {`}{\small\verb%prec%}{'} to hold the
precedence of the parent operator. If a rule does not explicitly modify this
parameter, it is passed on unchanged to recursive calls of the printer. The
braces of the set notation prevent any ambiguity, so we do not need to know
the precedence of the parent operator. If within a set we consider the
separating commas to have the lowest possible precedence, then the elements of
the set should not appear enclosed within parentheses. We force this by making
{`}{\small\verb%prec%}{'} have its highest possible value (which corresponds
to the lowest precedence) for all recursive calls of the printer.

\begin{window}\begin{verbatim}
prettyprinter sets =

abbreviations
   max_prec = {apply0 max_term_prec};

end abbreviations


rules
   'term'::CONST(EMPTY(),**) -> [<h 0> "{}"];

   'term'::[COMB(COMB(CONST(INSERT(),**),*elems),<>COMB(**))]
           COMB(COMB(CONST(INSERT(),**),*elem),CONST(EMPTY(),**)) ->
           [<h 0> "{" [<hv 0,0,0> **[<h 0> *elems with
                                                     prec := max_prec
                                                  end with
                                           ","]
                                  *elem with
                                           prec := max_prec
                                        end with] "}"];

end rules


end prettyprinter
\end{verbatim}\end{window}

\vfill

\noindent
\ml{max\_prec} is a value suitable for use within the pretty-printing
language. It is derived from the value of the \ML\ identifier
\ml{max\_term\_prec}. The value of \ml{max\_term\_prec} is the largest
possible precedence value (lowest precedence) for a \HOL\ `operator'. The
transformation from \ml{max\_term\_prec} to \ml{max\_prec} is explained in
Chapter~\ref{functions}.


\section{CAUTION!}

\setcounter{sessioncount}{1}
\newwindow{{\small\tt bad.pp}}

The previous section illustrates how the \HOL\ pretty-printer can be extended.
It should not be hard to see that the same methods could be used to
{\em modify\/} the \HOL\ pretty-printer. For example, consider the following
pretty-printer which performs an exceedingly undesirable transformation.

\begin{window}\begin{verbatim}
prettyprinter bad =

rules
   'term'::CONST(F(),**) -> [<h 0> "T"];

end rules


end prettyprinter
\end{verbatim}\end{window}

\noindent
We can make use of this in a \HOL\ session. First we enter \HOL\ and load the
library \ml{prettyp}.

\begin{session}\begin{verbatim}

          _  _    __    _      __    __
   |___   |__|   |  |   |     |__|  |__|
   |      |  |   |__|   |__   |__|  |__|
   
          Version 2

#load_library `prettyp`;;
Loading library `prettyp` ...
Updating help search path
.............................................................................
.............................................................................
.............................................................................
.............................................................
Library `prettyp` loaded.
() : void
\end{verbatim}\end{session}

\noindent
Now we look at the definition of {\it false}.

\begin{session}\begin{verbatim}
#let test = F_DEF;;
test = |- F = (!t. t)
\end{verbatim}\end{session}

\noindent
The new pretty-printer can be compiled, loaded and linked into the \HOL\
system:

\begin{session}\begin{verbatim}
#PP_to_ML false `bad` ``;;
() : void

#loadf `bad_pp`;;
..() : void

#top_print (\t. pp (bad_rules_fun then_try
#                   hol_thm_rules_fun then_try
#                   hol_term_rules_fun then_try
#                   hol_type_rules_fun) `thm` [] (pp_convert_thm t));;
- : (thm -> void)
\end{verbatim}\end{session}

\noindent
The result is a theorem which, although perfectly valid in the underlying
representation, appears to the user in a very unpleasant form.

\begin{session}\begin{verbatim}
#test;;
|- T = (!t. t)
\end{verbatim}\end{session}
			 % intro: getting and installing hol
   \chapter{Introduction to ML}
\label{ML}

This chapter is a brief introduction to the meta-language \ML.  The aim is just
to give a feel  for what  it is  like to  interact with  the language.   A more
detailed introduction can be found in \DESCRIPTION.

\section{How to interact with ML}

\ML\ is an interactive programming language like Lisp. At top level one can
evaluate expressions and perform declarations. The former results in the
expression's value and type being printed, the latter in a value being
bound to a name.  

The normal way to interact with \ML\ is to configure the workstation screen so that
there are two windows:
\begin{myenumerate}
\item An editor window into which \ML\ commands are initially typed and recorded.
\item A shell window (or non-Unix equivalent) which is used to evaluate the
commands.
\end{myenumerate}

\noindent A common way to achieve this is to work inside \ml{Emacs} with a text
window and a shell window.

After typing a command into the edit (text) window it can be transferred to the
shell and evaluated in \HOL\ by `cut-and-paste'. In \ml{Emacs} this is done by
copying the text into a buffer and then `yanking' it into the shell. On an {\small
IBM PC} it can be done using a proprietary window system; on a Macintosh it can be
done using the Mac's window system. The advantage of working via an editor is that
if the command has an error, then the text can simply be edited and used again; it
also records the commands in a file which can then be used again (via a batch load)
later. In \ml{Emacs}, the shell window also records the session, including both
input from the user and the system's response. The sessions in this tutorial were
produced this way. These sessions are split into segments displayed in boxes with a
number in their top right hand corner (to indicate their position in the complete
session).

The interactions in these boxes should be understood as occurring in
sequence.  For example, variable bindings made in earlier boxes are assumed
to persist to later ones.  To enter the \HOL\ system one types
{\small\verb%hol%} to Unix.\footnote{The Unix prompt is {\small\tt \%}.}
The \HOL\ system then prints a sign-on message and puts one into \ML.  The
\ML\ prompt is {\small\verb%#%}, so lines beginning with {\small\verb%#%}
are typed by the user and other lines are the system's response.

\setcounter{sessioncount}{1}
\begin{session}\begin{verbatim}
% hol

          _  _    __    _      __    __
   |___   |__|   |  |   |     |__|  |__|
   |      |  |   |__|   |__   |__|  |__|
   
          Version 2.0 (Sun3/Franz), built on Sep 1 1991 

#1.[2;3;4;5];;
[1; 2; 3; 4; 5] : int list
\end{verbatim}
\end{session}

The \ML\ expression {\small\verb%1.[2;3;4;5]%} has the form $e_1\ op\ e_2$ where
$e_1$ is the expression {\small\verb%1%} (whose value is the integer $1$), $e_2$ is
the expression {\small\verb%[2;3;4;5]%} (whose value is a list of four integers)
and $op$ is the infixed operator `{\small\verb%.%}' which is like Lisp's {\it cons}
function. Other list processing functions include {\small\verb%hd%} ($car$ in
Lisp), {\small\verb%tl%} ($cdr$ in Lisp) and {\small\verb%null%} ($null$ in Lisp).
The double semicolon `{\small\verb%;;%}' terminates a top-level phrase.  The
system's response is shown on the line not starting with a prompt.  It consists of
the value of the expression followed, after a colon, by its type. The \ML\
type 
checker infers the type of expressions using methods invented by Robin Milner
\cite{Milner-types}. The type {\small\verb%int list%} is the type of `lists of
integers'; {\small\verb%list%} is a unary type operator.  The type system of \ML\
is very similar to the type system of the \HOL\ logic which is explained in
Chapter~\ref{HOLlogic}.

The value of the last expression evaluated at top-level in \ML\ is always
remembered in a variable called {\small\verb%it%}.

\begin{session}
\begin{verbatim}
#let l = it;;
l = [1; 2; 3; 4; 5] : int list

#tl l;;
[2; 3; 4; 5] : int list

#hd it;;
2 : int

#tl(tl(tl(tl(tl l))));;
[] : int list
\end{verbatim}
\end{session}

Following standard $\lambda$-calculus usage, the application of a function
$f$ to an argument $x$ can be written without brackets as $f\ x$ (although
the more conventional $f${\small\verb%(%}$x${\small\verb%)%} is also
allowed).  The expression $f\ x_1\ x_2\ \cdots\ x_n$ abbreviates the less
intelligible expression {\small\verb%(%}$\cdots${\small\verb%((%}$f\ 
x_1${\small\verb%)%}$x_2${\small\verb%)%}$\cdots${\small\verb%)%}$x_n$
(function application is left associative).

Declarations have the form {\small\verb%let
%}$x_1${\small\verb%=%}$e_1${\small\verb% and %}$\cdots
${\small\verb% and %}$x_n${\small\verb%=%}$e_n$ and result in the value of
each expression $e_i$ being bound to the name $x_i$.

\begin{session}
\begin{verbatim}
#let l1 = [1;2;3] and l2 = [`a`;`b`;`c`];;
l1 = [1; 2; 3] : int list
l2 = [`a`; `b`; `c`] : string list
\end{verbatim}
\end{session}

\ML\ expressions like {\small\verb%`a`%}, {\small\verb%`b`%},
{\small\verb%`foo`%} \etc\ are {\it strings\/} and have type
{\small\verb%string%}. Any sequence of {\small ASCII} characters can be
written between the quotes. The function {\small\verb%words%} splits a
single string into a list of single character strings, using space as
separator.

\begin{session}
\begin{verbatim}
#words`a b c`;;
[`a`; `b`; `c`] : string list
\end{verbatim}
\end{session}

An expression of the form
{\small\verb%(%}$e_1${\small\verb%,%}$e_2${\small\verb%)%} evaluates to a
pair of the values of $e_1$ and $e_2$. If $e_1$ has type $\sigma_1$ and
$e_2$ has type $\sigma_2$ then
{\small\verb%(%}$e_1${\small\verb%,%}$e_2${\small\verb%)%} has type
$\sigma_1${\small\verb%#%}$\sigma_2$.  A tuple
{\small\verb%(%}$e_1${\small\verb%,%}$\ldots${\small\verb%,%}$e_n${\small\verb%)%}
is equivalent to
{\small\verb%(%}$e_1${\small\verb%,(%}$e_2${\small\verb%,%}$\ldots${\small\verb%,%}$e_n${\small\verb%))%}
(\ie\ `{\small\verb%,%}' is right associative).  The brackets around pairs
and tuples are optional; the system doesn't print them. The first and
second components of a pair can be extracted with the \ML\ functions
{\small\verb%fst%} and {\small\verb%snd%} respectively.

\begin{session}
\begin{verbatim}
#(1,true,`abc`);;
(1,true, `abc`) : (int # bool # string)

#snd it;;
(true, `abc`) : (bool # string)


#fst it;;
true : bool
\end{verbatim}
\end{session}

\noindent The \ML\ expressions {\small\verb%true%} and {\small\verb%false%}
denote the two truth values of type {\small\verb%bool%}.

\ML\ types can contain the {\it type variables\/} {\small\verb%*%},
{\small\verb%**%}, {\small\verb%***%}, \etc\ Such types are called {\it
polymorphic\/}. A function with a polymorphic type should be thought of as
possessing all the types obtainable by replacing type variables by types.
This is illustrated below with the function {\small\verb%zip%}.

Functions are defined with declarations of the form {\small\verb%let%}$\ f\
v_1\ \ldots\ v_n$ \ml{=} $e$ where each $v_i$ is either a variable or a pattern
built out of variables.\footnote{The chapters on ML in \DESCRIPTION\ give
exact details.} Recursive functions are declared with
{\small\verb%letrec%} instead of {\small\verb%let%}.

The function {\small\verb%zip%}, below, converts a pair of lists
{\small\verb%([%}$x_1${\small\verb%;%}$\ldots${\small\verb%;%}$x_n${\small\verb%],
[%}$y_1${\small\verb%;%}$\ldots${\small\verb%;%}$y_n${\small\verb%])%} to a
list of pairs
{\small\verb%[(%}$x_1${\small\verb%,%}$y_1${\small\verb%);%}$\ldots${\small\verb%;(%}$x_n${\small\verb%,%}$y_n${\small\verb%)]%}.

\begin{session}
\begin{verbatim}
#letrec zip(l1,l2) =
#if null l1 or null l2 
# then []
# else (hd l1,hd l2).zip(tl l1,tl l2);;
zip = - : ((* list # ** list) -> (* # **) list)

#zip([1;2;3],[`a`;`b`;`c`]);;
[(1,`a`); (2,`b`); (3,`c`)] : (int # string) list
\end{verbatim}
\end{session}

Functions may be {\it curried\/}, \ie\ take their arguments `one at a time'
instead of as a tuple.  This is illustrated with the function
{\small\verb%curried_zip%} below:

\begin{session}
\begin{verbatim}
#let curried_zip l1 l2 = zip(l1,l2);;
curried_zip = - : (* list -> ** list -> (* # **) list)

#let zip_num = curried_zip [0;1;2];;
zip_num = - : (* list -> (int # *) list)

#zip_num [`a`;`b`;`c`];;
[(0, `a`); (1, `b`); (2, `c`)] : (int # string) list
\end{verbatim}
\end{session}


Curried functions are useful because they can be `partially applied' as
illustrated above by the partial application of {\small\verb%curried_zip%}
to {\small\verb%[0;1;2]%} which results in the function
{\small\verb%zip_num%}.

The evaluation of an expression either {\it succeeds\/} or {\it fails\/}.
In the former case, the evaluation returns a value; in the latter case the
evaluation is aborted and a failure string (usually the name of the
function that caused the failure) is passed to whatever invoked the
evaluation. This context can either propagate the failure (this is the
default) or it can {\it trap\/} it. These two possibilities are illustrated
below.  A failure trap is an expression of the form
$e_1${\small\verb%?%}$e_2$. An expression of this form is evaluated by
first evaluating $e_1$. If the evaluation succeeds (\ie\ doesn't fail) then
the value of $e_1${\small\verb%?%}$e_2$ is the value of $e_1$.  If the
evaluation of $e_1$ fails, then the value of $e_1${\small\verb%?%}$e_2$ is
obtained by evaluating $e_2$.

\begin{session}
\begin{verbatim}
#hd[];;
evaluation failed     hd 

#hd[] ? `hd applied to empty list`;;
`hd applied to empty list` : string
\end{verbatim}
\end{session}

The sessions  above  are enough  to give  a feel  for \ML.   For  a much longer
example, which illustrates most of the main features  of the  language, see the
introduction to \ML\ in  \DESCRIPTION.

In the  next chapter,  the logic  supported by  the \HOL\  system (higher order
logic) will be introduced, together with the tools in \ML\ for manipulating it.











				 % intro to ml
   % Revised version of Part II, Chapter 9 of HOL DESCRIPTION
% Incorporates material from both of chapters 9 and 10 of the old
% version of DESCRIPTION
% Written by Andrew Pitts
% 8 March 1991
% revised August 1991
\chapter{Syntax and Semantics}\label{logic}

\section{Introduction}
\label{introduction}

This chapter describes the syntax and set-theoretic semantics of the
logic supported by the \HOL\ system, which is a variant of
Church's\index{Church, A.} simple theory of types \cite{Church} and
will henceforth be called the \HOL\ logic, or just \HOL.  The
meta-language for this description will be English, enhanced with
various mathematical notations and conventions.  The object language
of this description is the \HOL\ logic.  Note that there is a
`meta-language', in a different sense, associated with the \HOL\
logic, namely the programming language \ML.  This is the language used
to manipulate the \HOL\ logic by users of the system, and is described
in detail in Part~\ref{MLpart} of \DESCRIPTION.  It is hoped that because
of context, no confusion results from these two uses of the word
`meta-language'.  When \ML\ is described in Part~\ref{MLpart}, \ML\ is the
object language under consideration---and English is again the
meta-language!

The \HOL\ syntax contains syntactic categories of types and terms whose
elements are intended to denote respectively certain sets and elements
of sets. This set theoretic interpretation will be developed along side
the description of the \HOL\ syntax, and in the next chapter the \HOL\
proof system will be shown to be sound for reasoning about properties
of the set theoretic model.\footnote{There are other, `non-standard'
models of \HOL, which will not concern us here.} This model is given in
terms of a fixed set of sets $\cal U$, which will be called the {\em
universe\/}\index{universe, in semantics of HOL logic@universe, in
semantics of \HOL\ logic} and which is assumed to have the following
properties. 
\begin{description}

\item[Inhab] Each element of $\cal U$ is a non-empty set.  

\item[Sub] If $X\in{\cal U}$ and $\emptyset\not=Y\subseteq X$, then 
$Y\in{\cal U}$.

\item[Prod] If $X\in{\cal U}$ and $Y\in{\cal U}$, then $X\times
Y\in{\cal U}$. The set $X\times Y$ is the cartesian product,
consisting of ordered pairs $(x,y)$ with $x\in X$ and $y\in Y$, with
the usual set-theoretic coding of ordered pairs, \viz\ 
$(x,y)=\{\{x\},\{x,y\}\}$.

\item[Pow] If $X\in{\cal U}$, then the powerset
$P(X)=\{Y:Y\subseteq X\}$ is also an element of $\cal U$.

\item[Infty] $\cal U$ contains a distinguished infinite set $\inds$.

\item[Choice] There is a distinguished element $\ch\in\prod_{X\in{\cal
U}}X$. The elements of the product $\prod_{X\in{\cal U}}X$ are
(dependently typed) functions: thus for all $X\in{\cal U}$, $X$ is
non-empty by {\bf Inhab} and $\ch(X)\in X$ witnesses this.

\end{description}
There are some consequences of these assumptions which will be needed.
In set theory functions are identified with their graphs, which are
certain sets of ordered pairs. Thus the set $X\fun Y$ of all functions
from a set $X$ to a set $Y$ is a subset of $P(X\times Y)$; and it is a
non-empty set when $Y$ is non-empty. So {\bf Sub}, {\bf Prod} and {\bf
Pow} together imply that $\cal U$ also satisfies
\begin{description} 

\item[Fun] If $X\in{\cal U}$ and $Y\in{\cal U}$, then $X\fun Y\in{\cal U}$.

\end{description}
By iterating {\bf Prod}, one has that the cartesian product of any
finite, non-zero number of sets in $\cal U$ is again in $\cal U$.
$\cal U$ also contains the cartesian product of no sets, which is to
say that it contains a one-element set (by virtue of {\bf Sub} applied
to any set in ${\cal U}$---{\bf Infty} guarantees there is one); for
definiteness, a particular one-element set will be singled out.
\begin{description}

\item[Unit] $\cal U$ contains a distinguished one-element set $1=\{0\}$.

\end{description}
Similarly, because of {\bf Sub} and {\bf Infty}, $\cal U$ contains
two-element sets, one of which will be singled out.
\begin{description}

\item[Bool] $\cal U$ contains a distinguished two-element set
$\two=\{0,1\}$.

\end{description}

The above assumptions on $\cal U$ are weaker than those imposed on a
universe of sets by the axioms of
Zermelo-Fraenkel\index{Zermelo-Fraenkel set theory} set theory with the
Axiom of Choice (\theory{ZFC})\index{axiom of choice}\index{ZFC@\ml{ZFC}}, 
principally because $\cal U$ is not
required to satisfy any form of the Axiom of 
Replacement\index{axiom of replacement}. 
Indeed, it is possible to prove the existence of a set
$\cal U$ with the above properties from the axioms of \theory{ZFC}.
(For example one could take $\cal U$ to consist of all non-empty sets
in the von~Neumann cumulative hierarchy formed before stage
$\omega+\omega$.) Thus, as with many other pieces of mathematics, it is
possible in principal to give a completely formal version within
\theory{ZFC} set theory of the semantics of the \HOL\ logic to be given
below. 

\section{Types}
\label{types}

The types\index{type constraint!in HOL logic@in \HOL\ logic} of the
\HOL\ logic are expressions that denote sets (in the universe $\cal U$).
Following tradition, 
$\sigma$, possibly decorated with subscripts or primes, is used to
range over arbitrary types.

There are four kinds of types in the \HOL\ logic. These can be described
informally by the following {\small BNF} grammar, 
in which $\alpha$ ranges
over type variables, {\sl c} ranges over atomic types and {\sl op} ranges over
type operators.

\newlength{\ttX}
\settowidth{\ttX}{\tt X}
\newcommand{\tyvar}{\setlength{\unitlength}{\ttX}\begin{picture}(1,6)
\put(.5,0){\makebox(0,0)[b]{\footnotesize type variables}}
\put(0,1.5){\vector(0,1){4.5}}
\end{picture}}
\newcommand{\tyatom}{\setlength{\unitlength}{\ttX}\begin{picture}(1,6)
\put(.5,2.3){\makebox(0,0)[b]{\footnotesize atomic types}}
\put(.5,3.3){\vector(0,1){2.6}}
\end{picture}}
\newcommand{\funty}{\setlength{\unitlength}{\ttX}\begin{picture}(1,6)
\put(.5,1.5){\makebox(0,0)[b]{\footnotesize function types}}
\put(.5,0){\makebox(0,0)[b]{\footnotesize (domain $\sigma_1$, range $\sigma_2$)}}
\put(1,2.5){\vector(0,1){3.5}}
\end{picture}}
\newcommand{\cmpty}{\setlength{\unitlength}{\ttX}\begin{picture}(1,6)
\put(2,3.3){\makebox(0,0)[b]{\footnotesize compound types}}
\put(1.9,4.5){\vector(0,1){1.5}}
\end{picture}}
%
$$\sigma\quad ::=\quad {\mathord{\mathop{\alpha}\limits_{\tyvar}}}
        \quad\mid\quad{\mathord{\mathop{\sl c}\limits_{\tyatom}}}
        \quad\mid\quad\underbrace{(\sigma_1, \ldots , \sigma_n){\sl
        op}}_{\cmpty} 
        \quad\mid\quad\underbrace{\sigma_1\fun\sigma_2}_{\funty}$$

\noindent In more detail, the four kinds of types are as follows.

\begin{enumerate}

\item {\bf Type variables:}\index{type variables, in HOL logic@type variables, in \HOL\ logic!abstract form of} these stand for arbitrary
sets in the universe.  In Church's original formulation of simple type
theory, type variables are part of the meta-language and are used to
range over object language types.  Proofs containing type variables
were understood as proof schemes (\ie\ families of proofs). To support
such proof schemes {\it within} the \HOL\ logic, type variables have
been added to the object language type system.\footnote{This technique
was invented by Robin Milner for the object logic \PPL\ of his \LCF\
system.}

\item {\bf Atomic types:}\index{atomic types, in HOL logic@atomic types, in \HOL\ logic} these denote fixed sets in the universe. Each
theory determines a particular collection of atomic types.  For
example, the standard atomic types \ty{bool} and \ty{ind} denote,
respectively, the distinguished two-element set $\two$ and the
distinguished infinite set $\inds$.

\item {\bf Compound types:}\index{compound types, in HOL logic@compound types, in \HOL\ logic!abstract form of} These have the
form $(\sigma_1,\ldots,\sigma_n)\ty{op}$, where $\sigma_1$, $\dots$,
$\sigma_n$ are the argument types and $op$ is a {\it type operator\/}
of arity $n$.  Type operators denote operations for constructing sets.
The type $(\sigma_1,\ldots,\sigma_n)\ty{op}$ denotes the set resulting
from applying the operation denoted by $op$ to the sets denoted by
$\sigma_1$, $\dots$, $\sigma_n$.  For example,
\ty{list} is a type operator with arity 1.  It denotes the operation
of forming all finite lists of elements from a given set.  Another
example is the type operator \ty{prod} of arity 2 which denotes the
cartesian product operation.  The type $(\sigma_1,\sigma_2)\ty{prod}$
is written as $\sigma_1\times\sigma_2$.  

\item {\bf Function types:}\index{function types, in HOL logic@function types, in \HOL\ logic!abstract form of} If $\sigma_1$
and $\sigma_2$ are types, then $\sigma_1\fun\sigma_2$ is the function
type with {\it domain\/} $\sigma_1$ and {\it range} $\sigma_2$. It
denotes the set of all (total) functions from the set denoted by its
domain to the set denoted by its range. (In the literature
$\sigma_1\fun\sigma_2$ is written without the arrow and
backwards---\ie\ as $\sigma_2\sigma_1$.) Note that syntactically
$\fun$ is simply a distinguished type operator of arity 2 written with
infix notation. It is singled out in the definition of \HOL\ types
because it will always denote the same operation in any
model of a \HOL\ theory---in contrast to the other type operators which
may be interpreted differently in different models. (See
Section~\ref{semantics of types}.)


\end{enumerate}

It turns out to be convenient to identify atomic types with
compound types constructed with $0$-ary type operators.  For example,
the atomic type \ty{bool} of truth-values can be regarded as being an
abbreviation for $()\ty{bool}$.  This identification will be made in
the technical details that follow, but in the informal presentation
atomic types will continue to be distinguished from compound types,
and $()c$ will still be written as $c$.

\subsection{Type structures}
\label{type structures}
\index{type structure, in HOL logic@type structure, in \HOL\ logic}

The term `type constant' is used to cover both atomic types and type
operators.  It is assumed that an infinite set {\sf
TyNames} of the {\em names of type constants\/} is given.  The greek
letter $\nu$ is used to range over arbitrary members of {\sf TyNames},
{\sl c} will continue to be used to range over the names of atomic
types (\ie\ $0$-ary type constants), and {\sl op} is used to range
over the names of type operators (\ie\ $n$-ary type constants, where
$n>0$).

It is assumed that an infinite set {\sf TyVars} of {\em type
variables\/}\index{type variables, in HOL logic@type variables, in \HOL\ logic}
 is given.  Greek letters $\alpha,\beta,\ldots$, possibly with
subscripts or primes, are used to range over {\sf Tyvars}.  The sets
{\sf TyNames} and {\sf TyVars} are assumed disjoint.

A {\it type structure\/} is a set $\Omega$ of type constants. A {\it
type constant\/}\index{type constants, in HOL logic@type constants, in \HOL\ logic} is a pair $(\nu,n)$ where $\nu\in{\sf TyNames}$ is the
name of the constant and $n$ is its arity.  Thus $\Omega\subseteq{\sf
TyNames}\times\natnums$ (where $\natnums$ is the set of natural
numbers).  It is assumed that no two distinct type constants have the
same name,
\ie\ whenever $(\nu, n_1)\in\Omega$ and
$(\nu, n_2)\in\Omega$, then $n_1 = n_2$.

The set {\sf Types}$_{\Omega}$ of types over a structure ${\Omega}$
can now be defined as the smallest set such that:

\begin{itemize}

\item {\sf TyVars}$\ \subseteq\ ${\sf Types}$_{\Omega}$.

\item If $(\nu,0)\in\Omega$ then $()\nu\in{\sf Types}_{\Omega}$.

\item If $(\nu,n)\in\Omega$ and $\sigma_i\in{\sf Types}_{\Omega}$ for
$1\leq i\leq n$, then $(\sigma_1,\ \ldots\ ,\sigma_n)\nu\in{\sf
Types}_{\Omega}$.

\item If $\sigma_1\in{\sf Types}_{\Omega}$ and $\sigma_2\in{\sf
Types}_{\Omega}$ then $\sigma_1\fun\sigma_2\in{\sf Types}_{\Omega}$.


\end{itemize}
The type operator $\fun$ is assumed to associate\index{type operators, in HOL logic@type operators, in \HOL\ logic!associativity of} to the
right, so that
\[ 
\sigma_1\fun\sigma_2\fun\ldots\fun \sigma_n\fun\sigma 
\]
abbreviates
\[ 
\sigma_1\fun(\sigma_2\fun\ldots\fun (\sigma_n\fun\sigma)\ldots) 
\]
The notation $tyvars(\sigma)$ is used to denote the set of type
variables occurring in $\sigma$.

\subsection{Semantics of types}
\label{semantics of types}


A {\em model} $M$ of a type structure $\Omega$ is specified by giving
for each type constant $(\nu,n)$ an $n$-ary function
\[ 
M(\nu):{\cal U}^{n}\longrightarrow{\cal U} 
\]
Thus given sets $X_1,\ldots,X_n$ in the universe $\cal U$,
$M(\nu)(X_1,\ldots,X_n)$ is also a set in the universe.  In case $n=0$,
this amounts to specifying an element $M(\nu)\in{\cal U}$ for the
atomic type $\nu$.

Types containing no type variables are called {\it monomorphic},
whereas those that do contain type variables are called {\it
polymorphic}\index{polymorphic types, in HOL logic@polymorphic types, in \HOL\ logic}\index{types, in HOL logic@types, in \HOL\ logic!polymorphic}. What is the meaning of a polymorphic type? One can
only say what set a polymorphic type denotes once one has instantiated
its type variables to particular sets. So its overall meaning is not a
single set, but is rather a set-valued function, ${\cal
U}^{n}\longrightarrow{\cal U}$, assigning a set for each particular
assignment of sets to the relevant type variables. The arity $n$
corresponds to the number of type variables involved. It is convenient
in this connection to be able to consider a type variable to be
involved in the semantics of a type $\sigma$ whether or not it
actually occurs in $\sigma$, leading to the notion of a
type-in-context.

A {\em type context}\index{type context}, $\alpha\!s$, is simply a
finite (possibly empty) list of {\em distinct\/} type variables
$\alpha_{1},\ldots,\alpha_{n}$.  A {\em
type-in-context\/}\index{type-in-context} is a pair, written
$\alpha\!s.\sigma$, where $\alpha\!s$ is a type context, $\sigma$ is a
type (over some given type structure) and all the type variables
occurring in $\sigma$ appear somewhere in the list $\alpha\!s$. The
list $\alpha\!s$ may also contain type variables which do not occur in
$\sigma$. 

For each $\sigma$ there are minimal contexts $\alpha\!s$ for which
$\alpha\!s.\sigma$ is a type-in-context, which only differ by the order
in which the type variables of $\sigma$ are listed in $\alpha\!s$. In
order to select one such context, let us assume that  {\sf TyVars}
comes with a fixed total order and define the {\em
canonical}\index{canonical contexts, in HOL logic@canonical contexts, in \HOL\ logic!of types} context of the type $\sigma$ to consist of
exactly the type variables it contains, listed in order.\footnote{It is
possible to work with unordered contexts, specified by finite sets
rather than lists, but we choose not to do that since it mildly
complicates the definition of the semantics to be given
below.}

Let $M$ be a model of a type structure $\Omega$. For each
type-in-context
$\alpha\!s.\sigma$ over $\Omega$, define a function
\[ 
\den{\alpha\!s.\sigma}_{M}:{\cal U}^{n}\longrightarrow{\cal U}
\]
(where $n$ is the length of the context) by induction on the structure
of $\sigma$ as follows.
\begin{itemize}
 
\item If $\sigma$ is a type variable, it must be $\alpha_{i}$ for some unique
$i=1,\ldots,n$ and then $\den{\alpha\!s.\sigma}_{M}$ is the $i$\/th
projection function, which sends $(X_{1},\ldots,X_{n})\in{\cal U}^{n}$
to $X_{i}\in{\cal U}$.

\item If $\sigma$ is a function type\index{function types, in HOL logic@function types, in \HOL\ logic!formal semantics of}
$\sigma_{1}\fun\sigma_{2}$, then $\den{\alpha\!s.\sigma}_M$ sends
$X\!s\in{\cal U}^n$ to the set of all functions 
from $\den{\alpha\!s.\sigma_1}_M(X\!s)$ to
$\den{\alpha\!s.\sigma_2}_M(X\!s)$. (This makes
use of the property {\bf Fun} of $\cal U$.)  

\item If $\sigma$ is a
compound type $(\sigma_{1},\ldots,\sigma_{m})\nu$, then
$\den{\alpha\!s.\sigma}_{M}$ sends $X\!s$ to
$M(\nu)(S_{1},\ldots,S_{m})$ where each $S_{j}$ is
$\den{\alpha\!s.\sigma_{j}}_{M}(X\!s)$.
\end{itemize}
One can now define the meaning of a type $\sigma$ in a model $M$ to be
the function
\[ 
\den{\sigma}_{M}:{\cal U}^{n}\longrightarrow{\cal U} 
\]
given by $\den{\alpha\!s.\sigma}_{M}$, where $\alpha\!s$ is the
canonical context of $\sigma$. If $\sigma$ is monomorphic, then $n=0$
and $\den{\sigma}_{M}$ can be identified with the element
$\den{\sigma}_{M}()$ of $\cal U$. When the particular model $M$ is
clear from the context, $\den{\_}_{M}$ will be written $\den{\_}$.

To summarize, given a model in $\cal U$ of a type structure $\Omega$,
the semantics interprets monomorphic types over $\Omega$ as sets in
$\cal U$ and more generally, interprets polymorphic types\index{types, in HOL logic@types, in \HOL\ logic!polymorphic}\index{polymorphic types, in HOL logic@polymorphic types, in \HOL\ logic!formal semantics of} involving $n$ type variables as $n$-ary functions ${\cal
U}^{n}\longrightarrow{\cal U}$ on the universe.  Function types are
interpreted by full function sets.

\medskip

\noindent{\bf Examples\ } 
Suppose that $\Omega$ contains a type constant $({\sl b},0)$ and that
the model $M$ assigns the set $\two$ to $\sl b$. Then:
\begin{enumerate}
 
\item $\den{{\sl b}\fun{\sl b}\fun{\sl b}}=\two\fun\two\fun\two\in{\cal U}$.

\item $\den{(\alpha\fun{\sl b})\fun\alpha}:{\cal U}\longrightarrow{\cal U}$
is the function sending $X\in{\cal U}$ to $(X\fun\two)\fun X\in{\cal U}$.

\item $\den{\alpha,\beta . (\alpha\fun{\sl b})\fun\alpha}:{\cal
U}^{2}\longrightarrow{\cal U}$ is the function sending $(X,Y)\in{\cal
U}^{2}$ to $(X\fun\two)\fun X\in{\cal U}$.

\end{enumerate}

\medskip

\noindent{\bf Remark\ } 
A more traditional approach to the semantics would involve giving
meanings to types in the presence of `environments' assigning sets in
$\cal U$ to all type variables. The use of types-in-contexts is almost
the same as using partial environments with finite domains---it is
just that the context ties down the admissible domain to a particular
finite (ordered) set of type variables. At the level of types there is
not much to choose between the two approaches.  However for the syntax
and semantics of terms to be given below, where there is a dependency
both on type variables and on individual variables, the approach used
here seems best.

\subsection{Instances and substitution}
\label{instances-and-substitution}

If $\sigma$ and $\tau_1,\ldots,\tau_n$ are types over a type structure
$\Omega$, 
\[
\sigma[\tau_{1},\ldots,\tau_{p}/\beta_{1},\ldots,\beta_{p}]
\]
will denote the type resulting from the simultaneous substitution for
each $i=1,\ldots,p$ of
$\tau_i$ for the type variable $\beta_i$ in $\sigma$.
The resulting type is called an {\it instance\/}\index{types, in HOL logic@types, in \HOL\ logic!instantiation of} of $\sigma$. The
following lemma about instances will be useful later; it is proved by
induction on the structure of $\sigma$.

\medskip

\noindent{\bf Lemma 1\ }{\it
Suppose that $\sigma$ is a type containing distinct type variables
$\beta_1,\ldots,\beta_p$ and that
$\sigma'=\sigma[\tau_{1},\ldots,\tau_{n}/\beta_1,\ldots,\beta_p]$ is
an instance of $\sigma$.  Then the types $\tau_1,\ldots,\tau_p$ are
uniquely determined by $\sigma$ and $\sigma'$.}

\medskip

We also need to know how the semantics of types behaves with respect
to substitution:

\medskip

\noindent{\bf Lemma 2\ }{\it Given types-in-context $\beta\!s.\sigma$ and
$\alpha\!s.\tau_i$ ($i=1,\ldots,p$, where $p$ is the
length of $\beta\!s$), let $\sigma'$ be the instance
$\sigma[\tau\!s/\beta\!s]$. Then $\alpha\!s.\sigma'$ is also a
type-in-context and its meaning in any model $M$ is related to that of
$\beta\!s.\sigma$ as follows. For all $X\!s\in{\cal U}^n$ (where $n$
is the length of $\alpha\!s$)
\[
\den{\alpha\!s.\sigma'}(X\!s) =
\den{\beta\!s.\sigma}(\den{\alpha\!s.\tau_{1}}(X\!s), 
    \ldots ,\den{\alpha\!s.\tau_{p}}(X\!s))
\]
}
Once again, the lemma can be proved by induction on the structure of
$\sigma$.  

\section{Terms}
\label{terms}

The terms of the \HOL\ logic are expressions that denote elements of the sets
denoted by types. The meta-variable $t$
is used to range over arbitrary terms, possibly decorated
with subscripts or primes.

There are four kinds of terms in the \HOL\ logic. These can be
described approximately by the following {\small BNF} grammar, in
which $x$ ranges over variables and $c$ ranges over constants.
\index{combinations, in HOL logic@combinations, in \HOL\ logic!abstract form of}

\settowidth{\ttX}{\tt X}
\newcommand{\var}{\setlength{\unitlength}{\ttX}\begin{picture}(1,6)
\put(.5,0){\makebox(0,0)[b]{\footnotesize variables}}
\put(0,1.5){\vector(0,1){4.5}}
\end{picture}}
\newcommand{\const}{\setlength{\unitlength}{\ttX}\begin{picture}(1,6)
\put(.5,2.3){\makebox(0,0)[b]{\footnotesize constants}}
\put(.5,3.5){\vector(0,1){2.4}}
\end{picture}}
\newcommand{\app}{\setlength{\unitlength}{\ttX}\begin{picture}(1,6)
\put(.5,1.5){\makebox(0,0)[b]{\footnotesize function applications}}
\put(.5,0){\makebox(0,0)[b]{\footnotesize (function $t$, argument $t'$)}}
\put(0.5,2.5){\vector(0,1){3.5}}
\end{picture}}
\newcommand{\abs}{\setlength{\unitlength}{\ttX}\begin{picture}(1,6)
\put(1,3.3){\makebox(0,0)[b]{\footnotesize $\lambda$-abstractions}}
\put(0.7,4.5){\vector(0,1){1.5}}
\end{picture}}
%
$$ t \quad ::=\quad {\mathord{\mathop{x}\limits_{\var}}}
        \quad\mid\quad{\mathord{\mathop{\con{c}}\limits_{\const}}}
        \quad\mid\quad\underbrace{t\ t'}_{\app}
        \quad\mid\quad\underbrace{\lambda x .\ t}_{\abs}$$

Informally, a $\lambda$-term\index{lambda terms, in HOL logic@lambda terms, in \HOL\ logic}\index{function abstraction, in HOL logic@function abstraction, in \HOL\ logic!abstract form of} $\lambda x.\ t$ denotes
a function $v\mapsto t[v/x]$, where $t[v/x]$ denotes the result of
substituting $v$ for $x$ in $t$. An application\index{function application, in HOL logic@function application, in \HOL\ logic!abstract form of} $t\ t'$ denotes the result of applying the
function denoted by $t$ to the value denoted by $t'$. This will be
made more precise below.

The {\small BNF} grammar just given omits mention of types. In fact, each 
term in
the \HOL\ logic is associated with a unique type.
The notation $t_{\sigma}$ is
traditionally used to range over terms of type $\sigma$. A 
more accurate grammar of
terms is:

$$ t_{\sigma} \quad ::=\quad {\mathord{\mathop{x_{\sigma}}\limits_{}}}
\quad\mid\quad
{\mathord{\mathop{\con{c}_{\sigma}}\limits_{}}}
\quad\mid\quad (t_{\sigma'\fun\sigma}\ t'_{\sigma'})_{\sigma}
\quad\mid\quad(\lambda x_{\sigma_1} .\ t_{\sigma_2})
_{\sigma_1\fun\sigma_2}$$\index{constants, in HOL logic@constants, in \HOL\ logic!abstract form of}

In fact, just as the definition of types was relative to a particular
type structure $\Omega$, the formal definition of terms is relative to
a given collection of typed constants over $\Omega$.  Assume that an
infinite set {\sf Names} of names is given. A {\em constant\/} over
$\Omega$ is a pair $(\con{c}, \sigma)$, where $\con{c}\in{\sf Names}$
and $\sigma\in{\sf Types}_{\Omega}$. A {\em signature} over $\Omega$
is just a set $\Sigma_\Omega$ of such constants. 

The set {\sf Terms}$_{\Sigma_{\Omega}}$ of terms over
$\Sigma_{\Omega}$ is defined to be the smallest set closed under the
following rules of formation:
\begin{enumerate}

\item {\bf Constants:} If $(\con{c},\sigma)\in{\Sigma_{\Omega}}$ and
$\sigma'\in{\sf Types}_{\Omega}$ 
is an instance of $\sigma$, then $(\con{c},{\sigma'})\in{\sf
Terms}_{\Sigma_{\Omega}}$.  Terms formed in this way are called {\it
constants\/}\index{constants, in HOL logic@constants, in \HOL\ logic!abstract form of} and are written $\con{c}_{\sigma'}$.

\item {\bf Variables:}  If  $x\in{\sf  Names}$  and  $\sigma\in{\sf
Types}_{\Omega}$, then ${\tt var}\ x_{\sigma}\in{\sf
Terms}_{\Sigma_{\Omega}}$. Terms formed in this way are called {\it
variables}\index{variables, in HOL logic@variables, in \HOL\ logic!abstract form of}.  The marker {\tt var}\ is purely a device to
distinguish variables from constants with the same name.  A variable
${\tt var}\ x_{\sigma}$ will usually be written as $x_{\sigma}$, if it
is clear from the context that $x$ is a variable rather than a
constant.

\item {\bf Function applications:}  If $t_{\sigma'{\fun}\sigma}\in{\sf
Terms}_{\Sigma_{\Omega}}$ and $t'_{\sigma'}\in{\sf
Terms}_{\Sigma_{\Omega}}$, then $(t_{\sigma'\fun\sigma}\
t'_{\sigma'})_{\sigma}\in {\sf Terms}_{\Sigma_{\Omega}}$.
(Terms formed in this way are sometimes called {\it combinations}.)

\item {\bf $\lambda$-Abstractions:} If ${\tt var}\ x_{\sigma_1}
\in{\sf Terms}_{\Sigma_{\Omega}}$  and $t_{\sigma_2}\in{\sf
Terms}_{\Sigma_{\Omega}}$, then $(\lambda x_{\sigma_1}.\
t_{\sigma_2})_{\sigma_1\fun\sigma_2}
\in{\sf Terms}_{\Sigma_{\Omega}}$.

\end{enumerate}

Note that it is possible for constants and variables\index{variables, in HOL logic@variables, in \HOL\ logic!with same names} to have the
same name.  It is also possible for different variables to have the
same name, if they have different types. 

The type subscript on a term may be omitted if it is clear from the
structure of the term or the context in which it occurs what its type
must be.

Function application\index{function application, in HOL logic@function application, in \HOL\ logic!associativity of} is assumed to associate
to the left, so that $t\ t_1\ t_2\ \ldots\ t_n$ abbreviates $(\
\ldots\ ((t\ t_1)\ t_2)\ \ldots\ t_n)$.  

The notation $\lambda x_1\ x_2\ \cdots\ x_n.\ t$ abbreviates $\lambda
x_1.\ (\lambda x_2.\ \cdots\ (\lambda x_n.\ t)\ \cdots\ )$.

A term is called polymorphic\index{polymorphic terms, in HOL logic@polymorphic terms, in \HOL\ logic} if it contains a type
variable. Otherwise it is called monomorphic. Note that a term
$t_{\sigma}$ may be polymorphic even though $\sigma$ is
monomorphic---for example, $(f_{\alpha\fun{\sl b}}\ x_{\alpha})_{\sl
b}$, where $\sl b$ is an atomic type. The expression
$tyvars(t_{\sigma})$ denotes the set of type variables occurring in
$t_{\sigma}$.

An occurrence of a variable $x_{\sigma}$ is called {\it
bound\/}\index{bound variables, in HOL logic@bound variables, in \HOL\ logic}\index{variables, in HOL logic@variables, in \HOL\ logic!abstract form of}
 if it occurs within the scope of a textually enclosing
$\lambda x_{\sigma}$, otherwise the occurrence is called {\it
free\/}\index{free variables, in HOL logic@free variables, in \HOL\ logic!abstract form of}. Note that $\lambda x_{\sigma}$ does not bind
$x_{\sigma'}$ if $\sigma\neq \sigma'$.  A term in which all occurrences
of variables are bound is called {\it closed\/}. 

\subsection{Terms-in-context}
\label{terms-in-context}

A {\em context\/}\index{contexts, in semantics of HOL logic@contexts, in semantics of \HOL\ logic} $\alpha\!s,\!x\!s$ consists of a type
context $\alpha\!s$ together with a list $x\!s=x_{1},\ldots,x_{m}$ of
distinct variables whose types only contain type variables from the
list $\alpha\!s$.

The condition that $x\!s$ contains {\em distinct\/} variables needs
some comment. Since a variable is specified by both a name and a
type,  it is permitted for $x\!s$ to contain repeated
names\index{variables, in HOL logic@variables, in \HOL\ logic!with same names},
 so long as different types are attached to the
names. This aspect of the syntax means that one has to proceed with
caution when defining the meaning of type variable instantiation,
since instantiation may cause variables to become equal
`accidentally': see Section~\ref{term-substitution}.

A {\em term-in-context\/}\index{term-in-context}
$\:\;\alpha\!s,\!x\!s.t\;\:$ consists of a context together with a term
$t$ satisfying the following conditions.
\begin{itemize}

\item $\alpha\!s$ contains any type variable that occurs in $x\!s$ and $t$.
 
\item $x\!s$ contains any variable that occurs freely in $t$.
 
\item $x\!s$ does not contain any variable that occurs
bound in $t$.

\end{itemize}
The context $\alpha\!s,\!x\!s$ may contain (type) variables which do
not appear in $t$.  Note that the combination of the second and third
conditions implies that a variable cannot have both free and bound
occurrences in $t$. For an arbitrary term, there is always an
$\alpha$-equivalent term which satisfies this condition, obtained by
renaming the bound variables as necessary.\footnote{Recall that two
terms are said to be $\alpha$-equivalent if they differ only in the
names of their bound variables.} In the semantics of terms to be given
below we will restrict attention to such terms. Then the meaning of an
arbitrary term is taken to be the meaning of some $\alpha$-variant of
it having no variable both free and bound. (The semantics will equate
$\alpha$-variants, so it does not matter which is chosen.) Evidently
for such a term there is a minimal context $\alpha\!s,\!x\!s$, unique
up to the order in which variables are listed, for which
$\alpha\!s,\!x\!s.t$ is a term-in-context. As for type variables, we
will assume given a fixed total order on variables.  Then the unique
minimal context with variables listed in order will be called the {\em
canonical}\index{canonical contexts, in HOL logic@canonical contexts, in \HOL\ logic!of terms} context of the term $t$.

\subsection{Semantics of terms}
\label{semantics of terms}

Let $\Sigma_{\Omega}$ be a signature\index{signatures, of HOL logic@signatures, of \HOL\ logic!formal semantics of} over a type
structure $\Omega$ (see Section~\ref{terms}). A {\em model\/} $M$ of
$\Sigma_{\Omega}$ is specified by a model of the type structure plus
for each constant\index{primitive constants, of HOL logic@primitive constants, of \HOL\ logic} $(\con{c},\sigma)\in\Sigma_{\Omega}$ an
element
\[ 
M(\con{c},\sigma) \in 
\prod_{X\!s\in{\cal U}^{n}}\den{\sigma}_{M}(X\!s) 
\]
of the indicated cartesian product, where $n$ is the number of type
variables occurring in $\sigma$. In other words
$M(\con{c},\sigma)$ is a (dependently typed) function
assigning to each $X\!s\in{\cal U}^{n}$ an element of
$\den{\sigma}_{M}(X\!s)$. In the case that $n=0$ (so that
$\sigma$ is monomorphic), $\den{\sigma}_{M}$ was identified
with a set in $\cal U$ and then $M(c,\sigma)$ can be
identified with an element of that set.

The meaning of \HOL\ terms in such a model will now be described. The
semantics interprets closed terms involving no type variables as
elements of sets in $\cal U$ (the particular set involved being derived
from the type of the term as in Section~\ref{semantics of types}). More
generally, if the closed term involves $n$ type variables then it is
interpreted as an element of a product $\prod_{X\!s\in{\cal
U}^{n}}Y(X\!s)$, where the function $Y:{\cal U}^{n}\longrightarrow{\cal
U}$ is derived from the type of the term (in a type context derived
from the term). Thus the meaning of the term is a (dependently typed)
function which, when applied to any meanings chosen for the type
variables in the term, yields a meaning for the term as an element of a
set in $\cal U$. On the other hand, if the term involves $m$ free
variables but no type variables, then it is interpreted as a function
$Y_1\times\cdots\times Y_m\fun Y$ where the sets $Y_1,\ldots,Y_m$ in
$\cal U$ are the interpretations of the types of the free variables in
the term and the set $Y\in{\cal U}$ is the interpretation of the type
of the term; thus the meaning of the term is a function which, when
applied to any meanings chosen for the free variables in the term,
yields a meaning for the term. Finally, the most general case is of a
term involving $n$ type variables and $m$ free variables: it is
interpreted as an element of a product 
\[ 
\prod_{X\!s\in{\cal
U}^{n}}Y_{1}(X\!s)\times\cdots\times Y_{m}(X\!s) \;\fun\; Y(X\!s) 
\]
where the functions $Y_{1},\ldots,Y_{m},Y:{\cal
U}^{n}\longrightarrow{\cal U}$ are determined by the types of the free
variables and the type of the term (in a type context derived from the
term).

More precisely, given a term-in-context $\alpha\!s,\!x\!s.t$
over $\Sigma_{\Omega}$ suppose
\begin{itemize}
 
\item $t$ has type $\tau$
 
\item $x\!s=x_{1},\ldots,x_{m}$ and each $x_{j}$ has type $\sigma_{j}$
 
\item $\alpha\!s=\alpha_{1},\ldots,\alpha_{n}$.

\end{itemize}
Then since $\alpha\!s,\!x\!s.t$ is a term-in-context, $\alpha\!s.\tau$
and $\alpha\!s.\sigma_{j}$ are types-in-context, and hence give rise
to functions $\den{\alpha\!s.\tau}_{M}$ and
$\den{\alpha\!s.\sigma_{j}}_{M}$ from ${\cal U}^{n}$ to $\cal U$ as in
section~\ref{semantics of types}. The meaning of $\alpha\!s,\!x\!s.t$
in the model $M$ will be given by an element
\[ 
\den{\alpha\!s,\!x\!s.t}_{M} \in \prod_{X\!s\in{\cal U}^{n}}
\left(\prod_{j=1}^{m}\den{\alpha\!s.\sigma_{j}}_{M}(X\!s)\right) 
\fun \den{\alpha\!s.\tau}_{M}(X\!s) .
\]
In other words, given 
\begin{eqnarray*}
X\!s & = & (X_{1},\ldots,X_{n})\in{\cal U}^{n} \\
y\!s & = & (y_{1},\ldots,y_{m})\in\den{\alpha\!s.\sigma_{1}}_{M}(X\!s)
           \times\cdots\times \den{\alpha\!s.\sigma_{m}}_{M}(X\!s)
\end{eqnarray*}
one gets an element $\den{\alpha\!s,\!x\!s.t}_{M}(X\!s)(y\!s)$ of
$\den{\alpha\!s.\tau}_{M}(X\!s)$. The definition of
$\den{\alpha\!s,\!x\!s.t}_{M}$ proceeds by induction on the structure of
the term $t$, as follows. (As before, the subscript $M$ will be dropped from
the semantic brackets $\den{ \_ }$ when the particular model involved is
clear from the context.)
\begin{itemize}
 
\item 
If $t$ is a variable\index{variables, in HOL logic@variables, in \HOL\ logic!formal semantics of}, it must be $x_{j}$ for some unique 
$j=1,\ldots,m$, so $\tau=\sigma_{j}$ and then
$\den{\alpha\!s,\!x\!s.t}(X\!s)(y\!s)$ is defined to be $y_{j}$.
 
\item 
Suppose $t$ is a constant\index{constants, in HOL logic@constants, in \HOL\ logic!formal semantics of} $\con{c}_{\sigma'}$, where
$(\con{c},\sigma)\in\Sigma_{\Omega}$ and $\sigma'$ is an instance of
$\sigma$.  Then by Lemma~1 of \ref{instances-and-substitution},
$\sigma'=\sigma[\tau_{1},\ldots,\tau_{p}/\beta_{1},\ldots,\beta_{p}]$
for uniquely determined types $\tau_{1},\ldots,\tau_{p}$ (where
$\beta_{1},\ldots,\beta_{p}$ are the type variables occurring in
$\sigma$). Then define $\den{\alpha\!s,\!x\!s.t}(X\!s)(y\!s)$ to be
$M(\con{c},\sigma)(\den{\alpha\!s.\tau_{1}}(X\!s),
\ldots,\den{\alpha\!s.\tau_{p}}(X\!s))$,
which is an element of $\den{\alpha\!s.\tau}(X\!s)$ by Lemma~2 of
\ref{instances-and-substitution} (since $\tau$ is $\sigma'$).

\item 
Suppose $t$ is a function application\index{combinations, in HOL logic@combinations, in \HOL\ logic!formal semantics of} term $(t_{1}\
t_{2})$\index{function application, in HOL logic@function application, in \HOL\ logic!formal semantics of} where $t_{1}$ is of type
$\tau'\fun\tau$ and $t_{2}$ is of type $\tau'$. Then
$f=\den{\alpha\!s,\!x\!s.t_{1}}(X\!s)(y\!s)$, being an element of
$\den{\alpha\!s.\tau'\fun\tau}(X\!s)$, is a function from the set
$\den{\alpha\!s.\tau'}(X\!s)$ to the set $\den{\alpha\!s.\tau}(X\!s)$
which one can apply to the element
$y=\den{\alpha\!s,\!x\!s.t_{2}}(X\!s)(y\!s)$. Define
$\den{\alpha\!s,\!x\!s.t}(X\!s)(y\!s)$ to be $f(y)$.

\item Suppose $t$ is the abstraction\index{function abstraction, in HOL logic@function abstraction, in \HOL\ logic!formal semantics of}
term $\lambda x.t_{2}$where $x$ is of type $\tau_{1}$ and $t_{2}$ of
type $\tau_{2}$. Thus $\tau=\tau_{1}\fun\tau_{2}$ and
$\den{\alpha\!s.\tau}(X\!s)$ is the function set
$\den{\alpha\!s.\tau_{1}}(X\!s)\fun\den{\alpha\!s.\tau_{2}}(X\!s)$.
Define $\den{\alpha\!s,\!x\!s.t}(X\!s)(y\!s)$ to be the element of
this set which is the function sending
$y\in\den{\alpha\!s.\tau_{1}}(X\!s)$ to
$\den{\alpha\!s,\!x\!s,\!x.t_{2}}(X\!s)(y\!s,y)$. (Note that since
$\alpha\!s,\!x\!s.t$ is a term-in-context, by convention the bound
variable $x$ does not occur in $x\!s$ and thus
$\alpha\!s,\!x\!s,\!x.t_{2}$ is also a term-in-context.)
\end{itemize}
Now define the meaning of a term $t_{\tau}$ in a model $M$ to be the
dependently typed function
\[ \den{t_{\tau}} \in \prod_{X\!s\in{\cal U}^{n}}
   \left(\prod_{j=1}^{m}\den{\alpha\!s.\sigma_{j}}(X\!s)\right) 
   \fun \den{\alpha\!s.\tau}(X\!s) 
\]
given by $\den{\alpha\!s,\!x\!s.t_{\tau}}$, where $\alpha\!s,\!x\!s$ is the
canonical context of $t_{\tau}$. So $n$ is the number of type variables in
$t_{\tau}$, $\alpha\!s$ is a list of those type variables, $m$ is the
number of ordinary variables occurring freely in $t_{\tau}$ (assumed to be
distinct from the bound variables of $t_{\tau}$) and the $\sigma_{j}$ are
the types of those variables. (It is important to note that the list
$\alpha\!s$, which is part of the canonical context of $t$, may be strictly
bigger than the canonical type contexts of $\sigma_{j}$ or $\tau$. So it
would not make sense to write just $\den{\sigma_{j}}$ or $\den{\tau}$ in
the above definition.)

If $t_{\tau}$ is a closed term, then $m=0$ and for each $X\!s\in{\cal
U}^{n}$ one can identify $\den{t_{\tau}}$ with the element
$\den{t_{\tau}}(X\!s)()\in\den{\alpha\!s.\tau}(X\!s)$. So for closed terms
one gets
\[ \den{t_{\tau}} \in \prod_{X\!s\in{\cal U}^{n}}
\den{\alpha\!s.\tau}(X\!s) 
\]
where $\alpha\!s$ is the list of type variables occurring in $t_{\tau}$ and
$n$ is the length of that list. If moreover, no type variables occur in
$t_{\tau}$, then $n=0$ and $\den{t_{\tau}}$ can be identified with the
element $\den{t_{\tau}}()$ of the set $\den{\tau}\in{\cal U}$.

The semantics of terms appears somewhat complicated because of the
possible dependency of a term upon both type variables and ordinary
variables. Examples of how the definition of the semantics
works in practice can be found in Section~\ref{LOG}, where the meaning
of several terms denoting logical constants is given.

\subsection{Substitution}
\label{term-substitution}

Since terms may involve both type variables and
ordinary variables, there are two different operations of substitution
on terms which have to be considered---substitution of types for type
variables and substitution of terms for variables.

\subsubsection*{Substituting types for type variables in terms}
\index{substitution rule, in HOL logic@substitution rule, in \HOL\ logic!formal semantics of}

Suppose $t$ is a term, with canonical context $\alpha\!s,\!x\!s$ say,
where $\alpha\!s = \alpha_1,\ldots,\alpha_n$, $x\!s = x_1,\ldots,x_m$ 
and where for $j=1,\ldots,m$ the type of the variable $x_j$ is
$\sigma_j$. If $\alpha\!s'.\tau_{i}$ ($i=1,\ldots,n$) are
types-in-context, then substituting\index{type substitution, in HOL logic@type substitution, in \HOL\ logic!formal semantics of} the types
$\tau_{i}$ for the type variables $\alpha_{i}$ in the list $x\!s$, one
obtains a new list of variables $x\!s'$. Thus the $j$\/th entry of
$x\!s'$ has type $\sigma'_{j} = \sigma_{j}[\tau\!s/\alpha\!s]$. Only
substitutions with the following property will be considered.  
\begin{quote}  
In instantiating\index{types, in HOL logic@types, in \HOL\ logic!instantiation of} the type variables $\alpha\!s$ with the types
$\tau\!s$, no two distinct variables in the list $x\!s$ become equal in
the list $x\!s'$.\footnote{Such an identification of variables could
occur if the variables had the same name component and their types
became equal on instantiation.}  
\end{quote}  
This condition ensures that $\alpha\!s',x\!s'$ really is a context. Then
one obtains a new term-in-context $\alpha\!s',\!x\!s'.t'$ by
substituting the types $\tau\!s=\tau_{1},\ldots,\tau_{n}$ for the type
variables $\alpha\!s$ in $t$ (with suitable renaming of bound
occurrences of variables to make them distinct from the variables in
$x\!s'$). The notation 
\[ 
t[\tau\!s/\alpha\!s]  
\]  
is used for the term $t'$. 

\medskip

\noindent{\bf Lemma 3\ }{\it
The meaning of $\alpha\!s',\!x\!s'.t'$ in a model is  related to that
of $t$ as follows. For all $X\!s'\in{\cal U}^{n'}$ (where $n'$ is the
length of $\alpha\!s'$)} 
\[
\den{\alpha\!s',\!x\!s'.t'}(X\!s') = 
   \den{t}(\den{\alpha\!s'.\tau_{1}}(X\!s'),\ldots,
   \den{\alpha\!s'.\tau_{n}}(X\!s')).
\]

\medskip

Lemma~2 in \ref{instances-and-substitution} is needed to see that both
sides of the above equation are elements of the same set of functions.
The validity of the equation is proved by induction on the structure
of the term $t$.

\subsubsection*{Substituting terms for variables in terms}
\index{substitution rule, in HOL logic@substitution rule, in \HOL\ logic!formal semantics of}

Suppose $t$ is a term, with canonical context $\alpha\!s,\!x\!s$ say,
where $\alpha\!s = \alpha_1,\ldots,\alpha_n$, $x\!s = x_1,\ldots,x_m$
and where for $j=1,\ldots,m$ the type of the variable $x_j$ is
$\sigma_j$. If one has terms-in-context $\alpha\!s,\!x\!s'.t_{j}$ for
$j=1,\ldots,m$ with $t_{j}$ of the same type as $x_{j}$, say
$\sigma_{j}$, then one obtains a new term-in-context
$\alpha\!s,\!x\!s'.t''$ by substituting the terms
$t\!s=t_1,\ldots,t_m$ for the variables $x\!s$ in $t$ (with suitable
renaming of bound occurrences of variables to prevent the free
variables of the $t_{j}$ becoming bound after substitution). The
notation
\[ 
t[t\!s/x\!s]  
\]  
is used for the term $t''$. 

\medskip
 
\noindent{\bf Lemma 4\ }{\it
The meaning of $\alpha\!s,\!x\!s'.t''$ in a model is related to that of
$t$ as follows. For all $X\!s\in{\cal U}^{n}$ and all
$y\!s'\in\den{\alpha\!s.\sigma'_{1}} \times\cdots\times
\den{\alpha\!s.\sigma'_{m'}}$ (where $\sigma'_{j}$ is the type of
$x'_{j}$)}
\[ 
\den{\alpha\!s,\!x\!s'.t''}(X\!s)(y\!s') = \den{t}(X\!s)(
\den{\alpha\!s,\!x\!s'.t_{1}}(X\!s)(y\!s'),\ldots,
\den{\alpha\!s,\!x\!s'.t_{m}}(X\!s)(y\!s'))
\]

\medskip

Once again, this result is proved by induction on the structure of
the term $t$.


\section{Standard notions}

Up to now the syntax of types and terms  has been  very general.   To
represent the standard  formulas  of  logic  it  is  necessary  to 
impose  some specific structure. In  particular, every  type  structure
must contain  an atomic type \ty{bool} which  is  intended  to  denote
the  distinguished two-element set $\two\in{\cal U}$, regarded as a set
of  truth-values.  Logical formulas are then identified with
terms\index{type variables, in HOL logic@type variables, in \HOL\ logic!abstract form of}\index{terms, in HOL logic@terms, in \HOL\ logic!as logical formulas} of  type \ty{bool}.   In addition, various
logical  constants  are  assumed  to  be  in  all  signatures.    These
requirements are  formalized  by  defining  the  notion  of  a 
standard signature.

\subsection{Standard type structures}
\label{standard-type-structures}

A type structure $\Omega$ is {\em standard\/} if it contains the
atomic types \ty{bool} (of booleans or truth-values) and \ty{ind} (of
individuals).  (In the literature, the  symbol  $o$ is  often used 
instead of  \ty{bool} and $\iota$ instead of \ty{ind}.)

A model $M$ of $\Omega$ is {\em standard\/} if $M(\bool)$ and $M(\ind)$ are
respectively the distinguished sets $\two$ and $\inds$ in the universe
$\cal U$. 

It will be assumed from now on that type structures and their models
are standard.

\subsection{Standard signatures}
\label{standard-signatures}
\index{signatures, of HOL logic@signatures, of \HOL\ logic!standard}\index{standard signatures, of HOL logic@standard signatures, of \HOL\ logic} 

A signature $\Sigma_{\Omega}$ is {\em standard\/} if it contains the
following three primitive constants\index{primitive constants, of HOL logic@primitive constants, of \HOL\ logic}\index{constants, in HOL logic@constants, in \HOL\ logic!primitive, abstract form of}:
\[
{\imp}_{\ty{bool}\fun\ty{bool}\fun\ty{bool}}
\]
\[
{=}_{\alpha\fun\alpha\fun\ty{bool}}
\]
\[
\hilbert_{(\alpha\fun\ty{bool})\fun\alpha}
\]
The intended interpretation of these constants is that  $\imp$ denotes
implication, $=_{\sigma\fun\sigma\fun\ty{bool}}$ denotes equality on
the set denoted by $\sigma$, and
$\hilbert_{(\sigma\fun\ty{bool})\fun\sigma}$ denotes a choice function
on the set denoted by $\sigma$. More precisely, a model $M$ of
$\Sigma_{\Omega}$ will be called {\em standard\/}\index{standard models, of HOL logic@standard models, of \HOL\ logic} if
\begin{itemize}
 
\item 
$M({\imp},\bool\fun\bool\fun\bool)\in(\two\fun\two\fun\two)$ is the
standard implication\index{implication, in HOL logic@implication, in \HOL\ logic!formal semantics of} function, sending $b,b'\in\two$ to
\[ 
(b\imp b') = \left\{ \begin{array}{ll}
                           0 & \mbox{if $b=1$ and $b'=0$} \\
                           1 & \mbox{otherwise}
                          \end{array}
             \right.%} 
\]
 
\item 
$M({=},\alpha\fun\alpha\fun\bool)\in\prod_{X\in{\cal U}}.X\fun
X\fun\two$ is the function assigning to each $X\in{\cal U}$ the
equality\index{equality, in HOL logic@equality, in \HOL\ logic!formal semantics of} test function, sending $x,x'\in X$ to
\[ 
(x=_{X}x') = \left\{ \begin{array}{ll}
                           1 & \mbox{if $x=x'$} \\
                           0 & \mbox{otherwise}
                          \end{array}
             \right.%} 
\]
 
\item 
\index{epsilon operator}$M(\hilbert,(\alpha\fun\bool)\fun\alpha)\in\prod_{X\in{\cal 
U}}.(X\fun\two)\fun X$ is the function assigning to each $X\in{\cal
U}$ the choice\index{choice operator, in HOL logic@choice operator, in \HOL\ logic!formal semantics of} function sending $f\in(X\fun\two)$ to
\[
\ch_{X}(f) = \left\{ \begin{array}{ll}
                           \ch(f^{-1}\{1\})
                             & \mbox{if $f^{-1}\{1\}\not=\emptyset$}\\
                           \ch(X) & \mbox{otherwise}
                          \end{array}
             \right.%}
\]
where $f^{-1}\{1\}=\{x\in X : f(x)=1\}$. (Note that $f^{-1}\{1\}$ is in
$\cal U$ when it is non-empty, by the property {\bf Sub} of the
universe $\cal U$ given in Section~\ref{introduction}. The function
$\ch$ is given by property {\bf Choice}.)

\end{itemize}

It will be assumed from now on that signatures and their models are
standard. 
         
\medskip

\noindent{\bf Remark\ }
This particular choice of primitive constants is arbitrary.  The
standard collection of logical constants includes $\T$ (`true'), $\F$ 
(`false')\index{truth values, in HOL logic@truth values, in \HOL\ logic!abstract form of}, $\imp$ (`implies'), $\wedge$ (`and'), $\vee$
(`or'), $\neg$ (`not'), $\forall$ (`for all'), $\exists$ (`there
exists'), $=$ (`equals'), $\iota$ (`the'), and $\hilbert$ (`a'). This
set is redundant, since it can be defined (in a sense explained in
Section~\ref{defs}) from various subsets. In practice, it is
necessary to work with the full set of logical constants, and the
particular subset taken as primitive is not important. The interested
reader can explore this topic  further by reading Andrews'
book~\cite{Andrews} and the references it contains.

\medskip

Terms of type \ty{bool} are  called {\it formulas\/}\index{formulas as terms, in HOL logic@formulas as terms, in \HOL\ logic}.

The following notational abbreviations are used:

\begin{center}
\index{equality, in HOL logic@equality, in \HOL\ logic!abstract form of} 
\index{implication, in HOL logic@implication, in \HOL\ logic!abstract form of} 
\index{choice operator, in HOL logic@choice operator, in \HOL\ logic!abstract form of} 
\index{existential quantifier, in HOL logic@existential quantifier, in \HOL\ logic!abstract form of} 
\index{universal quantifier, in HOL logic@universal quantifier, in \HOL\ logic!abstract form of} 
\index{epsilon operator}
\begin{tabular}{|l|l|}\hline
{\rm Notation} & {\rm Meaning}\\ \hline
$t_{\sigma}=t'_{\sigma}$ &
  $=_{\sigma\fun\sigma\fun\ty{bool}}\ t_{\sigma}\ t'_{\sigma}$\\ \hline
$t\imp t'$ &
  $\imp_{\ty{bool}\fun\ty{bool}\fun\ty{bool}}\ t_\ty{bool}\
t'_\ty{bool}$\\ \hline 
$\hilbert x_{\sigma}.\ t$ &
  $ \hilbert_{(\sigma\fun\ty{bool})\fun\sigma}
(\lambda x_{\sigma}.\ t)$\\ \hline
\end{tabular}
\end{center}
These notations are special cases of general abbreviatory 
conventions supported by the \HOL\ system. The first two are infixes 
and the third is a binder (see Section~\ref{derived-terms}).


			 % the HOL logic
   \chapter{Introduction to Proof with HOL}
\label{proof}


For a logician, a formal proof is a sequence, each of whose  elements is
either an {\it axiom\/} or follows from earlier members of the sequence by a
{\it rule of inference\/}.  A theorem is the last element of a proof.

Theorems are represented in \HOL\ by values of an abstract
type\footnote{Abstract types are explained in \DESCRIPTION.}
{\small\verb%thm%}.  The  only way  to create theorems is by generating a
proof.  In \HOL\ (following \LCF), this consists in applying \ML\ functions
representing {\it rules of inference\/} to  axioms or previously generated
theorems.  The sequence of such applications  directly corresponds to a
logician's proof.

There are five axioms of the \HOL\ logic and eight primitive
inference rules. The axioms are bound to ML names. For example, the Law of
Excluded Middle is bound to the \ML\ name {\small\verb%BOOL_CASES_AX%}:

\begin{session}
\begin{verbatim}
#BOOL_CASES_AX;;
|- !t. (t = T) \/ (t = F)
\end{verbatim}
\end{session}

Theorems are printed with a preceding turnstile {\small\verb%|-%} as
illustrated above; the symbol `{\small\verb%!%}' is the universal
quantifier `$\forall$'.  Rules of inference are \ML\ functions that
return values of type {\small\verb%thm%}.  
An example of a rule of inference is {\it
specialization\/} (or $\forall$-elimination). 
In standard `natural deduction'
notation this is:

\[ \Gamma\turn \uquant{x}t\over \Gamma\turn t[t'/x]\]

\begin{itemize}
\item $t[t'/x]$ denotes the result of substituting $t'$ for free
occurrences of $x$ in $t$, with the restriction that no free variables in $t'$
become bound after substitution.
\end{itemize}

\noindent This rule is represented in \ML\ 
by a function
{\small\verb%SPEC%},\footnote{{\tt SPEC} is not a 
primitive rule of inference in the HOL logic, but is a derived rule. Derived rules
are described in Section~\ref{forward}.}
which takes as arguments a term
{\small\verb%"%}$a${\small\verb%"%} and a theorem 
{\small\verb%|- !%}$x${\small\verb%.%}$t[x]$ and returns the theorem 
{\small\verb%|- %}$t[a]$, the result of substituting $a$ for $x$ in $t[x]$. 

\setcounter{sessioncount}{1}
\begin{session}
\begin{verbatim}
#let Th1 = BOOL_CASES_AX;;
Th1 = |- !t. (t = T) \/ (t = F)

#let Th2 = SPEC "1 = 2" Th1;;
Th2 = |- ((1 = 2) = T) \/ ((1 = 2) = F)
\end{verbatim}
\end{session}

This session consists of a proof of two steps: using an axiom and
applying the rule \ml{SPEC}; it interactively performs the following proof:


\begin{enumerate}
\item $\turn \uquant{t} t=\top\ \disj\  t=\bot$ \hfill
[Axiom \ml{BOOL\_CASES\_AX}]
\item $\turn (1{=}2)=\top\ \disj\ (1{=}2)=\bot$\hfill [Specializing line 1 to `$1{=}2$']
\end{enumerate}

If the argument to an \ML\ function representing a rule of inference is of the
wrong kind, or violates a condition of the rule, then the application fails.
For example, $\ml{SPEC}\ t\ th$ will fail if $th$ is not of the form
$\ml{|-\ !}x\ml{.}\cdots$ 
or if it is of this form but the type of $t$ is not the same
as the type of $x$, or if the free variable restriction is not met.

\begin{session}
\begin{verbatim}
#SPEC "1=2" Th2;;
evaluation failed     SPEC

#SPEC "1" Th1;;
evaluation failed     SPEC
\end{verbatim}
\end{session}

\noindent As this session illustrates, the failure token does not always
indicate the exact reason for failure. The failure conditions for rules of
inference are given in \REFERENCE.

A proof in the \HOL\ system is constructed by repeatedly applying inference
rules to axioms or to previously proved theorems.
Since proofs may consist of millions of steps, it is necessary to provide
tools to make proof construction easier for the user.  The proof generating
tools in the \HOL\ system are just those of \LCF, and are described later.


The general form of a theorem is  $t_1,\ldots,t_n\ $\ml{|-}$\  t$, where $t_1$,
$\ldots$ , $t_n$ are boolean terms called  the {\it  assumptions} and  $t$ is a
boolean term called the {\it conclusion\/}.  Such a theorem asserts that if its
assumptions are true then so is its conclusion.  Its truth conditions are thus
the same as those for the single term 
$(t_1${\small\verb%/\%}$\ldots${\small\verb%/\%}$t_n$)\ml{==>}$t$.  
Theorems  with  no  assumptions are printed
out in the form \ml{|-}$\ t$.


The five  axioms and  eight primitive  inference rules  of the  \HOL\ logic are
described in  detail in  the document \DESCRIPTION.  Every
value of  type  \ml{thm}  in  the  \HOL\ system  can be  obtained by repeatedly
applying primitive inference rules to axioms.  When the \HOL\  system is built,
the eight  primitive rules  of inference  are defined  and the  five axioms are
bound to their \ML\ names, all other predefined theorems are proved using rules
of inference as the system is made.\footnote{This is a slight
over-simplification.} This is one of the reasons why making  \ml{hol} takes so
long.

In the rest of this chapter, the  process of  {\it forward  proof\/}, which has
just been sketched, is described in more detail.   In Chapter~\ref{tactics} {\it
goal directed proof\/} is  described, including  the important  notions of {\it
tactics\/} and {\it tacticals\/}, due to Robin Milner.

\section{Forward proof}
\label{forward}

Three of the primitive inference rules of the
\HOL\ logic are \ml{ASSUME} (assumption introduction),
\ml{DISCH} (discharging or assumption elimination) and \ml{MP} (Modus Ponens).
These rules will be used to illustrate forward proof and the writing of derived
rules.

The inference rule {\small\verb%ASSUME%} generates theorems of the form
$t${\small\verb% |- %}$t$. Note, however, that the \ML\ printer prints each
assumption as a dot (but this default can be changed; see below).  The function
{\small\verb%dest_thm%} decomposes a theorem into a pair consisting of list of
assumptions and the conclusion. The \ML\ type \ml{goal} abbreviates
{\small\verb%(term)list # term%}, this is motivated in Section~\ref{tactics}.

\begin{session}
\begin{verbatim}
#let Th3 = ASSUME "t1==>t2";;
Th3 = . |- t1 ==> t2

#dest_thm Th3;;
["t1 ==> t2"],"t1 ==> t2" : goal
\end{verbatim}
\end{session}

A sort of dual to \ml{ASSUME} is the primitive inference rule
\ml{DISCH} (discharging, assumption elimination) which infers from
a theorem of the form $\cdots t_1\cdots\ml{\ |-\ }t_2$ the new theorem
$\cdots\ \cdots\ \ml{|-}\ t_1\ml{==>}t_2$. \ml{DISCH} takes as arguments
the term to be discharged (\ie\ $t_1$) and the theorem from whose
assumptions it is to be discharged and returns the result of the discharging.
The following session illustrates this; it also illustrates what happens when
\ml{=>}, which is part of the syntax of conditional terms
(see the table on page~\pageref{logic-table}), 
is erroneously typed instead of the implication symbol \ml{==>}.

\begin{session}
\begin{verbatim}
#let Th4 = DISCH "t1=>t2" Th3;;
need 2 nd branch to conditional
skipping: t2 " Th3 ;; parse failed     

#let Th4 = DISCH "t1==>t2" Th3;;
Th4 = |- (t1 ==> t2) ==> t1 ==> t2
\end{verbatim}
\end{session}

\noindent Note that the term being discharged need not be in the assumptions; in
this case they will be unchanged.

\begin{session}\begin{verbatim}
#DISCH "1=2" Th3;;
. |- (1 = 2) ==> t1 ==> t2

#dest_thm it;;
(["t1 ==> t2"], "(1 = 2) ==> t1 ==> t2") : goal
\end{verbatim}\end{session}


In \HOL\,  the  rule  \ml{MP}  of  Modus  Ponens  is  specified in conventional
notation by:  

\[ \Gamma_1 \turn t_1 \imp t_2 \qquad\qquad \Gamma_2\turn t_1\over
\Gamma_1 \cup \Gamma_2 \turn t_2\]

\noindent The \ML\ function \ml{MP} takes argument theorems of the
form \ml{$\cdots\ $|-$\ t_1$\ ==>\ $t_2$} and \ml{$\cdots\ $|-$\ t_1$}
and returns \ml{$\cdots\ $|-$\ t_2$}. The next session illustrates the use of
\ml{MP} and also a common error, namely not supplying the \HOL\ logic type
checker with enough information. 

\begin{session}\begin{verbatim}
#let Th5 = ASSUME "t1";;
Indeterminate types:  "t1:?"

evaluation failed     types indeterminate in quotation

#let Th5 = ASSUME "t1:bool";;
Th5 = . |- t1

#let Th6 = MP Th3 Th5;;
Th6 = .. |- t2
\end{verbatim}\end{session}

The hypotheses of \ml{Th6} can be inspected with the \ML\ function \ml{hyp},
which returns the list of assumptions of a theorem (the conclusion is returned by
\ml{concl}).

\begin{session}\begin{verbatim}
#hyp Th6;;
["t1 ==> t2"; "t1"] : term list
\end{verbatim}\end{session}

\HOL\ can be made to print out hypotheses of theorems explicitly by
telling \ML\ to use the function \ml{print\_all\_thm} instead of
the default \ml{print\_thm}. 

\begin{session}\begin{verbatim}
#top_print print_all_thm;;
- : (thm -> void)

#Th5;;
t1 |- t1

#Th6;;
t1 ==> t2, t1 |- t2
\end{verbatim}\end{session}


\noindent Discharging \ml{Th6} twice establishes the theorem
\ml{|-\ t1 ==> (t1==>t2) ==> t2}.

\begin{session}\begin{verbatim}
#let Th7 = DISCH "t1==>t2" Th6;;
Th7 = t1 |- (t1 ==> t2) ==> t2

#let Th8 = DISCH "t1:bool" Th7;;
Th8 = |- t1 ==> (t1 ==> t2) ==> t2
\end{verbatim}\end{session}

The sequence of theorems: \ml{Th3},
\ml{Th5}, \ml{Th6}, \ml{Th7}, \ml{Th8} constitutes a proof in \HOL\ of
the theorem \ml{|-\ t1 ==> (t1 ==> t2) ==> t2}. In stardard logical notation this
proof could be written:

\begin{enumerate}
\item $ t_1\imp t_2\turn t_1\imp t_2$ \hfill
[Assumption introduction]
\item $ t_1\turn t_1$ \hfill
[Assumption introduction]
\item $t_1\imp t_2,\ t_1 \turn t_2 $ \hfill
[Modus Ponens applied to lines 1 and 2]
\item $t_1 \turn (t_1\imp t_2)\imp t_2$ \hfill
[Discharging the first assumption of line 3]
\item $\turn t_1 \imp (t_1 \imp t_2) \imp t_2$ \hfill
[Discharging the only assumption of line 4]
\end{enumerate}

\subsection{Derived rules}


A {\it proof from hypothesis $th_1, \ldots, th_n$} is a sequence each of whose
elements is either an axiom, or one of the hypotheses $th_i$, or follows from
earlier elements by a rule of inference.

For example, a proof of $\Gamma,\ t'\turn t$ from the hypothesis
$\Gamma\turn t$ is:


\begin{enumerate}
\item $t'\turn t'$ \hfill [Assumption introduction]
\item $\Gamma\turn t$ \hfill [Hypothesis]
\item $\Gamma\turn t'\imp t$ \hfill [Discharge $t'$ from line 2]
\item $\Gamma,\ t'\turn t$ \hfill [Modus Ponens applied to lines 3 and 1]
\end{enumerate}

\noindent This proof works for any hypothesis of the form $\Gamma\turn t$ 
and any boolean term $t'$ and
shows that the result of adding an arbitrary hypothesis to a theorem is another
theorem (because the four lines above can be added to any proof of
$\Gamma\turn t$ to get a proof of $\Gamma,\ t'\turn t$).\footnote{This property
of the logic is called {\it monotonicity}.} For example,
the next session uses this proof to add the hypothesis \ml{"t3"} to
\ml{Th6}.

\begin{session}\begin{verbatim}
#let Th9 = ASSUME "t3:bool";;
Th9 = t3 |- t3

#let Th10 = DISCH "t3:bool" Th6;;
Th10 = t1 ==> t2, t1 |- t3 ==> t2

#let Th11 = MP Th10 Th9;;
Th11 = t1 ==> t2, t1, t3 |- t2
\end{verbatim}\end{session}


A {\it derived rule\/} is an \ML\ procedure that generates a proof from given hypotheses
each time it is invoked. The hypotheses are the arguments of the rule.
To illustrate this, a rule, called \ml{ADD\_ASSUM}, will now
be defined as an \ML\ procedure that carries
out the proof above. In standard notation this would be described by:

\[ \Gamma\turn t\over \Gamma,\ t'\turn t \]

\noindent The \ML\ definition is:

\begin{session}\begin{verbatim}
#let ADD_ASSUM t th =
# let th9 = ASSUME t
# in
# let th10 = DISCH t th
# in
# MP th10 th9;;
ADD_ASSUM = - : (term -> thm -> thm)

#ADD_ASSUM "t3:bool" Th6;;
t1 ==> t2, t1, t3 |- t2
\end{verbatim}\end{session}

\noindent The body of \ml{ADD\_ASSUM} has been coded  to mirror  the proof done
in session~10 above, so as to show how an interactive proof  can be generalized
into a  procedure.   But \ml{ADD\_ASSUM}  can be  written much more
concisely as:

\begin{session}\begin{verbatim}
#let ADD_ASSUM t th = MP (DISCH t th) (ASSUME t);;
ADD_ASSUM = - : (term -> thm -> thm)

#ADD_ASSUM t3 Th6;;
t1 ==> t2, t1, t3 |- t2
\end{verbatim}\end{session}


Another example of  a derived  inference rule  is \ml{UNDISCH};  this moves the
antecedent of an implication to the assumptions.

\[ \Gamma\turn t_1\imp t_2 \over\Gamma,\ t_1\turn t_2 \]

\noindent An \ML\ derived rule that implements this is:


\begin{session}\begin{verbatim}
#let UNDISCH th =
# MP th (ASSUME(fst(dest_imp(concl th))));;
UNDISCH = - : (thm -> thm)

#Th10;;
t1 ==> t2, t1 |- t3 ==> t2

#UNDISCH Th10;;
t1 ==> t2, t1, t3 |- t2

#it = Th11;;
true : bool
\end{verbatim}\end{session}

\noindent Each time \ml{UNDISCH\ $\Gamma\turn t_1\imp t_2$} is executed,
the following proof is performed:

\begin{enumerate}
\item $t_1\turn t_1$ \hfill [Assumption introduction]
\item $\Gamma\turn t_1\imp t_2$ \hfill [Hypothesis]
\item $\Gamma,\ t_1\turn t_2$ \hfill [Modus Ponens applied to lines 2 and 1]
\end{enumerate}

The rules \ml{ADD\_ASSUM} and \ml{UNDISCH} are the first derived rules
 defined when the \HOL\ system is built. For a description
of the main rules see the section on derived rules in
\DESCRIPTION.

\section{Rewriting}

An important  derived  rule  is  \ml{REWRITE\_RULE}.    This  takes  a  list of
conjunctions of equations, \ie\ a list of theorems of the  form:  

\[ \Gamma\turn (u_1 = v_1) \conj (u_2 = v_2) \conj \ldots\ \conj (u_n  = v_n)\]

\noindent  and a theorem
$\Delta\turn t$  and  repeatedly  replaces  instances  of  $u_i$ in  $t$ by the
corresponding instance of $v_i$ until no further change occurs.   The result is
a theorem $\Gamma\cup\Delta\turn t'$ where $t'$ is the result  of rewriting $t$
in this way.  The session below illustrates the use of  \ml{REWRITE\_RULE}.  In
it the list of equations is a list \ml{rewrite\_list} containing the pre-proved
theorems \ml{ADD\_CLAUSES}  and   \ml{MULT\_CLAUSES}.     These  theorems  {\it
autoload\/} from the theory \ml{arithmetic}.   This  means that  the first time
they are mentioned in a session, which  is when  \ml{rewrite\_list} is defined,
they are automatically fetched and bound to their \ML\ name.

\begin{session}\begin{verbatim}
#let rewrite_list = [ADD_CLAUSES;MULT_CLAUSES];;
Theorem MULT_CLAUSES autoloaded from theory `arithmetic`.
MULT_CLAUSES = 
|- !m n.
    (0 * m = 0) /\
    (m * 0 = 0) /\
    (1 * m = m) /\
    (m * 1 = m) /\
    ((SUC m) * n = (m * n) + n) /\
    (m * (SUC n) = m + (m * n))

Theorem ADD_CLAUSES autoloaded from theory `arithmetic`.
ADD_CLAUSES = 
|- (0 + m = m) /\
   (m + 0 = m) /\
   ((SUC m) + n = SUC(m + n)) /\
   (m + (SUC n) = SUC(m + n))

rewrite_list = 
[|- (0 + m = m) /\
    (m + 0 = m) /\
    ((SUC m) + n = SUC(m + n)) /\
    (m + (SUC n) = SUC(m + n));
 |- !m n.
     (0 * m = 0) /\
     (m * 0 = 0) /\
     (1 * m = m) /\
     (m * 1 = m) /\
     ((SUC m) * n = (m * n) + n) /\
     (m * (SUC n) = m + (m * n))]
: thm list
\end{verbatim}\end{session}

\begin{session}\begin{verbatim}
#REWRITE_RULE rewrite_list (ASSUME "(m+0)<(1*n)+(SUC 0)");;
(m + 0) < ((1 * n) + (SUC 0)) |- m < (SUC n)
\end{verbatim}\end{session}

\noindent This can then be rewritten using a pre-proved theorem \ml{LESS\_THM}:

\begin{session}\begin{verbatim}
REWRITE_RULE [LESS_THM] it;;
Theorem LESS_THM autoloaded from theory `prim_rec`.
LESS_THM = |- !m n. m < (SUC n) = (m = n) \/ m < n

(m + 0) < ((1 * n) + (SUC 0)) |- (m = n) \/ m < n
\end{verbatim}\end{session}

\ml{REWRITE\_RULE} is not a primitive in \HOL, but is a derived rule. It is
inherited from Cambridge \LCF\ and was implemented by Larry Paulson (see
his paper \cite{lcp_rewrite} for details). In addition to the supplied equations,
\ml{REWRITE\_RULE} has some built in standard simplifications:

\begin{session}\begin{verbatim}
#REWRITE_RULE [] (ASSUME "(T /\ x) \/ F ==> F");;
T /\ x \/ F ==> F |- ~x
\end{verbatim}\end{session}

\noindent The complete list of these built-in rewrites is 
held in the \ML\ variable \ml{basic\_rewrites}:

\newpage % PBHACK

\begin{session}\begin{verbatim}
#basic_rewrites;;
[|- !x. (x = x) = T;
 |- !t.
     ((T = t) = t) /\ ((t = T) = t) /\ ((F = t) = ~t) /\ ((t = F) = ~t);
 |- (!t. ~~t = t) /\ (~T = F) /\ (~F = T);
 |- !t.
     (T /\ t = t) /\
     (t /\ T = t) /\
     (F /\ t = F) /\
     (t /\ F = F) /\
     (t /\ t = t);
 |- !t.
     (T \/ t = T) /\
     (t \/ T = T) /\
     (F \/ t = t) /\
     (t \/ F = t) /\
     (t \/ t = t);
 |- !t.
     (T ==> t = t) /\
     (t ==> T = T) /\
     (F ==> t = T) /\
     (t ==> t = T) /\
     (t ==> F = ~t);
 |- !t1 t2. ((T => t1 | t2) = t1) /\ ((F => t1 | t2) = t2);
 |- !t1 t2. t1 <=> t2 = (t1 = t2);
 |- !t. (!x. t) = t;
 |- !t. (?x. t) = t;
 |- !t1 t2. (\x. t1)t2 = t1;
 |- !x. FST x,SND x = x;
 |- !x y. FST(x,y) = x;
 |- !x y. SND(x,y) = y]
: thm list
\end{verbatim}\end{session}

There are elaborate facilities in \HOL\ for producing customized rewriting tools
which scan through terms in user programmed orders; \ml{REWRITE\_RULE} is the tip
of an iceberg, see \DESCRIPTION\ for more details.



\section{Tautologies}

ICL Defence Systems have supplied a tautology checker as a library program.
This takes any term \ml{"$t$"} and returns \ml{|-~$t$} if $t$ is an instance
of a tautology of the propositional calculus.
This checker is a derived rule and constructs (behind the scenes) a proof
of any tautology. It is in the library \ml{taut} which is loaded with
the function \ml{load\_library}.

\begin{session}\begin{verbatim}
#load_library `taut`;;
Loading library `taut` ...
evaluation failed     load -- foo/Library/taut/taut ml file not found
\end{verbatim}\end{session}

\noindent This error message shows that \HOL\ has not been installed
properly; see Chapter~\ref{install}. The \ml{hol} being used for this
session is on the directory {\small\verb%/tmp_mnt/home/flint/mjcg/hol%},
but:

\begin{session}\begin{verbatim}
#HOLdir;;
`wonk/foo/mumble` : string
\end{verbatim}\end{session}

\noindent which shows that \HOL\ thinks it is living on
{\small\verb%wonk/foo/mumble%}. This is easy to fix:

\begin{session}\begin{verbatim}
#install `/tmp_mnt/home/flint/mjcg/hol`;;
() : void

HOLdir = `/tmp_mnt/home/flint/mjcg/hol` : string
\end{verbatim}\end{session}

\noindent Now the library \ml{taut} will load properly.

\begin{session}\begin{verbatim}
#load_library `taut`;;
Loading library `taut` ...
[fasl /tmp_mnt/home/flint/mjcg/hol/Library/taut/taut_ml.o]
.......................
Library `taut` loaded.
() : void
\end{verbatim}\end{session}

The following session illustrates \ml{TAUT\_RULE}, which is one of the functions
loaded when the library \ml{taut} is loaded.

\begin{session}\begin{verbatim}
#TAUT_RULE "((A ==> B) ==> A) ==> A";;
|- ((A ==> B) ==> A) ==> A

#TAUT_RULE "((A ==> B) ==> A) ==> B";;
evaluation failed     TAC_PROOF -- TAUT_TAC - term not a tautology
\end{verbatim}\end{session}

\noindent \ml{TAUT\_RULE} has \ML\ type {\small\verb%term -> thm%} which
is abbreviated to \ml{conv} by \ML\ (the motivation for this name is 
discussed in the section on conversions in \DESCRIPTION).

\begin{session}\begin{verbatim}
#TAUT_RULE;;
- : conv
\end{verbatim}\end{session}

\noindent Such type abbreviations are discussed in the next chapter.


\chapter{Goal Oriented Proof: Tactics and Tacticals}
\label{backward}\label{tactics}

The style of forward proof described in the previous chapter is unnatural and
too `low level' for many applications. An important advance in proof generating
methodology was made by Robin Milner in the early 1970s when he invented the
notion of {\it tactics\/}. A tactic is a function that does two things.
\begin{myenumerate}
\item Splits a `goal' into `subgoals'.
\item Keeps track of the reason why solving the subgoals will solve the goal.
\end{myenumerate}

\noindent Consider, for example, the  rule of $\wedge$-introduction\footnote{In
higher order logic this is a derived rule; in first  order logic  it is usually
primitive.  In HOL the rule is called {\tt CONJ} and its derivation is given in
\DESCRIPTION.}  shown below:  

\[ \Gamma_1\turn
t_1\qquad\qquad\qquad\Gamma_2\turn t_2\over \Gamma_1\cup\Gamma_2 \turn t_1\conj
t_2 \]


\noindent In \HOL,  $\wedge$-introduction is  represented by  the \ML\ function
\ml{CONJ}:  

\[\ml{CONJ}\ (\Gamma_1\turn t_1)\ (\Gamma_2\turn t_2) \ \ \leadsto\
\ (\Gamma_1\cup\Gamma_2\turn  t_1\conj  t_2)\]

\noindent  This  is   illustrated  in  the
following new  session  (note  that  the  session  number  has  been  reset  to
{\small\sl 1}):

\setcounter{sessioncount}{1}
\begin{session}\begin{verbatim}
#top_print print_all_thm;;
- : (thm -> void)

#let Th1 = ASSUME "A:bool" and Th2 = ASSUME "B:bool";;
Th1 = A |- A
Th2 = B |- B

#let Th3 = CONJ Th1 Th2;;
Th3 = A, B |- A /\ B
\end{verbatim}\end{session}

Suppose the goal is to prove $A\conj B$, then this rule says 
that it is sufficient
to prove the two subgoals $A$ and $B$, because from $\turn A$ and $\turn B$
the theorem $\turn A\conj B$ can be deduced. Thus:

\begin{myenumerate}
\item To prove $\turn A \conj B$ it is sufficient to 
      prove $\turn A$ and $\turn B$.
\item The justification for the reduction of the 
goal  $\turn A \conj B$  to the two  subgoals  $\turn A$ 
and $\turn B$ is the rule of $\wedge$-introduction.
\end{myenumerate}

A {\it goal\/} in \HOL\ is a pair
\ml{([$t_1$;\ldots;$t_n$],$t$)} of \ML\ type
{\small\verb%term list # term%}. An {\it achievement\/} of such a goal
is a theorem
\ml{$t_1$,$\ldots$,$t_n$\ |-\ $t$}. 
A tactic is an \ML\ function that when applied to a goal generates subgoals
together with a {\it justification function\/} or {\it validation\/}, 
which will be an \ML\ derived inference
rule, that can be used to infer an achievement of the original goal from
achievements
of the subgoals. 



{\bf Aside:} \ML\ has a type abbreviating mechanism which is used to give mnemonic
names to the various types associated with goal oriented proof.  Type abbreviations
can be seen by evaluating \ml{print\_defined\_types()}:

\begin{session}\begin{verbatim}
#print_defined_types();;

  proof = (thm list -> thm)
  goal = (term list # term)
  tactic = ((term list # term) -> subgoals)
  thm_tactic = (thm -> tactic)
  thm_tactical = ((thm -> tactic) -> thm_tactic)
  conv = (term -> thm)
  term_net -- an abstract type
  subgoals = ((term list # term) list # proof)
  goalstack -- an abstract type
\end{verbatim}\end{session}

\noindent The left hand side of these abbreviations can be used anywhere that the
right hand side can (abstract types are explained elsewhere). The default is that
the printer will print out abbreviations at top-level, but this behaviour can
be changed (see the flag \ml{print\_lettypes} in \DESCRIPTION).
{\bf End aside.}

If $T$ is a tactic (\ie\ an \ML\ function of type \ml{tactic})  and $g$ 
is a goal (\ie\ an \ML\ function of type \ml{goal}), then
applying $T$ to $g$ (\ie\ evaluating the \ML\ 
expression $T\ g$) will result in
an object of \ML\ type {\small\verb%subgoals%}, \ie\ a pair whose 
first component is a list of 
goals and whose second component is a justification function, \ie\ has
\ML\ type {\small\verb%proof%}.


An example tactic is \ml{CONJ\_TAC} which implements (i) and (ii) above.
For example, consider the utterly trivial goal of showing {\small\verb%T /\ T%},
where \ml{T} is a constant that stands for $true$:

\begin{session}\begin{verbatim}
#let goal1 =([], "T /\ T");;
goal1 = ([], "T /\ T") : (* list # term)

#CONJ_TAC goal1;;
([([], "T"); ([], "T")], -) : subgoals

#let goal_list,just_fn = it;;
goal_list = [([], "T"); ([], "T")] : goal list
just_fn = - : proof
\end{verbatim}\end{session}

\noindent \ml{CONJ\_TAC} has produced a goal  list consisting  of two identical
subgoals of just showing \ml{([],"T")}.  Now, there  is a  preproved theorem in
\HOL, called \ml{TRUTH}, that achieves this goal:

\begin{session}\begin{verbatim}
#TRUTH;;
|- T
\end{verbatim}\end{session}

\noindent Applying the justification function \ml{just\_fn} to a list
of theorems achieving the goals in \ml{goal\_list} results
in a theorem achieving the original goal:

\begin{session}\begin{verbatim}
#just_fn [TRUTH;TRUTH];;
|- T /\ T
\end{verbatim}\end{session}

Although this  example  is trivial,  it does  illustrate the  essential idea of
tactics.  Note that  tactics are  not special  theorem-proving primitives; they
are just  \ML\  functions.   For example,  the definition  of \ml{CONJ\_TAC} is
simply:

\begin{hol}\begin{verbatim}
   let CONJ_TAC (asl,w) =
    let l,r = dest_conj w 
    in
    ([(asl,l); (asl,r)], \[th1;th2]. CONJ th1 th2)
\end{verbatim}\end{hol}

\noindent The \ML\ function \ml{dest\_conj} splits a conjunction into its
two conjuncts:
If \ml{(asl,"$t_1$}{\small\verb%/\%}\ml{$t_2$")} 
is a goal, then \ml{CONJ\_TAC} splits
it into the list of two subgoals \ml{(asl,$t_1$)} and
\ml{(asl,$t_2$)}. The justification function, 
{\small\verb%\[th1;th2]. CONJ th1 th2%}
takes
a list \ml{[$th_1$;$th_2$]} of theorems and applies the rule \ml{CONJ}
to $th_1$ and $th_2$.

To summarize:
if $T$ is a tactic and $g$ 
is a goal, then
applying $T$ to $g$ will result in
an object of \ML\ type {\small\verb%subgoals%}, \ie\ a pair whose 
first component is a list of 
goals and whose second component is a justification function, \ie\ has
\ML\ type {\small\verb%proof%}. 

Suppose $T\ g${\small\verb% = ([%}$g_1${\small\verb%;%}$\ldots${\small\verb%;%}$g_n${\small\verb%],%}$p${\small\verb%)%}. 
The idea is that $g_1$ , $\ldots$ , $g_n$ are subgoals and $p$
is a `justification' of the reduction of goal $g$ to subgoals 
$g_1$ , $\ldots$ , $g_n$.
Suppose further that the subgoals $g_1$ , $\ldots$ , $g_n$ have been solved. 
This would mean that 
theorems $th_1$ , $\ldots$ , $th_n$ had been proved
such that each $th_i$ ($1\leq i\leq n$) `achieves' the goal $g_i$. 
The justification $p$ (produced
by applying $T$ to $g$) is an \ML\ 
function which when applied to the list
{\small\verb%[%}$th_1${\small\verb%;%}$\ldots${\small\verb%;%}$th_n${\small\verb%]%} returns a theorem, $th$, 
which `achieves' the original goal $g$.
Thus $p$ is a function for converting a solution of the subgoals to a
solution of the original goal. If $p$ 
does this successfully, then the tactic $T$ is
called {\it valid\/}. 
Invalid tactics cannot result in the proof of invalid theorems;
the worst they can do is result in insolvable goals or unintended theorems
being proved.
If $T$ were invalid and were used
to reduce goal $g$ to subgoals $g_1$ , $\ldots$ , $g_n$,
then  effort might be spent proving
theorems $th_1$ , $\ldots$ , $th_n$ to
achieve the subgoals $g_1$ , $\ldots$ , $g_n$, 
only to find out after the work is done that this is a blind alley
because $p${\small\verb%[%}$th_1${\small\verb%;%}$\ldots${\small\verb%;%}$th_n${\small\verb%]%} 
doesn't achieve $g$ (\ie\ it fails, 
or else it achieves some other goal).

A theorem {\it achieves\/} a goal if the assumptions of the theorem are
included in the assumptions of the goal {\it and\/} if the conclusion of the
theorems is equal (up to the renaming of bound variables) to the conclusion of
the goal. More precisely, a theorem 
\begin{center}
$t_1$, $\dots$, $t_m${\small\verb% |- %}$t$
\end{center}

\noindent  achieves a goal
\begin{center}
{\small\verb%([%}$u_1${\small\verb%;%}$\ldots${\small\verb%;%}$u_n${\small\verb%],%}$u${\small\verb%)%} 
\end{center}

\noindent if and only if $\{t_1,\ldots,t_m\}$ 
is a subset of $\{u_1,\ldots,u_n\}$ and $t$ is equal to $u$ (up
to renaming of bound variables).  For example, the goal
{\small\verb%(["x=y";"y=z";"z=w"],"x=z")%} is achieved by the theorem
{\small\verb%x=y, y=z |- x=z%} (the assumption {\small\verb%"z=w"%} is not
needed).

A tactic {\it solves\/} a goal if it reduces the goal 
to the empty list
of subgoals. Thus $T$ solves $g$ if  $T\ g${\small\verb% = ([],%}$p${\small\verb%)%}.
If this is the case and if $T$ is valid, then $p${\small\verb%[]%} 
will evaluate to a theorem achieving $g$.
Thus if $T$ solves $g$ then the \ML\ expression 
{\small\verb%snd(%}$T\ g${\small\verb%)[]%} evaluates to
a theorem achieving $g$.

Tactics are specified using the following notation:

\begin{center}
\begin{tabular}{c} \\
$goal$ \\ \hline \hline
$goal_1\ \ \ goal_2 \ \ \ \cdots\ \ \ goal_n$ \\
\end{tabular}
\end{center}

\noindent For example, a tactic called {\small\verb%CONJ_TAC%} is described by

\begin{center}
\begin{tabular}{c} \\
$ t_1${\small\verb% /\ %}$t_2$ \\ \hline \hline
$t_1\ \ \ \ \ \ \ t_2$ \\
\end{tabular}
\end{center}



\noindent Thus {\small\verb%CONJ_TAC%} reduces a goal of the form 
{\small\verb%(%}$\Gamma${\small\verb%,"%}$t_1${\small\verb%/\%}$t_2${\small\verb%")%} 
to subgoals
{\small\verb%(%}$\Gamma${\small\verb%,"%}$t_1${\small\verb%")%} and {\small\verb%(%}$\Gamma${\small\verb%,"%}$t_2${\small\verb%")%}.
The fact that the assumptions of the top-level goal
are propagated unchanged to the two subgoals is indicated by the absence
of assumptions in the notation.

Another example is {\small\verb%INDUCT_TAC%}, the tactic for doing mathematical induction
on the natural numbers:

\begin{center}
\begin{tabular}{c} \\
{\small\verb%!%}$n${\small\verb%.%}$t[n]$ \\ \hline \hline
$t[${\small\verb%0%}$]$ {\small\verb%     %} $\{t[n]\}\ t[${\small\verb%SUC %}$n]$
\end{tabular}
\end{center}

{\small\verb%INDUCT_TAC%} reduces a goal 
{\small\verb%(%}$\Gamma${\small\verb%,"!%}$n${\small\verb%.%}$t[n]${\small\verb%")%} to a basis subgoal
{\small\verb%(%}$\Gamma${\small\verb%,"%}$t[${\small\verb%0%}$]${\small\verb%")%} 
and an induction step subgoal 
{\small\verb%(%}$\Gamma\cup\{${\small\verb%"%}$t[n]${\small\verb%"%}$\}${\small\verb%,"%}$t[${\small\verb%SUC %}$n]${\small\verb%")%}.
The extra induction assumption {\small\verb%"%}$t[n]${\small\verb%"%} 
is indicated in the tactic notation with set brackets.

\begin{session}\begin{verbatim}
#INDUCT_TAC([], "!m n. m+n = n+m");;
([([], "!n. 0 + n = n + 0");
  (["!n. m + n = n + m"], "!n. (SUC m) + n = n + (SUC m)")],
 -)
: subgoals
\end{verbatim}\end{session}

\noindent The first subgoal is the basis case and the second subgoal is
the step case.

Tactics generally fail (in the \ML\ sense) if they are applied to 
inappropriate
goals. For example, {\small\verb%CONJ_TAC%} will fail if it is applied to a goal whose
conclusion is not a conjunction. Some tactics never fail, for example
{\small\verb%ALL_TAC%}


\begin{center}
\begin{tabular}{c} \\
$t$ \\ \hline \hline
$t$
\end{tabular}
\end{center}

\noindent is the `identity tactic'; it reduces a goal 
{\small\verb%(%}$\Gamma${\small\verb%,%}$t${\small\verb%)%} 
to the single
subgoal {\small\verb%(%}$\Gamma${\small\verb%,%}$t${\small\verb%)%}---\ie\ 
it has no effect. {\small\verb%ALL_TAC%} is useful for writing
complex tactics using tacticals (\eg\ see the definition of
{\small\verb%REPEAT%} in Section~\ref{tacticals}).


\section{Using tactics to prove theorems}   
\label{using-tactics}

Suppose goal $g$  is to be solved. If $g$
is simple it might be possible
to immediately think up a tactic, $T$ 
say, which reduces it to the empty list of
subgoals. If this is the case then executing:


$\ ${\small\verb% let %}$gl${\small\verb%,%}$p${\small\verb% = %}$T\ g$


\noindent will bind $p$ to a function which when applied to the empty list
of theorems yields a theorem $th$ achieving $g$. 
(The declaration above
will also bind $gl$ to the empty list of goals.) Thus a theorem achieving 
$g$ can be computed by executing:

$\ ${\small\verb% let %}$th${\small\verb% = %}$p${\small\verb%[]%}


\noindent This will be illustrated using \ml{REWRITE\_TAC} which takes a list
of equations (empty in the example that follows) and tries to prove a goal
by rewriting with these equations together with
\ml{basic\_rewrites}:

\begin{session}\begin{verbatim}
#let goal2 = ([], "T /\ x ==> x \/ (y /\ F)");;
goal2 = ([], "T /\ x ==> x \/ y /\ F") : (* list # term)

#REWRITE_TAC [] goal2;;
([], -) : subgoals

#snd it [];;
|- T /\ x ==> x \/ y /\ F
\end{verbatim}\end{session}

\noindent Proved theorems are usually stored in the current theory 
so that
they can be used in subsequent sessions.

The built-in function
 \ml{prove\_thm} of
\ML\ type {\small\verb%(string # term # tactic) -> thm%} facilitates the use
of tactics:
{\small\verb%prove_thm(`foo`,%}$t${\small\verb%,%}$T${\small\verb%)%} proves
the goal   {\small\verb%([],%}$t${\small\verb%)%}   (\ie\  the   goal  with  no
assumptions and  conclusion  $t$)  using  tactic  $T$  and  saves the resulting
theorem with name {\small\verb%foo%} on the current theory.

If the theorem is not to be saved, the function \ml{prove} of type
{\small\verb%(term # tactic) -> thm%} can be used.  Evaluating
{\small\verb%prove(%}$t${\small\verb%,%}$T${\small\verb%)%} proves   the   goal
{\small\verb%([],%}$t${\small\verb%)%} using $T$ and returns the result without
saving it.  In both cases  the evaluation  fails if  $T$ does  not solve the
goal {\small\verb%([],%}$t${\small\verb%)%}.


When conducting a proof that involves many subgoals and tactics, it is necessary
to keep track of all the justification functions  
and compose them in the correct order.  While
this is feasible even in large proofs, it is tedious.  \HOL\ provides a package
for building and traversing the tree of subgoals, stacking the justification functions and
applying them properly; this package was originally implemented for \LCF\ by 
Larry Paulson.

The subgoal package implements a simple framework for interactive proof. A proof
tree is created and traversed top-down.  The current goal can be expanded
into subgoals using a tactic; the subgoals are pushed onto a goal
stack and the justification function onto a proof stack.
Subgoals can be considered in any order.  If the tactic solves a
subgoal (\ie\ returns an empty subgoal list), then the package proceeds to the
next subgoal in the tree. 

The function  \ml{set\_goal} of type \ml{goal -> void}
initializes the subgoal package with a new goal. Usually
top-level goals have no assumptions; the function \ml{g} is useful
in this case and is defined by:

\begin{hol}\begin{verbatim}
   let g t = set_goal([],t)
\end{verbatim}\end{hol}

To illustrate the subgoal package the trivial theorem
$\vdash \uquant{m}m+0=m$ will be proved from the definition of addition:

\begin{session}\begin{verbatim}
#ADD;;
Definition ADD autoloaded from theory `arithmetic`.
ADD = |- (!n. 0 + n = n) /\ (!m n. (SUC m) + n = SUC(m + n))

|- (!n. 0 + n = n) /\ (!m n. (SUC m) + n = SUC(m + n))
\end{verbatim}\end{session}

\noindent Notice that \ml{ADD} specifies
$0+m=m$ but not $m+0=m$. Of course, $\uquant{m\ n}m+n = n+m$ is true, but the first step of the proof is to show
$\uquant{m}m+0=m$ from the definition of addition.

\setcounter{sessioncount}{1}
\begin{session}\begin{verbatim}
#g "!m. m+0=m";;
"!m. m + 0 = m"

() : void
\end{verbatim}\end{session}

\noindent This sets up the goal. Next the goal is split into a basis and step case
with \ml{INDUCT\_TAC}. To do this the function \ml{e} (or, equivalently,
\ml{expand}) is used. This applies a tactic to the top goal on the stack, then
pushes the resulting subgoals onto the goal stack, then prints the resulting
subgoals. If there are no subgoals, the justification function is applied to the
theorems solving the subgoals that have been proved and the resulting theorems are
printed.

\begin{session}\begin{verbatim}
#e INDUCT_TAC;;
OK..
2 subgoals
"(SUC m) + 0 = SUC m"
    [ "m + 0 = m" ]

"0 + 0 = 0"
\end{verbatim}\end{session}

\noindent The top of the goal stack is printed last. The basis case
is an instance of the definition of addition, so is solved by rewriting with
\ml{ADD}.

\begin{session}\begin{verbatim}
#e(REWRITE_TAC[ADD]);;
OK..
goal proved
|- 0 + 0 = 0

Previous subproof:
"(SUC m) + 0 = SUC m"
    [ "m + 0 = m" ]
\end{verbatim}\end{session}

\noindent The basis is solved and the goal
stack popped so that its top is now the step case, namely showing
that {\small\verb%(SUC m) + 0 = SUC m%} under the assumption
{\small\verb%m + 0 = m%}. This goal can be solved by rewriting first
with the definition of addition:

\begin{session}\begin{verbatim}
#e(REWRITE_TAC[ADD]);;
OK..
"SUC(m + 0) = SUC m"
    [ "m + 0 = m" ]
\end{verbatim}\end{session}

\noindent and then with the assumption \ml{m+0=m}. The tactic
\ml{ASM\_REWRITE\_TAC} is used to rewrite with the assumptions of a goal. It is
just like \ml{REWRITE\_TAC} except that it adds the assumptions to the list of
equations used for rewriting. For the example here no equations besides the
assumptions are needed, so \ml{ASM\_REWRITE\_TAC} is given the empty list of
equations.

\begin{session}\begin{verbatim}
#e(ASM_REWRITE_TAC[]);;
OK..
goal proved
. |- SUC(m + 0) = SUC m
. |- (SUC m) + 0 = SUC m
|- !m. m + 0 = m

Previous subproof:
goal proved
\end{verbatim}\end{session}

\noindent The top goal is solved, hence the preceding goal (the step case)
is solved too, and since the basis is already solved, the main goal is solved.

The theorem achieving the goal can be extracted from the subgoal package with
\ml{top\_thm}:

\begin{session}\begin{verbatim}
#top_thm();;
|- !m. m + 0 = m
\end{verbatim}\end{session}

The proof just done can be `optimized'. For example, instead
of first rewriting with \ml{ADD} (box 4) and then with the assumptions
(box 5), a single rewriting with \ml{ADD} and the assumptions would suffice.
To illustrate, the last two steps of the proof will be `undone' using the function
\ml{backup} which restores the previous state of the goal and theorem stacks.

\begin{session}\begin{verbatim}
#backup();;
"SUC(m + 0) = SUC m"
    [ "m + 0 = m" ]

() : void

#backup();;
"(SUC m) + 0 = SUC m"
    [ "m + 0 = m" ]
\end{verbatim}\end{session}

\noindent The proof can now be completed in one step instead of two:

\begin{session}\begin{verbatim}
#e(ASM_REWRITE_TAC[ADD]);;
OK..
goal proved
. |- (SUC m) + 0 = SUC m
|- !m. m + 0 = m

Previous subproof:
goal proved
\end{verbatim}\end{session}


The order in which goals are attacked can be adjusted using \ml{rotate\ }$n$
which rotates the goal stack by $n$. For example:

\begin{session}\begin{verbatim}
#backup();backup();;
"(SUC m) + 0 = SUC m"
    [ "m + 0 = m" ]

2 subgoals
"(SUC m) + 0 = SUC m"
    [ "m + 0 = m" ]

"0 + 0 = 0"

() : void

#rotate 1;;
2 subgoals
"0 + 0 = 0"

"(SUC m) + 0 = SUC m"
    [ "m + 0 = m" ]
\end{verbatim}\end{session}

\noindent The top goal is now the step case not the basis case, so expanding
with a tactic will apply the tactic to the step case.

\begin{session}\begin{verbatim}
e(ASM_REWRITE_TAC[ADD]);;
OK..
goal proved
. |- (SUC m) + 0 = SUC m

Previous subproof:
"0 + 0 = 0"
\end{verbatim}\end{session}

It is possible to do the whole proof in one step, but this requires a compound
tactic built using the {\it tactical\/}\footnote{This word was invented by Robin
Milner: `tactical' is to `tactic` as `functional' is to `function'.} \ml{THENL}.
Tacticals are higher order operations for combining tactics.

\section{Tacticals}
\label{tacticals}

A {\it tactical\/} 
is an \ML\ function that returns a tactic (or tactics) as result.
Tacticals may take various parameters; this is reflected in the various
\ML\ types that the built-in tacticals have. Some important tacticals in 
the \HOL\ system
are listed below.


\subsection{\tt THENL : tactic -> tactic list -> tactic}
      
If tactic $T$ produces $n$ subgoals and $T_1$, $\ldots$ ,
$T_n$ are tactics
then $T${\small\verb% THENL%} {\small\verb%[%}$T_1${\small\verb%;%}$\ldots${\small\verb%;%}$T_n${\small\verb%]%} 
is a tactic which first applies $T$ and then
applies $T_i$ to the $i$th subgoal produced by $T$. 
The tactical {\small\verb%THENL%} is useful if one wants to do different
things to different subgoals.

\ml{THENL} can be illustrated by doing the proof of $\vdash \uquant{m}m+0=m$ in
one step.

\setcounter{sessioncount}{1}
\begin{session}\begin{verbatim}
#g "!m. m + 0 = m";;
"!m. m + 0 = m"

() : void

#e(INDUCT_TAC THENL [REWRITE_TAC[ADD];ASM_REWRITE_TAC[ADD]]);;
OK..
goal proved
|- !m. m + 0 = m

Previous subproof:
goal proved
() : void
\end{verbatim}\end{session}

\noindent The compound tactic 
{\small\verb%INDUCT_TAC THENL [REWRITE_TAC[ADD];ASM_REWRITE_TAC[ADD]]%}
first applies \ml{INDUCT\_TAC} and then applies
\ml{REWRITE\_TAC[ADD]} to the first subgoal (the basis) and
\ml{ASM\_REWRITE\_TAC[ADD]} to the second subgoal (the step). 

The tactical {\small\verb%THENL%} is useful for doing different things to different
subgoals. The tactical \ml{THEN} can be used to apply the same tactic to all
subgoals.

\subsection{\tt THEN : tactic -> tactic -> tactic}\label{THEN}


The tactical {\small\verb%THEN%} is an \ML\ infix. If $T_1$ and $T_2$ are tactics,
then the \ML\ expression $T_1${\small\verb% THEN %}$T_2$ evaluates to a tactic
which first applies $T_1$ and then applies $T_2$ to all the subgoals produced 
by $T_1$. 

In fact,
\ml{ASM\_REWRITE\_TAC[ADD]} will solve the basis as well as the step
case of the induction for $\uquant{m}m+0=m$, so there is an even
simpler one-step proof than the one above:
\setcounter{sessioncount}{1}
\begin{session}\begin{verbatim}
#g "!m. m+0 = m";;
"!m. m + 0 = m"

() : void

#e(INDUCT_TAC THEN ASM_REWRITE_TAC[ADD]);;
OK..
goal proved
|- !m. m + 0 = m

Previous subproof:
goal proved
\end{verbatim}\end{session}

\noindent This is typical: it is common to use a single tactic for several
goals. Here, for example, are the first four consequences of the definition
\ml{ADD} of addition that are pre-proved when the built-in theory
\ml{arithmetic} \HOL\ is made.

\begin{hol}\begin{verbatim}
   let ADD_0 =
    prove
     ("!m. m + 0 = m",
      INDUCT_TAC
       THEN ASM_REWRITE_TAC[ADD]);;
\end{verbatim}\end{hol}

\begin{hol}\begin{verbatim}
   let ADD_SUC =
    prove
     ("!m n. SUC(m + n) = m + SUC n",
      INDUCT_TAC
       THEN ASM_REWRITE_TAC[ADD]);;
\end{verbatim}\end{hol}

\begin{hol}\begin{verbatim}
   let ADD_CLAUSES =
    prove
     ("(0 + m = m)              /\
       (m + 0 = m)              /\
       (SUC m + n = SUC(m + n)) /\
       (m + SUC n = SUC(m + n))",
      REWRITE_TAC[ADD;ADD_0;ADD_SUC]);;
\end{verbatim}\end{hol}

\begin{hol}\begin{verbatim}
   let ADD_SYM =
    prove
    ("!m n. m + n = n + m",
     INDUCT_TAC
      THEN ASM_REWRITE_TAC[ADD_0;ADD;ADD_SUC]);;
\end{verbatim}\end{hol}


\noindent These proofs are performed when the \HOL\ system is made and the
theorems are saved in the theory \ml{arithmetic}. The complete list of
proofs for this built-in theory can be found in the file
\ml{theories/mk\_arith\_thms.ml}.




\subsection{\tt ORELSE : tactic -> tactic -> tactic}\label{ORELSE}

The tactical {\small\verb%ORELSE%} 
is an \ML\ infix. If $T_1$ and $T_2$ are tactics, 
\index{tacticals!for alternation}
then $T_1${\small\verb% ORELSE %}$T_2$ 
evaluates to a tactic which applies $T_1$ unless that fails;
if it fails,
it applies $T_2$. \ml{ORELSE} is defined in \ML\ 
as a curried infix by

\begin{hol}
   {\small\verb%(%}$T_1${\small\verb% ORELSE %}$T_2${\small\verb%)%} $g$ 
   {\small\verb%=%}  $T_1 g$ {\small\verb%?%} $T_2 g$ 
\end{hol}\index{alternation!of tactics|)}


\subsection{\tt REPEAT : tactic -> tactic}

If $T$ is a 
tactic then {\small\verb%REPEAT %}$T$ is a tactic which repeatedly applies
$T$ until it fails. This can be illustrated in conjunction with
\ml{GEN\_TAC}, which is specified by:


\begin{center}
\begin{tabular}{c} \\
{\small\verb%!%}$x${\small\verb%.%}$t[x]$
\\ \hline \hline
$t[x']$
\\
\end{tabular}
\end{center}

\begin{itemize}
\item Where $x'$ is a variant of $x$ 
not free in the goal or the assumptions.
\end{itemize}

\noindent \ml{GEN\_TAC} strips off one quantifier; 
\ml{REPEAT\ GEN\_TAC} strips off all quantifiers:

\begin{session}\begin{verbatim}
#g "!x y z. x+(y+z) = (x+y)+z";;
"!x y z. x + (y + z) = (x + y) + z"

() : void

#e GEN_TAC;;
OK..
"!y z. x + (y + z) = (x + y) + z"

() : void

#e(REPEAT GEN_TAC);;
OK..
"x + (y + z) = (x + y) + z"
\end{verbatim}\end{session}


\section{Some tactics built into HOL}

This section contains a summary of some of the tactics built into the \HOL\ system
(including those already discussed).
The tactics given here are those that are used in the parity checking case 
study.

Recall that the \ML\ type {\small\verb%thm_tactic%} abbreviates {\small\verb%theorem->tactic%}, 
and the type {\small\verb%conv%}\footnote{The type
{\small{\tt conv}} comes from Larry Paulson's theory of conversions
\cite{lcp_rewrite}.} abbreviates {\small\verb%term->thm%}.

\subsection{\tt REWRITE\_TAC : thm list -> tactic}
\label{rewrite}

\begin{itemize}
\item{\bf Summary:} {\small\verb%REWRITE_TAC[%}$th_1${\small\verb%;%}$\ldots${\small\verb%;%}$th_n${\small\verb%]%} 
simplifies the goal by rewriting
it with the explicitly given theorems $th_1$, $\ldots$ , $th_n$, 
and various built-in rewriting rules.


\begin{center}
\begin{tabular}{c} \\
$\{t_1, \ldots , t_m\}t$
\\ \hline \hline
$\{t_1, \ldots , t_m\}t'$
\\
\end{tabular}
\end{center}

\noindent where $t'$ is obtained from $t$ by rewriting with
\begin{enumerate}
\item  $th_1$, $\ldots$ , $th_n$ and
\item  the standard rewrites held in the \ML\ variable {\small\verb%basic_rewrites%}.
\end{enumerate}

\item{\bf Uses:} Simplifying goals using previously proved theorems.

\item{\bf Other rewriting tactics} (based on {\small\verb%REWRITE_TAC%}):
\begin{enumerate}
\item {\small\verb%ASM_REWRITE_TAC%} adds the assumptions of the goal to the list of
theorems used for rewriting.
\item {\small\verb%FILTER_ASM_REWRITE_TAC %}$p${\small\verb% [%}$th_1${\small\verb%;%}$\ldots${\small\verb%;%}$th_n${\small\verb%]%} 
simplifies the goal by rewriting
it with the explicitly given theorems $th_1$ , $\ldots$ , $th_n$ , 
together with those
assumptions of the goal which satisfy the predicate $p$ and also
the built-in rewrites in the \ML\ variable {\small\verb%basic_rewrites%}.
\item {\small\verb%PURE_ASM_REWRITE_TAC%} is like {\small\verb%ASM_REWRITE_TAC%}, but it
doesn't use any built-in rewrites.
\item {\small\verb%PURE_REWRITE_TAC%} uses neither the assumptions nor the built-in
rewrites.
\end{enumerate}
\end{itemize}


\subsection{\tt CONJ\_TAC : tactic}\label{CONJTAC}

\begin{itemize}

\item{\bf Summary:} Splits a 
goal {\small\verb%"%}$t_1${\small\verb%/\%}$t_2${\small\verb%"%} into two subgoals {\small\verb%"%}$t_1${\small\verb%"%} 
and {\small\verb%"%}$t_2${\small\verb%"%}.

\begin{center}
\begin{tabular}{c} \\
$t_1${\small\verb% /\ %}$t_2$
\\ \hline \hline
$t_1\ \ \ \ \ \ t_2$
\\
\end{tabular}
\end{center}

\item{\bf Uses:} Solving conjunctive goals. 
{\small\verb%CONJ_TAC%} is invoked by {\small\verb%STRIP_TAC%} (see below).

\end{itemize}



\subsection{\tt EQ\_TAC : tactic}\label{EQTAC}


\begin{itemize}

\item{\bf Summary:} 
{\small\verb%EQ_TAC%}
splits an equational goal into two implications (the `if-case' and
the `only-if' case):

\begin{center}



\begin{tabular}{c} \\
$u\ \ml{=}\ v$
\\ \hline \hline
$u\ \ml{==>}\ v\ \ \ \ \ v\ \ml{==>}\ u$
\\
\end{tabular}
\end{center}

\item{\bf Use:} Proving logical equivalences, \ie\ goals of the form
``$u$\ml{=}$v$'' where $u$ and $v$ are boolean terms.

\end{itemize}




\subsection{\tt DISCH\_TAC : tactic}\label{DISCHTAC}

\begin{itemize}

\item{\bf Summary:} Moves the antecedent  
of an implicative goal into the assumptions.

\begin{center}
\begin{tabular}{c} \\
$u${\small\verb% ==> %}$v$
\\ \hline \hline
$\{u\}v$
\\
\end{tabular}
\end{center}


\item{\bf Uses:} Solving goals of the form 
{\small\verb%"%}$u${\small\verb% ==> %}$v${\small\verb%"%} by assuming {\small\verb%"%}$u${\small\verb%"%} and then solving
{\small\verb%"%}$v${\small\verb%"%}.
{\small\verb%STRIP_TAC%} (see below) will invoke {\small\verb%DISCH_TAC%} on implicative goals.
\end{itemize}

\subsection{\tt GEN\_TAC : tactic}

\begin{itemize}

\item{\bf  Summary:} Strips off one universal quantifier.
   

\begin{center}
\begin{tabular}{c} \\
{\small\verb%!%}$x${\small\verb%.%}$t[x]$
\\ \hline \hline
$t[x']$
\\
\end{tabular}
\end{center}

\noindent Where $x'$ is a variant of $x$ 
not free in the goal or the assumptions.

\item{\bf   Uses:} Solving universally quantified goals. 
{\small\verb%REPEAT GEN_TAC%} strips off all
universal quantifiers and is often the first thing one does in a proof.
{\small\verb%STRIP_TAC%} (see below) applies {\small\verb%GEN_TAC%} to universally quantified goals.
\end{itemize}


\subsection{\tt IMP\_RES\_TAC : tactic}

\begin{itemize}

\item{\bf Summary:}  {\small\verb%IMP_RES_TAC %}$th$ does a limited amount of
automated theorem proving in the form of forward inference; it
`resolves' the theorem $th$ with the 
assumptions of the goal
and adds any new results to the assumptions. The specification for
\ml{IMP\_RES\_TAC} is:


\begin{center}
\begin{tabular}{c} \\
$\{t_1,\ldots,t_m\}t$
\\ \hline \hline
$\{t_1,\ldots,t_m,u_1,\ldots,u_n\}t$
\\
\end{tabular}
\end{center}

\noindent  where $u_1$, $\dots$, $u_n$ 
are derived by `resolving' the theorem $th$ with the existing assumptions 
$t_1$, $\dots$, $t_m$. 
Resolution in \HOL\ is not classical resolution, but just Modus Ponens with
one-way pattern matching (not unification) and term and type instantiation. The
general case is where $th$ is of the canonical form

$\ \ \ ${\small\verb%|- !%}$x_1$$\ldots x_p${\small\verb%.%}$v_1$ {\small\verb%==>%} $v_2$ {\small\verb%==>%} $\ldots$ {\small\verb%==>%} $v_q$ {\small\verb%==>%} $v$

\noindent {\small\verb%IMP_RES_TAC %}$th$ then tries to specialize $x_1$,
$\dots$, $x_p$ in succession so that $v_1$, $\dots$, $v_q$ match members of
$\{t_1,\ldots ,t_m\}$.  Each time a match is found for some antecedent $v_i$,
for $i$ successively equal to $1$, $2$, \dots, $q$, a term and type
instantiation is made and the rule of Modus Ponens is applied.  If all the
antecedents $v_i$ (for $1 \leq i \leq q$) can be dismissed in this way, then
the appropriate instance of $v$ is added to the assumptions. Otherwise, if only
some initial sequence $v_1$, \dots, $v_k$ (for some $k$ where $1 < k < q$) of
the assumptions can be dismissed, then the remaining implication:

$\ \ \ ${\small\verb%|- %} $v_{k+1}$ {\small\verb%==>%} $\ldots$ {\small\verb%==>%} $v_q$ {\small\verb%==>%} $v$

\noindent is added to the assumptions.

For a more detailed description of resolution and \ml{IMP\_RES\_TAC},
see \DESCRIPTION\ and \REFERENCE.  

\item{\bf Uses:} Deriving new results from a previously proved implicative
theorem, in combination with the current assumptions, so that
subsequent tactics can use these new results.

\end{itemize}


        
\subsection{\tt STRIP\_TAC : tactic}

\begin{itemize}

\item{\bf Summary:} Breaks a goal apart.
{\small\verb%STRIP_TAC%} removes one outer connective from the goal, using
{\small\verb%CONJ_TAC%}, {\small\verb%DISCH_TAC%}, {\small\verb%GEN_TAC%}, \etc\  
If the goal is
$t_1${\small\verb%/\%}$\cdots${\small\verb%/\%}$t_n${\small\verb% ==> %}$t$ 
then {\small\verb%DISCH_TAC%} makes each $t_i$ into a separate assumption.

\item{\bf Uses:} Useful for splitting a goal up into manageable pieces. 
Often the best thing to do first is {\small\verb%REPEAT STRIP_TAC%}.
\end{itemize}

\subsection{\tt SUBST\_TAC : thm list -> thm}

\begin{itemize}

\item{\bf Summary:}
{\small\verb%SUBST_TAC[|-%}$u_1${\small\verb%=%}$v_1${\small\verb%;%}$\ldots${\small\verb%;|-%}$u_n${\small\verb%=%}$v_n${\small\verb%]%}
converts a goal $t[u_1,\ldots ,u_n]$
to the subgoal form $t[v_1,\ldots ,v_n]$.

\item{\bf Uses:} 
To make replacements for terms in 
situations in which {\small\verb%REWRITE_TAC%} is too
general or would loop.
\end{itemize}


\subsection{\tt ACCEPT\_TAC : thm -> tactic}\label{ACCEPTTAC}


\begin{itemize}

\item{\bf Summary:} {\small\verb%ACCEPT_TAC %}$th$ 
is a tactic that solves any goal that is 
achieved by $th$.

\item{\bf  Use:} Incorporating forward proofs, or theorems already
proved, into goal directed proofs.
For example, one might reduce a goal $g$ to 
subgoals $g_1$, $\dots$, $g_n$ 
using a tactic $T$ and then prove theorems $th_1$ , $\dots$, $th_n$ 
respectively achieving 
these goals by forward proof. The tactic

\[\ml{  T THENL[ACCEPT\_TAC }th_1\ml{;}\ldots\ml{;ACCEPT\_TAC }th_n\ml{]}
\]

would then solve $g$, where \ml{THENL}
\index{THENL@\ml{THENL}} is the tactical that applies
the respective elements of the tactic list to the subgoals produced
by \ml{T}.

\end{itemize}



\subsection{\tt ALL\_TAC : tactic}

\begin{itemize}
\item{\bf Summary:} Identity tactic for the tactical {\small\verb%THEN%}
(see \DESCRIPTION).

\item{\bf Uses:}
\begin{enumerate}
\item Writing tacticals (see description of {\small\verb%REPEAT%} 
in \DESCRIPTION). 
\item With {\small\verb%THENL%}; for example, if tactic $T$ produces two subgoals 
and we want to apply $T_1$ 
to the first one but to do nothing to the second, then 
the tactic to use is $T${\small\verb% THENL[%}$T_1${\small\verb%;ALL_TAC]%}.
\end{enumerate}
\end{itemize}

\subsection{\tt NO\_TAC : tactic}

\begin{itemize}
\item{\bf Summary:} Tactic that always fails.

\item{\bf Uses:} Writing tacticals.
\end{itemize}








			 % intro to proof in HOL
   
\chapter{Example: a simple parity checker}\label{parity}

This chapter consists of a worked example: the specification and
verification of a simple sequential parity checker.  The intention is
to   accomplish two things:

\begin{myenumerate}
\item To present a complete piece of work with \HOL.
\item To give a flavour of what it is like to use the \HOL\
system for a tricky proof.  
\end{myenumerate}

Concerning (ii), note that although the theorems proved are, in fact,
rather simple, the way they are proved illustrates the kind of
intricate `proof engineering' that is typical.  The proofs could be
done more elegantly, but presenting them that way would defeat the
purpose of illustrating various features of \HOL. It is hoped that the
small example here will give the reader a feel for what it is like to
do a big one. 

Readers who are not interested in hardware verification
should be able to learn something about the
\HOL\ system even if they do not wish to penetrate the details of the
parity-checking example used here.  The specification and verification of a
slightly more complex parity checker is set as an exercise (a solution
is provided).

\section{Introduction}

This case study is supported by three files in
the \HOL\ distribution directory. These files are:

\begin{hol}\begin{verbatim}
   hol/Training/parity/PARITY.ml
   hol/Training/parity/RESET_REG.ml
   hol/Training/parity/RESET_PARITY.ml
\end{verbatim}\end{hol}

The file {\verb%PARITY.ml%} contains the \HOL\ sessions
in this chapter; the files {\verb%RESET_REG.ml%} and
{\verb%RESET_PARITY.ml%} contain the solutions
to the exercises described in Section~\ref{exercises}.

The goal of the case study is to illustrate detailed `proof hacking' on
a small and fairly simple example.

The boxes below contain a little session with the \HOL\ system.  The interactions
in these boxes should be understood as occurring in sequence.  For example,
variable bindings made in earlier boxes are assumed to persist to later ones.
These interactions comprise the specification and verification of a device that
computes the parity of a sequence of bits.  More specifically, a detailed
verification is given of a device with an input {\small\verb%in%}, an output
{\small\verb%out%} and the specification that the $n$th output on
{\small\verb%out%} is {\small\verb%T%} if and only if there have been an even
number of {\small\verb%T%}'s input on {\small\verb%in%}. A
theory named {\small\verb%PARITY%} is constructed; this contains the
specification and verification of the device. All the \ML\ input in the boxes below
can be found in the file {\small\verb%parity/PARITY.ml%}. It is suggested that the
reader interactively input this to get a `hands on' feel for the example.


\section{Specification}
\label{example}
The first step
is to start up the \HOL\ system and then enter draft mode for a new theory
called {\small\verb%PARITY%}.
The \HOL\ system
is entered by typing {\small\verb%hol%} to Unix.\footnote{The Unix 
prompt on the author's machine is {\small{\tt gwyndir\%}}.}
The \HOL\ system then prints a sign-on message
and puts the user into \ML.
The \ML\ prompt is {\small\verb%#%},  so lines beginning
with {\small\verb%#%} 
are typed by the user and other lines are the system's response.

\setcounter{sessioncount}{1}
\begin{session}\begin{verbatim}
gwyndir% hol

       _  _    __    _      __    __
|___   |__|   |  |   |     |__|  |__|
|      |  |   |__|   |__   |__|  |__|

          Version 2.0 (Sun3/Franz), built on Sep 1 1991 

#new_theory`PARITY`;;
() : void
\end{verbatim}
\end{session}

\noindent To specify the device, a primitive recursive
function {\small\verb%PARITY%} is defined 
so that for $n>0$, {\small\tt PARITY} $n f$ is true if the number
of {\small\verb%T%}'s in the sequence 
$f${\small\tt (}$1${\small\tt)}, $\ldots$ , $f${\small\tt (}$n${\small\tt)}
is even.

\begin{session}
\begin{verbatim}
#let PARITY_DEF =
# new_prim_rec_definition
#  (`PARITY_DEF`,
#   "(PARITY 0 f = T) /\
#    (PARITY(SUC n)f = (f(SUC n) => ~(PARITY n f) | PARITY n f))");;
PARITY_DEF = 
|- (!f. PARITY 0 f = T) /\
   (!n f. PARITY(SUC n)f = (f(SUC n) => ~PARITY n f | PARITY n f))
\end{verbatim}
\end{session}

\noindent 
The effect of {\small\verb%new_prim_rec_definition%} 
is to store the definition of {\small\verb%PARITY%}  on the  
current theory with
name {\small\verb%PARITY_DEF%} and to bind the defining theorem to the \ML\
variable with the same name.
Notice  that there  are two name spaces being
written into:  the names of constants in theories and the names of variables in
\ML.  The user is free to manage these names however he or  she wishes (subject
to the various lexical requirements), but a common  convention is  (as here) to
give the definition of a constant {\small\tt CON} the name
{\small\verb%CON_DEF%} in the theory  and also  in \ML.   Another commonly-used
convention is to use just {\small\verb%CON%}  for the  theory and  \ML\ name of
the definition  of  a  constant  {\small\verb%CON%}.   Unfortunately, the \HOL\
system does not use a uniform  convention, but  users are  recommended to adopt
one.


The specification of the parity checking device can now be given as:

{\small\baselineskip\HOLSpacing\begin{verbatim}
   !t. out t = PARITY t in
\end{verbatim}}

\noindent It is {\it intuitively\/} clear that this specification will be satisfied
if the signal\footnote{Signals are modelled as functions from numbers, representing
times, to booleans.} functions {\small\verb%in%} and {\small\verb%out%} satisfy:

{\small\baselineskip\HOLSpacing\begin{verbatim}
   out(0) = T
\end{verbatim}}

\noindent and

{\small\baselineskip\HOLSpacing\begin{verbatim}
   !t. out(t+1)  =  (in(t+1) => ~(out t) | out t)
\end{verbatim}}

\noindent This can be verified formally in \HOL\ by proving the
following lemma:

{\small\baselineskip\HOLSpacing\begin{verbatim}
   !in out. 
    (out 0 = T) /\ (!t. out(SUC t) = (in(SUC t) => ~(out t) | out t))
    ==>
    (!t. out t = PARITY t in)
\end{verbatim}}

\noindent The proof of this is done by Mathematical Induction and, although
trivial, is a good illustration of how such proofs are done.
The lemma is proved interactively using \HOL's subgoal 
package.
The proof is started by putting the goal
to be proved on a goal stack using the function
{\small\verb%set_goal%} which takes a goal as argument. 

\begin{session}
\begin{verbatim}
#set_goal
#([],
# "!in out. 
#   (out 0 = T) /\ (!t. out(SUC t) = (in(SUC t) => ~(out t) | out t))
#   ==>
#   (!t. out t = PARITY t in)");;
"!in out.
  (out 0 = T) /\ (!t. out(SUC t) = (in(SUC t) => ~out t | out t)) ==>
  (!t. out t = PARITY t in)"
\end{verbatim}
\end{session}


\noindent The subgoal package prints out the goal on the top of the goal stack.
The top  goal  is  expanded  by  stripping  off the  universal quantifier (with
{\small\verb%GEN_TAC%}) and then making the two conjuncts of  the antecedent of
the implication into assumptions  of the  goal (with {\small\verb%STRIP_TAC%}).
The \ML\ function {\small\verb%expand%} takes  a tactic  and applies  it to the
top goal; the resulting subgoals are pushed on to the goal stack.   The message
`{\small\verb%OK..%}' is printed out  just before  the tactic  is applied.
The resulting subgoal is then printed.


\begin{session}
\begin{verbatim}
#expand(REPEAT GEN_TAC THEN STRIP_TAC);;
OK..
"!t. out t = PARITY t in"
    [ "out 0 = T" ]
    [ "!t. out(SUC t) = (in(SUC t) => ~out t | out t)" ]
\end{verbatim}
\end{session}

\noindent Next induction on {\small\verb%t%} is done
using {\small\verb%INDUCT_TAC%}, which does
induction on the outermost universally quantified variable.

\begin{session}
\begin{verbatim}
#expand INDUCT_TAC;;
OK..
2 subgoals
"out(SUC t) = PARITY(SUC t)in"
    [ "out 0 = T" ]
    [ "!t. out(SUC t) = (in(SUC t) => ~out t | out t)" ]
    [ "out t = PARITY t in" ]

"out 0 = PARITY 0 in"
    [ "out 0 = T" ]
    [ "!t. out(SUC t) = (in(SUC t) => ~out t | out t)" ]
\end{verbatim}
\end{session}

\noindent The assumptions of the two subgoals
are shown in square brackets. The last goal printed is the one on the top of
the stack, which is the basis case. This is solved by rewriting with its
assumptions and the definition of {\small\verb%PARITY%}.


\begin{session}
\begin{verbatim}
#expand(ASM_REWRITE_TAC[PARITY_DEF]);;
OK..
goal proved
. |- out 0 = PARITY 0 in

Previous subproof:
"out(SUC t) = PARITY(SUC t)in"
    [ "out 0 = T" ]
    [ "!t. out(SUC t) = (in(SUC t) => ~out t | out t)" ]
    [ "out t = PARITY t in" ]
\end{verbatim}
\end{session}

The top goal is proved, so the system pops it from the goal stack
(and puts the proved theorem on a stack of theorems). The new top goal
is the step case of the induction. This goal is also solved by rewriting.

\begin{session}
\begin{verbatim}
#expand(ASM_REWRITE_TAC[PARITY_DEF]);;
OK..
goal proved
.. |- out(SUC t) = PARITY(SUC t)in
.. |- !t. out t = PARITY t in
|- !in out.
    (out 0 = T) /\ (!t. out(SUC t) = (in(SUC t) => ~out t | out t)) ==>
    (!t. out t = PARITY t in)

Previous subproof:
goal proved
\end{verbatim}
\end{session}

\noindent The goal is proved, \ie\ the empty list of subgoals is produced.
The system now applies
the justification functions produced by the 
tactics to the lists of theorems achieving the 
subgoals (starting with the empty list). 
These theorems are printed out in the order in which they are generated
(note that assumptions
of theorems are printed as dots).

The \ML\ function

{\small\baselineskip\HOLSpacing\begin{verbatim}
   save_top_thm : string -> thm
\end{verbatim}}

\noindent saves the theorem just proved (\ie\ on the top of the theorem stack)
in the current theory with a given name.

\begin{session}
\begin{verbatim}
#let UNIQUENESS_LEMMA = save_top_thm `UNIQUENESS_LEMMA`;;
UNIQUENESS_LEMMA = 
|- !in out.
    (out 0 = T) /\ (!t. out(SUC t) = (in(SUC t) => ~out t | out t)) ==>
    (!t. out t = PARITY t in)
\end{verbatim}
\end{session}

\noindent this saves the theorem in the theory {\small\verb%PARITY%} with name
{\small\verb%UNIQUENESS_LEMMA%} and also binds it to an \ML\ variable with the
 same name.

\section{Implementation}
\label{implementation}

The lemma just proved suggests that the parity checker can be implemented by
holding the parity value in a register and then complementing the contents
of the register whenever {\small\verb%T%} is input. To make the implementation
more interesting, it will be assumed that registers `power up' 
storing {\small\verb%F%}. Thus the output at time {\small\verb%0%} cannot be 
taken directly from a register,
because the output of the parity checker at time {\small\verb%0%} is 
specified to be {\small\verb%T%}. Another tricky thing to notice is that 
if {\small\verb%t>0%},
then the output of the parity checker at time {\small\verb%t%} is a 
function of the input at time {\small\verb%t%}. Thus there must be a 
combinational path from the input to the output.

The schematic diagram below shows the design of
a device that is intended to implement this specification.
(The leftmost input to \ml{MUX} is the selector.)
This works by storing the parity of the sequence input so far in the
lower of the two registers.  Each time {\small\verb%T%} is input at
{\small\verb%in%}, this stored value is complemented. Registers are assumed to
`power up' in a state in which they are storing {\small\verb%F%}.  The second
register (connected to {\small\verb%ONE%}) initially outputs
 {\small\verb%F%} and
then outputs {\small\verb%T%} forever.  Its role is just to ensure that the 
device
works during the first cycle by connecting the output {\small\verb%out%} to the
device {\small\verb%ONE%} via the lower multiplexer.  For all subsequent cycles
{\small\verb%out%} is connected to {\small\verb%l3%} and so either carries the
stored parity value (if the current input is {\small\verb%F%}) or the 
complement of this value (if the current input is {\small\verb%T%}).


\setlength{\unitlength}{5mm}
\begin{center}
\begin{picture}(14,30)(0,0.5)
\put(8,20){\framebox(2,2){\small{\tt NOT}}}
\put(6,16){\framebox(6,2){\small{\tt MUX}}}
\put(2,16){\framebox(2,2){\small{\tt ONE}}}
\put(2,12){\framebox(2,2){\small{\tt REG}}}
\put(6,8){\framebox(6,2){\small{\tt MUX}}}
\put(8,4){\framebox(2,2){\small{\tt REG}}}

\puthrule(9,24){4}
\puthrule(3,15){8}
\puthrule(3,11){4}
\puthrule(7,7){2}
\puthrule(9,3){4}

\putvrule(3,11){1}
\putvrule(3,14){2}
\putvrule(7,2){5}
\putvrule(7,10){1}
\putvrule(7,18){8}
\putvrule(9,3){1}
\putvrule(9,6){2}
\putvrule(9,10){6}
\putvrule(9,18){2}
\putvrule(9,22){2}
\putvrule(11,10){5}
\putvrule(11,18){6}
\putvrule(13,3){21}

\put(6,26){\makebox(2,2){\small{\tt in}}}
\put(6,0){\makebox(2,2){\small{\tt out}}}
\put(9,18){\makebox(1.8,2){\small{\tt l1}}}
\put(13,18){\makebox(1.8,2){\small{\tt l2}}}
\put(9,12){\makebox(1.8,2){\small{\tt l3}}}
\put(11,12){\makebox(1.8,2){\small{\tt l4}}}
\put(4,11){\makebox(3,1){\small{\tt l5}}}

\put(10,23){\makebox(2,2){$\bullet$}}
\put(8,6){\makebox(2,2){$\bullet$}}
\put(2,14){\makebox(2,2){$\bullet$}}

\end{picture}
\end{center}
\setlength{\unitlength}{1mm}


The devices making up this schematic will be modelled with predicates
\cite{Why-HOL-paper}. For example, the predicate {\small\verb%ONE%} is true 
of a signal {\small\verb%out%} if for all times {\small\verb%t%} the value of
{\small\verb%out%} is {\small\verb%T%}.

\begin{session}
\begin{verbatim}
#let ONE_DEF =
# new_definition
#  (`ONE_DEF`, "ONE(out:num->bool) = !t. out t = T");;
ONE_DEF = |- !out. ONE out = (!t. out t = T)
\end{verbatim}
\end{session}

\noindent Note that, as discussed above, `{\small\verb%ONE_DEF%}'  is used both
as an \ML\ variable and as the name of the definition in the theory
{\small\verb%PARITY%}.  Note  also  how  `{\small\verb%:num->bool%}' has been
added to resolve type ambiguities; without this (or some other type
information) the typechecker would not be able to  infer that  {\small\tt t} is
to have type {\small\tt num}.

The binary predicate {\small\verb%NOT%} is true of a pair of signals
{\small\verb%(in,out)%} 
if the value of {\small\verb%out%} is always the negation of the value of
{\small\verb%in%}. Inverters are thus modelled
as having no delay. This is appropriate
for a register-transfer level model, but not at a lower level.

\begin{session}
\begin{verbatim}
#let NOT_DEF =
# new_definition
#  (`NOT_DEF`, "NOT(in,out:num->bool) = !t. out t = ~(in t)");;
NOT_DEF = |- !in out. NOT(in,out) = (!t. out t = ~in t)
\end{verbatim}
\end{session}

\noindent The final combinational device needed is a multiplexer.
This is a `hardware conditional'; the input 
{\small\verb%sw%} selects which of the other
two inputs are to be connected to the output {\small\verb%out%}.

\begin{session}
\begin{verbatim}
#let MUX_DEF =
# new_definition
#  (`MUX_DEF`, 
#   "MUX(sw,in1,in2,out:num->bool) = 
#     !t. out t = (sw t => in1 t | in2 t)");;
MUX_DEF = 
|- !sw in1 in2 out.
    MUX(sw,in1,in2,out) = (!t. out t = (sw t => in1 t | in2 t))
\end{verbatim}
\end{session}

The remaining devices in the schematic are registers.
These are unit-delay elements; the values output at time {\small\verb%t+1%} are
the values input at the preceding time {\small\verb%t%}, 
except at time {\small\verb%0%}
when the register
outputs {\small\verb%F%}.\footnote{Time {\tt {\small 0}} 
represents when the device is switched on.}

\begin{session}
\begin{verbatim}
#let REG_DEF =
# new_definition
# (`REG_DEF`, "REG(in,out:num->bool) = 
#              !t. out t = ((t=0) => F | in(t-1))");;
REG_DEF = 
|- !in out. REG(in,out) = (!t. out t = ((t = 0) => F | in(t - 1)))
\end{verbatim}
\end{session}

The schematic diagram above can be represented as a predicate by
conjoining the relations holding between the various 
signals and then existentially quantifying the internal lines.
This technique is explained elsewhere 
(\eg\ see \cite{Camilleri-et-al,Why-HOL-paper}).

\begin{session}
\begin{verbatim}
#let PARITY_IMP_DEF =
# new_definition
#  (`PARITY_IMP_DEF`,
#   "PARITY_IMP(in,out) =
#    ?l1 l2 l3 l4 l5. 
#     NOT(l2,l1) /\ MUX(in,l1,l2,l3) /\ REG(out,l2) /\
#     ONE l4     /\ REG(l4,l5)       /\ MUX(l5,l3,l4,out)");;
PARITY_IMP_DEF = 
|- !in out.
    PARITY_IMP(in,out) =
    (?l1 l2 l3 l4 l5.
      NOT(l2,l1) /\
      MUX(in,l1,l2,l3) /\
      REG(out,l2) /\
      ONE l4 /\
      REG(l4,l5) /\
      MUX(l5,l3,l4,out))
\end{verbatim}
\end{session}\label{parity-imp}

\section{Verification}

The following theorem will eventually be proved:
{\small\baselineskip\HOLSpacing\begin{verbatim}
   |- !in out. PARITY_IMP(in,out) ==> (!t. out t = PARITY t in)
\end{verbatim}}
This states that {\it if\/} {\small\verb%in%} and {\small\verb%out%} 
are related as in the schematic
diagram (\ie\ as in the definition of {\small\verb%PARITY_IMP%}), 
{\it then\/} the 
pair of signals {\small\verb%(in,out)%} satisfies the specification.

First, the following lemma is proved; the correctness of the parity checker
 follows from this and 
{\small\verb%UNIQUENESS_LEMMA%} by the transitivity of {\small{\tt\verb+==>+}}.

\begin{session}
\begin{verbatim}
#set_goal
# ([], "!in out. 
#         PARITY_IMP(in,out) ==> 
#         (out 0 = T) /\ 
#         !t. out(SUC t) = (in(SUC t) => ~(out t) | out t)");;
"!in out.
  PARITY_IMP(in,out) ==>
  (out 0 = T) /\ (!t. out(SUC t) = (in(SUC t) => ~out t | out t))"
\end{verbatim}
\end{session}

The first step in proving this goal is to rewrite with definitions 
followed by a decomposition of the resulting goal using 
{\small\verb%STRIP_TAC%}. The rewriting
tactic {\small\verb%PURE_REWRITE_TAC%} is used; this
does no built-in simplifications, only the ones
explicitly given in the list of theorems supplied as an argument. 
One of the
built-in simplifications used by
{\small\verb%REWRITE_TAC%} is {\small\tt |-~(x~=~T)~=~x}. 
{\small\verb%PURE_REWRITE_TAC%} is used to prevent rewriting with this being done.
\begin{session}
\begin{verbatim}
#expand(PURE_REWRITE_TAC
#        [PARITY_IMP_DEF;ONE_DEF;NOT_DEF;MUX_DEF;REG_DEF]
#        THEN REPEAT STRIP_TAC);;
OK..
2 subgoals
"out(SUC t) = (in(SUC t) => ~out t | out t)"
    [ "!t. l1 t = ~l2 t" ]
    [ "!t. l3 t = (in t => l1 t | l2 t)" ]
    [ "!t. l2 t = ((t = 0) => F | out(t - 1))" ]
    [ "!t. l4 t = T" ]
    [ "!t. l5 t = ((t = 0) => F | l4(t - 1))" ]
    [ "!t. out t = (l5 t => l3 t | l4 t)" ]

"out 0 = T"
    [ "!t. l1 t = ~l2 t" ]
    [ "!t. l3 t = (in t => l1 t | l2 t)" ]
    [ "!t. l2 t = ((t = 0) => F | out(t - 1))" ]
    [ "!t. l4 t = T" ]
    [ "!t. l5 t = ((t = 0) => F | l4(t - 1))" ]
    [ "!t. out t = (l5 t => l3 t | l4 t)" ]
\end{verbatim}
\end{session}

The top goal is the one printed last; its conclusion is 
{\small\verb%out 0 = T%} and its assumptions are
equations relating the values on the lines in the circuit.
The natural next step is to expand the top goal
by rewriting with the assumptions.
However, if this is done the system will go into an infinite 
loop because the equations 
for {\small\verb%out%}, {\small\verb%l2%} and {\small\verb%l3%} are mutually
recursive. To prevent looping,
rewriting must be done with a non-looping subset of the assumptions. 

To enable the assumptions corresponding to particular lines to
be selected for rewriting, an \ML\ function {\small\verb%lines%} is defined
such that {\small\verb%lines `%}$l_1\ldots l_n${\small\verb%` %}$t$ is 
true whenever $t$ has the
form {\small\verb%"!t. %}$l_i${\small\verb% t = %}$\cdots${\small\verb% "%} 
for some $l_i$ in the
set specified by the string {\small\verb%`%}$l_1\ldots l_n${\small\verb%`%}. 
The functions
{\small\verb%words%} and {\small\verb%rator%} used below
are predeclared in \ML.
\begin{session}
\begin{verbatim}
#let lines tok t =
# (let x = fst(dest_var(rator(lhs(snd(dest_forall t)))))
#  in
#  mem x (words tok)) ? false;;
lines = - : (string -> term -> bool)
\end{verbatim}
\end{session}
{\small\verb%FILTER_ASM_REWRITE_TAC(lines`out l4 l5`)[]%}
is a tactic which rewrites 
with only those assumptions that involve {\small\verb%out%},
{\small\verb%l4%} and {\small\verb%l5%}.
\begin{session}
\begin{verbatim}
#expand(FILTER_ASM_REWRITE_TAC(lines`out l4 l5`)[]);;
OK..
goal proved
... |- out 0 = T

Previous subproof:
"out(SUC t) = (in(SUC t) => ~out t | out t)"
    [ "!t. l1 t = ~l2 t" ]
    [ "!t. l3 t = (in t => l1 t | l2 t)" ]
    [ "!t. l2 t = ((t = 0) => F | out(t - 1))" ]
    [ "!t. l4 t = T" ]
    [ "!t. l5 t = ((t = 0) => F | l4(t - 1))" ]
    [ "!t. out t = (l5 t => l3 t | l4 t)" ]
\end{verbatim}
\end{session}
The first of the two subgoals
is proved.  Inspecting the remaining 
goal it can be seen that it will be solved if its left hand side,
{\small\verb%out(SUC t)%}, is expanded using the assumption:
\medskip

{\small\verb%!t. out t = (l5 t => l3 t | l4 t)%}

\medskip
However, if this assumption is used for rewriting, 
then all the subterms of the form {\small\verb%out t%} will also
be expanded. To prevent this, the messy and \adhoc\ tactic shown
below will be used. 

{\small\baselineskip\HOLSpacing\begin{verbatim}
   FIRST_ASSUM
    (\th. if lines`out`(concl th) 
           then SUBST_TAC[SPEC "SUC t" th]
           else NO_TAC)
\end{verbatim}}

{\small\verb%FIRST_ASSUM%}~$f$ is a built-in tactical 
that applies $f$ in succession to the assumed assumptions of the goal;
the tactic resulting from the first successful application is applied.
The argument to {\small\verb%FIRST_ASSUM%} used here is:

{\small\baselineskip\HOLSpacing\begin{verbatim}
   (\th. if lines`out`(concl th) 
          then SUBST_TAC[SPEC "SUC t" th]
          else NO_TAC)
\end{verbatim}}

\noindent which converts a theorem of the form:

\medskip

{\small\verb%  |- !t. out t = %}$tm${\small\verb%[t]%}

\medskip

\noindent
to a tactic that replaces occurrences of
{\small\verb%out(SUC t)%} by $tm${\small\verb%[SUC t]%}.
All other theorems are mapped to {\small\verb%NO_TAC%}, 
a tactic that always fails.

\begin{session}
\begin{verbatim}
#expand
# (FIRST_ASSUM
#   (\th. if lines`out`(concl th) 
#          then SUBST_TAC[SPEC "SUC t" th]
#          else NO_TAC));;
OK..
"(l5(SUC t) => l3(SUC t) | l4(SUC t)) = (in(SUC t) => ~out t | out t)"
    [ "!t. l1 t = ~l2 t" ]
    [ "!t. l3 t = (in t => l1 t | l2 t)" ]
    [ "!t. l2 t = ((t = 0) => F | out(t - 1))" ]
    [ "!t. l4 t = T" ]
    [ "!t. l5 t = ((t = 0) => F | l4(t - 1))" ]
    [ "!t. out t = (l5 t => l3 t | l4 t)" ]
\end{verbatim}
\end{session}

The fact that this messy use
of {\small\verb%FIRST_ASSUM%} was resorted to in the proof shown above
illustrates both the strengths and weaknesses of \HOL.
Trivial deductions sometimes require elaborate tactics, but on the
other hand one never reaches an impasse. \HOL\ experts can prove 
arbitrarily complicated theorems if
they are willing to use sufficient ingenuity. Furthermore, 
the type discipline ensures
that no matter how complicated and \adhoc\ are the tactics, it is impossible
to prove an invalid theorem.

Inspecting the goal above, it can be seen that the next step is to unwind
the equations for lines {\small\verb%l1%}, 
{\small\verb%l3%}, {\small\verb%l4%} and {\small\verb%l5%}
and then, when this is done, unwind with the equation for line
{\small\verb%l2%}.
\vfill
\newpage

\begin{session}
\begin{verbatim}
#expand(FILTER_ASM_REWRITE_TAC(lines`l1 l3 l4 l5`)[]
#        THEN FILTER_ASM_REWRITE_TAC(lines`l2`)[]);;
OK..
"(((SUC t = 0) => F | T) => 
  (in(SUC t) => 
   ~((SUC t = 0) => F | out((SUC t) - 1)) | 
   ((SUC t = 0) => F | out((SUC t) - 1))) | 
  T) =
 (in(SUC t) => ~out t | out t)"
    [ "!t. l1 t = ~l2 t" ]
    [ "!t. l3 t = (in t => l1 t | l2 t)" ]
    [ "!t. l2 t = ((t = 0) => F | out(t - 1))" ]
    [ "!t. l4 t = T" ]
    [ "!t. l5 t = ((t = 0) => F | l4(t - 1))" ]
    [ "!t. out t = (l5 t => l3 t | l4 t)" ]
\end{verbatim}
\end{session}

This goal can now be solved by rewriting with two
standard built-in theorems:

{\small\baselineskip\HOLSpacing\begin{verbatim}
   NOT_SUC   =     |- !n. ~(SUC n = 0)

   SUC_SUB1  =     |- !m. (SUC m) - 1 = m
\end{verbatim}}

\begin{session}
\begin{verbatim}
#expand(REWRITE_TAC[NOT_SUC;SUC_SUB1]);;
OK..
Theorem SUC_SUB1 autoloaded from theory `arithmetic`.
SUC_SUB1 = |- !m. (SUC m) - 1 = m

OK..
goal proved
|- (((SUC t = 0) => F | T) => 
    (in(SUC t) => 
     ~((SUC t = 0) => F | out((SUC t) - 1)) | 
     ((SUC t = 0) => F | out((SUC t) - 1))) | 
    T) =
   (in(SUC t) => ~out t | out t)
..... |- (l5(SUC t) => l3(SUC t) | l4(SUC t)) =
         (in(SUC t) => ~out t | out t)
...... |- out(SUC t) = (in(SUC t) => ~out t | out t)
|- !in out.
    PARITY_IMP(in,out) ==>
    (out 0 = T) /\ (!t. out(SUC t) = (in(SUC t) => ~out t | out t))

Previous subproof:
goal proved
\end{verbatim}
\end{session}

\noindent The theorem just proved is named
{\small\verb%PARITY_LEMMA%} and saved in the current theory.

\begin{session}
\begin{verbatim}
#let PARITY_LEMMA = save_top_thm `PARITY_LEMMA`;;
PARITY_LEMMA = 
|- !in out.
    PARITY_IMP(in,out) ==>
    (out 0 = T) /\ (!t. out(SUC t) = (in(SUC t) => ~out t | out t))
\end{verbatim}
\end{session}

{\small\verb%PARITY_LEMMA%} could have been proved in one step with a single
compound tactic. This is illustrated below:

\begin{session}
\begin{verbatim}
#set_goal
# ([], "!in out. 
#         PARITY_IMP(in,out) ==> 
#         (out 0 = T) /\ 
#         !t. out(SUC t) = (in(SUC t) => ~(out t) | out t)");;
"!in out.
  PARITY_IMP(in,out) ==>
  (out 0 = T) /\ (!t. out(SUC t) = (in(SUC t) => ~out t | out t))"
\end{verbatim}
\end{session}

\noindent This goal can be expanded with a single tactic corresponding to the
sequence of tactics that were used interactively.

\begin{session}
\begin{verbatim}
#expand
# (PURE_REWRITE_TAC[PARITY_IMP_DEF;ONE_DEF;NOT_DEF;MUX_DEF;REG_DEF]
#  THEN REPEAT STRIP_TAC
#  THENL
#   [FILTER_ASM_REWRITE_TAC(lines`out l4 l5`)[];ALL_TAC]
#  THEN FIRST_ASSUM
#        (\th. if lines`out`(concl th) 
#               then SUBST_TAC[SPEC "SUC t" th]
#               else NO_TAC)
#  THEN FILTER_ASM_REWRITE_TAC(lines`l1 l3 l4 l5`)[]
#  THEN FILTER_ASM_REWRITE_TAC(lines`l2`)[]
#  THEN REWRITE_TAC[NOT_SUC;SUC_SUB1]);;
OK..
goal proved
|- !in out.
    PARITY_IMP(in,out) ==>
    (out 0 = T) /\ (!t. out(SUC t) = (in(SUC t) => ~out t | out t))

Previous subproof:
goal proved
\end{verbatim}
\end{session}

Armed with {\small\verb%PARITY_LEMMA%}, 
the final theorem is easily proved.
This will be done in one step using the \ML\ function
{\small\verb%prove_thm%}.
\vfill
\newpage

\begin{session}
\begin{verbatim}
#let PARITY_CORRECT =
# prove_thm
#  (`PARITY_CORRECT`,
#   "!in out. PARITY_IMP(in,out) ==> (!t. out t = PARITY t in)",
#   REPEAT GEN_TAC
#    THEN DISCH_TAC
#    THEN IMP_RES_TAC PARITY_LEMMA
#    THEN IMP_RES_TAC UNIQUENESS_LEMMA);;
PARITY_CORRECT = 
|- !in out. PARITY_IMP(in,out) ==> (!t. out t = PARITY t in)

#close_theory();;
() : void
\end{verbatim}
\end{session}

\noindent This completes the proof of the
parity checking device. 

\section{Exercises}
\label{exercises}

Two exercises are given in this section:
Exercise~1 is straightforward, but Exercise~2 is quite tricky and
might take a beginner several days to solve. The solutions to these exercises are
in the files:

\begin{hol}\begin{verbatim}
   hol/Training/parity/RESET_REG.ml
   hol/Training/parity/RESET_PARITY.ml
\end{verbatim}\end{hol}


\subsection{Exercise 1}

Using {\it only\/} the devices {\small\verb%ONE%}, {\small\verb%NOT%},
{\small\verb%MUX%} and {\small\verb%REG%} defined in Section~\ref{implementation},
design and verify a register {\small\verb%RESET_REG%} 
with an input {\small\verb%in%}, reset line {\small\verb%reset%},
output {\small\verb%out%} and behaviour specified as follows.
\begin{itemize}
\item If {\small\verb%reset%} is {\small\verb%T%} at time {\small\verb%t%},
then the value at {\small\verb%out%} at time {\small\verb%t%} is also
{\small\verb%T%}.
\item If {\small\verb%reset%} is {\small\verb%T%} at time {\small\verb%t%} or
{\small\verb%t+1%},
then the value output at {\small\verb%out%} at time {\small\verb%t+1%} is
{\small\verb%T%}, otherwise it is equal to
the value input at time {\small\verb%t%} on {\small\verb%in%}.
\end{itemize}
This is formalized in \HOL\ by the definition:

{\small\baselineskip\HOLSpacing\begin{verbatim}
   RESET_REG(reset,in,out) =
    (!t. reset t ==> (out t = T)) /\
    (!t. out(t+1) = ((reset t  \/ reset(t+1)) => T | in t))
\end{verbatim}}

\noindent Note that this specification is only partial; it doesn't specify the
output at time {\small\verb%0%} in the case that there is no reset.

The solution to the exercise should be a definition of a predicate
{\small\verb%RESET_REG_IMP%} as an existential quantification  of a conjunction
of applications  of  {\small\verb%ONE%}, {\small\verb%NOT%}, {\small\verb%MUX%}
and {\small\verb%REG%} to suitable line  names,\footnote{i.e.   a definition of
the same form as that of {\small\tt PARITY\_IMP} on page~\pageref{parity-imp}.}
together with a proof of:

{\small\baselineskip\HOLSpacing\begin{verbatim}
   RESET_REG_IMP(reset,in,out) ==> RESET_REG(reset,in,out)
\end{verbatim}}


\subsection{Exercise 2}

\begin{enumerate}
\item Formally specify a resetable parity checker that has two boolean
inputs {\small\tt reset}
and {\small\tt in}, and one boolean 
output {\small\tt out} with the following behaviour:
\begin{quote}
The value at {\small\tt out} is {\small\tt T} if and only if there
have been an even number of {\small\tt T}s input at {\small\tt in} since
the last time that {\small\tt T} was input at {\small\tt reset}.
\end{quote}
\item Design an implementation of this specification built
using {\it only\/} the devices {\small\verb%ONE%}, {\small\verb%NOT%},
{\small\verb%MUX%} and {\small\verb%REG%} defined in Section~\ref{implementation}.
\item Verify the correctness of your implementation in \HOL.
\end{enumerate}
                      % parity example
   \chapter{How to program a proof tool}\label{tool}

Users of \HOL\ can create their own theorem proving tools by combining
predefined rules and tactics. The \ML\ type-discipline
ensures that only logically sound methods can be used to create values
of type \ml{thm}.
In this chapter, a simple but real\footnote{The example
is `real' in that the need for it came up last week.} example is described. 

Several implementations of the tool are given to illustrate various styles
of proof programming. The first implementation is the obvious one, but
is very slow because of the `brute force' method used. The second
implementation produces a much more streamlined proof, but still has a
brute force component, namely the use of a tautology checker from the
library \ml{taut}. The third implementation replaces the general
tautology checker with a special purpose derived inference rule. The
fourth and final implementation uses an optimised implementation of
the special purpose rule; understanding it is left as an exercise in
using \DESCRIPTION.

The timings in this chapter are based on Version 1.12. Later versions
of \HOL{} have an optimised tautology checker library due to Richard
Boulton. This new tautology checker is actually faster than the
special purpose derived rule described in
Section~\ref{bogus-optimization}.  Thus with the new tautology checker
the so called ``even more efficient implementation'' is actually
slower than the program it replaces! This was only discovered (by
Juanito Camilleri) during the preparation of Version 2 of the
tutorial. Rather than completely rewriting the chapter, it was decided
to leave it essentially as it was (except for the addition of this
 paragraph). The methods
that are described are still useful, and there is an important lesson
here: optimizations can become obsolete.  The really dedicated reader
could learn a lot by studying the old and new tautology checker
({\small\verb%contrib/icl-taut%} and
{\small\verb%Library/taut%}, respectively) to find out how they work.
Besides improving the tautology library, Richard Boulton also
reimplemented rewriting using ideas from Tom Melham and Roger Fleming.
As a result, in versions later than 1.12 the various rewriting tools
are quite a bit faster and generate fewer intermediate theorems.


It is sometimes claimed that `\LCF-style' systems can never be
practical, because the efficiency needed to handle real examples can
only be obtained with decision procedures coded as primitive rules. It
is hoped that this chapter, as well as the \ml{taut} library, shows
that the truth of such claims is not obvious. Research is currently in
progress to see if a variety of practical decision algorithms can be
implemented as efficient derived rules.

The tool described here is a tactic that puts conjunctions into the
normal form obtained by right associating, sorting the conjuncts into
a canonical order and then removing repetitions. This canonical order
uses the built-in polymorphic infix \ml{<<}, which orders any pair of
\ML\ values with the same type.

\section{A simple implementation}

A first implementation uses `brute-force' rewriting with
the equations:

\begin{hol}\begin{verbatim}
   |- (t1 /\ t2) /\ t3 = t1 /\ (t2 /\ t3)     % Associativity          %

   |- t1 /\ t2 = t2 /\ t1                     % Symmetry (if t2 << t1) %
   |- t1 /\ (t2 /\ t3) = t2 /\ (t1 /\ t3)     % Symmetry (if t2 << t1) %

   |- t /\ t = t                              % Cancel repeated terms  %
   |- t1 /\ (t1 /\ t2) = t1 /\ t2             % Cancel repeated terms  %
\end{verbatim}\end{hol}

\noindent These equations are easily proved using the 
library \ml{taut}. Note that \HOL\ Version 1.12 is used in
this chapter. Versions of \HOL\ later than 1.12 contain improved
rewriting tools and a new version of the library \ml{taut} (the old version
of the library is preserved in the directory
{\small\verb%contrib/icl-taut%}).


\setcounter{sessioncount}{1}
\begin{session}\begin{verbatim}
scaup% hol
          _  _    __    _      __    __
   |___   |__|   |  |   |     |__|  |__|
   |      |  |   |__|   |__   |__|  |__|
   
          Version 1.12 (Sun3/Franz), built on Feb 23 1991

#load_library `taut`;;
Loading library `taut` ...
........................
Library `taut` loaded.
() : void
\end{verbatim}\end{session}

\noindent The library \ml{taut} defines \ml{TAUT\_RULE}\footnote{The function \ml{TAUT\_RULE} has been replaced by a function called \ml{TAUT\_PROVE} 
in the new version of the \ml{taut} library available in versions of 
\HOL\ later than 1.12}
which converts a term to the corresponding theorem, if the term is a tautology.
\vfill
\newpage
\begin{session}\begin{verbatim}
#let ASSOC = TAUT_RULE "(t1 /\ t2) /\ t3 = t1 /\ t2 /\ t3";;
ASSOC = |- (t1 /\ t2) /\ t3 = t1 /\ t2 /\ t3

#let SYM1 = TAUT_RULE "t1 /\ t2 = t2 /\ t1";;
SYM1 = |- t1 /\ t2 = t2 /\ t1

#let SYM2 = TAUT_RULE "t1 /\ t2 /\ t3 = t2 /\ t1 /\ t3";;
SYM2 = |- t1 /\ t2 /\ t3 = t2 /\ t1 /\ t3

#let CANCEL1 = TAUT_RULE "t /\ t = t";;
CANCEL1 = |- t /\ t = t

#let CANCEL2 = TAUT_RULE "t1 /\ t1 /\ t2 = t1 /\ t2";;
CANCEL2 = |- t1 /\ t1 /\ t2 = t1 /\ t2
\end{verbatim}\end{session}

\noindent One cannot just use \ml{REWRITE\_TAC} with \ml{SYM1} and
\ml{SYM2}, because it would loop.  What is needed is a special
rewriting tool that will only apply symmetry when terms are out of
order. Such a tool can be implemented as a {\it conversion\/}.

Conversions are described in detail in \DESCRIPTION. The idea, which
is due to Larry Paulson \cite{lcp_rewrite}, is that a conversion is an
\ML\ function that maps a term $t_1$ to an equation:

\medskip
{\small\verb%|- %}$t_1${\small\verb% = %}$t_2$.  
\medskip

\noindent The intention is that a conversion will only apply to a
subset of terms: on members of this subset it will generate an
equation, on all other terms it will fail. Because conversions are so
central to theorem-proving in \HOL, the \ML\ type
{\small\verb%term->thm%} is abbreviated to {\small\verb%conv%}.
Conversions are applied using the function:

\begin{hol}\begin{verbatim}
   REWR_CONV : thm -> conv
\end{verbatim}\end{hol}

\noindent This takes an equation {\small\verb%|- %}$t_1${\small\verb% = %}$t_2$
and generates a conversion (\ie\ \ML\ function of type {\small\verb%term->thm%}) 
that maps any term $u$ that matches $t_1$ to the theorem
{\small\verb%|- %}$u${\small\verb% = %}$v$, where $v$ is 
obtained by applying the substitution obtained by matching $u$ with $t_1$ to $t_2$.
If $u$ doesn't match $t_1$ then the application of \ml{REWR\_CONV} fails.


\begin{session}\begin{verbatim}
#REWR_CONV ASSOC "(A /\ B) /\ C";;
|- (A /\ B) /\ C = A /\ B /\ C

#REWR_CONV ASSOC "A /\ (B /\ C)";;
evaluation failed     REWR_CONV: lhs of theorem doesn't match term

#REWR_CONV SYM1 "B /\ A";;
|- B /\ A = A /\ B

#REWR_CONV SYM1 "A \/ B";;
evaluation failed     REWR_CONV: lhs of theorem doesn't match term
\end{verbatim}\end{session}

\noindent For our application, the required conversion should map
a conjunction 

\medskip
$t_1${\small\verb% /\ (%}$t_{2_1}${\small\verb% /\ %}$t_{2_2}${\small\verb%)%} 
\medskip

\noindent in which 
$t_{2_1}${\small\verb% << %}$t_1$ to the equational theorem:

\medskip

{\small\verb%|- %}$t_1${\small\verb% /\ (%}$t_{2_1}${\small\verb% /\ %}$t_{2_2}${\small\verb%)  =  %} $t_{2_1}${\small\verb% /\ (%}$t_1${\small\verb% /\ %}$t_{2_2}${\small\verb%)%} 

\medskip

\noindent If $t_1${\small\verb% << %}$t_{2_1}$ then the conversion fails
(in the \ML\ sense) when applied to
$t_1${\small\verb% /\ (%}$t_{2_1}${\small\verb% /\ %}$t_{2_2}${\small\verb%)%}.
In addition, if the right conjunct is not itself a conjunction, then
the conversion should reorder if necessary. More precisely, if the conversion
is applied to $t_1${\small\verb% /\ %}$t_2$ where $t_2$ is not a conjunction and
$t_2${\small\verb% << %}$t_1$, then it should generate the equation:

\medskip
{\small\verb%|- %}$t_1${\small\verb% /\ %}$t_2${\small\verb%  =  %}$t_2${\small\verb% /\ %}$t_1$
\medskip

\noindent Such a conversion is easily implemented in \ML\ using
\ml{SYM1} and \ml{SYM2} proved above, together with the \ML\ syntax
processing functions \ml{is\_conj} and \ml{dest\_conj}, where:

\begin{hol}\begin{verbatim}
   is_conj   : term -> bool 
   dest_conj : term -> (term # term)
\end{verbatim}\end{hol}

\noindent These are functions that test whether a term is a conjunction, and
splits a term into its two conjuncts, respectively. For example:

\begin{session}\begin{verbatim}
#is_conj "A /\ B";;
true : bool

#is_conj "A \/ B";;
false : bool

#dest_conj "A /\ B";;
("A", "B") : (term # term)

#dest_conj "A \/ B";;
evaluation failed     dest_conj
\end{verbatim}\end{session}

The implementation of the special purpose conversion, 
\ml{CONJ\_ORD\_CONV}, is now straightforward.


\begin{session}\begin{verbatim}
#let CONJ_ORD_CONV t =
# let t1,t2 = dest_conj t
# in
# if is_conj t2
#   then (let t21,t22 = dest_conj t2
#         in
#         if t21 << t1 then REWR_CONV SYM2 t else fail)
#   else (if t2  << t1 then REWR_CONV SYM1 t else fail);;
CONJ_ORD_CONV = - : conv
\end{verbatim}\end{session}

\noindent This is illustrated by:

\begin{session}\begin{verbatim}
#"A:bool" << "B:bool";;
true : bool

#"B:bool" << "C:bool";;
true : bool

#CONJ_ORD_CONV "B /\ A";;
|- B /\ A = A /\ B

#CONJ_ORD_CONV "A /\ B";;
evaluation failed     fail
\end{verbatim}\end{session}

The process of normalizing a conjunction can be split into four phases:

\begin{enumerate}
\item Right associate the conjunction by repeatedly applying:
\begin{quote}
\ml{REWR\_CONV\ ASSOC}
\end{quote}
\item Put the conjuncts in canonical order by repeatedly applying:
\begin{quote}
\ml{CONJ\_ORD\_CONV}
\end{quote}
\item Remove repetitions of $t$ of the form $t${\small\verb% /\ %}$t$ 
by repeatedly applying:
\begin{quote}
\ml{REWR\_CONV\ CANCEL1}
\end{quote}
\item Remove repetitions of $t_1$ in
$t_1${\small\verb% /\ (%}$t_1${\small\verb% /\ %}$t_2${\small\verb%)%}
by repeatedly applying:
\begin{quote}
\ml{REWR\_CONV\ CANCEL2}
\end{quote}
\end{enumerate}


To implement this, a method of repeatedly applying a conversion to
subterms of a term is needed. This is provided by the operator

\begin{hol}\begin{verbatim}
   TOP_DEPTH_CONV : conv -> conv
\end{verbatim}\end{hol}

\noindent If $c$ is a conversion then \ml{TOP\_DEPTH\_CONV}~$c$ is a
conversion that repeatedly applies $c$ to all subterms until $c$ is no
longer applicable to any subterms. The function \ml{TOP\_DEPTH\_CONV}
is one of a family of operators that apply conversions throughout 
terms. Members of this family differ in the order in which subterms
are visited and the amount of repetition that is done. For more
details, see the chapter on conversions in \DESCRIPTION.

\begin{session}\begin{verbatim}
#let ex1 = "A /\ (B /\ C /\ A) /\ (C /\ A /\ D) /\ D";;
ex1 = "A /\ (B /\ C /\ A) /\ (C /\ A /\ D) /\ D" : term

#REWR_CONV ASSOC ex1;;
evaluation failed     REWR_CONV: lhs of theorem doesn't match term

#TOP_DEPTH_CONV (REWR_CONV ASSOC) ex1;;
|- A /\ (B /\ C /\ A) /\ (C /\ A /\ D) /\ D =
   A /\ B /\ C /\ A /\ C /\ A /\ D /\ D
\end{verbatim}\end{session}

\noindent The right hand side of this theorem is \ml{ex1} in right-associated
form. The conclusion of a theorem can be extracted with the \ML\ function
\ml{concl} and the right hand side of an equation can be extracted with
\ml{rhs}. Thus, continuing the session:

\begin{session}\begin{verbatim}
#let ex2 = rhs(concl it);;
ex2 = "A /\ B /\ C /\ A /\ C /\ A /\ D /\ D" : term

#TOP_DEPTH_CONV CONJ_ORD_CONV ex2;;
|- A /\ B /\ C /\ A /\ C /\ A /\ D /\ D =
   A /\ A /\ A /\ B /\ C /\ C /\ D /\ D
\end{verbatim}\end{session}

\noindent The right hand side of this is the result of canonicalizing
the order of the conjuncts in the left hand side. Next, the repetitions
can be eliminated using \ml{CANCEL1} and \ml{CANCEL2}.

\begin{session}\begin{verbatim}
#let ex3 = rhs(concl it);;
ex3 = "A /\ A /\ A /\ B /\ C /\ C /\ D /\ D" : term

#TOP_DEPTH_CONV (REWR_CONV CANCEL1) ex3;;
|- A /\ A /\ A /\ B /\ C /\ C /\ D /\ D =
   A /\ A /\ A /\ B /\ C /\ C /\ D
\end{verbatim}\end{session}

\begin{session}\begin{verbatim}
#let ex4 = rhs(concl it);;
ex4 = "A /\ A /\ A /\ B /\ C /\ C /\ D" : term

#TOP_DEPTH_CONV (REWR_CONV CANCEL2) ex4;;
|- A /\ A /\ A /\ B /\ C /\ C /\ D = A /\ B /\ C /\ D
\end{verbatim}\end{session}


To make the conjunction normalizer, the four stages just described
must be performed in sequence. Conversions can be applied in sequence using
the infixed function:

\begin{hol}\begin{verbatim}
   THENC : conv -> conv -> conv
\end{verbatim}\end{hol}


\noindent If $c_1\ t_1$ evaluates to $\ml{ |- }t_1\ml{=}t_2$ and
$c_2\ t_2$ evaluates to $\ml{ |- }t_2\ml{=}t_3$, then
$\ml{(}c_1\ \ml{THENC}\ c_2\ml{)}\ t_1$ evaluates to
$\ml{\ |-\ }t_1\ml{=}t_3$. If the
evaluation of $c_1\ t_1$ or the evaluation of $c_2\ t_2$ fails,
then so does the evaluation of $c_1\ \ml{THENC}\ c_2$. \ml{THENC} is
justified by the transitivity of equality.

Using \ml{THENC}, the normalizer is defined by

\begin{session}\begin{verbatim}
#let CONJ_NORM_CONV =
# TOP_DEPTH_CONV(REWR_CONV ASSOC)   THENC
# TOP_DEPTH_CONV CONJ_ORD_CONV         THENC
# TOP_DEPTH_CONV(REWR_CONV CANCEL1) THENC
# TOP_DEPTH_CONV(REWR_CONV CANCEL2);;

CONJ_NORM_CONV = - : conv

#CONJ_NORM_CONV ex1;;
|- A /\ (B /\ C /\ A) /\ (C /\ A /\ D) /\ D = A /\ B /\ C /\ D
\end{verbatim}\end{session}

This conversion can now be converted to a rule or tactic using the functions
\ml{CONV\_RULE} or \ml{CONV\_TAC}, respectively.


\begin{hol}
\begin{verbatim}
   CONV_RULE : conv -> thm -> thm
   CONV_TAC  : conv -> tactic
\end{verbatim}
\end{hol}

\noindent $\ml{CONV\_RULE}\ c\ \ml{(|- }t\ml{)}$ returns $\ml{|- }t'$, where
$c\ t$ evaluates to the equation
$\ml{|-}\ t\ml{=}t'$. 
$\ml{CONV\_TAC}\ c$ is a tactic that
converts the conclusion of a goal using $c$. For more details see \DESCRIPTION.

\begin{session}\begin{verbatim}
#let CONJ_NORM_TAC = CONV_TAC CONJ_NORM_CONV;;
CONJ_NORM_TAC = - : tactic
\end{verbatim}\end{session}

Here is an example. It uses {\it antiquotation\/}: if $x$ is an \ML\
indentifier bound to term, then occurrences of
{\small\verb%^%}$x$ inside a quotation
denotes the term bound to $x$.

\begin{session}\begin{verbatim}
#g "^ex1 ==> B";;
"A /\ (B /\ C /\ A) /\ (C /\ A /\ D) /\ D ==> B"

() : void

#e CONJ_NORM_TAC;;
OK..
"A /\ B /\ C /\ D ==> B"
\end{verbatim}\end{session}

To summarize, here is the \ML\ code implementing the normalizer:

\begin{hol}\begin{verbatim}
   load_library `taut`;;

   let ASSOC   = TAUT_RULE "(t1 /\ t2) /\ t3 = t1 /\ t2 /\ t3"
   and SYM1    = TAUT_RULE "t1 /\ t2 = t2 /\ t1"
   and SYM2    = TAUT_RULE "t1 /\ t2 /\ t3 = t2 /\ t1 /\ t3"
   and CANCEL1 = TAUT_RULE "t /\ t = t"
   and CANCEL2 = TAUT_RULE "t1 /\ t1 /\ t2 = t1 /\ t2";;

   let CONJ_ORD_CONV t =
    let t1,t2 = dest_conj t
    in
    if is_conj t2
      then (let t21,t22 = dest_conj t2
            in
            if t21 << t1 then REWR_CONV SYM2 t else fail)
      else (if t2  << t1 then REWR_CONV SYM1 t else fail);;

   let CONJ_NORM_CONV =
    TOP_DEPTH_CONV(REWR_CONV ASSOC)   THENC
    TOP_DEPTH_CONV CONJ_ORD_CONV         THENC
    TOP_DEPTH_CONV(REWR_CONV CANCEL1) THENC
    TOP_DEPTH_CONV(REWR_CONV CANCEL2);;

   let CONJ_NORM_TAC = CONV_TAC CONJ_NORM_CONV;;
\end{verbatim}\end{hol} 

\section{A more efficient implementation}

The normalizer just given is rather slow. This can be shown by switching on the
system timer using the function:

\begin{hol}\begin{verbatim}
   timer : bool -> bool
\end{verbatim}\end{hol}

\noindent Evaluating \ml{timer~true} switches on timing; evaluating
\ml{timer~false} switches it off (the previous value of the timing flag
is returned). Garbage collection times are also shown, together with a
count of the number of intermediate theorems that are generated (which
gives an estimate of the number of primitive inferences done).

\begin{session}\begin{verbatim}
#timer true;;
false : bool
Run time: 0.0s

#CONJ_NORM_CONV ex1;;
|- A /\ (B /\ C /\ A) /\ (C /\ A /\ D) /\ D = A /\ B /\ C /\ D
Run time: 1.1s
Garbage collection time: 0.5s
Intermediate theorems generated: 73
\end{verbatim}\end{session}

\noindent Here is a bigger example:

\begin{session}\begin{verbatim}
#CONJ_NORM_CONV "^ex1 /\ (^ex1 /\ (^ex1 /\ ^ex1 /\ ^ex1) /\ ^ex1)";;
|- (A /\ (B /\ C /\ A) /\ (C /\ A /\ D) /\ D) /\
   (A /\ (B /\ C /\ A) /\ (C /\ A /\ D) /\ D) /\
   ((A /\ (B /\ C /\ A) /\ (C /\ A /\ D) /\ D) /\
    (A /\ (B /\ C /\ A) /\ (C /\ A /\ D) /\ D) /\
    A /\
    (B /\ C /\ A) /\
    (C /\ A /\ D) /\
    D) /\
   A /\
   (B /\ C /\ A) /\
   (C /\ A /\ D) /\
   D =
   A /\ B /\ C /\ D
Run time: 38.3s
Garbage collection time: 11.5s
Intermediate theorems generated: 16761
\end{verbatim}\end{session}

The reason that \ml{CONJ\_CANON\_CONV} is slow is because of the
repeated pattern matching done during rewriting. A much more efficient
approach is to normalize the conjunction by \ML\ programming outside
the logic, and then to prove that the normalized term is equal to the
original one. An even more efficient approach, which is not explored
here, would be to avoid having to do this proof by verifing the
normalization code by some sort of meta-theoretic
reasoning about \ML. How to do this in
\HOL\ is not clear, but work on this approach has been done in the
context of {\small FOL} \cite{FOL}, the Boyer-Moore prover \cite{BoyerMoore}
and Nuprl \cite{Nuprl}. These approaches all use logically
sophisticated extra axioms, called reflection principles, that enable
metatheorems to be `reflected' into the logic as object level
theorems.

To normalize the term by \ML\ programming, the conjuncts are extracted,
repeated elements are deleted and the resulting list is sorted.

\HOL\ already has a predefined function:

\begin{hol}\begin{verbatim}
   conjuncts : term -> term list
\end{verbatim}\end{hol}

\noindent for extracting conjuncts.
\HOL\ also has a predefined \ML\ function for removing repeated elements of
a list:


\begin{hol}\begin{verbatim}
   setify : * list -> * list
\end{verbatim}\end{hol}

\noindent Both \ml{conjuncts} and \ml{setify} are illustrated below:

\begin{session}\begin{verbatim}
#timer false;;
true : bool

#conjuncts ex1;;
["A"; "B"; "C"; "A"; "C"; "A"; "D"; "D"] : term list

#setify it;;
["B"; "C"; "A"; "D"] : term list

#let ex1_list = it;;
ex1_list = ["B"; "C"; "A"; "D"] : term list
\end{verbatim}\end{session}

There is a predefined sorting function in \ML:

\begin{session}\begin{verbatim}
#sort;; 
sort = - : (((* # *) -> bool) -> * list -> * list)

#sort $< [3;2;5;6;1;1;7;9;3];;
[1; 1; 2; 3; 3; 5; 6; 7; 9] : int list

#sort $<< ex1_list;;
["A"; "B"; "C"; "D"] : term list
\end{verbatim}\end{session}

Using this function, the list of conjuncts of the normalization of a
term is easily computed.  The predefined \ML\ function:

\begin{hol}\begin{verbatim}
   list_mk_conj : term list -> term
\end{verbatim}\end{hol}

\noindent can then be used to build the normalized conjunction.


\begin{session}\begin{verbatim}
#ex1;;
"A /\ (B /\ C /\ A) /\ (C /\ A /\ D) /\ D" : term

#let ex1_norm = list_mk_conj(sort $<< (setify(conjuncts ex1)));;
ex1_norm = "A /\ B /\ C /\ D" : term
\end{verbatim}\end{session}

\noindent The calculation of \ml{ex1\_norm} from \ml{ex1} has been
done by (unverified) \ML\ code. What is required is the theorem
asserting that they are equal. This can be proved using the tautology
checker.

\begin{session}\begin{verbatim}
#TAUT_RULE "^ex1 = ^ex1_norm";;
|- A /\ (B /\ C /\ A) /\ (C /\ A /\ D) /\ D = A /\ B /\ C /\ D
\end{verbatim}\end{session}

\noindent A conversion that normalizes conjunctions is thus:

\begin{session}\begin{verbatim}
#let CONJ_NORM_CONV2 t =
# if is_conj t
#  then TAUT_RULE "^t = ^(list_mk_conj(sort $<< (setify(conjuncts t))))"
#  else fail;;
CONJ_NORM_CONV2 = - : conv

#CONJ_NORM_CONV2 ex1;;
|- A /\ (B /\ C /\ A) /\ (C /\ A /\ D) /\ D = A /\ B /\ C /\ D
\end{verbatim}\end{session}

\noindent \ml{CONJ\_CANON\_CONV2} is more than an order of magnitude faster 
than \ml{CONJ\_CANON\_CONV}:


\begin{session}\begin{verbatim}
#timer true;;
false : bool
Run time: 0.0s

#CONJ_NORM_CONV2 "^ex1 /\ (^ex1 /\ (^ex1 /\ ^ex1 /\ ^ex1) /\ ^ex1)";;
|- (A /\ (B /\ C /\ A) /\ (C /\ A /\ D) /\ D) /\
   (A /\ (B /\ C /\ A) /\ (C /\ A /\ D) /\ D) /\
   ((A /\ (B /\ C /\ A) /\ (C /\ A /\ D) /\ D) /\
    (A /\ (B /\ C /\ A) /\ (C /\ A /\ D) /\ D) /\
    A /\
    (B /\ C /\ A) /\
    (C /\ A /\ D) /\
    D) /\
   A /\
   (B /\ C /\ A) /\
   (C /\ A /\ D) /\
   D =
   A /\ B /\ C /\ D
Run time: 1.9s
Garbage collection time: 0.5s
Intermediate theorems generated: 1273
\end{verbatim}\end{session}

\section{An even more efficient implementation}\label{bogus-optimization}

Although the implementation just given is much faster than the first
naive one, it can be improved further by replacing the call to the
general tautology checker with a special purpose conjunction-equivalence
prover.

To see how this works, the equivalence of \ml{ex1} and \ml{ex1\_norm} will first
be proved manually. The general form of this proof will then be abstracted into
a derived rule.

The goal is to prove that \ml{ex1} and \ml{ex1\_norm} are equal.

\begin{session}\begin{verbatim}
#timer false;;
true : bool

#g "^ex1 = ^ex1_norm";;
"A /\ (B /\ C /\ A) /\ (C /\ A /\ D) /\ D = A /\ B /\ C /\ D"

() : void
\end{verbatim}\end{session}

\noindent The predefined tactic \ml{EQ\_TAC} splits an equation into two
implications (see Section~\ref{EQTAC}).

\begin{session}\begin{verbatim}
#e EQ_TAC;;
OK..
2 subgoals
"A /\ B /\ C /\ D ==> A /\ (B /\ C /\ A) /\ (C /\ A /\ D) /\ D"

"A /\ (B /\ C /\ A) /\ (C /\ A /\ D) /\ D ==> A /\ B /\ C /\ D"

() : void
\end{verbatim}\end{session}

\noindent Each of these can be solved by:
\begin{enumerate}
\item moving the antecedent of the
implication to the assumption list (using \ml{DISCH\_TAC}, see
Section~\ref{DISCHTAC});
\item breaking up the remaining
goal (the consequent of the implication) into one subgoal per conjunct (using
\ml{CONJ\_TAC}, see Section~\ref{CONJTAC});
\item  solving each of these
conjuncts using the antecedent (which is now an assumption)
\end{enumerate}

\noindent Step 1--3 are now done interactively.

\begin{session}\begin{verbatim}
#e DISCH_TAC;;
OK..
"A /\ B /\ C /\ D"
    [ "A /\ (B /\ C /\ A) /\ (C /\ A /\ D) /\ D" ]

() : void
\end{verbatim}\end{session}

\noindent \ml{CONJ\_TAC} is repeated using the tactical \ml{REPEAT}
described in Section~\ref{THEN}.

\begin{session}\begin{verbatim}
#e (REPEAT CONJ_TAC);;
OK..
4 subgoals
"D"
    [ "A /\ (B /\ C /\ A) /\ (C /\ A /\ D) /\ D" ]

"C"
    [ "A /\ (B /\ C /\ A) /\ (C /\ A /\ D) /\ D" ]

"B"
    [ "A /\ (B /\ C /\ A) /\ (C /\ A /\ D) /\ D" ]

"A"
    [ "A /\ (B /\ C /\ A) /\ (C /\ A /\ D) /\ D" ]

() : void
\end{verbatim}\end{session}

\noindent The final step is to use the assumption
{\small\verb%"A /\ (B /\ C /\ A) /\ (C /\ A /\ D) /\ D"%} to solve each goal.
To do this, the assumption is grabbed using the tactical:

\begin{hol}\begin{verbatim}
   POP_ASSUM : (thm -> tactic) -> tactic
\end{verbatim}\end{hol}

\noindent Given a function \ml{$f$ : thm -> tactic}, the tactic
\ml{POP\_ASSUM}\ $f$ applies $f$ to the (assumed) first
assumption of a goal
and then applies the tactic created thereby to the original goal
minus its top assumption:

\begin{hol}\begin{alltt}
   POP_ASSUM \(f\) ([\(t\sb{1}\);\(\ldots\);\(t\sb{n}\)],\(t\)) = \(f\) (ASSUME \(t\sb{1}\)) ([\(t\sb{2}\);\(\ldots\);\(t\sb{n}\)],\(t\))
\end{alltt}\end{hol}

\noindent \ML\ functions such as $f$,
with type \ml{thm -> tactic} are abbreviated to \ml{thm\_tactic} (see
\DESCRIPTION\ for further details).

After grabbing the assumption, it is split into its individual conjunctions
using the predefined derived rule:

\begin{hol}\begin{verbatim}
   CONJUNCTS : thm -> thm list
\end{verbatim}\end{hol}

\noindent For example:


\begin{session}\begin{verbatim}
#CONJUNCTS(ASSUME "A /\ (B /\ C /\ A) /\ (C /\ A /\ D) /\ D");;
[. |- A; . |- B; . |- C; . |- A; . |- C; . |- A; . |- D; . |- D]
: thm list
\end{verbatim}\end{session}

\noindent Among the individual conjuncts is the goal, which can thus be
solved immediately using \ml{ACCEPT\_TAC} (see Section~\ref{ACCEPTTAC}).
The appropriate assumption can be chosen with the predefined tactical
\ml{MAP\_FIRST}, which
is characterized by:

\begin{hol}\begin{alltt}
   MAP_FIRST \(f\) [\(x\sb{1}\); \(\ldots\) ;\(x\sb{n}\)]  =  \(f\)(\(x\sb{1}\)) ORELSE \(\ldots\) ORELSE \(f\)(\(x\sb{n}\))
\end{alltt}\end{hol}

\noindent Returning to the proof: the final step is now performed by
popping the assumption and applying to it the function obtained by
composing \ml{CONJUNCTS} and \ml{MAP\_FIRST} using the \ML\ infixed
function composition operator \ml{o} 
(where \ml{(}$f$~\ml{o}~$g$\ml{)}$x$~\ml{=}~$g$\ml{(}$f$\ml{(}$x$\ml{))}).

\begin{session}\begin{verbatim}
#e(POP_ASSUM(MAP_FIRST ACCEPT_TAC o CONJUNCTS));;
OK..
goal proved
. |- A

Previous subproof:
3 subgoals
"D"
    [ "A /\ (B /\ C /\ A) /\ (C /\ A /\ D) /\ D" ]

"C"
    [ "A /\ (B /\ C /\ A) /\ (C /\ A /\ D) /\ D" ]

"B"
    [ "A /\ (B /\ C /\ A) /\ (C /\ A /\ D) /\ D" ]

() : void
\end{verbatim}\end{session}


\noindent The remaining subgoals are solved identically. Stitching together
the tactics just used results in:

\begin{hol}\begin{verbatim}
   EQ_TAC          THEN 
   DISCH_TAC       THEN
   REPEAT CONJ_TAC THEN 
   POP_ASSUM(MAP_FIRST ACCEPT_TAC o CONJUNCTS)
\end{verbatim}\end{hol}

\noindent With this, the entire proof can be done in one step.

\begin{session}\begin{verbatim}
#g "^ex1 = ^ex1_norm";;
"A /\ (B /\ C /\ A) /\ (C /\ A /\ D) /\ D = A /\ B /\ C /\ D"

() : void

#e(EQ_TAC          THEN 
#  DISCH_TAC       THEN
#  REPEAT CONJ_TAC THEN 
#  POP_ASSUM(MAP_FIRST ACCEPT_TAC o CONJUNCTS));;
OK..
goal proved
|- A /\ (B /\ C /\ A) /\ (C /\ A /\ D) /\ D = A /\ B /\ C /\ D

Previous subproof:
goal proved
() : void
\end{verbatim}\end{session}

Using this tactic, a derived rule \ml{CONJ\_EQ} can be defined that proves
two conjunctions equal. This is what is needed to replace the call to
\ml{TAUT\_RULE}.
\ml{CONJ\_EQ} is defined with the predefined function:

\begin{hol}\begin{verbatim}
   PROVE : term # tactic -> theorem
\end{verbatim}\end{hol}

\noindent \ml{PROVE}\ml{(}$t$\ml{,}$T$\ml{)} applies the tactic $T$ to
the goal \ml{([],}$t$\ml{)}; if this goal is proved by $T$ then the
resulting justification is applied to \ml{[]} to obtain the theorem
\ml{|-}~$t$, which is returned. If $T$ does not solve the goal, then
the application of \ml{PROVE} fails. Using \ml{PROVE}, the definition
of \ml{CONJ\_EQ} is:

\begin{hol}\begin{verbatim}
   let CONJ_EQ t1 t2 = 
    PROVE ("^t1 = ^t2", 
           EQ_TAC          THEN 
           DISCH_TAC       THEN
           REPEAT CONJ_TAC THEN 
           POP_ASSUM(MAP_FIRST ACCEPT_TAC o CONJUNCTS))
\end{verbatim}\end{hol}


\noindent Replacing the call to \ml{TAUT\_RULE} in the definition
of \ml{CONJ\_NORM\_CONV2} results in:

\begin{hol}\begin{verbatim}
   let CONJ_NORM_CONV3 t =
    if is_conj t
     then CONJ_EQ t (list_mk_conj(sort $<< (setify(conjuncts t))))
     else fail
\end{verbatim}\end{hol}

\noindent Continuing the session:

\begin{session}\begin{verbatim}
#let CONJ_EQ t1 t2 = 
# PROVE ("^t1 = ^t2", 
#        EQ_TAC          THEN 
#        DISCH_TAC       THEN
#        REPEAT CONJ_TAC THEN 
#        POP_ASSUM(MAP_FIRST ACCEPT_TAC o CONJUNCTS));;
CONJ_EQ = - : (term -> conv)

#let CONJ_NORM_CONV3 t =
# if is_conj t
#  then CONJ_EQ t (list_mk_conj(sort $<< (setify(conjuncts t))))
#  else fail;;
CONJ_NORM_CONV3 = - : conv
\end{verbatim}\end{session}

\noindent \ml{CONJ\_NORM\_CONV3} is almost twice
as efficient as \ml{CONJ\_NORM\_CONV2}. To show this, the timer is switched 
back on.

\begin{session}\begin{verbatim}
#timer true;;
false : bool
Run time: 0.0s
\end{verbatim}\end{session}

\noindent Here is the big example with \ml{CONJ\_NORM\_CONV3}:

\begin{session}\begin{verbatim}
#CONJ_NORM_CONV3 "^ex1 /\ (^ex1 /\ (^ex1 /\ ^ex1 /\ ^ex1) /\ ^ex1)";;
|- (A /\ (B /\ C /\ A) /\ (C /\ A /\ D) /\ D) /\
   (A /\ (B /\ C /\ A) /\ (C /\ A /\ D) /\ D) /\
   ((A /\ (B /\ C /\ A) /\ (C /\ A /\ D) /\ D) /\
    (A /\ (B /\ C /\ A) /\ (C /\ A /\ D) /\ D) /\
    A /\
    (B /\ C /\ A) /\
    (C /\ A /\ D) /\
    D) /\
   A /\
   (B /\ C /\ A) /\
   (C /\ A /\ D) /\
   D =
   A /\ B /\ C /\ D
Run time: 1.0s
Garbage collection time: 0.5s
Intermediate theorems generated: 775
\end{verbatim}\end{session}

\section{Further optimizations}

Further improvements are still possible. As an exercise the reader
might want to decipher the following highly optimized definition of
\ml{CONJ\_EQ}.

The function \ml{PROVE\_CONJ}, defined below, 
converts a term $t$ to the theorem \ml{|-}~$t$
if that theorem occurs in a supplied list of theorems (\ml{ths} in the
code below), or $t$ is a conjunction each of whose conjuncts occurs in
the list. The definition of \ml{PROVE\_CONJ} uses the following
predefined \ML\ functions:

\begin{itemize}
\item \ml{uncurry}~$f$~\ml{(}$x$\ml{,}$y$\ml{)}~~\ml{=}~~$f$~$x$~$y$

\item \ml{(}$f${\small\verb% # %}$g$\ml{)}\ml{(}$x$\ml{,}$y$\ml{)}~~\ml{=}~~\ml{(}$f\ x$~\ml{,}~$g\ y$\ml{)}

\item \ml{find}~$p$~\ml{[}$x_1\ml{;}\ldots\ml{;}x_n$\ml{]}~~=~~{\it the first $x_i$ for which $p\ x_i$ is true\/}

\end{itemize}

\noindent and the inference rule \ml{CONJ}:


\[ \Gamma_1\turn
t_1\qquad\qquad\qquad\Gamma_2\turn t_2\over \Gamma_1\cup\Gamma_2 \turn t_1\conj
t_2 \]


\noindent Here is the definition of \ml{PROVE\_CONJ}:

\begin{hol}\begin{verbatim}
   letrec PROVE_CONJ ths tm =
    (uncurry CONJ ((PROVE_CONJ ths # PROVE_CONJ ths) (dest_conj tm))) ?
    find (\th. concl th = tm) ths
\end{verbatim}\end{hol}

\noindent Using this, the optimized \ml{CONJ\_EQ}, called
\ml{CONJ\_EQ2}, is defined using \ml{IMP\_ANTISYM\_RULE} (a predefined rule):


\[ \Gamma_1 \turn t_1 \imp t_2 \qquad\qquad \Gamma_2\turn t_2 \imp t_1\over
\Gamma_1 \cup \Gamma_2 \turn t_1 = t_2\]

\noindent The definition is:

\begin{hol}\begin{verbatim}
   let CONJ_EQ2 t1 t2 =
    let imp1 = DISCH t1 (PROVE_CONJ (CONJUNCTS(ASSUME t1)) t2)
    and imp2 = DISCH t2 (PROVE_CONJ (CONJUNCTS(ASSUME t2)) t1) 
    in IMP_ANTISYM_RULE imp1 imp2
\end{verbatim}\end{hol}

\noindent Loading these \ML\ function definitions into \HOL:

\begin{session}\begin{verbatim}
#letrec PROVE_CONJ ths tm =
# (uncurry CONJ ((PROVE_CONJ ths # PROVE_CONJ ths) (dest_conj tm))) ?
# find (\th. concl th = tm) ths;;
PROVE_CONJ = - : (thm list -> conv)
Run time: 0.0s

#let CONJ_EQ2 t1 t2 =
# let imp1 = DISCH t1 (PROVE_CONJ (CONJUNCTS(ASSUME t1)) t2)
# and imp2 = DISCH t2 (PROVE_CONJ (CONJUNCTS(ASSUME t2)) t1) 
# in IMP_ANTISYM_RULE imp1 imp2;;
CONJ_EQ2 = - : (term -> conv)
Run time: 0.0s
\end{verbatim}\end{session}


\noindent A version of \ml{CONJ\_NORM\_CONV} that
uses \ml{CONJ\_EQ2} is defined by:

\begin{hol}\begin{verbatim}
   let CONJ_NORM_CONV4 t =
    if is_conj t
     then CONJ_EQ2 t (list_mk_conj(sort $<< (setify(conjuncts t))))
     else fail
\end{verbatim}\end{hol}


\noindent Loading this into \ML:

\begin{session}\begin{verbatim}
#let CONJ_NORM_CONV4 t =
# if is_conj t
#  then CONJ_EQ2 t (list_mk_conj(sort $<< (setify(conjuncts t))))
#  else fail;;
CONJ_NORM_CONV4 = - : conv
Run time: 0.0s
\end{verbatim}\end{session}

\noindent This is even faster than \ml{CONJ\_NORM\_CONV3}:

\begin{session}\begin{verbatim}
#CONJ_NORM_CONV4 "^ex1 /\ (^ex1 /\ (^ex1 /\ ^ex1 /\ ^ex1) /\ ^ex1)";;
|- (A /\ (B /\ C /\ A) /\ (C /\ A /\ D) /\ D) /\
   (A /\ (B /\ C /\ A) /\ (C /\ A /\ D) /\ D) /\
   ((A /\ (B /\ C /\ A) /\ (C /\ A /\ D) /\ D) /\
    (A /\ (B /\ C /\ A) /\ (C /\ A /\ D) /\ D) /\
    A /\
    (B /\ C /\ A) /\
    (C /\ A /\ D) /\
    D) /\
   A /\
   (B /\ C /\ A) /\
   (C /\ A /\ D) /\
   D =
   A /\ B /\ C /\ D
Run time: 0.4s
Intermediate theorems generated: 155
\end{verbatim}\end{session}

\section{Normalizing all subterms}

There is an important difference in the functionality of
\ml{CONJ\_NORM\_CONV} and the various optimised versions of it. The
difference is that \ml{CONJ\_NORM\_CONV} applies to any term,
normalizing all subterms that are conjunctions. However the functions
\ml{CONJ\_NORM\_CONV}{\small $n$} (where {\small $n = 2,3,4$}) all 
fail on non-conjunctions.

\begin{session}\begin{verbatim}
#CONJ_NORM_CONV "^ex1 ==> ^ex1";;
|- A /\ (B /\ C /\ A) /\ (C /\ A /\ D) /\ D ==>
   A /\ (B /\ C /\ A) /\ (C /\ A /\ D) /\ D =
   A /\ B /\ C /\ D ==> A /\ B /\ C /\ D
Run time: 2.0s
Garbage collection time: 0.6s
Intermediate theorems generated: 1307

#CONJ_NORM_CONV4 "^ex1 => ^ex1";;
need 2 nd branch to conditional
skipping: ex1 " ;; parse failed     

#CONJ_NORM_CONV4 "^ex1 ==> ^ex1";;
evaluation failed     fail
\end{verbatim}\end{session}

What is needed is a function that will apply a conversion to all
conjunctive subterms of a term. Such a function is \ml{TOP\_DEPTH\_CONV}, however

\begin{hol}\begin{verbatim}
   TOP_DEPTH_CONV CONJ_NORM_CONV4
\end{verbatim}\end{hol}

\noindent would loop, because \ml{CONJ\_NORM\_CONV4} never fails on
conjunctions, so \ml{TOP\_DEPTH\_CONV} would keep applying it forever!
This is easily got around using:

\begin{hol}\begin{verbatim}
   CHANGED_CONV : conv -> conv
\end{verbatim}\end{hol}

\noindent \ml{CHANGED\_CONV}~$c$ behaves like $c$, except that if $c$
has no effect, then \ml{CHANGED\_CONV}~$c$ fails.

\begin{session}\begin{verbatim}
#TOP_DEPTH_CONV(CHANGED_CONV CONJ_NORM_CONV4) "^ex1 ==> ^ex1";;
|- A /\ (B /\ C /\ A) /\ (C /\ A /\ D) /\ D ==>
   A /\ (B /\ C /\ A) /\ (C /\ A /\ D) /\ D =
   A /\ B /\ C /\ D ==> A /\ B /\ C /\ D
Run time: 0.5s
Intermediate theorems generated: 292
\end{verbatim}\end{session}

Although this works, the scanning through subterms done by
\ml{TOP\_DEPTH\_CONV} is a bit `brute force'. Further efficiency can
be obtained by writing a special scanning function that
just applies a conversion to maximal conjunctive subterms.
This is provided by the code below. 
The understanding of this is a fairly hard exercise for the reader. The
section on conversions in \DESCRIPTION\ should be helpful.

First, an auxiliary derived rule for combining two equations into
a single equation by conjoining  the left hand sides and  the
right hand sides.

\begin{session}\begin{verbatim}
#let MK_CONJ th1 th2 = MK_COMB(AP_TERM "$/\" th1, th2);;
MK_CONJ = - : (thm -> thm -> thm)
Run time: 0.0s
\end{verbatim}\end{session}

\noindent Next a function that conjoins the left hand sides and right
hand sides of lists of equations.

\begin{session}\begin{verbatim}
#letrec MK_CONJL l =
# if null l     then fail
# if null(tl l) then hd l
#               else MK_CONJ (hd l) (MK_CONJL(tl l));;
MK_CONJL = - : proof
Run time: 0.0s
\end{verbatim}\end{session}

\noindent Next, a function that applies a conversion $c$ to all
conjunctive subterms of a term. This uses the \ML\ function:

\bigskip
\ml{map}~$f$~\ml{[}$x_1\ml{;}\ldots\ml{;}x_n$\ml{]}~~=~~\ml{[}$f\ x_1\ml{;}\ldots\ml{;}f\ x_n$\ml{]}

\bigskip

\noindent and the rules \ml{MK\_COMB}:


\[ \Gamma_1 \turn f = g \qquad\qquad \Gamma_2\turn x = y \over
\Gamma_1 \cup \Gamma_2 \turn f\ x = g\ y\]

\noindent and \ml{MK\_ABS}:


\[ \Gamma \turn \forall x.\ t_1 = t_2 \over
\Gamma \turn (\lambda x.\ t_1) = (\lambda x.\ t_2)\]

\noindent and \ml{GEN}:

$$\Gamma\turn t\over\Gamma\turn\uquant{x} t$$
\begin{itemize}
\item Where $x$ is not free in $\Gamma$.
\end{itemize}


\noindent and \ml{REFL}: 

$$ \over\turn t = t$$

\begin{itemize}
\item\ml{REFL}~$t$~~\ml{=}~~ $\turn t = t$.
\end{itemize}

\noindent The definition of \ml{CONJ\_DEPTH\_CONV} also uses:

\begin{hol}\begin{verbatim}
   is_comb   : term -> bool
   dest_comb : term -> (term # term)

   is_abs    : term -> bool
   dest_abs  : term -> (term # term)
\end{verbatim}\end{hol}

\noindent which are the tests and destructors for combinations and
abstractions, respectively.

\begin{session}\begin{verbatim}
#letrec CONJ_DEPTH_CONV c tm =
# if is_conj tm 
#  then (c THENC (MK_CONJL o map (CONJ_DEPTH_CONV c) o conjuncts)) tm
# if is_comb tm
#  then (let rator,rand = dest_comb tm in
#         MK_COMB (CONJ_DEPTH_CONV c rator, CONJ_DEPTH_CONV c rand))
# if is_abs tm 
#  then (let bv,body = dest_abs tm in
#         let bodyth = CONJ_DEPTH_CONV c body in
#         MK_ABS (GEN bv bodyth))
#   else (REFL tm);;
CONJ_DEPTH_CONV = - : (conv -> conv)
Run time: 0.0s
\end{verbatim}\end{session}

\noindent The next session shows that \ml{CONJ\_DEPTH\_CONV} is an
improvement over \ml{TOP\_DEPTH\_CONV}.

\begin{session}\begin{verbatim}
#CONJ_DEPTH_CONV CONJ_NORM_CONV4 "^ex1 ==> ^ex1";;
|- A /\ (B /\ C /\ A) /\ (C /\ A /\ D) /\ D ==>
   A /\ (B /\ C /\ A) /\ (C /\ A /\ D) /\ D =
   A /\ B /\ C /\ D ==> A /\ B /\ C /\ D
Run time: 0.4s
Intermediate theorems generated: 95
\end{verbatim}\end{session}

\noindent However, the figures show that we are getting to a point of
diminishing returns.

Finally, the tactic for normalizing all conjunctions in a goal is:

\begin{session}\begin{verbatim}
#let CONJ_NORM_TAC = CONV_TAC (CONJ_DEPTH_CONV CONJ_NORM_CONV4);;
CONJ_NORM_TAC = - : tactic
Run time: 0.0s
\end{verbatim}\end{session}

\noindent This is illustrated by:

\begin{session}\begin{verbatim}
#g "^ex1 ==> ^ex1 /\ ^ex1_norm";;
"A /\ (B /\ C /\ A) /\ (C /\ A /\ D) /\ D ==>
 (A /\ (B /\ C /\ A) /\ (C /\ A /\ D) /\ D) /\ A /\ B /\ C /\ D"

() : void
Run time: 0.1s

#e CONJ_NORM_TAC;;
OK..
"A /\ B /\ C /\ D ==> A /\ B /\ C /\ D"

() : void
Run time: 0.3s
Intermediate theorems generated: 110
\end{verbatim}\end{session}

\noindent To show how much faster the optimized version is, here is
the last step repeated with the first version of the tool. The \ML\
function \ml{b} backs up to the last goal.

\begin{session}\begin{verbatim}
#b();;
"A /\ (B /\ C /\ A) /\ (C /\ A /\ D) /\ D ==>
 (A /\ (B /\ C /\ A) /\ (C /\ A /\ D) /\ D) /\ A /\ B /\ C /\ D"

() : void
Run time: 0.1s
\end{verbatim}\end{session}

\noindent Expanding with the slow tactic:

\begin{session}\begin{verbatim}
#e(CONV_TAC CONJ_NORM_CONV);;
OK..
"A /\ B /\ C /\ D ==> A /\ B /\ C /\ D"

() : void
Run time: 3.5s
Garbage collection time: 1.0s
Intermediate theorems generated: 1932
\end{verbatim}\end{session}

\noindent it is 10 times slower and generates almost 20 times as many
primitive inference steps!


Here is the complete \ML\ program for the optimized normalizer.

\begin{hol}\begin{verbatim}
   letrec insert ord x l =
    if null l 
     then [x]
    if ord(x,hd l)
     then x.l
     else hd l.(insert ord x (tl l));;

   letrec sort ord l =
    if null l
     then []
     else insert ord (hd l) (sort ord (tl l));;
\end{verbatim}\end{hol}

\begin{hol}\begin{verbatim}
   letrec PROVE_CONJ ths tm =
    (uncurry CONJ ((PROVE_CONJ ths # PROVE_CONJ ths) (dest_conj tm))) ?
    find (\th. concl th = tm) ths;;

   let CONJ_EQ t1 t2 =
    let imp1 = DISCH t1 (PROVE_CONJ (CONJUNCTS(ASSUME t1)) t2)
    and imp2 = DISCH t2 (PROVE_CONJ (CONJUNCTS(ASSUME t2)) t1) 
    in IMP_ANTISYM_RULE imp1 imp2;;
\end{verbatim}\end{hol}

\begin{hol}\begin{verbatim}
   let CONJ_NORM_CONV t =
    if is_conj t
     then CONJ_EQ t (list_mk_conj(sort $<< (setify(conjuncts t))))
     else fail;;
\end{verbatim}\end{hol}

\begin{hol}\begin{verbatim}
   let MK_CONJ th1 th2 = MK_COMB(AP_TERM "$/\" th1, th2);;

   letrec MK_CONJL l =
    if null l     then fail
    if null(tl l) then hd l
                  else MK_CONJ (hd l) (MK_CONJL(tl l));;
\end{verbatim}\end{hol}

\begin{hol}\begin{verbatim}
   letrec CONJ_DEPTH_CONV c tm =
    if is_conj tm 
     then (c THENC (MK_CONJL o map (CONJ_DEPTH_CONV c) o conjuncts)) tm
    if is_comb tm
     then (let rator,rand = dest_comb tm in
            MK_COMB (CONJ_DEPTH_CONV c rator, CONJ_DEPTH_CONV c rand))
    if is_abs tm 
     then (let bv,body = dest_abs tm in
            let bodyth = CONJ_DEPTH_CONV c body in
            MK_ABS (GEN bv bodyth))
     else (REFL tm);;
\end{verbatim}\end{hol}

\begin{hol}\begin{verbatim}
   let CONJ_NORM_TAC = CONV_TAC (CONJ_DEPTH_CONV CONJ_NORM_CONV);;
\end{verbatim}\end{hol}


Although the optimized implementation is much more efficient, it uses
less general methods. An advantage of the simple implementation based
on rewriting is that essentially the same algorithm can be used to
normalize terms built out of any associative, commutative and
idenpotent operation. The two exercises that follow (whose solutions are not
supplied) suggest that the reader try to extract general principles from the
conjunction normalizer and use these to implement generic tools.

\subsection{Exercise 1}

Implement a normalizer for any associative and commutative operator.

\begin{hol}\begin{verbatim}
   AC_CANON_CONV : thm # thm -> conv
\end{verbatim}\end{hol}

\noindent The two theorem arguments should be the
associative and commutative laws for the operator. For example:

\begin{hol}\begin{verbatim}
   AC_CANON_CONV(ASSOC,SYM1)
\end{verbatim}\end{hol}

\noindent would be a canonicalizer for conjunctions.
Use the `brute force' rewriting method described at the beginning of this chapter.

\subsection{Exercise 2}


Implement an optimized canonicalizer:

\begin{hol}\begin{verbatim}
   FAST_AC_CANON_CONV : thm # thm -> conv \end{verbatim}\end{hol}
\noindent This should use tuned rewriting (\eg\ a generalization of
\ml{CONJ\_DEPTH\_CONV}) and be as fast as possible. Try to think up
tricks to minimise the amount of general matching and to make every
inference count.


                        % conjunction canonicalization tool
   % binomial.tex %
% The Binomial Theorem proven in HOL %

\setlength{\unitlength}{1mm}
% \vskip10mm
% \begin{center}\LARGE\bf
% The Binomial Theorem proven in HOL.
% \end{center}
% \vskip10mm

\def\obj#1{\mbox{\tt#1}}

% ---------------------------------------------------------------------
% Parameters customising the document; set by whoever \input's it
% ---------------------------------------------------------------------
% \self{} is one word denoting this document, such as "article" or "chapter"
% \path{} is the path denoting the case-study directory
%
% Typical examples:
% \def\self{article}
% \def\path{\verb%Training/studies/binomial%}
% ---------------------------------------------------------------------

\section{Pascal's Triangle and the Binomial Theorem}

Pascal's Triangle\footnote{
According to Knuth \cite{knuth73}, the triangle gets its name from
its appearance in Blaise Pascal's {\em Trait\'e du triangle arithm\'etique}
in 1653, although the binomial coefficients were known long before then.
The earliest surviving description is from the tenth century,
in Hal\={a}yudha's commentary on the Hindu classic, the Chandah-S\^{u}tra.
The Binomial Theorem itself was first reported in 1676 by Isaac Newton.
}
is the infinite triangle of numbers which starts like this:
\[\begin{array}{ccccccc}
1  \\
1 & 1 \\
1 & 2 & 1 \\
1 & 3 & 3 & 1 \\
1 & 4 & 6 & 4 & 1 \\
1 & 5 & 10 & 10 & 5 & 1 \\
\cdot & \cdot & \cdot & \cdot & \cdot & \cdot & \cdot \\
\end{array}\]
The construction is as follows. The numbers at the edges 
are $1$s. Each number in the interior is the sum of the number immediately 
above it and the number to its left in the previous row.

The numbers in Pascal's Triangle are called Binomial Coefficients. 
The number in the $n$th row and $k$th column (where $0 \leq k \leq n$, and the 
topmost row and the leftmost column are numbered $0$) is written 
$n \choose k$ and pronounced ``$n$ choose $k$''.  The coefficients are 
pronounced this way because $n \choose k$ turns out to be the number of 
ways to choose $k$ things out of $n$ things, but that is another story
(see for instance Knuth's book \cite{knuth73}).

A simple form of the Binomial Theorem \cite{maclane67} \cite{mostow63} 
is the following equation, where $a$ and $b$ are integers and $n$ is any 
natural number,
\[
(a + b)^n = \sum_{k=0}^n {n \choose k} a^k b^{n-k}
\]
The rest of this \self{} describes how the Binomial Theorem can be proven 
in \HOL{}.  In fact, a more general theorem is proven, where $a$ and $b$ 
are elements of an arbitrary commutative ring, but the form displayed here 
can be derived from the general result.

The motivation for the proof of the Binomial Theorem in \HOL{} is 
tutorial.
% ---to be a medium sized worked example whose subject matter is 
% more widely known than any specific piece of hardware or software.
The proof illustrates specific ways in \HOL{} to represent mathematical 
notations and manage theorems proven about general structures such as 
monoids, groups and rings. The following files accompany this \self{} in 
directory \path{}:
\begin{description}
\item[{\tt mk\_BINOMIAL.ml}]
    The \ML{} program containing the definitions, lemmas and theorems\\
    needed to prove the Binomial Theorem in \HOL{}.
\item
    [{\tt BINOMIAL.th}]
    The theory file which is generated by \verb@mk_BINOMIAL.ml@.
\item
    [{\tt BINOMIAL.ml}]
    An \ML{} program which loads the theory \verb@BINOMIAL@ and declares
    a few derived rules and one tactic used in the proof.
    The Binomial Theorem is bound to the \ML{} identifier \verb@BINOMIAL@.
\end{description}
% The aim of this \self{} is to introduce the small amount of algebra and 
% mathematical notation needed to state and prove the Binomial Theorem, show 
% how this is rendered in \HOL{}, and outline the structure of the proof 
% contained in \verb@mk_BINOMIAL.ml@.
The \self{} is meant to be intelligible 
without examination of any of the accompanying files.
To avoid clutter not every detail is spelt out.
Often definitions and theorems are indicated 
in the form displayed by \HOL{}, rather than as \ML{} source text. Specific 
details of the \ML{} definitions or tactics can be found in the file 
\verb@mk_BINOMIAL.ml@.

The way to work with this case study is first to study this \self{} to see
the structure of the proof, and then to work interactively with \HOL{}.
Start \HOL{} and say:
\begin{session}
\begin{verbatim}
#loadt `BINOMIAL`;;
...
File BINOMIAL loaded
() : void

#BINOMIAL;;
|- !plus times.
    RING(plus,times) ==>
    (!a b n.
      POWER times n(plus a b) =
      SIGMA(plus,0,SUC n)(BINTERM plus times a b n))
\end{verbatim}
\end{session}
(This is the first display of an interaction with \HOL{}.  The difference
between lines input to and lines output from \HOL{} is that only the former
are preceded by the \HOL{} prompt ``\verb@#@''.  A line of ellipsis
``\verb@...@'' stands for output from \HOL{} which has been omitted from the 
display.)

The first input line in the display above loads in all the definitions, 
theorems, derived rules and the tactic defined for the proof.  The second 
input line asks \HOL{} to print theorem \verb@BINOMIAL@.  The constants 
used in \verb@BINOMIAL@ are explained later in the \self{}.  The \HOL{} 
session is now set up either to make use of \verb@BINOMIAL@ by specialising 
it to a specific ring, or to study \verb@mk_BINOMIAL.ml@ by using the subgoal 
package to work through proofs contained in it.

The remainder of the \self{} follows the plan:
\begin{description}
\item[Section~\ref{BinomialCoefficients}]
    shows how to define the number $n \choose k$ in \HOL{} as the term 
    $\obj{CHOOSE}\,n\,k$, where $\obj{CHOOSE}$ is a function defined by 
    primitive recursion.
\item[Section~\ref{MonoidsGroupsRings}]
    shows how to define elementary algebraic concepts like associativity, 
    commutativity, monoid, group and ring in \HOL{}, and how to apply 
    theorems proved in the general case to particular situations.
\item[Section~\ref{PowersReductionsRangesSums}]
    shows how to define the iterated operations of raising a term to a 
    power and reducing (summing) a list of terms. These operations are 
    part of the everyday mathematical language used to state the Binomial 
    Theorem, but they are not built-in to \HOL{} and so need to be defined.  
    This section and the previous two together describe enough definitions 
    in \HOL{} to state the Binomial Theorem.
\item[Section~\ref{BinomialTheorem}]
    shows how the Binomial Theorem is proven in \HOL{}.
    The proof is by induction, and depends on three main lemmas.
\item[Section~\ref{BinomialTheoremForIntegers}]
    outlines how to apply the general theorem to a particular case:
    the ring of integers.
\item[Appendix~\ref{PrincipalLemmas}]
    states the principal lemmas used to prove the theorem.
    Some of these are {\em ad hoc} but others could be reused elsewhere.
\end{description}

% ----------------------------------------------------------------------------

\section{The Binomial Coefficients}
\label{BinomialCoefficients}

The definition of the binomial coefficients as the numbers in Pascal's Triangle
is formalised in three equations:
\begin{eqnarray*}
n \choose 0 &=& 1 \\
0 \choose {k+1} &=& 0 \\
{n+1} \choose {k+1} &=& {{n} \choose {k+1}} + {{n} \choose {k}}
\end{eqnarray*}
(These actually define $n \choose k$ to be $0$ if $k>n$, but this is 
consistent with Pascal's Triangle: think of the spaces to the right of 
the triangle as filled with $0$'s.) The desire is to define a constant 
\verb@CHOOSE@ in \HOL{} such that for all numbers $n$, $k$ in the type 
{\tt num}, $\verb@CHOOSE@\,n\,k$ denotes the number $n \choose k$.  
Unfortunately these three equations cannot be used directly as the definition 
of \verb@CHOOSE@ in \HOL{} because they are not in the form of a base case 
(when $n=0$) together with an inductive case (when $n>0$).

To work towards the definition, consider a term that specifies a base case 
and an inductive case intended to be equivalent to the three equations above:
\begin{session}
\begin{verbatim}
let base_and_inductive: term =
    "(CHOOSE 0 k = ((0 = k) => 1 | 0)) /\
     (CHOOSE (SUC n) k =
         ((0 = k) => 1 | (CHOOSE n (k-1)) + (CHOOSE n k)))";;
\end{verbatim}
\end{session}
The theory \verb@prim_rec@ contains a primitive recursion theorem which 
says that any base case and inductive case identifies a unique function, 
\verb@fn@:
\begin{session}
\begin{verbatim}
#num_Axiom;;
|- !e f. ?! fn.
    (fn 0 = e) /\
    (!n. fn (SUC n) = f (fn n) n)
\end{verbatim}
\end{session}
Given the theorem \verb@prim_rec@, the builtin function 
\verb@prove_rec_fn_exists@ can prove that there is a function that satisfies 
the property specified by the term \verb@base_and_inductive@:
\begin{session}
\begin{verbatim}
#let RAW_CHOOSE_DEF = prove_rec_fn_exists num_Axiom base_and_inductive;;
RAW_CHOOSE_DEF = 
|- ?CHOOSE.
    (!k. CHOOSE 0 k = ((0 = k) => 1 | 0)) /\
    (!n k.
      CHOOSE(SUC n)k = ((0 = k) => 1 | (CHOOSE n(k - 1)) + (CHOOSE n k)))
\end{verbatim}
\end{session}
These equations would not do as the only definition of \verb@CHOOSE@ due 
to the presence of the conditional operator, which makes rewriting 
difficult.  But having obtained the theorem \verb@RAW_CHOOSE_DEF@ it is 
easy to prove there is a function which satisfies the original three 
equations:
\begin{session}
\begin{verbatim}
CHOOSE_DEF_LEMMA = 
|- ?CHOOSE.
    (!n. CHOOSE n 0 = 1) /\
    (!k. CHOOSE 0(SUC k) = 0) /\
    (!n k. CHOOSE(SUC n)(SUC k) = (CHOOSE n(SUC k)) + (CHOOSE n k))
\end{verbatim}
\end{session}
Now the constant \verb@CHOOSE@ can be specified as originally desired:
\begin{session}
\begin{verbatim}
#let CHOOSE_DEF =
#    new_specification `CHOOSE_DEF` [`constant`,`CHOOSE`] CHOOSE_DEF_LEMMA;;
CHOOSE_DEF = 
|- (!n. CHOOSE n 0 = 1) /\
   (!k. CHOOSE 0(SUC k) = 0) /\
   (!n k. CHOOSE(SUC n)(SUC k) = (CHOOSE n(SUC k)) + (CHOOSE n k))
\end{verbatim}
\end{session}
Two elementary properties of the binomial coefficients can now be proven
by induction:
\begin{session}
\begin{verbatim}
CHOOSE_LESS = |- !n k. n < k ==> (CHOOSE n k = 0)
CHOOSE_SAME = |- !n. CHOOSE n n = 1
\end{verbatim}
\end{session}

% ----------------------------------------------------------------------------

\section{Monoids, Groups and Commutative Rings}
\label{MonoidsGroupsRings}
%A simple way to define algebraic structures in \HOL{}.

\subsection{Associativity and Commutativity}

An {\em algebraic structure} is a set equipped with some operators that 
obey some laws.  An elementary example of such a structure is the pair 
$(A,+)$, where $A$ is a set and $+$ is a binary operator on $A$ that obeys 
the laws of associativity,
\[
a + (b + c) = (a + b) + c
\]
and commutativity,
\[
a + b = b + a
\]

How might these laws be represented in \HOL{}?  The simplest scheme is 
to deal with structures of the form $(\sigma, +)$, where $\sigma$ is a 
type of the \HOL{} logic, and $+$ is a \HOL{} term of type 
$\sigma\to\sigma\to\sigma$.  Notice that the set in the structure is 
represented as a \HOL{} type: see the theory \verb@group@ presented in 
Elsa Gunter's case study of Modular Arithmetic for a more sophisticated 
approach where the set in a structure can be a subset of a \HOL{} type.

The two laws can be defined as two constants in \HOL{}:
\begin{session}
\begin{verbatim}
#let ASSOCIATIVE =
#    new_definition (`ASSOCIATIVE`,
#      "!plus: *->*->*. ASSOCIATIVE plus =
#          (!a b c. plus a (plus b c) = plus (plus a b) c)");;
...
#let COMMUTATIVE =
#    new_definition (`COMMUTATIVE`,
#      "!plus: *->*->*. COMMUTATIVE plus =
#          (!a b. plus a b = plus b a)");;
\end{verbatim}
\end{session}
By instantiating the type variable \verb@*@ and specialising the variable 
\verb@plus@ these constants are applicable to any particular structure 
of the form $(\sigma, +: \sigma\to\sigma\to\sigma)$.

A simple example to illustrate this methodology is a permutation theorem 
about an arbitrary associative commutative operator (which is called 
\verb@PERM@ in file \verb@mk_BINOMIAL.ml@).  It is worthwhile going 
through the proof in detail because it illustrates how a difficulty in 
making use of assumptions like \verb@ASSOCIATIVE plus@ and \verb@COMMUTATIVE 
plus@ can be solved.
\begin{session}
\begin{verbatim}
#g "!plus: *->*->*.
#       ASSOCIATIVE plus /\ COMMUTATIVE plus ==>
#           !a b c. plus (plus a b) c = plus b (plus a c)";;
\end{verbatim}
\end{session}
The first step is to strip off the quantifiers, and break up the implication:
\begin{session}
\begin{verbatim}
#e (REPEAT STRIP_TAC);;
OK..
"plus(plus a b)c = plus b(plus a c)"
    [ "ASSOCIATIVE plus" ]
    [ "COMMUTATIVE plus" ]
\end{verbatim}
\end{session}
The next step is to rewrite \verb@plus a b@ to \verb@plus b a@ on the 
left-hand side, from the assumption \verb@COMMUTATIVE plus@. If the theorem 
was about a specific operator, such as \verb@$+: num->num->num@ from the 
theory of numbers in \HOL{}, the rewriting could be done with 
\verb@REWRITE_TAC@ and the specific commutativity theorem, such as 
\verb@ADD_SYM@:
\[
\mbox{\verb@|- !m n. m + n = n + m@}.
\]
But in the general case all that is available is the definition of
\verb@COMMUTATIVE plus@ which cannot be used directly with
\verb@REWRITE_TAC@ to swap the terms \verb@a@ and \verb@b@.
One approach is first to prove the
following equation from the definition \verb@COMMUTATIVE plus@,
\[
\mbox{\verb@COMMUTATIVE plus |- !a b. plus a b = plus b a@},
\]
and then to use the equation with \verb@REWRITE_TAC@ in the same way as one
might use \verb@ADD_SYM@.  This is a valid tactic because the
assumption \verb@COMMUTATIVE plus@ is already present in the assumption 
list.  A little forward proof achieves the first step, the derivation of the
equation:
\begin{session}
\begin{verbatim}
#top_print print_all_thm;;
- : (thm -> void)

#let forwards = fst(EQ_IMP_RULE (SPEC "plus: *->*->*" COMMUTATIVE));;
forwards = |- COMMUTATIVE plus ==> (!a b. plus a b = plus b a)

#let eqn = UNDISCH forwards;;
eqn = COMMUTATIVE plus |- !a b. plus a b = plus b a
\end{verbatim}
\end{session}
This forward proof is a special case of a derived
rule of the logic,
\[
\Gamma\turn \uquant{x} t_1 = t_2
\over
\Gamma, t_1[t'/x] \turn t_2[t'/x]
\]
which can be coded as the \ML{} function \verb@SPEC_EQ@:
\begin{session}
\begin{verbatim}
#let SPEC_EQ v th = UNDISCH(fst(EQ_IMP_RULE (SPEC v th)));;
SPEC_EQ = - : (term -> thm -> thm)
\end{verbatim}
\end{session}
Now the equation \verb@eqn@ can be used to prove swap terms
\verb@a@ and \verb@b@:
\begin{session}
\begin{verbatim}
#e(SUBST1_TAC (SPECL ["a:*";"b:*"] eqn));;
OK..
"plus(plus b a)c = plus b(plus a c)"
    [ "ASSOCIATIVE plus" ]
    [ "COMMUTATIVE plus" ]
\end{verbatim}
\end{session}
Now that the terms appear in both sides in the same order but with different 
grouping, associativity can be used to make the grouping on both sides 
the same.  Again the definition of \verb@ASSOCIATIVE@ cannot be used
directly, but the derived rule \verb@SPEC_EQ@ obtains from the definition
an equation analogous to \verb@eqn@ which can be used:
\begin{session}
\begin{verbatim}
#e(REWRITE_TAC [SPEC_EQ "plus: *->*->*" ASSOCIATIVE]);;
OK..
goal proved
...
|- !plus.
    ASSOCIATIVE plus /\ COMMUTATIVE plus ==>
    (!a b c. plus(plus a b)c = plus b(plus a c))
\end{verbatim}
\end{session}

The point of this example was to illustrate how to prove theorems about 
an arbitrary operator whose behaviour is specified by general predicates 
like \verb@ASSOCIATIVE@ and \verb@COMMUTATIVE@. The main difficulty is 
in deriving equations for use with rewriting or substitution.  The solution 
used in the example and extensively in the proof of the Binomial Theorem 
is to derive equations using the derived rule \verb@SPEC_EQ@. Another derived 
rule, \verb@SPEC_IMP@, is used extensively and is a variant of \verb@SPEC_EQ@:
\[
\Gamma\turn \uquant{x} t_1 \Rightarrow t_2
\over
\Gamma, t_1[t'/x] \turn t_2[t'/x]
\]
\begin{session}
\begin{verbatim}
#let SPEC_IMP v th = UNDISCH(SPEC v th);;
SPEC_IMP = - : (term -> thm -> thm)
\end{verbatim}
\end{session}
Two following two rules are versions of \verb@SPEC_EQ@ and \verb@SPEC_IMP@ 
generalised to take lists of variables rather than just one:
\begin{session}
\begin{verbatim}
SPECL_EQ = - : (term list -> thm -> thm)
SPECL_IMP = - : (term list -> thm -> thm)
\end{verbatim}
\end{session}
The final new derived rule used in the proof of the Binomial Theorem
is \verb@STRIP_SPEC_IMP@, a variant of \verb@SPEC_IMP@, which
splits up a conjunction in the antecedent into separate assumptions:
\[
\Gamma\turn \uquant{x} (t_1 \wedge \cdots \wedge t_n) \Rightarrow t_{n+1}
\over
\Gamma, t_1[t'/x], \ldots, t_n[t'/x] \turn t_{n+1}[t'/x]
\]
\begin{session}
\begin{verbatim}
#let STRIP_SPEC_IMP v th =
#    UNDISCH_ALL(SPEC v (CONV_RULE (ONCE_DEPTH_CONV ANTE_CONJ_CONV) th));;
STRIP_SPEC_IMP = - : (term -> thm -> thm)
\end{verbatim}
\end{session}

There are other methods for proving theorems which are conditional on 
predicates such as \verb@ASSOCIATIVE@ and \verb@COMMUTATIVE@.  One is to 
rewrite with the definition before assuming the predicate (see, for instance, 
the proof of \verb@RIGHT_CANCELLATION@ in \verb@mk_BINOMIAL.ml@). This 
method is fine for small proofs, but in larger proofs it presents the problem 
of how to specify a particular assumption on the assumption list.  Another 
method is to put the predicates on the assumption list, and use resolution 
to extract the body of the definition (see the proof of 
\verb@UNIQUE_LEFT_INV@).  When the assumption list is not short this is 
rather a blunt instrument, in the sense that resolution can find many more 
theorems than the one desired, and is also rather computationally costly 
when compared to carefully constructing a theorem with \verb@SPEC_EQ@ and 
its variants.

\subsection{Monoids}

A {\em monoid} is a structure $(M,+)$ where $M$ is a set, $+$ is an 
associative binary operator on $M$, such that there is an {\em identity 
element}, $u \in M$, which is a left and right identity of $+$, that is, 
it satisfies $u+a = a+u = a$ for all $a \in M$. When the operator is written 
as $+$, the structure $(M,+)$ is called an {\em additive monoid} and the 
identity element is written as $0$; otherwise if the operator is written 
as $\times$, the structure $(M,\times)$ is called a {\em multiplicative 
monoid} and the identity element is written as $1$.
\begin{session}
\begin{verbatim}
#let RAW_MONOID =
#    new_definition (`RAW_MONOID`,
#      "!plus: *->*->*. MONOID plus =
#          ASSOCIATIVE plus /\
#          (?u. (!a. plus a u = a) /\ (!a. plus u a = a))");;
\end{verbatim}
\end{session}
This definition is inconvenient to manipulate as it stands, because of 
the presence of the existential quantifier.  However it is possible via 
Hilbert's $\epsilon$-operator to specify a function \verb@Id@ such that 
\verb@Id plus@ stands for an identity element of \verb@plus@, if such 
exists.  In fact it is easy to prove that the identity element in a monoid 
is unique.  The first step is to prove that there exists a function \verb@Id@ 
with the property just described:
\begin{session}
\begin{verbatim}
ID_LEMMA = 
|- ?Id.
    !plus.
     (?u. (!a. plus u a = a) /\ (!a. plus a u = a)) ==>
     (!a. plus(Id plus)a = a) /\ (!a. plus a(Id plus) = a)
\end{verbatim}
\end{session}
The proof of \verb@ID_LEMMA@ uses Hilbert's $\epsilon$-operator to
construct a witness for the outer existential quantifier,
\begin{verbatim}
    \plus.@u:*.(!a:*. (plus u a = a)) /\ (!a. (plus a u = a))
\end{verbatim}
Given \verb@ID_LEMMA@, the constant \verb@Id@ is specified as follows:
\begin{session}
\begin{verbatim}
#let ID = new_specification `Id` [`constant`, `Id`] ID_LEMMA;;
ID = 
|- !plus.
    (?u. (!a. plus u a = a) /\ (!a. plus a u = a)) ==>
    (!a. plus(Id plus)a = a) /\ (!a. plus a(Id plus) = a)
\end{verbatim}
\end{session}
The condition that \verb@Id plus@ is a left and right identity is expressed 
by the predicates \verb@LEFT_ID@ and \verb@RIGHT_ID@, respectively:
\begin{session}
\begin{verbatim}
LEFT_ID = |- !plus. LEFT_ID plus = (!a. plus(Id plus)a = a)
RIGHT_ID = |- !plus. RIGHT_ID plus = (!a. plus a(Id plus) = a)
\end{verbatim}
\end{session}
Given these abbreviations it is easy to prove in \HOL{} that
if a left and right identity element exists, then it is unique:
\begin{session}
\begin{verbatim}
UNIQUE_LEFT_ID =
|- !plus.
    LEFT_ID plus /\ RIGHT_ID plus ==>
    (!l. (!a. plus l a = a) ==> (Id plus = l))

UNIQUE_RIGHT_ID =
|- !plus.
    LEFT_ID plus /\ RIGHT_ID plus ==>
    (!r. (!a. plus a r = a) ==> (Id plus = r))
\end{verbatim}
\end{session}
The use of these last two theorems is to help identify the element \verb@Id 
plus@ for some specific \verb@plus@, such as \verb@$+@ from the theory 
of \verb@arithmetic@.  The abbreviations \verb@LEFT_ID@ and \verb@RIGHT_ID@ 
admit a new characterisation of the predicate \verb@MONOID@, which is easier 
to manipulate in \HOL{} than \verb@RAW_MONOID@:
\begin{session}
\begin{verbatim}
MONOID =
|- !plus. MONOID plus = LEFT_ID plus /\ RIGHT_ID plus /\ ASSOCIATIVE plus
\end{verbatim}
\end{session}

\subsection{Groups}

A {\em group} is a monoid in which every element is invertible.
\begin{session}
\begin{verbatim}
RAW_GROUP = 
|- !plus.
    Group plus =
    MONOID plus /\
    (!a. ?b. (plus b a = Id plus) /\ (plus a b = Id plus))
\end{verbatim}
\end{session}
An alternative characterisation of \verb@Group@\footnote{
The mixed case identifier \verb@Group@ is used rather than \verb@GROUP@, 
because the latter is already defined in Elsa Gunter's theory \verb@group@, 
a parent of her theory \verb@integer@, which needs to coexist with the 
present theory so that integers can be used as an example ring 
(Section~\ref{BinomialTheoremForIntegers}).
} which makes the 
existential quantification implicit can be had by specifying a function 
\verb@Inv@ so that, if \verb@Group plus@ holds, then the element \verb@Inv 
plus a@ is an inverse of \verb@a@.  The function \verb@Inv@ is specified 
in a way analogous to the specification of \verb@Id@.  First a lemma is 
proven which says there does exist a function \verb@Inv@ with the property 
desired:
\begin{session}
\begin{verbatim}
INV_LEMMA = 
|- ?Inv.
    !plus.
     (!a. ?b. (plus b a = Id plus) /\ (plus a b = Id plus)) ==>
     (!a.
       (plus(Inv plus a)a = Id plus) /\ (plus a(Inv plus a) = Id plus))
\end{verbatim}
\end{session}
Given this lemma, which is proven by an easy tactic almost the same
as the one for \verb@ID_LEMMA@, the constant \verb@Inv@ can be specified:
\begin{session}
\begin{verbatim}
#let INV =
#    new_specification `Inv` [`constant`, `Inv`] INV_LEMMA;;
\end{verbatim}
\end{session}
The predicates which say that \verb@Inv@ produces left and right inverses
are defined as new constants:
\begin{session}
\begin{verbatim}
LEFT_INV =
|- !plus. LEFT_INV plus = (!a. plus(Inv plus a)a = Id plus)
RIGHT_INV =
|- !plus. RIGHT_INV plus = (!a. plus a(Inv plus a) = Id plus)
\end{verbatim}
\end{session}
Finally, the alternative characterisation of \verb@Group@ can be proven:
\begin{session}
\begin{verbatim}
GROUP = |- Group plus = MONOID plus /\ LEFT_INV plus /\ RIGHT_INV plus
\end{verbatim}
\end{session}
The theory of groups has been developed only as far as needed to prove 
the Binomial Theorem; Appendix~\ref{PrincipalLemmas} shows the one lemma, 
\verb@RIGHT_CANCELLATION@, specifically about groups.

\subsection{Rings}

A {\em ring} is a structure $(R,+,\times)$ such that structure $(R,+)$ 
is a group, structure $(R,\times)$ is a monoid, {\em addition}, $+$, is 
commutative, and {\em multiplication}, $\times$, distributes (on both sides) 
over addition.  A {\em commutative ring} is a ring in which multiplication 
is commutative.  The word ring is used in the remainder of this \self{} 
and in the \HOL{} code to mean a commutative ring:
\begin{session}
\begin{verbatim}
RING = 
|- !plus times.
    RING(plus,times) =
    Group plus /\
    COMMUTATIVE plus /\
    MONOID times /\
    COMMUTATIVE times /\
    LEFT_DISTRIB(plus,times) /\
    RIGHT_DISTRIB(plus,times)
\end{verbatim}
\end{session}
\verb@LEFT_DISTRIB@ and \verb@RIGHT_DISTRIB@ are new constants
defined by the theorems:
\begin{session}
\begin{verbatim}
LEFT_DISTRIB = 
|- !plus times.
    LEFT_DISTRIB(plus,times) =
    (!a b c. times a(plus b c) = plus(times a b)(times a c))

RIGHT_DISTRIB = 
|- !plus times.
    RIGHT_DISTRIB(plus,times) =
    (!a b c. times(plus a b)c = plus(times a c)(times b c))
\end{verbatim}
\end{session}
Notice that the definition of \verb@RING@ abbreviates a conjunction of 
predicates, some of which are atomic, in the sense that they are some basic 
property of \verb@plus@ or \verb@times@ or both, and some are compound, 
in the sense that they abbreviate a further conjunction of properties.  
The definition is a tree of predicates with atomic ones like 
\verb@LEFT_DISTRIB@ and \verb@ASSOCIATIVE@, at the leaves, and compound 
ones like \verb@Group@ and \verb@MONOID@, at the branches.  Many theorems 
conditional on \verb@RING(plus,times)@ need to make use of both atomic 
and compound predicates contained in the tree.  To make access to all these 
predicates easy, the following lemma is proven, which makes explicit all 
the predicates in the tree: 
\begin{session}
\begin{verbatim}
RING_LEMMA = 
|- RING(plus,times) ==>
   RING(plus,times) /\
   Group plus /\
   MONOID plus /\
   LEFT_ID plus /\
   RIGHT_ID plus /\
   ASSOCIATIVE plus /\
   LEFT_INV plus /\
   RIGHT_INV plus /\
   COMMUTATIVE plus /\
   MONOID times /\
   LEFT_ID times /\
   RIGHT_ID times /\
   ASSOCIATIVE times /\
   COMMUTATIVE times /\
   LEFT_DISTRIB(plus,times) /\
   RIGHT_DISTRIB(plus,times)
\end{verbatim}
\end{session}
Goals of the form,
\[
\mbox{\verb@!plus times. RING(plus,times) ==> ...@}
\]
are consistently tackled with the following tactic, which uses \verb@RING_LEMMA@
to add all the atomic and compound predicates in the tree to the assumption 
list:
\begin{session}
\begin{verbatim}
#let RING_TAC =
#    GEN_TAC THEN GEN_TAC THEN
#    DISCH_THEN (\th. STRIP_ASSUME_TAC (MP RING_LEMMA th));;
\end{verbatim}
\end{session}
The one lemma about rings needed here, \verb@RING_0@, is stated in
Appendix~\ref{PrincipalLemmas}.

% ----------------------------------------------------------------------------

\section{Powers, Reductions, Ranges and Sums}
\label{PowersReductionsRangesSums}
%Translation of mathematical definitions such as sums of series into HOL

The aim of this section is to explain how the mathematical notation 
$\sum_{k=0}^n {n \choose k} a^k b^{n-k}$ can be rendered in \HOL{}.
If this is written out more explicitly than necessary for human
consumption it looks like this:
\[
\sum_{k=0}^n {n \choose k} \cdot (a^k \times b^{n-k})
\]
where $\times$ is multiplication in the ring, and where exponentiation 
and $\cdot$ stand for the scalar powers of multiplication and addition, 
respectively.  Since binomial coefficients were treated in 
Section~\ref{BinomialCoefficients}, what remain to be defined in this section 
are scalar powers and the $\Sigma$-notation.  The latter is tackled in 
three steps: reduction of a list of monoid elements; generation of lists 
comprising subranges of natural numbers, $[n,n+1,\ldots,n+k-1]$; and finally 
the reduction of a list of ring elements indexed by a subrange of the natural 
numbers.

\subsection{Scalar powers in a monoid}

If $(M,+)$ and $(M,\times)$ are additive and multiplicative
monoids respectively, the scalar powers of addition and multiplication are
defined informally by the following equations, where $n>0$:
\begin{eqnarray*}
n \cdot a &=& \overbrace{a + \cdots + a}^{n} \\
a^n &=& \overbrace{a \times \cdots \times a}^{n}
\end{eqnarray*}
When $n=0$, $n \cdot a = 0$ and $a^n = 1$.

Scalar power is defined in \HOL{} via the constant \verb@POWER@ which
is defined by primitive recursion:
\begin{session}
\begin{verbatim}
POWER = 
|- (!plus a. POWER plus 0 a = Id plus) /\
   (!plus n a. POWER plus(SUC n)a = plus a(POWER plus n a))
\end{verbatim}
\end{session}
Three lemmas about \verb@POWER@ are stated in Appendix~\ref{PrincipalLemmas}.

\subsection{Reduction in a monoid}

This section makes use of the standard \HOL{} theory of lists, in particular 
the constants \verb@NIL@, \verb@CONS@, \verb@APPEND@ and \verb@MAP@, 
definition by primitive recursion and proof by induction. The theory of 
lists is described in \DESCRIPTION.

If $(M,+)$ is a monoid, then the reduction of a list $[a_1,\ldots,a_n]$,
where each $a_i$ is an element of $M$, is the sum,
\[
a_1 + \cdots + a_n,
\]
or the monoid's identity element if $n=0$.

Reduction is represented in \HOL{} by the constant \verb@REDUCE@,
with type,
\begin{verbatim}
REDUCE : (*->*->*) -> (*)list -> *
\end{verbatim}
and which is defined by primitive recursion on lists on its second argument:
\begin{session}
\begin{verbatim}
#let REDUCE =
#  new_list_rec_definition (`REDUCE`,
#    "(!plus. (REDUCE plus NIL) = (Id plus):*) /\
#     (!plus (a:*) as. REDUCE plus (CONS a as) = plus a (REDUCE plus as))");;
REDUCE = 
|- (!plus. REDUCE plus[] = Id plus) /\
   (!plus a as. REDUCE plus(CONS a as) = plus a(REDUCE plus as))
\end{verbatim}
\end{session}
Three lemmas about \verb@REDUCE@ are stated in Appendix~\ref{PrincipalLemmas}.

\subsection{Subranges of the natural numbers}

The constant \verb@RANGE@, which is a curried function of two numbers,
is defined by primitive recursion so that,
\[
\verb@RANGE@\,m\,n = [m, m+1, \ldots, m+n-1]
\]
This definition means that the subrange is the first $n$ numbers starting 
at $m$.  In particular, when $n=0$ the list is empty.
Appendix~\ref{PrincipalLemmas} states two simple properties of \verb@RANGE@

\verb@RANGE@ generates a subrange given its least element and length;
an alternative method is to generate the subrange given its two endpoints,
via a constant \verb@INTERVAL@ which satisfies the equation:
\[
\verb@INTERVAL@\,m\,n =
    \left\{\begin{array}{ll}
    [m, m+1, \ldots, n] & \mbox{if $m \leq n$} \\
    {}[] & \mbox{otherwise}
    \end{array}\right.
\]

For comparison, the two methods are definable in \HOL{} as follows:
\begin{session}
\begin{verbatim}
#let RANGE = 
#    new_recursive_definition false num_Axiom `RANGE`
#      "(RANGE m 0 = NIL) /\
#       (RANGE m (SUC n) = CONS m (RANGE (SUC m) n))";;
#
#let INTERVAL =
#    new_recursive_definition false num_Axiom `INTERVAL`
#      "(INTERVAL m 0 =
#          (m > 0) => NIL | [0]) /\
#      (INTERVAL m (SUC n) =
#          (m > (SUC n)) => NIL | APPEND (INTERVAL m n) ([SUC n]))";;
\end{verbatim}
\end{session}

\subsection{$\Sigma$-notation}

The standard $\Sigma$-notation is defined informally by the equation,
\[
\sum_{i=m}^n a_i =
    \left\{\begin{array}{ll}
    a_m + a_{m+1} + \cdots + a_n & \mbox{if $m \leq n$} \\
    0 & \mbox{otherwise}
    \end{array}\right.
\]
Therefore the standard $\Sigma$-notation is naturally definable via
a subrange generated by \verb@INTERVAL@.

An alternative notation, naturally definable via \verb@RANGE@, can be
defined informally as follows:
\[
\sum_{i=m}^{(n)} a_i =
    \left\{\begin{array}{ll}
    a_m + a_{m+1} + \cdots + a_{m+n-1} & \mbox{if $n > 0$} \\
    0 & \mbox{otherwise}
    \end{array}\right.
\]
The brackets around the $n$ above the $\Sigma$ distinguish this
notation from the standard one.

This second notation is used in the \HOL{} proof of the Binomial Theorem.  
The difference between the two is just that the standard notation is 
definable via \verb@INTERVAL@, and the second via \verb@RANGE@. The second 
notation was chosen entirely because the constant \verb@RANGE@ is easier 
to manipulate in \HOL{} than \verb@INTERVAL@, which is because the latter 
has a more complex definition. The disadvantage of this choice is that 
the proof of the theorem uses a non-standard notation, but that's all the 
difference is: notation.  If desired, the Binomial Theorem stated in the 
standard notation could be trivially deduced from its non-standard statement.

The non-standard $\Sigma$-notation is defined in \HOL{} as the following
abbreviation:
\begin{session}
\begin{verbatim}
#let SIGMA =
#    new_definition (`SIGMA`,
#      "SIGMA (plus,m,n) f = REDUCE plus (MAP f (RANGE m n)): *");;
\end{verbatim}
\end{session}
Note that an indexed term $a_i$ is represented in \HOL{} as a function
\verb@f@ of type \verb@num -> *@.

Appendix~\ref{PrincipalLemmas} states several properties of \verb@SIGMA@; 
their proofs make use of properties of \verb@MAP@ from the theory 
\verb@list@, and also the lemmas about \verb@REDUCE@ and \verb@RANGE@.

The functions \verb@REDUCE@ and \verb@RANGE@ are needed here only to define 
\verb@SIGMA@, and in fact could be omitted if \verb@SIGMA@ were to be defined 
directly  using primitive recursion.  It is worth going to the trouble 
of defining \verb@REDUCE@ and \verb@RANGE@, however, because they are likely 
to be useful in other situations. 

% ----------------------------------------------------------------------------

\section{The Binomial Theorem for a Commutative Ring}
\label{BinomialTheorem}
%Division of a medium sized proof into manageable lemmas.

This section outlines the proof.  For full details see the tactics in
\verb@mk_BINOMIAL.ml@.

The statement and proof of the Binomial Theorem use a new \HOL{} constant, 
\verb@BINTERM@, to abbreviate indexed terms of the form $\lambda k.\,{n 
\choose k} \cdot a^{n-k} \times b^k$.  \verb@BINTERM@ is defined as a
curried function as follows:
\begin{session}
\begin{verbatim}
#let BINTERM_DEF =
#    new_definition (`BINTERM_DEF`,
#      "BINTERM (plus: *->*->*) (times: *->*->*) a b n k =
#          POWER plus (CHOOSE n k)
#              (times (POWER times (n-k) a) (POWER times k b))");;
\end{verbatim}
\end{session}
This abbreviation lets the Binomial Theorem be stated and proven in \HOL{} as
follows:
\begin{session}
\begin{verbatim}
BINOMIAL = 
|- !plus times.
    RING(plus,times) ==>
    (!a b n.
      POWER times n(plus a b) =
      SIGMA(plus,0,SUC n)(BINTERM plus times a b n))
\end{verbatim}
\end{session}
In outline, the proof proceeds as follows.  The following equation is
proven by induction on $n$:
\begin{eqnarray*}
(a+b)^n &=& \sum_{k=0}^{(n+1)} {n \choose k} a^{n-k} b^k
\end{eqnarray*}
When $n=0$ both sides reduce to the multiplicative identity element, $1$, 
of the ring. For the inductive step, the equation is assumed for $n$, and 
the following goal must be proven:
\begin{eqnarray*}
(a+b)^{n+1} &=& \sum_{k=0}^{(n+2)} {{n+1} \choose k} a^{n+1-k} b^k
\end{eqnarray*}
Three lemmas are used in the inductive step:
\[\begin{array}{rcll}
\displaystyle
\sum_{k=1}^{(n)} {{n+1} \choose k} a^{n+1-k} b^k &=&
\displaystyle
    a \times \sum_{k=1}^{(n)} {n \choose k} a^{n+1-k} b^k +
    b \times \sum_{k=0}^{(n)} {n \choose k} a^{n+1-k} b^k
    & \mbox{({\tt LEMMA\_1})} \\
\displaystyle
a \times \sum_{k=0}^{(n+1)} {n \choose k} a^{n-k} b^k &=&
\displaystyle
    a^{n+1} +
    a \times \sum_{k=1}^{(n)} {{n+1} \choose k} a^{n+1-k} b^k
    & \mbox{({\tt LEMMA\_2})} \\
\displaystyle
b \times \sum_{k=0}^{(n+1)} {n \choose k} a^{n-k} b^k &=&
\displaystyle
    b \times \sum_{k=0}^{(n)} {{n+1} \choose k} a^{n+1-k} b^k
    + b^{n+1}
    & \mbox{({\tt LEMMA\_3})}
\end{array}\]
\verb@LEMMA_1@ is the key to the inductive step. The three stages of the 
inductive step are indicated (a), (b) and (c) in \ML{} comments.  Step 
(a) expands the right-hand side of the goal with \verb@LEMMA_1@:
\begin{eqnarray*}
\mbox{rhs} &=&
    a^{n+1} +
    a \times \sum_{k=1}^{(n)} {{n+1} \choose k} a^{n+1-k} b^k +
    b \times \sum_{k=0}^{(n)} {{n+1} \choose k} a^{n+1-k} b^k +
    b^{n+1}
\end{eqnarray*}
Step (b) expands the left-hand side with the induction hypothesis:
\begin{eqnarray*}
\mbox{lhs} &=&
    (a+b)\sum_{k=0}^{(n+1)} {n \choose k} a^{n-k} b^k
\end{eqnarray*}
Finally, in step (c) the expansions of the two sides are shown to be the 
same, using distributivity of $\times$ over $+$, associativity of $+$,
and \verb@LEMMA_2@ and \verb@LEMMA_3@:
\begin{eqnarray*}
\mbox{lhs} &=&
    a \times \sum_{k=0}^{(n+1)} {n \choose k} a^{n-k} b^k
    +
    b \times \sum_{k=0}^{(n+1)} {n \choose k} a^{n-k} b^k \\
  &=&
    a^{n+1} +
    a \times \sum_{k=1}^{(n)} {{n+1} \choose k} a^{n+1-k} b^k
    +
    b \times \sum_{k=0}^{(n)} {{n+1} \choose k} a^{n+1-k} b^k
    + b^{n+1} \\
  &=&
      \mbox{rhs}
\end{eqnarray*}

% ----------------------------------------------------------------------------

\section{The Binomial Theorem for Integers}
\label{BinomialTheoremForIntegers}

The previous sections have dealt with the material in theory \verb@BINOMIAL@,
concerning how to prove the Binomial Theorem for an arbitrary ring.  A 
natural next step is to produce some specific commutative ring, and prove
a Binomial Theorem for it, as an instance of the general case.

To see how to do so for Elsa Gunter's theory of integers, take a look 
at the file {\tt mk\_BINOMIAL\_integer.ml}, contained in directory \path{}.
When loaded into \HOL{}, this \ML{} file proves that the integers are a 
ring---by quoting laws contained in the theory \verb@integer@---and then 
derives a Binomial Theorem for the integers---by Modus Ponens from the 
general theorem \verb@BINOMIAL@.
                    % Andy Gordon's Binomial Thm in HOL
   
\begin{thebibliography}{1}

\bibitem{PPML}
P.~Borras.
\newblock {\em {PPML} Reference Manual \& Compiler Implementation}.
\newblock Sema Group, INRIA Sophia-Antipolis, 2004 Route des Lucioles, 06565
  Valbonne CEDEX, France, June 1989.
\newblock Appears in {\it The CENTAUR Reference Manual}. Contact: CENTAUR
  Distribution (centaur@mirsa.inria.fr).

\bibitem{description} % OK
{\small DSTO} and {\small SRI} International, 
{\it The HOL System: DESCRIPTION}, (1991).

\end{thebibliography}


			 % references

\end{document}


