\chapter{Pre-proved Theorems}
\label{theorems}

The sections that follow list all theorems in the \ml{finite\_sets} library,
including definitions. The theorems are grouped into sections according to
subject matter. Some theorems could be classified under more than one subject,
but each theorem is listed in only one section. The reader may therefore have
to consult more than one section when searching for any particular theorem.

When the \ml{finite\_sets} library is loaded, all the theorems listed in this
chapter (including definitions) are set up to autoload when their names are
mentioned in \ML.


\section{Numeral theory}
\THEOREM BASEN numeral
|- (!radix. BASEN radix[] = 0) /\
   (!radix digit digits.
     BASEN radix(CONS digit digits) =
     (digit * (radix EXP (LENGTH digits))) + (BASEN radix digits))
\ENDTHEOREM
\THEOREM BASEN\_11 numeral
|- !r l1 l2.
    1 < r ==>
    IS_BASEN r l1 ==>
    IS_BASEN r l2 ==>
    (LENGTH l1 = LENGTH l2) ==>
    (BASEN r l1 = BASEN r l2) ==>
    (l1 = l2)
\ENDTHEOREM
\THEOREM BASEN\_APPEND numeral
|- !r l m.
    BASEN r(APPEND l m) =
    ((r EXP (LENGTH m)) * (BASEN r l)) + (BASEN r m)
\ENDTHEOREM
\THEOREM BASEN\_CONS\_0 numeral
|- !r l. BASEN r(CONS 0 l) = BASEN r l
\ENDTHEOREM
\THEOREM BASEN\_DIGITS numeral
|- !n m r.
    1 < r ==>
    (LENGTH(BASEN_DIGITS r n m) = n) /\
    (BASEN r(BASEN_DIGITS r n m) = m MOD (r EXP n))
\ENDTHEOREM
\THEOREM BASEN\_DIGIT\_EQ\_DIGIT numeral
|- !r e. BASEN r[e] = e
\ENDTHEOREM
\THEOREM BASEN\_EMPTY\_EQ\_0 numeral
|- !r l. 1 < r ==> IS_NORMALIZED l ==> ((BASEN r l = 0) = (l = []))
\ENDTHEOREM
\THEOREM BASEN\_EXP\_LESS numeral
|- !r l.
    IS_BASEN r l ==>
    IS_NORMALIZED l ==>
    ~NULL l ==>
    1 < r ==>
    ((r EXP ((LENGTH l) - 1)) - 1) < (BASEN r l)
\ENDTHEOREM
\THEOREM BASEN\_EXP\_LESS\_OR\_EQ numeral
|- !r l.
    1 < r ==>
    ~NULL l ==>
    IS_NORMALIZED l ==>
    IS_BASEN r l ==>
    (r EXP ((LENGTH l) - 1)) <= (BASEN r l)
\ENDTHEOREM
\THEOREM BASEN\_EXP\_N numeral
|- !r n. BASEN r(CONS 1(REPLICATE n 0)) = r EXP n
\ENDTHEOREM
\THEOREM BASEN\_LEADING numeral
|- !r l.
    1 < r ==>
    IS_BASEN r l ==>
    ~NULL l ==>
    (BASEN r(BUTLAST l) = (BASEN r l) DIV r)
\ENDTHEOREM
\THEOREM BASEN\_LESS\_EXP\_LENGTH numeral
|- !r l. 1 < r ==> IS_BASEN r l ==> (BASEN r l) < (r EXP (LENGTH l))
\ENDTHEOREM
\THEOREM BASEN\_LESS\_OR\_EQ\_EXP\_LENGTH numeral
|- !r l.
    1 < r ==> IS_BASEN r l ==> (BASEN r l) <= ((r EXP (LENGTH l)) - 1)
\ENDTHEOREM
\THEOREM BASEN\_ONTO numeral
|- !r l. ?n. BASEN r l = n
\ENDTHEOREM
\THEOREM BASEN\_SNOC numeral
|- !r e l. BASEN r(SNOC e l) = ((BASEN r l) * r) + e
\ENDTHEOREM
\THEOREM BASEN\_TRAILING numeral
|- !r l.
    1 < r ==>
    IS_BASEN r l ==>
    ~NULL l ==>
    (BASEN r(TL l) = (BASEN r l) MOD (r EXP ((LENGTH l) - 1)))
\ENDTHEOREM
\THEOREM BASEN\_ZEROS numeral
|- !r n. BASEN r(REPLICATE n 0) = 0
\ENDTHEOREM
\THEOREM BINARY numeral
|- BINARY = BASEN 2
\ENDTHEOREM
\THEOREM BINARY\_11 numeral
|- !l1 l2.
    IS_BINARY l1 ==>
    IS_BINARY l2 ==>
    (LENGTH l1 = LENGTH l2) ==>
    (BINARY l1 = BINARY l2) ==>
    (l1 = l2)
\ENDTHEOREM
\THEOREM BINARY\_CONS\_0 numeral
|- !l. BINARY(CONS 0 l) = BINARY l
\ENDTHEOREM
\THEOREM BINARY\_DIGIT\_EQ\_DIGIT numeral
|- !e. BINARY[e] = e
\ENDTHEOREM
\THEOREM BINARY\_EMPTY\_EQ\_0 numeral
|- !l. IS_NORMALIZED l ==> ((BINARY l = 0) = (l = []))
\ENDTHEOREM
\THEOREM BINARY\_EXP\_LESS numeral
|- !l.
    IS_BINARY l ==>
    IS_NORMALIZED l ==>
    ~NULL l ==>
    ((2 EXP ((LENGTH l) - 1)) - 1) < (BINARY l)
\ENDTHEOREM
\THEOREM BINARY\_EXP\_LESS\_OR\_EQ numeral
|- !l.
    ~NULL l ==>
    IS_NORMALIZED l ==>
    IS_BINARY l ==>
    (2 EXP ((LENGTH l) - 1)) <= (BINARY l)
\ENDTHEOREM
\THEOREM BINARY\_EXP\_N numeral
|- !n. BINARY(CONS 1(REPLICATE n 0)) = 2 EXP n
\ENDTHEOREM
\THEOREM BINARY\_LESS\_EXP\_LENGTH numeral
|- !l. IS_BINARY l ==> (BINARY l) < (2 EXP (LENGTH l))
\ENDTHEOREM
\THEOREM BINARY\_LESS\_OR\_EQ\_EXP\_LENGTH numeral
|- !l. IS_BINARY l ==> (BINARY l) <= ((2 EXP (LENGTH l)) - 1)
\ENDTHEOREM
\THEOREM BINARY\_ONTO numeral
|- !l. ?n. BINARY l = n
\ENDTHEOREM
\THEOREM BINARY\_ZEROS numeral
|- !n. BINARY(REPLICATE n 0) = 0
\ENDTHEOREM
\THEOREM DECIMAL numeral
|- DECIMAL = BASEN 10
\ENDTHEOREM
\THEOREM DECIMAL\_11 numeral
|- !l1 l2.
    IS_DECIMAL l1 ==>
    IS_DECIMAL l2 ==>
    (LENGTH l1 = LENGTH l2) ==>
    (DECIMAL l1 = DECIMAL l2) ==>
    (l1 = l2)
\ENDTHEOREM
\THEOREM DECIMAL\_CONS\_0 numeral
|- !l. DECIMAL(CONS 0 l) = DECIMAL l
\ENDTHEOREM
\THEOREM DECIMAL\_DIGIT\_EQ\_DIGIT numeral
|- !e. DECIMAL[e] = e
\ENDTHEOREM
\THEOREM DECIMAL\_EMPTY\_EQ\_0 numeral
|- !l. IS_NORMALIZED l ==> ((DECIMAL l = 0) = (l = []))
\ENDTHEOREM
\THEOREM DECIMAL\_EXP\_LESS numeral
|- !l.
    IS_DECIMAL l ==>
    IS_NORMALIZED l ==>
    ~NULL l ==>
    ((10 EXP ((LENGTH l) - 1)) - 1) < (DECIMAL l)
\ENDTHEOREM
\THEOREM DECIMAL\_EXP\_LESS\_OR\_EQ numeral
|- !l.
    ~NULL l ==>
    IS_NORMALIZED l ==>
    IS_DECIMAL l ==>
    (10 EXP ((LENGTH l) - 1)) <= (DECIMAL l)
\ENDTHEOREM
\THEOREM DECIMAL\_EXP\_N numeral
|- !n. DECIMAL(CONS 1(REPLICATE n 0)) = 10 EXP n
\ENDTHEOREM
\THEOREM DECIMAL\_LESS\_EXP\_LENGTH numeral
|- !l. IS_DECIMAL l ==> (DECIMAL l) < (10 EXP (LENGTH l))
\ENDTHEOREM
\THEOREM DECIMAL\_LESS\_OR\_EQ\_EXP\_LENGTH numeral
|- !l. IS_DECIMAL l ==> (DECIMAL l) <= ((10 EXP (LENGTH l)) - 1)
\ENDTHEOREM
\THEOREM DECIMAL\_ONTO numeral
|- !l. ?n. DECIMAL l = n
\ENDTHEOREM
\THEOREM DECIMAL\_ZEROS numeral
|- !n. DECIMAL(REPLICATE n 0) = 0
\ENDTHEOREM
\THEOREM HEX numeral
|- HEX = BASEN 16
\ENDTHEOREM
\THEOREM HEX\_11 numeral
|- !l1 l2.
    IS_HEX l1 ==>
    IS_HEX l2 ==>
    (LENGTH l1 = LENGTH l2) ==>
    (HEX l1 = HEX l2) ==>
    (l1 = l2)
\ENDTHEOREM
\THEOREM HEX\_CONS\_0 numeral
|- !l. HEX(CONS 0 l) = HEX l
\ENDTHEOREM
\THEOREM HEX\_DIGIT\_EQ\_DIGIT numeral
|- !e. HEX[e] = e
\ENDTHEOREM
\THEOREM HEX\_EMPTY\_EQ\_0 numeral
|- !l. IS_NORMALIZED l ==> ((HEX l = 0) = (l = []))
\ENDTHEOREM
\THEOREM HEX\_EXP\_LESS numeral
|- !l.
    IS_HEX l ==>
    IS_NORMALIZED l ==>
    ~NULL l ==>
    ((16 EXP ((LENGTH l) - 1)) - 1) < (HEX l)
\ENDTHEOREM
\THEOREM HEX\_EXP\_LESS\_OR\_EQ numeral
|- !l.
    ~NULL l ==>
    IS_NORMALIZED l ==>
    IS_HEX l ==>
    (16 EXP ((LENGTH l) - 1)) <= (HEX l)
\ENDTHEOREM
\THEOREM HEX\_EXP\_N numeral
|- !n. HEX(CONS 1(REPLICATE n 0)) = 16 EXP n
\ENDTHEOREM
\THEOREM HEX\_LESS\_EXP\_LENGTH numeral
|- !l. IS_HEX l ==> (HEX l) < (16 EXP (LENGTH l))
\ENDTHEOREM
\THEOREM HEX\_LESS\_OR\_EQ\_EXP\_LENGTH numeral
|- !l. IS_HEX l ==> (HEX l) <= ((16 EXP (LENGTH l)) - 1)
\ENDTHEOREM
\THEOREM HEX\_ONTO numeral
|- !l. ?n. HEX l = n
\ENDTHEOREM
\THEOREM HEX\_ZEROS numeral
|- !n. HEX(REPLICATE n 0) = 0
\ENDTHEOREM
\THEOREM IS\_BASEN numeral
|- !radix digits. IS_BASEN radix digits = EVERY($> radix)digits
\ENDTHEOREM
\THEOREM IS\_BASEN\_APPEND numeral
|- !r l m. IS_BASEN r(APPEND l m) = IS_BASEN r l /\ IS_BASEN r m
\ENDTHEOREM
\THEOREM IS\_BASEN\_CONS numeral
|- !r l e. 1 < r ==> (IS_BASEN r(CONS e l) = e < r /\ IS_BASEN r l)
\ENDTHEOREM
\THEOREM IS\_BASEN\_CONS\_EQ numeral
|- !r l e. IS_BASEN r(CONS e l) = e < r /\ IS_BASEN r l
\ENDTHEOREM
\THEOREM IS\_BASEN\_CONS\_IMP\_IS\_BASEN numeral
|- !r l e. 1 < r ==> IS_BASEN r(CONS e l) ==> IS_BASEN r l
\ENDTHEOREM
\THEOREM IS\_BASEN\_CONS\_IMP\_LESS numeral
|- !r l e. 1 < r ==> IS_BASEN r(CONS e l) ==> e < r
\ENDTHEOREM
\THEOREM IS\_BASEN\_NIL numeral
|- !r. IS_BASEN r[]
\ENDTHEOREM
\THEOREM IS\_BINARY numeral
|- IS_BINARY = IS_BASEN 2
\ENDTHEOREM
\THEOREM IS\_BINARY\_CONS numeral
|- !l e. IS_BINARY(CONS e l) = e < 2 /\ IS_BINARY l
\ENDTHEOREM
\THEOREM IS\_BINARY\_CONS\_IMP\_IS\_BINARY numeral
|- !l e. IS_BINARY(CONS e l) ==> IS_BINARY l
\ENDTHEOREM
\THEOREM IS\_BINARY\_CONS\_IMP\_LESS numeral
|- !l e. IS_BINARY(CONS e l) ==> e < 2
\ENDTHEOREM
\THEOREM IS\_BINARY\_NIL numeral
|- IS_BINARY[]
\ENDTHEOREM
\THEOREM IS\_BINARY\_NORMALIZED numeral
|- !digits.
    IS_BINARY_NORMALIZED digits =
    IS_BINARY digits /\ IS_NORMALIZED digits
\ENDTHEOREM
\THEOREM IS\_DECIMAL numeral
|- IS_DECIMAL = IS_BASEN 10
\ENDTHEOREM
\THEOREM IS\_DECIMAL\_CONS numeral
|- !l e. IS_DECIMAL(CONS e l) = e < 10 /\ IS_DECIMAL l
\ENDTHEOREM
\THEOREM IS\_DECIMAL\_CONS\_IMP\_IS\_DECIMAL numeral
|- !l e. IS_DECIMAL(CONS e l) ==> IS_DECIMAL l
\ENDTHEOREM
\THEOREM IS\_DECIMAL\_CONS\_IMP\_LESS numeral
|- !l e. IS_DECIMAL(CONS e l) ==> e < 10
\ENDTHEOREM
\THEOREM IS\_DECIMAL\_NIL numeral
|- IS_DECIMAL[]
\ENDTHEOREM
\THEOREM IS\_DECIMAL\_NORMALIZED numeral
|- !digits.
    IS_DECIMAL_NORMALIZED digits =
    IS_DECIMAL digits /\ IS_NORMALIZED digits
\ENDTHEOREM
\THEOREM IS\_HEX numeral
|- IS_HEX = IS_BASEN 16
\ENDTHEOREM
\THEOREM IS\_HEX\_CONS numeral
|- !l e. IS_HEX(CONS e l) = e < 16 /\ IS_HEX l
\ENDTHEOREM
\THEOREM IS\_HEX\_CONS\_IMP\_IS\_HEX numeral
|- !l e. IS_HEX(CONS e l) ==> IS_HEX l
\ENDTHEOREM
\THEOREM IS\_HEX\_CONS\_IMP\_LESS numeral
|- !l e. IS_HEX(CONS e l) ==> e < 16
\ENDTHEOREM
\THEOREM IS\_HEX\_NIL numeral
|- IS_HEX[]
\ENDTHEOREM
\THEOREM IS\_HEX\_NORMALIZED numeral
|- !digits.
    IS_HEX_NORMALIZED digits = IS_HEX digits /\ IS_NORMALIZED digits
\ENDTHEOREM
\THEOREM IS\_NORMALIZED numeral
|- !digits. IS_NORMALIZED digits = (digits = []) \/ 0 < (HD digits)
\ENDTHEOREM
\THEOREM IS\_NORMALIZED\_CONS numeral
|- !e l. IS_NORMALIZED(CONS e l) = 0 < e
\ENDTHEOREM
\THEOREM IS\_NORMALIZED\_NIL numeral
|- IS_NORMALIZED[]
\ENDTHEOREM
\THEOREM IS\_OCTAL numeral
|- IS_OCTAL = IS_BASEN 8
\ENDTHEOREM
\THEOREM IS\_OCTAL\_CONS numeral
|- !l e. IS_OCTAL(CONS e l) = e < 8 /\ IS_OCTAL l
\ENDTHEOREM
\THEOREM IS\_OCTAL\_CONS\_IMP\_IS\_HEX numeral
|- !l e. IS_OCTAL(CONS e l) ==> IS_OCTAL l
\ENDTHEOREM
\THEOREM IS\_OCTAL\_CONS\_IMP\_LESS numeral
|- !l e. IS_OCTAL(CONS e l) ==> e < 8
\ENDTHEOREM
\THEOREM IS\_OCTAL\_NIL numeral
|- IS_OCTAL[]
\ENDTHEOREM
\THEOREM IS\_OCTAL\_NORMALIZED numeral
|- !digits.
    IS_OCTAL_NORMALIZED digits = IS_OCTAL digits /\ IS_NORMALIZED digits
\ENDTHEOREM
\THEOREM LOG numeral
|- !r n. LOG r n = (@x. (r EXP x) <= n /\ n < (r EXP (x + 1)))
\ENDTHEOREM
\THEOREM LOG\_1 numeral
|- !r. 1 < r ==> (LOG r 1 = 0)
\ENDTHEOREM
\THEOREM NORMALIZED\_BASEN\_11 numeral
|- !l1 l2 r.
    1 < r ==>
    IS_BASEN r l1 ==>
    IS_BASEN r l2 ==>
    IS_NORMALIZED l1 ==>
    IS_NORMALIZED l2 ==>
    (BASEN r l1 = BASEN r l2) ==>
    (l1 = l2)
\ENDTHEOREM
\THEOREM NORMALIZED\_LENGTHS numeral
|- !l1 l2 r.
    1 < r ==>
    IS_BASEN r l1 ==>
    IS_BASEN r l2 ==>
    IS_NORMALIZED l1 ==>
    IS_NORMALIZED l2 ==>
    (BASEN r l1 = BASEN r l2) ==>
    (LENGTH l1 = LENGTH l2)
\ENDTHEOREM
\THEOREM NORMALIZED\_LENGTHS\_LEMMA numeral
|- !l1 l2 r.
    ~(1 < r /\
      IS_BASEN r l1 /\
      IS_BASEN r l2 /\
      IS_NORMALIZED l1 /\
      IS_NORMALIZED l2 /\
      (BASEN r l1 = BASEN r l2) /\
      (LENGTH l1) < (LENGTH l2))
\ENDTHEOREM
\THEOREM OCTAL numeral
|- OCTAL = BASEN 8
\ENDTHEOREM
\THEOREM OCTAL\_11 numeral
|- !l1 l2.
    IS_OCTAL l1 ==>
    IS_OCTAL l2 ==>
    (LENGTH l1 = LENGTH l2) ==>
    (OCTAL l1 = OCTAL l2) ==>
    (l1 = l2)
\ENDTHEOREM
\THEOREM OCTAL\_CONS\_0 numeral
|- !l. OCTAL(CONS 0 l) = OCTAL l
\ENDTHEOREM
\THEOREM OCTAL\_DIGIT\_EQ\_DIGIT numeral
|- !e. OCTAL[e] = e
\ENDTHEOREM
\THEOREM OCTAL\_EMPTY\_EQ\_0 numeral
|- !l. IS_NORMALIZED l ==> ((OCTAL l = 0) = (l = []))
\ENDTHEOREM
\THEOREM OCTAL\_EXP\_LESS numeral
|- !l.
    IS_OCTAL l ==>
    IS_NORMALIZED l ==>
    ~NULL l ==>
    ((8 EXP ((LENGTH l) - 1)) - 1) < (OCTAL l)
\ENDTHEOREM
\THEOREM OCTAL\_EXP\_LESS\_OR\_EQ numeral
|- !l.
    ~NULL l ==>
    IS_NORMALIZED l ==>
    IS_OCTAL l ==>
    (8 EXP ((LENGTH l) - 1)) <= (OCTAL l)
\ENDTHEOREM
\THEOREM OCTAL\_EXP\_N numeral
|- !n. OCTAL(CONS 1(REPLICATE n 0)) = 8 EXP n
\ENDTHEOREM
\THEOREM OCTAL\_LESS\_EXP\_LENGTH numeral
|- !l. IS_OCTAL l ==> (OCTAL l) < (8 EXP (LENGTH l))
\ENDTHEOREM
\THEOREM OCTAL\_LESS\_OR\_EQ\_EXP\_LENGTH numeral
|- !l. IS_OCTAL l ==> (OCTAL l) <= ((8 EXP (LENGTH l)) - 1)
\ENDTHEOREM
\THEOREM OCTAL\_ONTO numeral
|- !l. ?n. OCTAL l = n
\ENDTHEOREM
\THEOREM OCTAL\_ZEROS numeral
|- !n. OCTAL(REPLICATE n 0) = 0
\ENDTHEOREM
\THEOREM SNOC\_APPEND numeral
|- !h l. SNOC h l = APPEND l[h]
\ENDTHEOREM
