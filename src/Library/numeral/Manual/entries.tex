\chapter{ML Functions in the numeral Library}
This chapter provides documentation on all the \ML\ functions that are made
available in \HOL\ when the \ml{more\_arithmetic} library is loaded.  This
documentation is also available online via the \ml{help} facility.



\DOC{BASEN\_ADD\_CONV}

\TYPE {\small\verb%BASEN_ADD_CONV : term -> thm%}\egroup

\SYNOPSIS
Proves the result of adding one numeral to another.

\DESCRIBE
If {\small\verb%m%} and {\small\verb%n%} are lists of digits in base {\small\verb%r%}, and {\small\verb%p%} is the list of digits
in the base {\small\verb%r%} numeral representing the sum of {\small\verb%m%} and {\small\verb%n%}, then
{\small\verb%BASEN_ADD_CONV%} returns the theorem:
{\par\samepage\setseps\small
\begin{verbatim}
        |- (BASEN r m + BASEN r n) = BASEN r p
\end{verbatim}
}
If the addend and augend are not normalized, the resulting theorem expresses
them normalized anyway.

\FAILURE
The argument to {\small\verb%BASEN_ADD_CONV%} must be of the form
{\small\verb%"BASEN r [...] + BASEN r [...]"%}, the radix of the two numerals must be the
same, and the radix must be a numeric constant that is at least 2, and all
elements of the digit lists must be numeric constants, and must be less than
the radix.

\EXAMPLE
{\par\samepage\setseps\small
\begin{verbatim}
#BASEN_ADD_CONV "BASEN 10 [3;4;5] + BASEN 10 [3;4;5]";;
|- BASEN 10 [3;4;5] + BASEN 10 [3;4;5] = BASEN 10 [6;9;0]

#BASEN_ADD_CONV "BASEN 2 [1;1;1;1] + BASEN 2 [1;1;1;0]";;
|- BASEN 2 [1;1;1;1] + BASEN 2 [1;1;1;0]) = BASEN 2 [1;1;1;0;1]

#BASEN_ADD_CONV "BASEN 16 [8;10;14] + BASEN 16 [12;4]";;
|- BASEN 16 [8;10;14] + BASEN 16 [12;4]) = BASEN 16 [9;7;2]
\end{verbatim}
}
\SEEALSO
BASEN_SUC_CONV.

\ENDDOC
\DOC{BASEN\_COMPARE\_CONV}

\TYPE {\small\verb%BASEN_COMPARE_CONV : term -> thm%}\egroup

\SYNOPSIS
Given a comparison of two numerals, returns a theorem relating their values
with {\small\verb%"<"%}, {\small\verb%">"%}, or {\small\verb%"="%}.

\DESCRIBE
Given a term of the form {\small\verb%"BASEN r [...] op BASEN r [...]"%}, returns a theorem
of the form
{\par\samepage\setseps\small
\begin{verbatim}
   |- ( BASEN r [...] ?? BASEN r [...] ) = T
\end{verbatim}
}
where {\small\verb%??%} is {\small\verb%<%}, {\small\verb%>%} or {\small\verb%=%}.  Note that {\small\verb%op%} from the input term is ignored.

\EXAMPLE
{\par\samepage\setseps\small
\begin{verbatim}
#BASEN_COMPARE_CONV "BASEN 10 [1;2;3] < BASEN 10 [3;4;5]";;
|- (BASEN 10[1;2;3]) < (BASEN 10[3;4;5]) = T

#BASEN_COMPARE_CONV "BASEN 10 [7;8;9] < BASEN 10 [3;4;5]";;
|- (BASEN 10[7;8;9]) > (BASEN 10[3;4;5]) = T
\end{verbatim}
}
\FAILURE
Fails if the numerals' radices are different, or if any element of either
list is not a numeric constant, or is not less than the radix, or if the
radix is 0 or 1.

\SEEALSO
BASEN_EQ_CONV, BASEN_LT_CONV, BASEN_LE_CONV, BASEN_GT_CONV, BASEN_GE_CONV.

\ENDDOC
\DOC{BASEN\_DIV\_CONV}

\TYPE {\small\verb%BASEN_DIV_CONV : term -> thm%}\egroup

\SYNOPSIS
Proves the result of dividing one numeral by another.

\DESCRIBE
If {\small\verb%m%} and {\small\verb%n%} are lists of digits in base {\small\verb%r%}, and {\small\verb%p%} is the list of digits
in the base {\small\verb%r%} numeral representing the sum of {\small\verb%m%} and {\small\verb%n%}, then
{\small\verb%BASEN_DIV_CONV%} returns the theorem:
{\par\samepage\setseps\small
\begin{verbatim}
        |- (BASEN r m DIV BASEN r n) = BASEN r p
\end{verbatim}
}
\FAILURE
The argument term must be of the form {\small\verb%"BASEN r [...] DIV BASEN r [...]"%}, and
the radix of the two numerals must be the same, and the radix must be a numeric
constant that is at least 2, and all elements of the digit lists must be
numeric constants, and must be less than the radix, and the divisor must be
nonzero.

\EXAMPLE
{\par\samepage\setseps\small
\begin{verbatim}
#BASEN_DIV_CONV "BASEN 10 [3;4;5] DIV BASEN 10 [3;4;5]";;
|- BASEN 10 [3;4;5] DIV BASEN 10 [3;4;5] = BASEN 10 [1]

#BASEN_DIV_CONV "BASEN 2 [1;1;1;1] DIV BASEN 2 [1;1;1;0]";;
|- BASEN 2 [1;1;1;1] DIV BASEN 2 [1;1;1;0]) = BASEN 2 [1]

#BASEN_DIV_CONV "BASEN 16 [8;10;14] DIV BASEN 16 [12;4]";;
|- BASEN 16 [8;10;14] DIV BASEN 16 [12;4]) = BASEN 16 [11]
\end{verbatim}
}
\SEEALSO
BASEN_MOD_CONV

\ENDDOC
\DOC{BASEN\_EQ\_CONV}

\TYPE {\small\verb%BASEN_EQ_CONV : term -> thm%}\egroup

\SYNOPSIS
Given a term equating two numerals, returns a theorem saying whether the
equation is true or false.

\DESCRIBE
Given a term of the form {\small\verb%"BASEN r [...] = BASEN r [...]"%}, returns a theorem
of the form {\small\verb%|- (BASEN r [...] op BASEN r [...]) = b%}, where {\small\verb%b%} is {\small\verb%"T"%} or
{\small\verb%"F"%}.

\FAILURE
The argument must be of the expected form, and the radix of the two numerals
must be the same, and the radix must be a numeric constant that is at least
2, and all elements of the digit lists must be numeric constants, and must be
less than the radix.

\EXAMPLE
{\par\samepage\setseps\small
\begin{verbatim}
#BASEN_EQ_CONV "BASEN 10 [3;4;5] = BASEN 10 [3;4;5]";;
|- (BASEN 10 [3;4;5] = BASEN 10 [3;4;5]) = T

#BASEN_LT_CONV "BASEN 2 [1;1;1;1] < BASEN 2 [1;1;1;0]";;
|- (BASEN 2[1;1;1;1]) < (BASEN 2[1;1;1;0]) = F

#BASEN_GE_CONV "BASEN 16 [8;10;14] >= BASEN 16 [12;4]";;
|- (BASEN 16[8;10;14]) >= (BASEN 16[12;4]) = T
\end{verbatim}
}
\SEEALSO
BASEN_COMPARE_CONV.

\ENDDOC
\DOC{BASEN\_EXP\_CONV}

\TYPE {\small\verb%BASEN_EXP_CONV : term -> thm%}\egroup

\SYNOPSIS
Proves the result of raising one numeral to the power specified by another.

\DESCRIBE
If {\small\verb%m%} and {\small\verb%n%} are lists of digits in base {\small\verb%r%}, and {\small\verb%p%} is the list of digits
in the base {\small\verb%r%} numeral representing the sum of {\small\verb%m%} and {\small\verb%n%}, then
{\small\verb%BASEN_EXP_CONV%} returns the theorem:
{\par\samepage\setseps\small
\begin{verbatim}
        |- (BASEN r m EXP BASEN r n) = BASEN r p
\end{verbatim}
}
\FAILURE
The argument to {\small\verb%BASEN_EXP_CONV%} must be of the form
{\small\verb%"BASEN r [...] EXP BASEN r [...]"%}, and the radix of the two numerals must be
the same, and the radix must be a numeric constant that is at least 2, and all
elements of the digit lists must be numeric constants, and must be less than
the radix.

\EXAMPLE
{\par\samepage\setseps\small
\begin{verbatim}
#BASEN_EXP_CONV "BASEN 10 [2] EXP BASEN 10 [1;2]";;
|- BASEN 10 [2] EXP BASEN 10 [1;2] = BASEN 10 [4;0;9;6]

#BASEN_EXP_CONV "BASEN 2 [1;1] EXP BASEN 2 [1;0]";;
|- BASEN 2 [1;1] EXP BASEN 2 [1;0]) = BASEN 2 [1;0;0;1]

#BASEN_EXP_CONV "BASEN 16 [1;0] EXP BASEN 16 [4]";;
|- BASEN 16 [1;0] EXP BASEN 16 [4]) = BASEN 16 [1;0;0;0;0]
\end{verbatim}
}
\ENDDOC
\DOC{BASEN\_GE\_CONV}

\TYPE {\small\verb%BASEN_GE_CONV : term -> thm%}\egroup

\SYNOPSIS
Given a term comparing two numerals, returns a theorem saying whether one is
greater than or equal to the other.

\DESCRIBE
Given a term of the form {\small\verb%"BASEN r [...] >= BASEN r [...]"%}, returns a theorem
of the form {\small\verb%|- (BASEN r [...] >= BASEN r [...]) = b%}, where {\small\verb%b%} is {\small\verb%"T"%} or
{\small\verb%"F"%}.

\FAILURE
The argument must be of the expected form, and the radix of the two numerals
must be the same, and the radix must be a numeric constant that is at least
2, and all elements of the digit lists must be numeric constants, and must be
less than the radix.

\EXAMPLE
{\par\samepage\setseps\small
\begin{verbatim}
#BASEN_GE_CONV "BASEN 16 [8;10;14] >= BASEN 16 [12;4]";;
|- (BASEN 16[8;10;14]) >= (BASEN 16[12;4]) = T
\end{verbatim}
}
\SEEALSO
BASEN_COMPARE_CONV, BASEN_GT_CONV, BASEN_LE_CONV, BASEN_LT_CONV.

\ENDDOC
\DOC{BASEN\_GT\_CONV}

\TYPE {\small\verb%BASEN_GT_CONV : term -> thm%}\egroup

\SYNOPSIS
Given a term comparing two numerals, returns a theorem saying whether one is
greater than the other.

\DESCRIBE
Given a term of the form {\small\verb%"BASEN r [...] > BASEN r [...]"%}, returns a theorem
of the form {\small\verb%|- (BASEN r [...] > BASEN r [...]) = b%}, where {\small\verb%b%} is {\small\verb%"T"%} or
{\small\verb%"F"%}.

\FAILURE
The argument must be of the expected form, and the radix of the two numerals
must be the same, and the radix must be a numeric constant that is at least
2, and all elements of the digit lists must be numeric constants, and must be
less than the radix.

\EXAMPLE
{\par\samepage\setseps\small
\begin{verbatim}
#BASEN_GT_CONV "BASEN 16 [8;10;14] > BASEN 16 [12;4]";;
|- (BASEN 16[8;10;14]) > (BASEN 16[12;4]) = T
\end{verbatim}
}
\SEEALSO
BASEN_COMPARE_CONV, BASEN_GE_CONV, BASEN_LE_CONV, BASEN_LT_CONV.

\ENDDOC
\DOC{BASEN\_LE\_CONV}

\TYPE {\small\verb%BASEN_LE_CONV : term -> thm%}\egroup

\SYNOPSIS
Given a term comparing two numerals, returns a theorem saying whether one is
less than or equal to the other.

\DESCRIBE
Given a term of the form {\small\verb%"BASEN r [...] <= BASEN r [...]"%}, returns a theorem
of the form {\small\verb%|- (BASEN r [...] <= BASEN r [...]) = b%}, where {\small\verb%b%} is {\small\verb%"T"%} or
{\small\verb%"F"%}.

\FAILURE
The argument must be of the expected form, and the radix of the two numerals
must be the same, and the radix must be a numeric constant that is at least
2, and all elements of the digit lists must be numeric constants, and must be
less than the radix.

\EXAMPLE
{\par\samepage\setseps\small
\begin{verbatim}
#BASEN_LE_CONV "BASEN 16 [8;10;14] <= BASEN 16 [12;4]";;
|- (BASEN 16[8;10;14]) <= (BASEN 16[12;4]) = F
\end{verbatim}
}
\SEEALSO
BASEN_COMPARE_CONV, BASEN_GE_CONV, BASEN_GT_CONV, BASEN_LT_CONV.

\ENDDOC
\DOC{BASEN\_LT\_CONV}

\TYPE {\small\verb%BASEN_LT_CONV : term -> thm%}\egroup

\SYNOPSIS
Given a term comparing two numerals, returns a theorem saying whether one is
less than the other.

\DESCRIBE
Given a term of the form {\small\verb%"BASEN r [...] < BASEN r [...]"%}, returns a theorem
of the form {\small\verb%|- (BASEN r [...] < BASEN r [...]) = b%}, where {\small\verb%b%} is {\small\verb%"T"%} or
{\small\verb%"F"%}.

\FAILURE
The argument must be of the expected form, and the radix of the two numerals
must be the same, and the radix must be a numeric constant that is at least
2, and all elements of the digit lists must be numeric constants, and must be
less than the radix.

\EXAMPLE
{\par\samepage\setseps\small
\begin{verbatim}
#BASEN_LT_CONV "BASEN 16 [8;10;14] < BASEN 16 [12;4]";;
|- (BASEN 16[8;10;14]) < (BASEN 16[12;4]) = F
\end{verbatim}
}
\SEEALSO
BASEN_COMPARE_CONV, BASEN_GE_CONV, BASEN_GT_CONV, BASEN_LE_CONV.

\ENDDOC
\DOC{BASEN\_MOD\_CONV}

\TYPE {\small\verb%BASEN_MOD_CONV : term -> thm%}\egroup

\SYNOPSIS
Proves the result of taking the modulus of one numeral by another.

\DESCRIBE
If {\small\verb%m%} and {\small\verb%n%} are lists of digits in base {\small\verb%r%}, and {\small\verb%p%} is the list of digits
in the base {\small\verb%r%} numeral representing the sum of {\small\verb%m%} and {\small\verb%n%}, then
{\small\verb%BASEN_MOD_CONV%} returns the theorem:
{\par\samepage\setseps\small
\begin{verbatim}
        |- (BASEN r m MOD BASEN r n) = BASEN r p
\end{verbatim}
}
\FAILURE
The argument term must be of the form {\small\verb%BASEN r [...] MOD BASEN r [...]"%}, and
the radix of the two numerals must be the same, and the radix must be a numeric
constant that is at least 2, and all elements of the digit lists must be
numeric constants, and must be less than the radix, and the divisor must be
nonzero.

\EXAMPLE
{\par\samepage\setseps\small
\begin{verbatim}
#BASEN_MOD_CONV "BASEN 10 [3;4;5] MOD BASEN 10 [3;4;5]";;
|- BASEN 10 [3;4;5] MOD BASEN 10 [3;4;5] = BASEN 10 []

#BASEN_MOD_CONV "BASEN 2 [1;1;1;1] MOD BASEN 2 [1;1;1;0]";;
|- BASEN 2 [1;1;1;1] MOD BASEN 2 [1;1;1;0]) = BASEN 2 [1]

#BASEN_MOD_CONV "BASEN 16 [8;10;14] MOD BASEN 16 [12;4]";;
|- BASEN 16 [8;10;14] MOD BASEN 16 [12;4]) = BASEN 16 [4;2]
\end{verbatim}
}
\SEEALSO
BASEN_DIV_CONV.

\ENDDOC
\DOC{BASEN\_MUL\_CONV}

\TYPE {\small\verb%BASEN_MUL_CONV : term -> thm%}\egroup

\SYNOPSIS
Proves the result of multiplying one numeral by another.

\DESCRIBE
If {\small\verb%m%} and {\small\verb%n%} are lists of digits in base {\small\verb%r%}, and {\small\verb%p%} is the list of digits
in the base {\small\verb%r%} numeral representing the sum of {\small\verb%m%} and {\small\verb%n%}, then
{\small\verb%BASEN_MUL_CONV%} returns the theorem:
{\par\samepage\setseps\small
\begin{verbatim}
        |- (BASEN r m * BASEN r n) = BASEN r p
\end{verbatim}
}
\FAILURE
The argument to {\small\verb%BASEN_MUL_CONV%} must be of the form
{\small\verb%"BASEN r [...] * BASEN r [...]"%}, and the radix of the two numerals must be
the same, and the radix must be a numeric constant that is at least 2, and all
elements of the digit lists must be numeric constants, and must be less than
the radix.

\EXAMPLE
{\par\samepage\setseps\small
\begin{verbatim}
#BASEN_MUL_CONV "BASEN 10 [3;4;5] * BASEN 10 [3;4;5]";;
|- BASEN 10 [3;4;5] * BASEN 10 [3;4;5] = BASEN 10 [1;1;9;0;2;5]

#BASEN_MUL_CONV "BASEN 2 [1;1;1;1] * BASEN 2 [1;1;1;0]";;
|- BASEN 2 [1;1;1;1] * BASEN 2 [1;1;1;0]) = BASEN 2 [1;1;0;1;0;0;1;0]

#BASEN_MUL_CONV "BASEN 16 [1;0] * BASEN 16 [4]";;
|- BASEN 16 [1;0] * BASEN 16 [4]) = BASEN 16 [4;0]
\end{verbatim}
}
\ENDDOC
\DOC{BASEN\_NORMALIZE\_CONV}

\TYPE {\small\verb%BASEN_NORMALIZE_CONV : term -> thm%}\egroup

\SYNOPSIS
Proves that a numeral is equal to itself with leading zeros removed.

\DESCRIBE
If {\small\verb%m%} is a list of base {\small\verb%r%} digits, and {\small\verb%n%} is the same list with leading
zeros removed, then {\small\verb%BASEN_NORMALIZE_CONV "BASEN r m"%} returns the theorem:
{\par\samepage\setseps\small
\begin{verbatim}
        |- BASEN r m = BASEN r n
\end{verbatim}
}
\FAILURE
The argument to {\small\verb%BASEN_CONV%} must be of the form {\small\verb%"BASEN r [...]"%}, and the
radix must be a numeric constant with a value of at least 2, and all elements
of the digit list must be numeric constants less than the radix.  If the
argument is already normalized, {\small\verb%BASEN_NORMALIZE_CONV%} returns a reflexive
theorem.

\EXAMPLE
{\par\samepage\setseps\small
\begin{verbatim}
#BASEN_NORMALIZE_CONV "BASEN 16 [0;12;0;0]";;
|- BASEN 16 [0;12;0;0] = BASEN 16 [12;0;0]

#BASEN_NORMALIZE_CONV "BASEN 16 [0;0;0;0;3]";;
|- BASEN 16 [0;0;0;0;3] = BASEN 16 [3]

#BASEN_NORMALIZE_CONV "BASEN 8 [7;7;5;3;4]";;
|- BASEN 8 [7;7;5;3;4] = BASEN 8 [7;7;5;3;4]
\end{verbatim}
}
\SEEALSO
BASEN_NORMALIZE_OR_FAIL_CONV, IS_NORMALIZED_CONV, IS_BASEN_CONV.

\ENDDOC
\DOC{BASEN\_NORMALIZE\_OR\_FAIL\_CONV}

\TYPE {\small\verb%BASEN_NORMALIZE_OR_FAIL_CONV : term -> thm%}\egroup

\SYNOPSIS
Proves that a numeral is equal to itself with leading zeros removed.

\DESCRIBE
If {\small\verb%m%} is a list of base {\small\verb%r%} digits, and {\small\verb%n%} is the same list with leading
zeros removed, then {\small\verb%BASEN_NORMALIZE_OR_FAIL_CONV "BASEN r m"%} returns the
theorem:
{\par\samepage\setseps\small
\begin{verbatim}
        |- BASEN r m = BASEN r n
\end{verbatim}
}
\FAILURE
The argument to {\small\verb%BASEN_CONV%} must be of the form {\small\verb%"BASEN r [...]"%}, and the
radix must be a numeric constant with a value of at least 2, and all elements
of the digit list must be numeric constants less than the radix.  If the
argument is already normalized, {\small\verb%BASEN_NORMALIZE_OR_FAIL_CONV%} fails.

\EXAMPLE
{\par\samepage\setseps\small
\begin{verbatim}
#BASEN_NORMALIZE_OR_FAIL_CONV "BASEN 16 [0;12;0;0]";;
|- BASEN 16 [0;12;0;0] = BASEN 16 [12;0;0]

#BASEN_NORMALIZE_OR_FAIL_CONV "BASEN 16 [0;0;0;0;3]";;
|- BASEN 16 [0;0;0;0;3] = BASEN 16 [3]

#BASEN_NORMALIZE_OR_FAIL_CONV "BASEN 8 [7;7;5;3;4]";;
evaluation failed       BASEN_NORMALIZE_OR_FAIL_CONV
\end{verbatim}
}
\COMMENTS
{\small\verb%BASEN_NORMALIZE_OR_FAIL_CONV%} is needed to avoid looping in higher-level
operations (like {\small\verb%NUM_ARITH_CONV%}).

\SEEALSO
BASEN_NORMALIZE_CONV, IS_NORMALIZED_CONV, IS_BASEN_CONV.

\ENDDOC
\DOC{basen\_of\_int}

\TYPE {\small\verb%basen_of_int : term -> thm%}\egroup

\SYNOPSIS
Given a radix and a value, returns a term containing a BASEN numeral in that
radix, with that value.

\DESCRIBE
If {\small\verb%r%} and {\small\verb%n%} are ML integers, and {\small\verb%r%} is at least 2, and {\small\verb%n%} is positive,
then {\small\verb%basen_of_int r n%} returns the term containing the canonical BASEN numeral
in that radix, of that value.

\FAILURE
The radix must have a value of at least 2, and {\small\verb%n%} must be positive.

\EXAMPLE
{\par\samepage\setseps\small
\begin{verbatim}
#basen_of_int (10, 34567);;
"BASEN 10[3;4;5;6;7]" : term

#basen_of_int (2, 23);;
"BASEN 2[1;0;1;1;1]" : term

#basen_of_int (16, 3072);;
"BASEN 16[12;0;0]" : term
\end{verbatim}
}
\SEEALSO
BASEN_OF_NUM_CONV, int_of_basen, mk_basen.

\ENDDOC
\DOC{BASEN\_OF\_NUM\_CONV}

\TYPE {\small\verb%BASEN_OF_NUM_CONV : conv%}\egroup

\SYNOPSIS
Proves what the numeral is that represents a given numeric value in a given
radix.

\DESCRIBE
If {\small\verb%m%} and {\small\verb%r%} are numeric constants, and {\small\verb%n%} is a list of base {\small\verb%r%} digits with
a value of {\small\verb%m%} (in base {\small\verb%r%}), then {\small\verb%BASEN_OF_NUM_CONV "r" "m"%} returns the
theorem:
{\par\samepage\setseps\small
\begin{verbatim}
        |- BASEN_DIGITS r m = n
\end{verbatim}
}
\FAILURE
The radix must be a numeric constant with a value of at least 2, and the value
must be a numeric constant.

\EXAMPLE
{\par\samepage\setseps\small
\begin{verbatim}
#BASEN_OF_NUM_CONV "2" "32";;
|- 32 = BASEN 2 [1;0;0;0;0;0]

#BASEN_OF_NUM_CONV "10" "265";;
|- 265 = BASEN 10 [2;6;5]

#BASEN_OF_NUM_CONV "16" "3072";;
|- 3072 = BASEN 16 [12;0;0]
\end{verbatim}
}
\SEEALSO
mk_basen, BASEN_CONV, basen_of_int.

\ENDDOC
\DOC{BASEN\_PRE\_CONV}

\TYPE {\small\verb%BASEN_PRE_CONV : term -> thm%}\egroup

\SYNOPSIS
Converts the predecessor function on a numeral into a subtract-one in the same
radix.

\DESCRIBE
If {\small\verb%m%} is a list of digits in base {\small\verb%r%}, and {\small\verb%m%} is a list of digits in base
{\small\verb%r%}, then {\small\verb%BASEN_PRE_CONV "PRE (BASEN r m)"%} returns the theorem:
{\par\samepage\setseps\small
\begin{verbatim}
        |- PRE (BASEN r m) = BASEN r m - BASEN r [1]
\end{verbatim}
}
\FAILURE
The argument to {\small\verb%BASEN_PRE_CONV%} must be of the form {\small\verb%"PRE (BASEN r [...])"%},
and the radix must be a numeric constant that is at least 2, and all elements
of the digit lists must be numeric constants, and must be less than the radix.

\EXAMPLE
{\par\samepage\setseps\small
\begin{verbatim}
#BASEN_PRE_CONV "PRE (BASEN 10 [3;4;5])";;
|- PRE(BASEN 10[3;4;5]) = BASEN 10[3;4;4]

#BASEN_PRE_CONV "PRE (BASEN 10 [])";;
|- PRE(BASEN 10[]) = BASEN 10[]

#BASEN_PRE_CONV "PRE (BASEN 2 [1;1;1;1])";;
|- PRE(BASEN 2[1;1;1;1]) = BASEN 2[1;1;1;0]

#BASEN_PRE_CONV "PRE (BASEN 16 [8;10;14])";;
|- PRE(BASEN 16[8;0;0]) = BASEN 16[7;15;15]
\end{verbatim}
}
\SEEALSO
BASEN_SUB_CONV, BASEN_SUC_CONV.

\ENDDOC
\DOC{BASEN\_SUB\_CONV}

\TYPE {\small\verb%BASEN_SUB_CONV : term -> thm%}\egroup

\SYNOPSIS
Proves the result of subtracting one numeral from another.

\DESCRIBE
If {\small\verb%m%} and {\small\verb%n%} are lists of digits in base {\small\verb%r%}, and {\small\verb%p%} is the list of digits
in the base {\small\verb%r%} numeral representing the sum of {\small\verb%m%} and {\small\verb%n%}, then
{\small\verb%BASEN_SUB_CONV%} returns the theorem:
{\par\samepage\setseps\small
\begin{verbatim}
        |- (BASEN r m - BASEN r n) = BASEN r p
\end{verbatim}
}
\FAILURE
The argument to {\small\verb%BASEN_SUB_CONV%} must be of the form
{\small\verb%"BASEN r [...] - BASEN r [...]"%}, and the radix of the two numerals must be
the same, and the radix must be a numeric constant that is at least 2, and all
elements of the digit lists must be numeric constants, and must be less than
the radix.

\EXAMPLE
{\par\samepage\setseps\small
\begin{verbatim}
#BASEN_SUB_CONV "BASEN 10 [3;4;5] - BASEN 10 [3;4;5]";;
|- BASEN 10 [3;4;5] - BASEN 10 [3;4;5] = BASEN 10 []

#BASEN_SUB_CONV "BASEN 2 [1;1;1;1] - BASEN 2 [1;1;1;0]";;
|- BASEN 2 [1;1;1;1] - BASEN 2 [1;1;1;0]) = BASEN 2 [1]

#BASEN_SUB_CONV "BASEN 16 [10;14] - BASEN 16 [8;4]";;
|- BASEN 16 [10;14] - BASEN 16 [8;4]) = BASEN 16 []
\end{verbatim}
}
\COMMENTS
This is natural number arithmetic, so subtracting a large number from a small
number returns 0.

\SEEALSO
BASEN_PRE_CONV.

\ENDDOC
\DOC{BASEN\_SUC\_CONV}

\TYPE {\small\verb%BASEN_SUC_CONV : term -> thm%}\egroup

\SYNOPSIS
Converts the successor function on a numeral into an add-one in the same radix.

\DESCRIBE
If {\small\verb%m%} is a list of digits in base {\small\verb%r%}, and {\small\verb%m%} is a list of digits in base
{\small\verb%r%}, then {\small\verb%BASEN_SUC_CONV "SUC (BASEN r m)"%} returns the theorem:
{\par\samepage\setseps\small
\begin{verbatim}
        |- SUC (BASEN r m) = BASEN r m + BASEN r [1]
\end{verbatim}
}
\FAILURE
The argument to {\small\verb%BASEN_SUC_CONV%} must be of the form {\small\verb%"SUC (BASEN r [...])"%},
and the radix must be a numeric constant that is at least 2, and all elements
of the digit lists must be numeric constants, and must be less than the radix.

\EXAMPLE
{\par\samepage\setseps\small
\begin{verbatim}
#BASEN_SUC_CONV "SUC (BASEN 10 [3;4;5])";;
|- SUC(BASEN 10[3;4;5]) = BASEN 10[3;4;6]

#BASEN_SUC_CONV "SUC (BASEN 2 [1;1;1;1])";;
|- SUC(BASEN 2[1;1;1;1]) = BASEN 2 [1;0;0;0;0]

#BASEN_SUC_CONV "SUC (BASEN 16 [8;10;14])";;
|- SUC(BASEN 16[8;10;14]) = BASEN 16[8;10;15]
\end{verbatim}
}
\SEEALSO
BASEN_ADD_CONV, BASEN_PRE_CONV.

\ENDDOC
\DOC{dest\_basen}

\TYPE {\small\verb%dest_basen : term -> thm%}\egroup

\SYNOPSIS
Takes apart a BASEN numeral.

\DESCRIBE
Given a term containing such a BASEN numeral, {\small\verb%dest_basen%} returns a term
containing the radix and a term containing the list of digits.

\COMMENTS
This is not exactly an inverse of {\small\verb%mk_basen%}, in that {\small\verb%mk_basen%} takes a list
of digit terms, and {\small\verb%dest_basen%} returns a single term of the digit list.

\FAILURE
{\small\verb%dest_basen%} fails if the term is not a BASEN numeral.

\ENDDOC
\DOC{dest\_binary\_basen\_comb}

\TYPE {\small\verb%dest_binary_basen_comb : term -> thm%}\egroup

\SYNOPSIS
Takes apart a term containing a binary operation on BASEN numerals, and returns
the parts.

\DESCRIBE
Given a term containing an operation applied to two BASEN numerals, returns the
operation, the radix of the two numerals, the numerals themselves, and the
numerals' digits (in two forms).

\EXAMPLE
{\par\samepage\setseps\small
\begin{verbatim}
#dest_binary_basen_comb "$*" "BASEN 10 [2;3;4] * BASEN 10 [6;7;8;9]";;
("10",
 "BASEN 10[2;3;4]",
 "[2;3;4]",
 ["2"; "3"; "4"],
 "BASEN 10[6;7;8;9]",
 "[6;7;8;9]",
 ["6"; "7"; "8"; "9"])
: (term # term # term # term list # term # term # term list)
\end{verbatim}
}
\FAILURE
Fails if the numerals have different radices.

\ENDDOC
\DOC{dest\_unary\_basen\_comb}

\TYPE {\small\verb%dest_unary_basen_comb  : term -> thm%}\egroup

\SYNOPSIS
Takes apart a term containing a unary operation on BASEN numerals, and returns
the parts.

\DESCRIBE
Given a term containing an operation applied to one BASEN numeral, returns the
operation, the radix of the numeral, the numeral itself, and the numeral's
digits (in two forms).

\EXAMPLE
{\par\samepage\setseps\small
\begin{verbatim}
#dest_unary_basen_comb "SUC" "SUC (BASEN 10 [2;3;4])";;
("10",
 "BASEN 10[2;3;4]",
 "[2;3;4]",
 ["2"; "3"; "4"])
: (term # term # term # term list)
\end{verbatim}
}
\FAILURE
Never fails.

\ENDDOC
\DOC{int\_of\_basen}

\TYPE {\small\verb%int_of_basen : term -> thm%}\egroup

\SYNOPSIS
Given a term containing a BASEN numeral, return the value of that numeral.

\DESCRIBE
If {\small\verb%r%} is an ML integer and n is an ML integer list, then
{\small\verb%int_of_basen "BASEN r n"%} returns the value of the base {\small\verb%r%} numeral with
digits {\small\verb%n%}.

\FAILURE
Never fails.

\EXAMPLE
{\par\samepage\setseps\small
\begin{verbatim}
#int_of_basen "BASEN 10[3;4;5;6;7]";;
34567 : int

#int_of_basen "BASEN 2[1;0;1;1;1]";;
23 : int

#int_of_basen "BASEN 16[12;0;0]";;
3072 : int
\end{verbatim}
}
\SEEALSO
dest_basen, BASEN_CONV, basen_of_int.

\ENDDOC
\DOC{is\_basen}

\TYPE {\small\verb%is_basen : term -> thm%}\egroup

\SYNOPSIS
Tests whether a term is a BASEN numeral.

\DESCRIBE
Given a term, {\small\verb%is_basen%} returns {\small\verb%true%} if the term is a BASEN numeral, and
returns {\small\verb%false%} otherwise.

\FAILURE
Never fails.

\ENDDOC
\DOC{IS\_BASEN\_CONV}

\TYPE {\small\verb%IS_BASEN_CONV : term -> thm%}\egroup

\SYNOPSIS
Proves that a numeral's digits are or are not all less than its radix.

\DESCRIBE
If {\small\verb%r%} is a radix and {\small\verb%m%} is a list of radix {\small\verb%r%} digits (that is, each digit is
less than {\small\verb%r%}), then {\small\verb%IS_BASEN_CONV "IS_BASEN r m"%} returns the theorem:
{\par\samepage\setseps\small
\begin{verbatim}
        |- IS_BASEN r m = T
\end{verbatim}
}
and if one or more digits of {\small\verb%m%} is not a radix {\small\verb%r%} digit (that is, if some
digit in {\small\verb%m%} is equal to or greater than {\small\verb%r%}), then
{\small\verb%IS_BASEN_CONV "IS_BASEN r m"%} returns the theorem:
{\par\samepage\setseps\small
\begin{verbatim}
        |- IS_BASEN r m = F
\end{verbatim}
}
\FAILURE
The argument to {\small\verb%IS_BASEN_CONV%} must be of the form {\small\verb%"IS_BASEN r [...]"%}, and
the radix must be a numeric constant that is at least 2, and all elements of
the digit list must be numeric constants.

\EXAMPLE
{\par\samepage\setseps\small
\begin{verbatim}
#IS_BASEN_CONV "IS_BASEN 10 [2;3;5;0]";;
|- IS_BASEN 10 [2;3;5;0] = T

#IS_BASEN_CONV "IS_BASEN 10 [2;3;15;0]";;
|- IS_BASEN 10 [2;3;15;0] = F

#IS_BASEN_CONV "IS_BASEN 16 [2;3;15;0]";;
|- IS_BASEN 16 [2;3;15;0] = F
\end{verbatim}
}
\SEEALSO
IS_NORMALIZED_CONV, BASEN_NORMALIZE_CONV.

\ENDDOC
\DOC{IS\_NORMALIZED\_CONV}

\TYPE {\small\verb%IS_NORMALIZED_CONV : term -> thm%}\egroup

\SYNOPSIS
Proves that a list of digits is normalized (that is, that it has no leading
zeros).

\DESCRIBE
If {\small\verb%m%} is a list of digits, and {\small\verb%m%} is empty or the first digit of {\small\verb%m%} is
nonzero, then {\small\verb%IS_NORMALIZED_CONV "m"%} returns the theorem:
{\par\samepage\setseps\small
\begin{verbatim}
        |- IS_NORMALIZED m = T
\end{verbatim}
}
and if the first digit of {\small\verb%m%} is zero, then {\small\verb%IS_NORMALIZED_CONV "m"%} returns
the theorem:
{\par\samepage\setseps\small
\begin{verbatim}
        |- IS_NORMALIZED m = F
\end{verbatim}
}
\FAILURE
The argument to {\small\verb%IS_NORMALIZED_CONV%} must be a list of type num list, and
either the list must be empty or the first element must be a numeric constant
added with {\small\verb%CONS%}.

\EXAMPLE
{\par\samepage\setseps\small
\begin{verbatim}
#IS_NORMALIZED_CONV "IS_NORMALIZED [2;3;5;0]";;
|- IS_NORMALIZED [2;3;5;0] = T

#IS_NORMALIZED_CONV "IS_NORMALIZED [0;2;3;5;0]";;
|- IS_NORMALIZED [0;2;3;5;0] = F

#IS_NORMALIZED_CONV "IS_NORMALIZED []";;
|- IS_NORMALIZED [] = T
\end{verbatim}
}
\SEEALSO
IS_BASEN_CONV, BASEN_NORMALIZE_CONV.

\ENDDOC
\DOC{mk\_basen}

\TYPE {\small\verb%mk_basen : term -> thm%}\egroup

\SYNOPSIS
Constructs a BASEN numeral.

\DESCRIBE
Given a term containing a numeric constant to be used as a radix, and a list
of terms to be taken as the digits, {\small\verb%mk_basen%} constructs a term containing a
BASEN numeral in that radix with those digits.

\FAILURE
{\small\verb%mk_basen%} fails if the terms are of the wrong types.

\ENDDOC
\DOC{ONCE\_BASEN\_NORMALIZE\_CONV}

\TYPE {\small\verb%ONCE_BASEN_NORMALIZE_CONV : term -> thm%}\egroup

\SYNOPSIS
Proves that a numeral is equal to itself with leading zeros removed.

\DESCRIBE
If {\small\verb%m%} is a list of base {\small\verb%r%} digits, and {\small\verb%n%} is the same list with a single
leading zero removed, then {\small\verb%ONCE_BASEN_NORMALIZE_CONV "BASEN r m"%} returns the
theorem:
{\par\samepage\setseps\small
\begin{verbatim}
        |- BASEN r m = BASEN r n
\end{verbatim}
}
\FAILURE
The argument to {\small\verb%BASEN_CONV%} must be of the form {\small\verb%"BASEN r [...]"%}, and the
radix must be a numeric constant with a value of at least 2, and all elements
of the digit list must be numeric constants less than the radix.  If the
argument is already normalized, {\small\verb%ONCE_BASEN_NORMALIZE_CONV%} fails.
theorem.

\EXAMPLE
{\par\samepage\setseps\small
\begin{verbatim}
#ONCE_BASEN_NORMALIZE_CONV "BASEN 16 [0;12;0;0]";;
|- BASEN 16 [0;12;0;0] = BASEN 16 [12;0;0]

#ONCE_BASEN_NORMALIZE_CONV "BASEN 16 [0;0;0;0;3]";;
|- BASEN 16 [0;0;0;0;3] = BASEN 16 [3]

#ONCE_BASEN_NORMALIZE_CONV "BASEN 8 [7;7;5;3;4]";;
|- BASEN 8 [7;7;5;3;4] = BASEN 8 [7;7;5;3;4]
\end{verbatim}
}
\SEEALSO
BASEN_NORMALIZE_OR_FAIL_CONV, IS_BASEN_CONV, IS_NORMALIZED_CONV.

\ENDDOC
\DOC{sanity\_test}

{\small
\begin{verbatim}
sanity_test : ((* -> **) ->
string ->
(* -> ** -> *** -> bool) ->
(string # * # ***) list ->
string)
\end{verbatim}
}\egroup

\SYNOPSIS
Test a function to ensure that it returns the expected values.

\DESCRIBE
Its inputs are:
  f             a function to test
  f\_name        the name of that function as a string
  relation      a function that compares the input, the actual output, and
                the expected output and returns true if the actual output
                was correct
  test\_cases    a list of triples, each of which is a test case, consisting
                of a string that identifies the test case and is to be printed
                out in case of failure, an input, and an expected output.
If the function signals failure when given the input on a test case, that test
case is considered to have failed.  If all test cases pass, the output is nil.
If any test case fails, a list is printed of the identification strings of all
failing test cases, and then the whole function fails.

\FAILURE
None.

\EXAMPLE
{\par\samepage\setseps\small
\begin{verbatim}
#let increment x = x + 1;;
increment = - : (int -> int)

#sanity_test increment `increment` relation [
        `test 1: increment zero`,
                0,      1;
        `test 2: increment one`,
                1,      3;
        `test 3: increment a relatively small number`,
                20,     21;
        `test 4: increment causing a carry`,
                9,      13;
        `test 5: increment a large number`,
                999999, 1000000 ]
where relation input actual expected =
  actual = expected;;

Failures testing increment --
  test 2: increment one     test 4: increment causing a carry
evaluation failed     sanity test
\end{verbatim}
}
\COMMENTS
{\small\verb%sanity_test%} cannot be used to test the failure behaviour of a function.  It
should be replaced by something that can.

\ENDDOC
