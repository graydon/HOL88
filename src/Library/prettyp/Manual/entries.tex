\chapter{ML Functions in the prettyp Library}
This chapter provides documentation on all the \ML\ functions that are made
available in \HOL\ when the \ml{more\_arithmetic} library is loaded.  This
documentation is also available online via the \ml{help} facility.



\DOC{Address}

\TYPE {\small\verb%Address : (int list -> address)%}\egroup

\SYNOPSIS
Type constructor for sub-tree addresses.

\DESCRIBE
{\small\verb%Address il%} denotes the address of a sub-tree within a tree. The integer
list {\small\verb%il%} is the path that has to be followed from the root node of the tree
in order to reach the sub-tree.

\FAILURE
Never fails.

\EXAMPLE
The ML value {\small\verb%Address [1;2]%} is the address within the tree:
{\par\samepage\setseps\small
\begin{verbatim}
         a
        / \
       b   c
      / \   \
     d   e   f
        / \
       g   h
\end{verbatim}
}
\noindent of the sub-tree:
{\par\samepage\setseps\small
\begin{verbatim}
         e   
        / \
       g   h
\end{verbatim}
}
\noindent The sub-tree is the second child of the first child of the main tree.

\USES
Sub-tree addresses are maintained as far as possible during the pretty-printing
process. They can thus be used to determine which sub-tree of the original
parse-tree was used to generate some specified part of the pretty-printed
text.

\SEEALSO
No_address.

\ENDDOC
\DOC{All\_types}

\TYPE {\small\verb%All_types : type_selection%}\egroup

\SYNOPSIS
Value used to control the amount of type information included in the print-tree
of a term.

\DESCRIBE
{\small\verb%All_types%} is a value used to instruct the term-to-print-tree conversion
function to include type information in the tree for every variable and
constant in the term.

\SEEALSO
No_types, Hidden_types, Useful_types, term_to_print_tree.

\ENDDOC
\DOC{apply0}

\TYPE {\small\verb%apply0 : (* -> (string # int) list -> print_binding -> *)%}\egroup

\SYNOPSIS
Function for constructing environment accessing functions.

\DESCRIBE
{\small\verb%apply0%} is a higher-order function which can be used to simplify the ML code
required for user-defined pretty-printer environment accessing functions.
Instead of having to mention the parameter list and binding explicitly as in:
{\par\samepage\setseps\small
\begin{verbatim}
   \params. \pbind. f
\end{verbatim}
}
\noindent one can use {\small\verb%apply0%}:
{\par\samepage\setseps\small
\begin{verbatim}
   apply0 f
\end{verbatim}
}
\FAILURE
Cannot fail when given only one argument. However, the resulting function may
fail. This will depend on the value of the argument.

\EXAMPLE
A function for testing whether the parameter `{\small\verb%test%}' has value 1 can be
written as:
{\par\samepage\setseps\small
\begin{verbatim}
   apply2 (curry $=) (bound_number `test`) (apply0 1)
\end{verbatim}
}
\noindent instead of:
{\par\samepage\setseps\small
\begin{verbatim}
   \params. \pbind. (bound_number `test` params pbind) = 1
\end{verbatim}
}
\noindent In this example it is not clear that use of {\small\verb%apply0%} and {\small\verb%apply2%} is
beneficial. However, it illustrates their usage.

\SEEALSO
apply1, apply2.

\ENDDOC
\DOC{apply1}

{\small
\begin{verbatim}
apply1 : ((* -> **) ->
          ((string # int) list -> print_binding -> *) ->
          ((string # int) list -> print_binding -> **))
\end{verbatim}
}\egroup

\SYNOPSIS
Function for constructing environment accessing functions.

\DESCRIBE
{\small\verb%apply1%} is a higher-order function which can be used to simplify the ML code
required for user-defined pretty-printer environment accessing functions.
Instead of having to mention the parameter list and binding explicitly as in:
{\par\samepage\setseps\small
\begin{verbatim}
   \params. \pbind. f (val params pbind)
\end{verbatim}
}
\noindent one can use {\small\verb%apply1%}:
{\par\samepage\setseps\small
\begin{verbatim}
   apply1 f val
\end{verbatim}
}
\FAILURE
Cannot fail when given no more than two arguments. However, the resulting
function may fail. This will depend on the values of the arguments.

\EXAMPLE
Suppose a function is required which evaluates the length of the node-name
bound to the metavariable {\small\verb%***x%}. The ML code for this is:
{\par\samepage\setseps\small
\begin{verbatim}
   \params. \pbind. (length o explode) (bound_name `x` params pbind)
\end{verbatim}
}
\noindent The function takes a parameter list and a binding as arguments. It
uses these to find the node-name bound to the metavariable with name `{\small\verb%x%}'. The
resulting string is then exploded into a list of single character strings and
the length of this list is computed. Using {\small\verb%apply1%}, the code can be written
more simply as:
{\par\samepage\setseps\small
\begin{verbatim}
   apply1 (length o explode) (bound_name `x`)
\end{verbatim}
}
\SEEALSO
apply0, apply2.

\ENDDOC
\DOC{apply2}

{\small
\begin{verbatim}
apply2 : ((* -> ** -> ***) ->
          ((string # int) list -> print_binding -> *) ->
          ((string # int) list -> print_binding -> **) ->
          ((string # int) list -> print_binding -> ***))
\end{verbatim}
}\egroup

\SYNOPSIS
Function for constructing environment accessing functions.

\DESCRIBE
{\small\verb%apply2%} is a higher-order function which can be used to simplify the ML code
required for user-defined pretty-printer environment accessing functions.
Instead of having to mention the parameter list and binding explicitly as in:
{\par\samepage\setseps\small
\begin{verbatim}
   \params. \pbind. f (val1 params pbind) (val2 params pbind)
\end{verbatim}
}
\noindent one can use {\small\verb%apply2%}:
{\par\samepage\setseps\small
\begin{verbatim}
   apply2 f val1 val2
\end{verbatim}
}
\FAILURE
Cannot fail when given no more than three arguments. However, the resulting
function may fail. This will depend on the values of the arguments.

\EXAMPLE
A function for testing whether the parameter `{\small\verb%test%}' has value 1 can be
written as:
{\par\samepage\setseps\small
\begin{verbatim}
   apply2 (curry $=) (bound_number `test`) (\params. \pbind. 1)
\end{verbatim}
}
\noindent instead of:
{\par\samepage\setseps\small
\begin{verbatim}
   \params. \pbind. (bound_number `test` params pbind) = 1
\end{verbatim}
}
\noindent In this example it is not clear that use of {\small\verb%apply2%} is beneficial.
However, it illustrates its usage.

\SEEALSO
apply0, apply1.

\ENDDOC
\DOC{bound\_child}

{\small
\begin{verbatim}
bound_child : (string ->
               (string # int) list ->
               print_binding ->
               print_tree)
\end{verbatim}
}\egroup

\SYNOPSIS
Obtains the print-tree bound to a pretty-printer metavariable.

\DESCRIBE
{\small\verb%bound_child%} can be used to obtain the data item bound to a named
metavariable. It takes the name of a metavariable (less the preceding {\small\verb%*%},
{\small\verb%**%}, or {\small\verb%***%}) as its first argument and returns a function of type:
{\par\samepage\setseps\small
\begin{verbatim}
   (string # int) list -> print_binding -> print_tree
\end{verbatim}
}
\noindent When given the current environment as arguments, this function
yields the print-tree bound to the specified metavariable. The parameter list
is not used, but is present for consistency.

\FAILURE
The function fails if the specified metavariable is not bound to a print-tree.
It also fails if the metavariable cannot be found in the binding.

\SEEALSO
bound_children, bound_name, bound_names, is_a_member_of, bound_number,
bound_context.

\ENDDOC
\DOC{bound\_children}

{\small
\begin{verbatim}
bound_children : (string ->
                  (string # int) list ->
                  print_binding ->
                  print_tree list)
\end{verbatim}
}\egroup

\SYNOPSIS
Obtains the print-trees bound to a pretty-printer metavariable.

\DESCRIBE
{\small\verb%bound_children%} can be used to obtain the data item bound to a named
metavariable. It takes the name of a metavariable (less the preceding {\small\verb%*%},
{\small\verb%**%}, or {\small\verb%***%}) as its first argument and returns a function of type:
{\par\samepage\setseps\small
\begin{verbatim}
   (string # int) list -> print_binding -> print_tree list
\end{verbatim}
}
\noindent When given the current environment as arguments, this function yields
the list of print-trees bound to the specified metavariable. The parameter
list is not used, but is present for consistency.

\FAILURE
The function fails if the specified metavariable is not bound to a list of
print-trees. It also fails if the metavariable cannot be found in the binding.

\SEEALSO
bound_child, bound_name, bound_names, is_a_member_of, bound_number,
bound_context.

\ENDDOC
\DOC{bound\_context}

\TYPE {\small\verb%bound_context : ((string # int) list -> print_binding -> string)%}\egroup

\SYNOPSIS
Obtains the value of the current context from the pretty-printer environment.

\DESCRIBE
To make it easier to extract the value of the current context from its rather
contorted representation in the parameter list, there is a function called
{\small\verb%bound_context%}. When presented with the current environment by way of its
arguments, {\small\verb%bound_context%} returns the character string representing the
current context. The binding is not used, but is present for consistency.

\FAILURE
The function will not fail unless it is given an invalid parameter list.

\SEEALSO
is_a_member_of, bound_name, bound_names, bound_child, bound_children,
bound_number.

\ENDDOC
\DOC{bound\_name}

\TYPE {\small\verb%bound_name : (string -> (string # int) list -> print_binding -> string)%}\egroup

\SYNOPSIS
Obtains the node-name bound to a pretty-printer metavariable.

\DESCRIBE
{\small\verb%bound_name%} can be used to obtain the data item bound to a named metavariable.
It takes the name of a metavariable (less the preceding {\small\verb%*%}, {\small\verb%**%}, or {\small\verb%***%}) as
its first argument and returns a function of type:
{\par\samepage\setseps\small
\begin{verbatim}
   (string # int) list -> print_binding -> string
\end{verbatim}
}
\noindent When given the current environment as arguments, this function yields
the node-name bound to the specified metavariable. The parameter list is not
used, but is present for consistency.

\FAILURE
The function fails if the specified metavariable is not bound to a single
node-name. It also fails if the metavariable cannot be found in the binding.

\SEEALSO
bound_names, bound_child, bound_children, is_a_member_of, bound_number,
bound_context.

\ENDDOC
\DOC{bound\_names}

{\small
\begin{verbatim}
bound_names : (string ->
               (string # int) list ->
               print_binding ->
               string list)
\end{verbatim}
}\egroup

\SYNOPSIS
Obtains the node-names bound to a pretty-printer metavariable.

\DESCRIBE
{\small\verb%bound_names%} can be used to obtain the data item bound to a named
metavariable. It takes the name of a metavariable (less the preceding {\small\verb%*%},
{\small\verb%**%}, or {\small\verb%***%}) as its first argument and returns a function of type:
{\par\samepage\setseps\small
\begin{verbatim}
   (string # int) list -> print_binding -> string list
\end{verbatim}
}
\noindent When given the current environment as arguments, this function yields
the list of node-names bound to the specified metavariable. The parameter list
is not used, but is present for consistency.

\FAILURE
The function fails if the specified metavariable is not bound to a list of
node-names. It also fails if the metavariable cannot be found in the binding.

\SEEALSO
bound_name, bound_child, bound_children, is_a_member_of, bound_number,
bound_context.

\ENDDOC
\DOC{bound\_number}

\TYPE {\small\verb%bound_number : (string -> print_int_exp)%}\egroup

\SYNOPSIS
Obtains the value bound to a pretty-printer parameter.

\DESCRIBE
{\small\verb%bound_number%} takes the name of a pretty-printer parameter as its first
argument (a string) and returns a function of type:
{\par\samepage\setseps\small
\begin{verbatim}
   (string # int) list -> print_binding -> int
\end{verbatim}
}
\noindent This function yields the integer value associated with the parameter,
when it is presented with an environment via its two arguments. The binding is
not used, but is present for consistency.

\FAILURE
The function fails if the parameter is not present in the parameter list.

\SEEALSO
is_a_member_of, bound_name, bound_names, bound_child, bound_children,
bound_context.

\ENDDOC
\DOC{get\_margin}

\TYPE {\small\verb%get_margin : (void -> int)%}\egroup

\SYNOPSIS
Returns the limit on the width of the output produced by the standard HOL
pretty-printer.

\FAILURE
Never fails.

\EXAMPLE
{\par\samepage\setseps\small
\begin{verbatim}
#get_margin ();;
72 : int
\end{verbatim}
}
\SEEALSO
set_margin.

\ENDDOC
\DOC{Hidden\_types}

\TYPE {\small\verb%Hidden_types : type_selection%}\egroup

\SYNOPSIS
Value used to control the amount of type information included in the print-tree
of a term.

\DESCRIBE
{\small\verb%Hidden_types%} is a value used to instruct the term-to-print-tree conversion
function as to how much type information to include in the tree. Type
information is only included for variables which, although free, without type
information appear to be bound. An example of such a variable is {\small\verb%"x:num"%} in
the term:
{\par\samepage\setseps\small
\begin{verbatim}
   "\(x:bool). (x:num)"
\end{verbatim}
}
\noindent Without types, this term appears as {\small\verb%"\x. x"%}. However, the two
occurrences of {\small\verb%x%} are different.

\SEEALSO
No_types, Useful_types, All_types, term_to_print_tree.

\ENDDOC
\DOC{hol\_rules\_fun}

\TYPE {\small\verb%hol_rules_fun : print_rule_function%}\egroup

\SYNOPSIS
Pretty-printing rules (as a function) for HOL types, terms and theorems.

\FAILURE
Fails if none of the rules match the input. However, this function should not
be applied `by hand'; it should only be used as an argument to one of the
pretty-printing functions.

\SEEALSO
hol_type_rules_fun, hol_term_rules_fun, hol_thm_rules_fun, raw_tree_rules_fun,
then_try, pretty_print, pp, pp_write.

\ENDDOC
\DOC{hol\_term\_rules\_fun}

\TYPE {\small\verb%hol_term_rules_fun : print_rule_function%}\egroup

\SYNOPSIS
Pretty-printing rules (as a function) for HOL terms. {\small\verb%hol_type_rules_fun%} is
required for {\small\verb%hol_term_rules_fun%} to function correctly.

\FAILURE
Fails if none of the rules match the input. However, this function should not
be applied `by hand'; it should only be used as an argument to one of the
pretty-printing functions.

\SEEALSO
hol_type_rules_fun, hol_thm_rules_fun, hol_rules_fun, raw_tree_rules_fun,
then_try, pretty_print, pp, pp_write.

\ENDDOC
\DOC{hol\_thm\_rules\_fun}

\TYPE {\small\verb%hol_thm_rules_fun : print_rule_function%}\egroup

\SYNOPSIS
Pretty-printing rules (as a function) for HOL theorems. The rules
{\small\verb%hol_type_rules_fun%} and {\small\verb%hol_term_rules_fun%} are required for
{\small\verb%hol_thm_rules_fun%} to function correctly.

\FAILURE
Fails if none of the rules match the input. However, this function should not
be applied `by hand'; it should only be used as an argument to one of the
pretty-printing functions.

\SEEALSO
hol_type_rules_fun, hol_term_rules_fun, hol_rules_fun, raw_tree_rules_fun,
then_try, pretty_print, pp, pp_write.

\ENDDOC
\DOC{hol\_type\_rules\_fun}

\TYPE {\small\verb%hol_type_rules_fun : print_rule_function%}\egroup

\SYNOPSIS
Pretty-printing rules (as a function) for HOL types.

\FAILURE
Fails if none of the rules match the input. However, this function should not
be applied `by hand'; it should only be used as an argument to one of the
pretty-printing functions.

\SEEALSO
hol_term_rules_fun, hol_thm_rules_fun, hol_rules_fun, raw_tree_rules_fun,
then_try, pretty_print, pp, pp_write.

\ENDDOC
\DOC{is\_a\_member\_of}

\TYPE {\small\verb%$is_a_member_of : (string -> string list -> print_test)%}\egroup

\SYNOPSIS
Function for testing a node-name metavariable in a pretty-printing rule.

\DESCRIBE
{\small\verb%is_a_member_of%} forms a {\small\verb%print_test%} which yields {\small\verb%true%} only if the
metavariable whose name is the first argument to {\small\verb%is_a_member_of%} is bound to
a node-name which appears in the second argument. This evaluation to a Boolean
value is only performed when the {\small\verb%print_test%} is applied to a parameter list
and a binding.

\FAILURE
The function fails if the metavariable named is bound to anything other than
a single node-name.

\EXAMPLE
An example of the use of this function is the rule:
{\par\samepage\setseps\small
\begin{verbatim}
   ''::***node(*,*) where
          {`node` is_a_member_of [`plus`;`minus`;`mult`;`div`]} ->
       [<h 0> ***node];
\end{verbatim}
}
\SEEALSO
bound_number, bound_name, bound_names, bound_child, bound_children,
bound_context.

\ENDDOC
\DOC{max\_term\_prec}

\TYPE {\small\verb%max_term_prec : int%}\egroup

\SYNOPSIS
Lowest precedence (maximum value) used by the pretty-printer for HOL function
constants and syntactic constructs in terms.

\SEEALSO
min_term_prec, term_prec.

\ENDDOC
\DOC{max\_type\_prec}

\TYPE {\small\verb%max_type_prec : int%}\egroup

\SYNOPSIS
Lowest precedence (maximum value) used by the pretty-printer for HOL type
operators.

\SEEALSO
min_type_prec, type_prec.

\ENDDOC
\DOC{min\_term\_prec}

\TYPE {\small\verb%min_term_prec : int%}\egroup

\SYNOPSIS
Highest precedence (minimum value) used by the pretty-printer for HOL function
constants and syntactic constructs in terms.

\SEEALSO
max_term_prec, term_prec.

\ENDDOC
\DOC{min\_type\_prec}

\TYPE {\small\verb%min_type_prec : int%}\egroup

\SYNOPSIS
Highest precedence (minimum value) used by the pretty-printer for HOL type
operators.

\SEEALSO
max_type_prec, type_prec.

\ENDDOC
\DOC{new\_child}

{\small
\begin{verbatim}
new_child : ((print_tree -> print_tree) ->
             string ->
             (string # int) list ->
             print_binding ->
             metavar_binding)
\end{verbatim}
}\egroup

\SYNOPSIS
Function for transforming a print-tree bound to a metavariable.

\DESCRIBE
Within the metavariable transformation part of a pretty-printing rule, a
typical requirement is to `declare' a new metavariable to be bound to the
result of performing a transformation on a single existing metavariable. The
type of function required is:
{\par\samepage\setseps\small
\begin{verbatim}
   (string # int) list -> print_binding -> metavar_binding
\end{verbatim}
}
\noindent There are four functions available to facilitate this, corresponding
to the four different types of data which can be bound to a metavariable.
{\small\verb%new_child%} is the function for use when the data is a single print-tree.

The first argument is the transformation function. The second argument is the
name of the metavariable which is bound to the value to be transformed. When
provided with these arguments and a pretty-printer environment, {\small\verb%new_child%}
extracts the item bound to the named metavariable and then applies the
transformation function to it. The result is then made into a form suitable
for binding to a metavariable, that is it is made into an object of type
{\small\verb%metavar_binding%}.

\FAILURE
{\small\verb%new_child%} fails if the named metavariable does not exist or is bound to an
item of the wrong type.

\SEEALSO
new_children, new_name, new_names, bound_child.

\ENDDOC
\DOC{new\_children}

{\small
\begin{verbatim}
new_children :
   (((print_tree # address) list -> (print_tree # address) list) ->
    string ->
    (string # int) list ->
    print_binding ->
    metavar_binding)
\end{verbatim}
}\egroup

\SYNOPSIS
Function for transforming a list of print-trees bound to a metavariable.

\DESCRIBE
Within the metavariable transformation part of a pretty-printing rule, a
typical requirement is to `declare' a new metavariable to be bound to the
result of performing a transformation on a single existing metavariable. The
type of function required is:
{\par\samepage\setseps\small
\begin{verbatim}
   (string # int) list -> print_binding -> metavar_binding
\end{verbatim}
}
\noindent There are four functions available to facilitate this, corresponding
to the four different types of data which can be bound to a metavariable.
{\small\verb%new_children%} is the function for use when the data is a list of print-trees.

The first argument is the transformation function. The second argument is the
name of the metavariable which is bound to the value to be transformed. When
provided with these arguments and a pretty-printer environment, {\small\verb%new_children%}
extracts the item bound to the named metavariable and then applies the
transformation function to it. The result is then made into a form suitable
for binding to a metavariable, that is it is made into an object of type
{\small\verb%metavar_binding%}.

Note that the transformation function has to deal with sub-tree addresses in
addition to the print-trees. If the transformation function is polymorphic, as
is for example a function to reverse the list, this will not cause any
difficulties. The addresses have to be dealt with by the transformation
function because the system cannot know how to re-assign addresses to the
values in the result list.

\FAILURE
{\small\verb%new_children%} fails if the named metavariable does not exist or is bound to an
item of the wrong type.

\SEEALSO
new_child, new_name, new_names, bound_children.

\ENDDOC
\DOC{new\_name}

{\small
\begin{verbatim}
new_name : ((string -> string) ->
            string ->
            (string # int) list ->
            print_binding ->
            metavar_binding)
\end{verbatim}
}\egroup

\SYNOPSIS
Function for transforming a node-name bound to a metavariable.

\DESCRIBE
Within the metavariable transformation part of a pretty-printing rule, a
typical requirement is to `declare' a new metavariable to be bound to the
result of performing a transformation on a single existing metavariable. The
type of function required is:
{\par\samepage\setseps\small
\begin{verbatim}
   (string # int) list -> print_binding -> metavar_binding
\end{verbatim}
}
\noindent There are four functions available to facilitate this, corresponding
to the four different types of data which can be bound to a metavariable.
{\small\verb%new_name%} is the function for use when the data is a single node-name.

The first argument is the transformation function. The second argument is the
name of the metavariable which is bound to the value to be transformed. When
provided with these arguments and a pretty-printer environment, {\small\verb%new_name%}
extracts the item bound to the named metavariable and then applies the
transformation function to it. The result is then made into a form suitable
for binding to a metavariable, that is it is made into an object of type
{\small\verb%metavar_binding%}.

\FAILURE
{\small\verb%new_name%} fails if the named metavariable does not exist or is bound to an
item of the wrong type.

\SEEALSO
new_names, new_child, new_children, bound_name.

\ENDDOC
\DOC{new\_names}

{\small
\begin{verbatim}
new_names : (((string # address) list -> (string # address) list) ->
             string ->
             (string # int) list ->
             print_binding ->
             metavar_binding)
\end{verbatim}
}\egroup

\SYNOPSIS
Function for transforming a list of node-names bound to a metavariable.

\DESCRIBE
Within the metavariable transformation part of a pretty-printing rule, a
typical requirement is to `declare' a new metavariable to be bound to the
result of performing a transformation on a single existing metavariable. The
type of function required is:
{\par\samepage\setseps\small
\begin{verbatim}
   (string # int) list -> print_binding -> metavar_binding
\end{verbatim}
}
\noindent There are four functions available to facilitate this, corresponding
to the four different types of data which can be bound to a metavariable.
{\small\verb%new_names%} is the function for use when the data is a list of node-names.

The first argument is the transformation function. The second argument is the
name of the metavariable which is bound to the value to be transformed. When
provided with these arguments and a pretty-printer environment, {\small\verb%new_names%}
extracts the item bound to the named metavariable and then applies the
transformation function to it. The result is then made into a form suitable
for binding to a metavariable, that is it is made into an object of type
{\small\verb%metavar_binding%}.

Note that the transformation function has to deal with sub-tree addresses in
addition to the node-names. If the transformation function is polymorphic, as
is for example a function to reverse the list, this will not cause any
difficulties. The addresses have to be dealt with by the transformation
function because the system cannot know how to re-assign addresses to the
values in the result list.

\FAILURE
{\small\verb%new_names%} fails if the named metavariable does not exist or is bound to an
item of the wrong type.

\SEEALSO
new_name, new_child, new_children, bound_names.

\ENDDOC
\DOC{No\_address}

\TYPE {\small\verb%No_address : address%}\egroup

\SYNOPSIS
Type constructor for sub-tree addresses.

\DESCRIBE
{\small\verb%No_address%} is used to indicate that there is no valid address information
for a sub-tree.

\SEEALSO
Address.

\ENDDOC
\DOC{No\_types}

\TYPE {\small\verb%No_types : type_selection%}\egroup

\SYNOPSIS
Value used to control the amount of type information included in the print-tree
of a term.

\DESCRIBE
{\small\verb%No_types%} is a value used to instruct the term-to-print-tree conversion
function to include no type information in the tree.

\SEEALSO
Hidden_types, Useful_types, All_types, term_to_print_tree.

\ENDDOC
\DOC{pp}

{\small
\begin{verbatim}
pp : (print_rule_function -> string -> (string # int) list ->
      print_tree -> void)
\end{verbatim}
}\egroup

\SYNOPSIS
One of the main pretty-printing functions. For use with the standard HOL
pretty-printer.

\DESCRIBE
{\small\verb%pp%} invokes the pretty-printer. It can be used for merging output with text
produced by the standard HOL pretty-printer. Instead of ending each line of
text by printing a new-line, it sends its output to the standard HOL printer
in the form of a pretty-printing block. The arguments to the function are:
(1) pretty-printing rules expressed as a function, (2) the initial context,
(3) initial parameters, (4) tree to be printed. {\small\verb%pp%} uses as its maximum width
the width for the standard HOL printer, as specified by the function
{\small\verb%set_margin%}. The initial offset from the left margin is taken to be zero.

\FAILURE
Failure or incorrect behaviour can be caused by mistakes in the pretty-printing
rules or by inappropriate arguments to the printing function. The most common
errors are use of uninitialised parameters and reference to unknown
metavariables. The latter are due to metavariables appearing in the format of a
rule, but not in the pattern. Errors also occur if a metavariable is used in a
place inappropriate for the value it is bound to. An example of this is an
attempt to compare a string with a metavariable that is bound to a tree rather
than a node-name.

Use of negative indentations in formats may cause text to overflow the left
margin, and an exception to be raised. Any user defined function may also
cause a run-time error.

The printing functions have been designed to trap exceptions and to print
{\small\verb%*error*%}. This does not indicate what caused the error, but it may give some
indication of where the error occurred. However, this is not the main reason
for trapping exceptions. The ML directive {\small\verb%top_print%} installs a user print
function. If an exception is raised within this function, it does not appear
at the top-level of ML. Instead, an obscure Lisp error is produced. Since the
pretty-printing functions are normally used with {\small\verb%top_print%}, it is best to
avoid raising exceptions. For this reason the printing functions display
{\small\verb%*error*%} instead.

\SEEALSO
pretty_print, pp_write.

\ENDDOC
\DOC{pp\_convert\_all\_thm}

\TYPE {\small\verb%pp_convert_all_thm : (thm -> print_tree)%}\egroup

\SYNOPSIS
Function for converting a HOL theorem into a print-tree.

\DESCRIBE
{\small\verb%pp_convert_all_thm%} converts a theorem into a print-tree. The hypotheses
(assumptions) of the theorem are included in the print-tree. Instances of the
HOL constant {\small\verb%UNCURRY%} in the theorem are converted into an appropriate use of
ordered pairs in the print-tree. The amount of type information included in
the print-tree is determined by the value of the HOL system flag {\small\verb%show_types%}.
If {\small\verb%show_types%} is {\small\verb%true%}, then `useful' types are included in the print-tree.
Otherwise, only `hidden' types are included.

`Useful' type information is type information on the bound variables of
abstractions and on one occurrence of every free variable. Type information is
only included for constants if the constant is a function and it is not fully
applied.

`Hidden' types are rare. They only occur on variables which, although free,
without type information appear to be bound.

\FAILURE
Never fails.

\SEEALSO
pp_convert_thm, pp_convert_type, pp_convert_term, thm_to_print_tree.

\ENDDOC
\DOC{pp\_convert\_term}

\TYPE {\small\verb%pp_convert_term : (term -> print_tree)%}\egroup

\SYNOPSIS
Function for converting a HOL term into a print-tree.

\DESCRIBE
{\small\verb%pp_convert_term%} converts a term into a print-tree. Instances of the HOL
constant {\small\verb%UNCURRY%} in the term are converted into an appropriate use of
ordered pairs in the print-tree. The amount of type information included in the
print-tree is determined by the value of the HOL system flag {\small\verb%show_types%}.
If {\small\verb%show_types%} is {\small\verb%true%}, then `useful' types are included in the print-tree.
Otherwise, only `hidden' types are included.

`Useful' type information is type information on the bound variables of
abstractions and on one occurrence of every free variable. Type information is
only included for constants if the constant is a function and it is not fully
applied.

`Hidden' types are rare. They only occur on variables which, although free,
without type information appear to be bound.

\FAILURE
Never fails.

\SEEALSO
pp_convert_type, pp_convert_thm, pp_convert_all_thm, term_to_print_tree.

\ENDDOC
\DOC{pp\_convert\_thm}

\TYPE {\small\verb%pp_convert_thm : (thm -> print_tree)%}\egroup

\SYNOPSIS
Function for converting a HOL theorem into a print-tree.

\DESCRIBE
{\small\verb%pp_convert_thm%} converts a theorem into a print-tree. The hypotheses
(assumptions) of the theorem are not included in the print-tree. Instances of
the HOL constant {\small\verb%UNCURRY%} in the theorem are converted into an appropriate use
of ordered pairs in the print-tree. The amount of type information included in
the print-tree is determined by the value of the HOL system flag {\small\verb%show_types%}.
If {\small\verb%show_types%} is {\small\verb%true%}, then `useful' types are included in the print-tree.
Otherwise, only `hidden' types are included.

`Useful' type information is type information on the bound variables of
abstractions and on one occurrence of every free variable. Type information is
only included for constants if the constant is a function and it is not fully
applied.

`Hidden' types are rare. They only occur on variables which, although free,
without type information appear to be bound.

\FAILURE
Never fails.

\SEEALSO
pp_convert_all_thm, pp_convert_type, pp_convert_term, thm_to_print_tree.

\ENDDOC
\DOC{pp\_convert\_type}

\TYPE {\small\verb%pp_convert_type : (type -> print_tree)%}\egroup

\SYNOPSIS
Function for converting a HOL type into a print-tree.

\DESCRIBE
{\small\verb%pp_convert_type%} has an identical specification to {\small\verb%type_to_print_tree%}.

\FAILURE
Never fails.

\SEEALSO
type_to_print_tree, pp_convert_term, pp_convert_thm, pp_convert_all_thm.

\ENDDOC
\DOC{pp\_print\_all\_thm}

\TYPE {\small\verb%pp_print_all_thm : (thm -> void)%}\egroup

\SYNOPSIS
Print function for HOL theorems. Simulates the HOL system function
{\small\verb%print_all_thm%}.

\FAILURE
Never fails.

\SEEALSO
pp_print_thm, pp_print_type, pp_print_term.

\ENDDOC
\DOC{pp\_print\_term}

\TYPE {\small\verb%pp_print_term : (term -> void)%}\egroup

\SYNOPSIS
Print function for HOL terms. Simulates the HOL system function {\small\verb%print_term%}.

\FAILURE
Never fails.

\SEEALSO
pp_print_type, pp_print_thm, pp_print_all_thm.

\ENDDOC
\DOC{pp\_print\_theory}

\TYPE {\small\verb%pp_print_theory : (string -> void)%}\egroup

\SYNOPSIS
Print function for HOL theories.

\DESCRIBE
{\small\verb%pp_print_theory%} simulates the HOL system function {\small\verb%print_theory%} using the
pretty-printer library. The function takes a theory-segment name as argument.
The following information is displayed: the parents of the theory, types
defined within the theory, constants of the theory, the binders and infixes
(subsets of the constants), the axioms, the definitions, and the derived
theorems.

\FAILURE
Fails if the named theory does not exist or is not an ancestor of the current
theory.

\SEEALSO
pp_print_type, pp_print_term, pp_print_thm, pp_print_all_thm.

\ENDDOC
\DOC{pp\_print\_thm}

\TYPE {\small\verb%pp_print_thm : (thm -> void)%}\egroup

\SYNOPSIS
Print function for HOL theorems. Simulates the HOL system function {\small\verb%print_thm%}.

\FAILURE
Never fails.

\SEEALSO
pp_print_all_thm, pp_print_type, pp_print_term.

\ENDDOC
\DOC{pp\_print\_type}

\TYPE {\small\verb%pp_print_type : (type -> void)%}\egroup

\SYNOPSIS
Print function for HOL types. Simulates the HOL system function {\small\verb%print_type%}.

\FAILURE
Never fails.

\SEEALSO
pp_print_term, pp_print_thm, pp_print_all_thm.

\ENDDOC
\DOC{PP\_to\_ML}

\TYPE {\small\verb%PP_to_ML : (bool -> string -> string -> void)%}\egroup

\SYNOPSIS
Function to compile pretty-printing rules into ML datastructures.

\DESCRIBE
The function {\small\verb%PP_to_ML%} invokes the parser for the pretty-printing language.
Its first argument indicates whether or not the output is to be appended to the
destination file. If the argument is {\small\verb%false%} and the destination file existed
previously, the file is overwritten. The second and third arguments specify
the names of the source and destination files respectively. For example, the
ML function call:
{\par\samepage\setseps\small
\begin{verbatim}
   PP_to_ML false `xxxx.pp` ``;;
\end{verbatim}
}
\noindent compiles the file {\small\verb%xxxx.pp%} to a file called {\small\verb%xxxx_pp.ml%},
overwriting any previous version.

The `{\small\verb%.pp%}' extension can be omitted. So, the following has precisely the same
effect as the previous `command':
{\par\samepage\setseps\small
\begin{verbatim}
   PP_to_ML false `xxxx` ``;;
\end{verbatim}
}
\noindent If the last argument is anything other than the empty string, it is
used as the name of the destination file. So,
{\par\samepage\setseps\small
\begin{verbatim}
   PP_to_ML false `xxxx` `test.ml`;;
\end{verbatim}
}
\noindent compiles the file {\small\verb%xxxx.pp%} to the file {\small\verb%test.ml%}.

\FAILURE
The compiler may fail to parse the source code. In this case the error message
specifies the kind of symbol the compiler was expecting and the kind of symbol
it received. In addition, the compiler displays a few lines of the source file
following the point at which the failure occurred. This should facilitate the
location of the fault.

The second kind of error occurs after the parse has completed successfully. At
this point the compiler is converting the parse-tree into ML. Faults at this
point are due to additional restrictions not being met, and the error messages
are correspondingly ad hoc. The part of the parse-tree under conversion is
printed in the pretty-printing language. This may or may not be helpful,
depending on the size of the tree.

\ENDDOC
\DOC{pp\_write}

{\small
\begin{verbatim}
pp_write : (string -> int -> int -> print_rule_function -> string ->
            (string # int) list -> print_tree -> void)
\end{verbatim}
}\egroup

\SYNOPSIS
One of the main pretty-printing functions. Function for printing to files.

\DESCRIBE
{\small\verb%pp_write%} invokes the pretty-printer. The arguments to this function are:
(1) file handle (port) of the file to be written to,
(2) maximum width of output permitted, (3) initial offset from left margin,
(4) pretty-printing rules expressed as a function, (5) the initial context,
(6) initial parameters, (7) tree to be printed.

\FAILURE
Failure or incorrect behaviour can be caused by mistakes in the pretty-printing
rules or by inappropriate arguments to the printing function. The most common
errors are use of uninitialised parameters and reference to unknown
metavariables. The latter are due to metavariables appearing in the format of a
rule, but not in the pattern. Errors also occur if a metavariable is used in a
place inappropriate for the value it is bound to. An example of this is an
attempt to compare a string with a metavariable that is bound to a tree rather
than a node-name.

Use of negative indentations in formats may cause text to overflow the left
margin, and an exception to be raised. Any user defined function may also
cause a run-time error.

The printing functions have been designed to trap exceptions and to print
{\small\verb%*error*%}. This does not indicate what caused the error, but it may give some
indication of where the error occurred. However, this is not the main reason
for trapping exceptions. The ML directive {\small\verb%top_print%} installs a user print
function. If an exception is raised within this function, it does not appear
at the top-level of ML. Instead, an obscure Lisp error is produced. Since the
pretty-printing functions are normally used with {\small\verb%top_print%}, it is best to
avoid raising exceptions. For this reason the printing functions display
{\small\verb%*error*%} instead.

\SEEALSO
pretty_print, pp.

\ENDDOC
\DOC{pretty\_print}

{\small
\begin{verbatim}
pretty_print : (int -> int -> print_rule_function -> string ->
                (string # int) list -> print_tree -> void)
\end{verbatim}
}\egroup

\SYNOPSIS
One of the main pretty-printing functions. This one writes directly to the
terminal, independently of the standard HOL printer.

\DESCRIBE
{\small\verb%pretty_print%} invokes the pretty-printer. The arguments to this function are:
(1) maximum width of output permitted, (2) initial offset from left margin,
(3) pretty-printing rules expressed as a function, (4) the initial context,
(5) initial parameters, (6) tree to be printed.

\FAILURE
Failure or incorrect behaviour can be caused by mistakes in the pretty-printing
rules or by inappropriate arguments to the printing function. The most common
errors are use of uninitialised parameters and reference to unknown
metavariables. The latter are due to metavariables appearing in the format of a
rule, but not in the pattern. Errors also occur if a metavariable is used in a
place inappropriate for the value it is bound to. An example of this is an
attempt to compare a string with a metavariable that is bound to a tree rather
than a node-name.

Use of negative indentations in formats may cause text to overflow the left
margin, and an exception to be raised. Any user defined function may also
cause a run-time error.

The printing functions have been designed to trap exceptions and to print
{\small\verb%*error*%}. This does not indicate what caused the error, but it may give some
indication of where the error occurred. However, this is not the main reason
for trapping exceptions. The ML directive {\small\verb%top_print%} installs a user print
function. If an exception is raised within this function, it does not appear
at the top-level of ML. Instead, an obscure Lisp error is produced. Since the
pretty-printing functions are normally used with {\small\verb%top_print%}, it is best to
avoid raising exceptions. For this reason the printing functions display
{\small\verb%*error*%} instead.

\SEEALSO
pp, pp_write.

\ENDDOC
\DOC{Print\_node}

\TYPE {\small\verb%Print_node : ((string # print_tree list) -> print_tree)%}\egroup

\SYNOPSIS
Constructor function for print-trees (parse-trees).

\DESCRIBE
{\small\verb%Print_node%} takes a node label and a list of sub-trees and uses them to
construct a new print-tree. Leaf nodes have an empty sub-tree (child) list.

\FAILURE
Never fails.

\SEEALSO
print_tree_name, print_tree_children.

\ENDDOC
\DOC{print\_tree\_children}

\TYPE {\small\verb%print_tree_children : (print_tree -> print_tree list)%}\egroup

\SYNOPSIS
Function to extract the sub-trees (children) of the root node of a print-tree.

\FAILURE
Never fails.

\SEEALSO
print_tree_name, Print_node.

\ENDDOC
\DOC{print\_tree\_name}

\TYPE {\small\verb%print_tree_name : (print_tree -> string)%}\egroup

\SYNOPSIS
Function to extract the name (label) of the root node of a print-tree.

\FAILURE
Never fails.

\SEEALSO
print_tree_children, Print_node.

\ENDDOC
\DOC{raw\_tree\_rules\_fun}

\TYPE {\small\verb%raw_tree_rules_fun : print_rule_function%}\egroup

\SYNOPSIS
Pretty-printing rules (as a function) for raw print-trees (parse-trees).

\DESCRIBE
In the event of no pretty-printing rules matching the tree to be printed, a
default set of rules are used. These rules always match, and the output
generated is a textual representation of the structure of the tree. The default
rules are available to the user as {\small\verb%raw_tree_rules_fun%}.

\FAILURE
Never fails.

\SEEALSO
hol_type_rules_fun, hol_term_rules_fun, hol_thm_rules_fun, hol_rules_fun,
then_try, pretty_print, pp, pp_write.

\ENDDOC
\DOC{term\_prec}

\TYPE {\small\verb%term_prec : (string -> int)%}\egroup

\SYNOPSIS
Precedence table for HOL terms (as a function).

\DESCRIBE
{\small\verb%term_prec%} is a function which given the name of a HOL function constant,
returns the precedence used by the pretty-printer. The precedences of
abstractions ({\small\verb%`\\`%}) and type annotations ({\small\verb%`:`%}) are also included.

\FAILURE
Never fails.

\SEEALSO
min_term_prec, max_term_prec, type_prec.

\ENDDOC
\DOC{term\_to\_print\_tree}

\TYPE {\small\verb%term_to_print_tree : (bool -> type_selection -> term -> print_tree)%}\egroup

\SYNOPSIS
Function for converting a HOL term into a print-tree.

\DESCRIBE
The first argument to {\small\verb%term_to_print_tree%} is a flag. If the flag is {\small\verb%true%},
the function converts instances of the HOL constant {\small\verb%UNCURRY%} in the term
into an appropriate use of ordered pairs in the print-tree. If the flag is
{\small\verb%false%}, {\small\verb%UNCURRY%} is treated in the same way as any other HOL constant. The
conversion is necessary because the representation of tuples of bound
variables in a HOL term is so unlike the syntax of the tuples that the
pretty-printer cannot handle them. So, normally, the flag should be set to
{\small\verb%true%}.

The second argument to {\small\verb%term_to_print_tree%} controls the amount of type
information included in the print-tree of the term. If {\small\verb%No_types%} is given as
the argument, then the print-tree will contain no type information. If
{\small\verb%All_types%} is given as the argument, the tree will contain type information
for every variable and constant. Use of {\small\verb%Useful_types%} instructs
{\small\verb%term_to_print_tree%} to attach type information to the bound variables of
abstractions, and to one occurrence of every free variable. Type information
is only included for constants if the constant is a function and it is not
fully applied. So, the equals sign in {\small\verb%"1 = 2"%} would not be adorned with type
information, but in {\small\verb%"$= 1"%} it would be.

Finally, using {\small\verb%Hidden_types%} as the second argument to {\small\verb%term_to_print_tree%}
causes type information to be attached only to variables which, although free,
without type information appear to be bound. An example of such a variable is
{\small\verb%"x:num"%} in the term:
{\par\samepage\setseps\small
\begin{verbatim}
   "\(x:bool). (x:num)"
\end{verbatim}
}
\noindent Without types, this term appears as {\small\verb%"\x. x"%}. However, the two
occurrences of {\small\verb%x%} are different.

\FAILURE
Never fails.

\EXAMPLE
{\par\samepage\setseps\small
\begin{verbatim}
#term_to_print_tree true No_types "\x. x /\ T";;
Print_node(`term`,
           [Print_node(`ABS`,
                       [Print_node(`VAR`, [Print_node(`x`, [])]);
                        Print_node(`COMB`,
                                   [Print_node(`COMB`,
                                               [Print_node(`CONST`,
                                                           [Print_node(`/\`,
                                                                       [])]);
                                                Print_node(`VAR`,
                                                           [Print_node(`x`,
                                                                       [])])]);
                                    Print_node(`CONST`,
                                               [Print_node(`T`, [])])])])])
: print_tree
\end{verbatim}
}
\SEEALSO
type_to_print_tree, thm_to_print_tree, pp_convert_term.

\ENDDOC
\DOC{then\_try}

{\small
\begin{verbatim}
$then_try : (print_rule_function ->
             print_rule_function ->
             print_rule_function)
\end{verbatim}
}\egroup

\SYNOPSIS
Function for composing print-rule functions.

\DESCRIBE
{\small\verb%then_try%} is an infix function which forms the composite of two print-rule
functions, say {\small\verb%prf1%} and {\small\verb%prf2%}. The result is a new print-rule function
which, when given a tree to match, first tries the rules of {\small\verb%prf1%}; if none of
these match, it then tries the rules of {\small\verb%prf2%}.

\FAILURE
Cannot fail when given two print-rule functions as arguments. However, the
resulting function may fail when used, with this depending on the failure
properties of the two argument functions.

\SEEALSO
hol_type_rules_fun, hol_term_rules_fun, hol_thm_rules_fun, hol_rules_fun,
raw_tree_rules_fun, pretty_print, pp, pp_write.

\ENDDOC
\DOC{thm\_to\_print\_tree}

{\small
\begin{verbatim}
thm_to_print_tree :
   (bool -> bool -> type_selection -> thm -> print_tree)
\end{verbatim}
}\egroup

\SYNOPSIS
Function for converting a HOL theorem into a print-tree.

\DESCRIBE
The first argument to {\small\verb%thm_to_print_tree%} determines whether or not the
hypotheses (assumptions) of the theorem are included in the print-tree in full.

The second argument to {\small\verb%thm_to_print_tree%} is a flag. If the flag is {\small\verb%true%},
the function converts instances of the HOL constant {\small\verb%UNCURRY%} in the theorem
into an appropriate use of ordered pairs in the print-tree. If the flag is
{\small\verb%false%}, {\small\verb%UNCURRY%} is treated in the same way as any other HOL constant. The
conversion is necessary because the representation of tuples of bound
variables in a HOL term is so unlike the syntax of the tuples that the
pretty-printer cannot handle them. So, normally, the flag should be set to
{\small\verb%true%}.

The third argument to {\small\verb%thm_to_print_tree%} controls the amount of type
information included in the print-tree of the theorem. If {\small\verb%No_types%} is given
as the argument, then the print-tree will contain no type information. If
{\small\verb%All_types%} is given as the argument, the tree will contain type information
for every variable and constant. Use of {\small\verb%Useful_types%} instructs
{\small\verb%thm_to_print_tree%} to attach type information to the bound variables of
abstractions, and to one occurrence of every free variable. Type information
is only included for constants if the constant is a function and it is not
fully applied. So, the equals sign in {\small\verb%"1 = 2"%} would not be adorned with type
information, but in {\small\verb%"$= 1"%} it would be.

Finally, using {\small\verb%Hidden_types%} as the third argument to {\small\verb%thm_to_print_tree%}
causes type information to be attached only to variables which, although free,
without type information appear to be bound. An example of such a variable is
{\small\verb%"x:num"%} in the term:
{\par\samepage\setseps\small
\begin{verbatim}
   "\(x:bool). (x:num)"
\end{verbatim}
}
\noindent Without types, this term appears as {\small\verb%"\x. x"%}. However, the two
occurrences of {\small\verb%x%} are different.

\FAILURE
Never fails.

\EXAMPLE
{\par\samepage\setseps\small
\begin{verbatim}
#thm_to_print_tree false true No_types (UNDISCH (SPEC_ALL FALSITY));;
Print_node(`thm`,
           [Print_node(`term`,
                       [Print_node(`VAR`, [Print_node(`t`, [])])]);
            Print_node(`dots`, [Print_node(`dot`, [])])])
: print_tree

#thm_to_print_tree true true No_types (UNDISCH (SPEC_ALL FALSITY));;
Print_node(`thm`,
           [Print_node(`term`,
                       [Print_node(`VAR`, [Print_node(`t`, [])])]);
            Print_node(`hyp`,
                       [Print_node(`term`,
                                   [Print_node(`CONST`,
                                               [Print_node(`F`, [])])])])])
: print_tree
\end{verbatim}
}
\SEEALSO
type_to_print_tree, term_to_print_tree, pp_convert_thm, pp_convert_all_thm.

\ENDDOC
\DOC{type\_prec}

\TYPE {\small\verb%type_prec : (string -> int)%}\egroup

\SYNOPSIS
Precedence table for HOL types (as a function).

\DESCRIBE
{\small\verb%type_prec%} is a function which given the name of a HOL type operator, returns
the precedence used by the pretty-printer. The standard infix type operators
should be referred to by {\small\verb%fun%}, {\small\verb%prod%} and {\small\verb%sum%}, rather than by the symbolic
forms.

\FAILURE
Never fails.

\SEEALSO
min_type_prec, max_type_prec, term_prec.

\ENDDOC
\DOC{type\_to\_print\_tree}

\TYPE {\small\verb%type_to_print_tree : (type -> print_tree)%}\egroup

\SYNOPSIS
Function for converting a HOL type into a print-tree.

\FAILURE
Never fails.

\EXAMPLE
{\par\samepage\setseps\small
\begin{verbatim}
#type_to_print_tree ":* -> bool";;
Print_node(`type`,
           [Print_node(`OP`,
                       [Print_node(`fun`, []);
                        Print_node(`VAR`, [Print_node(`*`, [])]);
                        Print_node(`OP`, [Print_node(`bool`, [])])])])
: print_tree
\end{verbatim}
}
\SEEALSO
term_to_print_tree, thm_to_print_tree, pp_convert_type.

\ENDDOC
\DOC{Useful\_types}

\TYPE {\small\verb%Useful_types : type_selection%}\egroup

\SYNOPSIS
Value used to control the amount of type information included in the print-tree
of a term.

\DESCRIBE
{\small\verb%Useful_types%} is a value used to instruct the term-to-print-tree conversion
function to attach type information to the bound variables of abstractions,
and to one occurrence of every free variable. Type information is only
included for constants if the constant is a function and it is not fully
applied. So, the equals sign in {\small\verb%"1 = 2"%} would not be adorned with type
information, but in {\small\verb%"$= 1"%} it would be.

\SEEALSO
No_types, Hidden_types, All_types, term_to_print_tree.

\ENDDOC
