\chapter{The trs Library}

\section{Introduction}

This document describes the facilities provided by the \ml{trs} library
for the \HOL\ system~\cite{description}.

One of the most time-consuming activities when using the \HOL\ System,
particularly for people new to the system, is discovering the names of
built-in theorems required for a proof. Typically this involves manually
searching for the theorem in the documentation.

If the user can guess roughly what the name of the theorem is, then it may be
possible to use the operating system to search for strings within theory
files, and hence locate the theorem. It would be better to search for the
theorem by specifying its structure, since the user should have a fairly good
idea of what this is. The Theorem Retrieval System provides such searching by
structure. It does this from within a \HOL\ session.

The system uses \HOL\ terms as a means of specifying the structure of the
required theorem. The terms given are interpreted as templates or patterns
for the theorem. Variables in the template term can match subterms in the
theorem. The user may only know the form of some subterm of the theorem, so
it is not sufficient just to be able to match a template against the whole of
the conclusion of a theorem. The retrieval system allows searches for theorems
containing subterms which match a pattern. Logical connectives are also
available to build up more sophisticated search specifications.

It is not just the names of built-in theorems that may not be known. A user
may have forgotten the name under which he/she stored a personal theorem.
For this reason, the retrieval system allows user theories to be searched
as well as the built-in theories.


\section{Using the library}

The \ml{trs} library can be loaded into a user's \HOL\ session using the
function \ml{load\_library}\index{load\_library@{\ptt load\_library}} (see the
\HOL\ manual for a general description of library loading). The first action
in the load sequence initiated by \ml{load\_library} is to update the \HOL\
help\index{help!updating search path} search path. The help search path is
updated with pathnames to online help files for the \ML\ functions in the
library. After updating the help search path, the \ML\ functions in the
library are loaded into \HOL.

The following session shows how the \ml{trs} library may be loaded using
\ml{load\_library}:

\setcounter{sessioncount}{1}

\begin{session}\begin{verbatim}
#load_library `trs`;;
Loading library `trs` ...
Updating help search path
.............................................................................
......................
Library `trs` loaded.
() : void

#
\end{verbatim}\end{session}

Below are some examples of the kind of searches that can be performed using
the \ml{trs} library.

Search for axioms in the theory {\small\verb%theory1%}:

\begin{boxed}\begin{verbatim}
#FT (kind Axiom)
#   (Paths [Theory `theory1`]);;
\end{verbatim}\end{boxed}

Search for theorems whose name contains the string {\small\verb%`CONJ`%}.
{\small\verb%theory1%} and all its ancestors are searched, excluding
{\small\verb%theory2%} and its ancestors:

\begin{boxed}\begin{verbatim}
#FT (thmname `*CONJ*`)
#   (Paths [Ancestors ([`theory1`],[`theory2`])]);;
\end{verbatim}\end{boxed}

Search the theory {\small\verb%theory1%} and all of its ancestors for theorems
with a conclusion matching the pattern {\small\verb%"!x y. x /\ y = y /\ x"%}.
The hypotheses of the theorem can be anything:

\begin{boxed}\begin{verbatim}
#FT (conc "!x y. x /\ y = y /\ x")
#   (Paths [Ancestors ([`theory1`],[])]);;
\end{verbatim}\end{boxed}

Search for theorems which contain a term of the form {\small\verb%"a /\ b"%}
(where {\small\verb%a%} and {\small\verb%b%} are arbitrary Boolean-valued
terms) somewhere within the structure of the conclusion. This is done by
matching any conclusion using the pattern {\small\verb%"x:bool"%} and then
performing a side-condition test on {\small\verb%"x:bool"%} to see if it
contains a term of the required form. The theory {\small\verb%theory1%} is
searched first, followed by {\small\verb%theory2%} and all of its ancestors.
Note that:

\begin{small}\begin{verbatim}
   (conc "x:bool") Where ("x:bool" contains "a /\ b")
\end{verbatim}\end{small}

\noindent
can be abbreviated to:

\begin{small}\begin{verbatim}
   "conc:bool" contains "a /\ b"
\end{verbatim}\end{small}

\begin{boxed}\begin{verbatim}
#FT ((conc "x:bool") Where ("x:bool" contains "a /\ b"))
#   (Paths [Theory `theory1`;Ancestors ([`theory2`],[])]);;
\end{verbatim}\end{boxed}

Search the list of theorems \ml{fthml} for theorems with at least one
hypothesis:

\begin{boxed}\begin{verbatim}
#FT (hypP ["x:bool"])
#   (List fthml);;
\end{verbatim}\end{boxed}

Search for theorems with exactly one hypothesis. The source for the search is
extracted from the search step \ml{s}, which was obtained from a previous
search:

\begin{boxed}\begin{verbatim}
#FT (hypF ["x:bool"])
#   (List_from s);;
\end{verbatim}\end{boxed}

Search {\small\verb%theory1%} for theorems with exactly two hypotheses, both
of which are equations, and where the right-hand side of one of the equations
is either the number {\small\verb%1%} or a variable:

\begin{boxed}\begin{verbatim}
#FT ((hypF ["a:* = b";"c:** = d"])
#      Where (("b:*" matches "1") Orelse (test1term is_var "b:*")))
#   (Paths [Theory `theory1`]);;
\end{verbatim}\end{boxed}

Search {\small\verb%theory1%} for theorems whose conclusions contain exactly
two conjunctions:

\begin{boxed}\begin{verbatim}
#FT (test1term (\t. (cnt_conj t) = 2) "conc:bool")
#   (Paths [Theory `theory1`]);;
\end{verbatim}\end{boxed}

\noindent
The function \ml{cnt\_conj} is defined by:

\begin{small}\begin{verbatim}
   letrec (cnt_conj : term -> int) t =
      let n = if (is_conj t) then 1 else 0
      in  (n + (cnt_conj (body t))) ?
          (n + (cnt_conj (rator t)) + (cnt_conj (rand t))) ?
          n;;
\end{verbatim}\end{small}

\noindent
Given a term, it returns an integer. The value of the integer is the number of
conjunctions in the term.


\chapter{Example Sessions}

This chapter consists of several \HOL\ sessions. These are intended to provide
a tutorial introduction to the Theorem Retrieval System. At the beginning of
each session the Theorem Retrieval System is loaded into \HOL. The material
displayed within boxes is part of the session. The other text is a commentary
on what is happening within the session.

The following diagram (taken from the \HOL\ System
Description~\cite{description}) of the built-in theory hierarchy of the \HOL\
system may be found useful.

\begin{center}
\setlength{\unitlength}{1mm}           % unit of length = 1mm
\begin{picture}(65,115)

\thicklines


% -----------------------------------------------------------
% Lines in theory hierarchy graph
% -----------------------------------------------------------

\put(40,5){\line(-4,1){20}}      % HOL --> tydefs
\put(40,5){\line(0,1){5}}        % HOL --> sum
\put(40,5){\line(4,1){20}}       % HOL --> one

\put(20,15){\line(0,1){5}}       % tydefs --> ltree
\put(40,15){\line(-2,3){10}}     % sum    --> combin

\put(20,25){\line(-2,1){10}}     % ltree --> tree
\put(20,25){\line(2,1){10}}      % ltree --> combin

\put(10,35){\line(0,1){5}}       % tree --> list
\put(10,45){\line(0,1){5}}       % list --> arithmetic
\put(10,55){\line(0,1){5}}       % arithmetics --> prim_rec
\put(10,65){\line(0,1){5}}       % prim_rec --> num

\put(10,75){\line(4,1){20}}      % num --> BASIC-HOL

\put(30,85){\line(0,1){5}}       % BASIC-HOL --> ind
\put(30,95){\line(0,1){5}}       % ind --> bool
\put(30,105){\line(0,1){5}}      % bool --> PPLAMB
\put(30,35){\line(0,1){45}}      % combin --> BASIC-HOL
\put(60,20){\line(-1,2){30}}     % one --> BASIC-HOL
\put(60,15){\line(0,1){5}}       % one --> BASIC-HOL



% -----------------------------------------------------------
% Theory names:
% -----------------------------------------------------------

\put(40,2.5){\makebox(0,0){\verb!HOL!}}

\put(20,12.5){\makebox(0,0){\verb!tydefs!}}
\put(40,12.5){\makebox(0,0){\verb!sum!}}
\put(60,12.5){\makebox(0,0){\verb!one!}}

\put(20,22.5){\makebox(0,0){\verb!ltree!}}

\put(30,32.5){\makebox(0,0){\verb!combin!}}
\put(10,32.5){\makebox(0,0){\verb!tree!}}

\put(10,42.5){\makebox(0,0){\verb!list!}}
\put(10,52.5){\makebox(0,0){\verb!arithmetic!}}
\put(10,62.5){\makebox(0,0){\verb!prim\_rec!}}
\put(10,72.5){\makebox(0,0){\verb!num!}}
\put(30,82.5){\makebox(0,0){\verb!BASIC-HOL!}}
\put(30,92.5){\makebox(0,0){\verb!ind!}}
\put(30,102.5){\makebox(0,0){\verb!bool!}}
\put(30,112.5){\makebox(0,0){\verb!PPLAMB!}}

\end{picture}
\end{center}


\section{Simple examples}

This session illustrates some typical uses of the retrieval system.

\setcounter{sessioncount}{1}

\begin{session}\begin{verbatim}

          _  _    __    _      __    __
   |___   |__|   |  |   |     |__|  |__|
   |      |  |   |__|   |__   |__|  |__|
   
          Version 2

#load_library `trs`;;
Loading library `trs` ...
Updating help search path
.............................................................................
......................
Library `trs` loaded.
() : void
\end{verbatim}\end{session}

First we search for the commutativity of multiplication within the built-in
theories of the \HOL\ system. We want the search\index{searching!steps} to
stop as soon as a matching theorem has been found. However, the search cannot
stop in the middle of a theory.

The pattern\index{patterns!for terms} matches whether or not the theorem is
universally quantified. The variables used in the theorem need not be
{\small\verb%a%} and {\small\verb%b%}. In fact, the pattern will also match
any theorems in which the {\small\verb%a%} and {\small\verb%b%} are replaced
by arbitrary terms of the appropriate type, but such theorems are unlikely to
exist.

The first function call begins the search. Only one theory is examined
initially. The result of the function is a list of theorems found, as well as
a function to continue the search.

\begin{session}\begin{verbatim}
#find_theorems ("conc:bool" has_body "a * b = b * a")
#              (Paths [Ancestors ([`HOL`],[])]);;
Searching theory HOL
Step([], -) : searchstep
\end{verbatim}\end{session}

\noindent
We now continue\index{searching!continuation} the search until the theorem
required is found:

\begin{session}\begin{verbatim}
#search_until_find it;;
Searching theory tydefs
Searching theory sum
Searching theory one
Searching theory BASIC-HOL
Searching theory ltree
Searching theory combin
Searching theory ind
Searching theory tree
Searching theory bool
Searching theory list
Searching theory PPLAMB
Searching theory arithmetic
Step([((Theorem), `arithmetic`, `MULT_SYM`, |- !m n. m * n = n * m)], -)
: searchstep
\end{verbatim}\end{session}

More general searches may also be attempted. The next function call searches
for commutative laws which occur in the theory {\small\verb%arithmetic%}.
{\small\verb%C%} is a function in the logic. It is likely that it will match
an infix operator.

\begin{session}\begin{verbatim}
#find_theorems ("conc:bool" has_body "(C:*->*->**) a b = C b a")
#              (Paths [Theory `arithmetic`]);;
Searching theory arithmetic
Step([((Theorem), `arithmetic`, `MULT_SYM`, |- !m n. m * n = n * m);
      ((Theorem), `arithmetic`, `ADD_SYM`, |- !m n. m + n = n + m)],
     -)
: searchstep
\end{verbatim}\end{session}

Theorems concerning a specific operator\index{searching!for operators} can
be obtained, though for many operators the list returned will be very large.

\begin{session}\begin{verbatim}
#full_search ("conc:bool" contains ">")
#            (Paths [Ancestors ([`HOL`],[])]);;
Searching theory HOL
Searching theory tydefs
Searching theory sum
Searching theory one
Searching theory BASIC-HOL
Searching theory ltree
Searching theory combin
Searching theory ind
Searching theory tree
Searching theory bool
Searching theory list
Searching theory PPLAMB
Searching theory arithmetic
Searching theory prim_rec
Searching theory num
[((Definition),
  `arithmetic`,
  `GREATER_OR_EQ`,
  |- !m n. m >= n = m > n \/ (m = n));
 ((Definition), `arithmetic`, `GREATER`, |- !m n. m > n = n < m)]
: foundthm list
\end{verbatim}\end{session}

\noindent
This is the end of the first session.

\begin{session}\begin{verbatim}
#quit();;
$
\end{verbatim}\end{session}


\section{Patterns}

The following session illustrates the construction and use of patterns.

\setcounter{sessioncount}{1}

\begin{session}\begin{verbatim}

          _  _    __    _      __    __
   |___   |__|   |  |   |     |__|  |__|
   |      |  |   |__|   |__   |__|  |__|
   
          Version 2

#load_library `trs`;;
Loading library `trs` ...
Updating help search path
.............................................................................
......................
Library `trs` loaded.
() : void
\end{verbatim}\end{session}

\index{patterns!for terms}
A common search requirement is to test the conclusion of a theorem. The
constructor \ml{conc} attempts to match the term given as its argument to the
{\em whole\/} of the conclusion of a theorem. The argument term is used as a
pattern in which variables can match any term of a matching type. We begin by
searching for theorems with a conclusion matching
{\small\verb%"!x. SUC x = y"%}. The function {\small\verb%full_search%}
completes the search in one step.

\begin{session}\begin{verbatim}
#full_search (conc "!x. SUC x = y") (Paths [Ancestors ([`HOL`],[])]);;
Searching theory HOL
Searching theory tydefs
Searching theory sum
Searching theory one
Searching theory BASIC-HOL
Searching theory ltree
Searching theory combin
Searching theory ind
Searching theory tree
Searching theory bool
Searching theory list
Searching theory PPLAMB
Searching theory arithmetic
Searching theory prim_rec
Searching theory num
[((Theorem), `arithmetic`, `ADD1`, |- !m. SUC m = m + 1);
 ((Definition),
  `num`,
  `SUC_DEF`,
  |- !m. SUC m = ABS_num(SUC_REP(REP_num m)))]
: foundthm list
\end{verbatim}\end{session}

\index{matching!within a term}
It is more common to want to search for a pattern {\em within\/} the
conclusion of a theorem, because the exact form of the theorem is not known.
This is far more computationally intensive than direct matching because the
pattern must be tested against every subterm of the conclusion, not just
against the conclusion itself.

In the following example, we search the theory {\small\verb%arithmetic%} for
theorems with a conclusion {\em containing\/} the greater-than operator. The
list of theorems found is extracted from the {\small\verb%searchstep%} and is
bound to \ml{found}.

\vfill
\begin{session}\begin{verbatim}
#let found = show_step (find_theorems ("conc:bool" contains "x > y")
#                                     (Paths [Theory `arithmetic`]));;
Searching theory arithmetic
found = 
[((Definition),
  `arithmetic`,
  `GREATER_OR_EQ`,
  |- !m n. m >= n = m > n \/ (m = n));
 ((Definition), `arithmetic`, `GREATER`, |- !m n. m > n = n < m)]
: foundthm list
\end{verbatim}\end{session}
\vfill

\index{searching!the result of a previous search}
A search may yield more theorems than required because the original pattern
was not specific enough. We can filter out the unwanted theorems by searching
the list of theorems obtained from the previous search with a new pattern.
We now filter out theorems in the list bound to \ml{found} if they do not
contain the greater-than-or-equal-to operator.

\vfill
\begin{session}\begin{verbatim}
#find_theorems ("conc:bool" contains ">=") (List found);;
Endofsearch[((Definition),
             `arithmetic`,
             `GREATER_OR_EQ`,
             |- !m n. m >= n = m > n \/ (m = n))]
: searchstep
\end{verbatim}\end{session}
\vfill

\index{kind@{\ptt kind}}
Sometimes one wants to restrict searches to theorems which are either axioms,
definitions, or derived theorems. In the next example we search for axioms
within the built-in \HOL\ theories.

\begin{session}\begin{verbatim}
#full_search (kind Axiom) (Paths [Ancestors ([`HOL`],[])]);;
Searching theory HOL
Searching theory tydefs
Searching theory sum
Searching theory one
Searching theory BASIC-HOL
Searching theory ltree
Searching theory combin
Searching theory ind
Searching theory tree
Searching theory bool
Searching theory list
Searching theory PPLAMB
Searching theory arithmetic
Searching theory prim_rec
Searching theory num
[((Axiom), `ind`, `INFINITY_AX`, |- ?f. ONE_ONE f /\ ~ONTO f);
 ((Axiom), `bool`, `SELECT_AX`, |- !P x. P x ==> P($@ P));
 ((Axiom), `bool`, `ETA_AX`, |- !t. (\x. t x) = t);
 ((Axiom),
  `bool`,
  `IMP_ANTISYM_AX`,
  |- !t1 t2. (t1 ==> t2) ==> (t2 ==> t1) ==> (t1 = t2));
 ((Axiom), `bool`, `BOOL_CASES_AX`, |- !t. (t = T) \/ (t = F));
 ((Axiom), `bool`, `ARB_THM`, |- $= = $=)]
: foundthm list
\end{verbatim}\end{session}
\vfill

\index{patterns!for names}
The retrieval system also provides simple pattern matching on the names of
theorems and the names of theory segments to which the theorems belong. It is
however envisaged that these facilities will not be greatly used, given the
far more powerful structural matching available.

As an example, we search the theory segment {\small\verb%sum%} for theorems
with names containing two characters followed by a letter `L'. The function
\ml{thmname} tests for theorem names matching the string pattern given as its
argument. A {\small\verb%*%} in the string pattern means `match zero or more
characters'. A {\small\verb%?%} means `match exactly one character'. All other
characters in the pattern must be matched literally.

\begin{session}\begin{verbatim}
#find_theorems (thmname `*??L*`) (Paths [Theory `sum`]);;
Searching theory sum
Step([((Definition), `sum`, `OUTL`, |- !x. OUTL(INL x) = x);
      ((Definition),
       `sum`,
       `ISL`,
       |- (!x. ISL(INL x)) /\ (!y. ~ISL(INR y)));
      ((Definition),
       `sum`,
       `INL_DEF`,
       |- !e. INL e = ABS_sum(\b x y. (x = e) /\ b));
      ((Theorem), `sum`, `INL`, |- !x. ISL x ==> (INL(OUTL x) = x));
      ((Theorem), `sum`, `ISL_OR_ISR`, |- !x. ISL x \/ ISR x)],
     -)
: searchstep
\end{verbatim}\end{session}

Since the search specifications are built up entirely from \ML\ functions, it
is easy to make abbreviations for parts of the specification. Here we bind
\ml{s} to a {\small\verb%source%} for searching the theory (segment)
{\small\verb%sum%}. We then search this {\small\verb%source%} for theorems
which have {\em both\/} a sub-string consisting of two characters and the
letter `L', and a sub-string consisting of two characters and the letter `R',
in their name.

\begin{session}\begin{verbatim}
#let s = Paths [Theory `sum`];;
s = Paths[Theory `sum`] : source

#find_theorems ((thmname `*??L*`) Andalso (thmname `*??R*`)) s;;
Searching theory sum
Step([((Theorem), `sum`, `ISL_OR_ISR`, |- !x. ISL x \/ ISR x)], -)
: searchstep

#quit();;
$
\end{verbatim}\end{session}

\noindent
The function \ml{Andalso}\index{Andalso@{\ptt Andalso}} can be used to combine
two patterns into a single one such that both the sub-patterns must match for
the composite pattern to match. There are also functions
\ml{Orelse}\index{Orelse@{\ptt Orelse}} and \ml{Not}\index{Not@{\ptt Not}}
which implement the obvious logical operations on patterns.


\section{Search paths}

This session searches the ancestry of the theory {\small\verb%HOL%}. The
patterns used are chosen so that they do not match any theorem. The purpose of
the session is to illustrate the route taken through a theory hierarchy during
a search.

\setcounter{sessioncount}{1}

\begin{session}\begin{verbatim}

          _  _    __    _      __    __
   |___   |__|   |  |   |     |__|  |__|
   |      |  |   |__|   |__   |__|  |__|
   
          Version 2

#load_library `trs`;;
Loading library `trs` ...
Updating help search path
.............................................................................
......................
Library `trs` loaded.
() : void
\end{verbatim}\end{session}

\index{searching!end of}
Attempting to continue a search beyond its end causes an exception. We
illustrate this by binding the \ML\ identifier \ml{none} to a pattern which
never matches. We then search a single theory with this pattern.

\begin{session}\begin{verbatim}
#let none = thmname ``;;
none = - : thmpattern

#find_theorems none (Paths [Theory `HOL`]);;
Searching theory HOL
Step([], -) : searchstep

#continue_search it;;
Endofsearch [] : searchstep

#continue_search it;;
evaluation failed     continue_search
\end{verbatim}\end{session}

The above example searches a single theory. The next example searches three
theories in sequence, in the order specified.

\begin{session}\begin{verbatim}
#full_search none (Paths [Theory `one`; Theory `num`; Theory `list`]);;
Searching theory one
Searching theory num
Searching theory list
[] : foundthm list
\end{verbatim}\end{session}

\index{searching!a theory ancestry}
We can also search the ancestry of a theory. This is done breadth-first
starting from the specified theory. Its parents are then searched in the order
in which they appear as parents. Then the parents of the parents are searched,
and so on, until no ancestors remain. Since the ancestry is a graph, a theory
may be encountered more than once. It is ignored on all but the first
encounter.

\begin{session}\begin{verbatim}
#full_search none (Paths [Ancestors ([`HOL`],[])]);;
Searching theory HOL
Searching theory tydefs
Searching theory sum
Searching theory one
Searching theory BASIC-HOL
Searching theory ltree
Searching theory combin
Searching theory ind
Searching theory tree
Searching theory bool
Searching theory list
Searching theory PPLAMB
Searching theory arithmetic
Searching theory prim_rec
Searching theory num
[] : foundthm list
\end{verbatim}\end{session}

\index{exclusion!of parts of an ancestry}
Observe that the theory {\small\verb%BASIC-HOL%} is searched quite early on.
This is because although one may `feel' that it is deep down in the hierarchy,
it is actually only two steps away from {\small\verb%HOL%}. We can force
{\small\verb%BASIC-HOL%} and its parents to be searched last by excluding them
from the first ancestry search and then searching them explicitly afterwards.

\begin{session}\begin{verbatim}
#full_search none (Paths [Ancestors ([`HOL`],[`BASIC-HOL`]);
#                         Ancestors ([`BASIC-HOL`],[])]);;
Searching theory HOL
Searching theory tydefs
Searching theory sum
Searching theory one
Searching theory ltree
Searching theory combin
Searching theory tree
Searching theory list
Searching theory arithmetic
Searching theory prim_rec
Searching theory num
Searching theory BASIC-HOL
Searching theory ind
Searching theory bool
Searching theory PPLAMB
[] : foundthm list
\end{verbatim}\end{session}

\index{searching!user theories}
\ml{Ancestry} is a constructor for searching a hierarchy excluding
{\small\verb%HOL%} and its ancestors. This is useful when searching user theory
hierarchies. If {\small\verb%HOL%} is not excluded, then because it is a
parent of every user theory, the search soon dives into the built-in hierarchy.

In the following example, no theories are searched.

\begin{session}\begin{verbatim}
#full_search none (Paths [Ancestry [`HOL`]]);;
[] : foundthm list
\end{verbatim}\end{session}

\index{searching!order of}
Two theories which do not have a common descendant cannot be `active' at the
same time. So, they cannot both be specified in a single search. However,
two ancestors of the current theory can be specified in a single search. For
example, we can do a breadth-first search of {\small\verb%sum%}, then
{\small\verb%tydefs%}, then the parents of {\small\verb%sum%}, then the
parents of {\small\verb%tydefs%}, and so on. {\small\verb%BASIC-HOL%} and its
ancestors are excluded.

\begin{session}\begin{verbatim}
#full_search none (Paths [Ancestors ([`sum`;`tydefs`],[`BASIC-HOL`])]);;
Searching theory sum
Searching theory tydefs
Searching theory combin
Searching theory ltree
Searching theory tree
Searching theory list
Searching theory arithmetic
Searching theory prim_rec
Searching theory num
[] : foundthm list

#quit();;
$
\end{verbatim}\end{session}


\section{Side-conditions}

Sometimes pattern matching is not sufficiently expressive. In this session,
side-condition tests are used to express complex queries. The tools for
generating side-conditions are also illustrated here.

\setcounter{sessioncount}{1}

\begin{session}\begin{verbatim}

          _  _    __    _      __    __
   |___   |__|   |  |   |     |__|  |__|
   |      |  |   |__|   |__   |__|  |__|
   
          Version 2

#load_library `trs`;;
Loading library `trs` ...
Updating help search path
.............................................................................
......................
Library `trs` loaded.
() : void

#let s = Paths [Theory `arithmetic`];;
s = Paths[Theory `arithmetic`] : source
\end{verbatim}\end{session}

Having bound \ml{s} to a {\small\verb%source%} consisting of the theory
{\small\verb%arithmetic%}, we search for theorems with an equation as body,
such that the right-hand side of the equation is a variable.

\begin{session}\begin{verbatim}
#find_theorems ((conc "x:bool") Where ("x:bool" has_body "(a:*) = b")
#                               Where (test1term is_var "b:*")) s;;
Searching theory arithmetic
Step([((Theorem), `arithmetic`, `ADD_SUB`, |- !a c. (a + c) - c = a);
      ((Theorem), `arithmetic`, `MULT_RIGHT_1`, |- !m. m * 1 = m);
      ((Theorem), `arithmetic`, `MULT_LEFT_1`, |- !m. 1 * m = m);
      ((Theorem), `arithmetic`, `SUC_SUB1`, |- !m. (SUC m) - 1 = m);
      ((Theorem), `arithmetic`, `ADD_0`, |- !m. m + 0 = m)],
     -)
: searchstep
\end{verbatim}\end{session}

\index{side-conditions!abbreviation}
\noindent
Since matching the {\small\verb%"x:bool"%} against the conclusion serves no
purpose, the above search can be abbreviated as follows:

\begin{session}\begin{verbatim}
#find_theorems (("conc:bool" has_body "(a:*) = b")
#                  Where (test1term is_var "b:*")) s;;
Searching theory arithmetic
Step([((Theorem), `arithmetic`, `ADD_SUB`, |- !a c. (a + c) - c = a);
      ((Theorem), `arithmetic`, `MULT_RIGHT_1`, |- !m. m * 1 = m);
      ((Theorem), `arithmetic`, `MULT_LEFT_1`, |- !m. 1 * m = m);
      ((Theorem), `arithmetic`, `SUC_SUB1`, |- !m. (SUC m) - 1 = m);
      ((Theorem), `arithmetic`, `ADD_0`, |- !m. m + 0 = m)],
     -)
: searchstep
\end{verbatim}\end{session}

\noindent
The function \ml{test1term}\index{test1term@{\ptt test1term}} obtains the term
bound to the variable {\small\verb%b%}. It then applies the function
\ml{is\_var} to the term. The Boolean result determines the outcome of the
side-condition.

We now search for theorems with an equation as body, such that the right-hand
side of the equation is a variable or the number {\small\verb%0%}.

\begin{session}\begin{verbatim}
#find_theorems (("conc:bool" has_body "(a:*) = b")
#                  Where (("b:*" matches "0")
#                     Orelse (test1term is_var "b:*"))) s;;
Searching theory arithmetic
Step([((Theorem), `arithmetic`, `SUB_EQUAL_0`, |- !c. c - c = 0);
      ((Theorem), `arithmetic`, `ADD_SUB`, |- !a c. (a + c) - c = a);
      ((Theorem), `arithmetic`, `MOD_ONE`, |- !k. k MOD (SUC 0) = 0);
      ((Theorem), `arithmetic`, `MULT_RIGHT_1`, |- !m. m * 1 = m);
      ((Theorem), `arithmetic`, `MULT_LEFT_1`, |- !m. 1 * m = m);
      ((Theorem), `arithmetic`, `MULT_0`, |- !m. m * 0 = 0);
      ((Theorem), `arithmetic`, `SUC_SUB1`, |- !m. (SUC m) - 1 = m);
      ((Theorem), `arithmetic`, `ADD_0`, |- !m. m + 0 = m)],
     -)
: searchstep
\end{verbatim}\end{session}

\index{side-conditions!user functions}
In the next example, we search the theory {\small\verb%bool%} for theorems
with a conclusion containing exactly two conjunctions. To do this, we first
define a function which counts the number of conjunctions in a term. We then
form a function which yields \ml{true} of a term if it contains exactly two
conjunctions. This is used as an argument to \ml{test1term}.

\vfill
\begin{session}\begin{verbatim}
#letrec cnt_conj t =
#   let n = if (is_conj t) then 1 else 0
#   in  (n + (cnt_conj (snd (dest_abs t)))) ?
#       (n + (cnt_conj (rator t)) + (cnt_conj (rand t))) ?
#       n;;
cnt_conj = - : (term -> int)

#find_theorems (test1term (\t. (cnt_conj t) = 2) "conc:bool")
#              (Paths [Theory `bool`]);;
Searching theory bool
Step([((Definition),
       `bool`,
       `EXISTS_UNIQUE_DEF`,
       |- $?! = (\P. $? P /\ (!x y. P x /\ P y ==> (x = y))))],
     -)
: searchstep
\end{verbatim}\end{session}
\vfill

\index{side-conditions!abbreviation}
\noindent
The use of {\small\verb%"conc:bool"%} as the second argument to \ml{test1term}
allows only the side-condition to be given as the pattern. The variable
{\small\verb%"conc:bool"%} is always bound to the conclusion of the theorem
being tested.

We now define a side-condition for comparing the number of distinct free
variables in two terms, and use it to search for equations in which there are
more variables on the left-hand side than on the right. The function
\ml{test2terms}\index{test2terms@{\ptt test2terms}} behaves in a similar
manner to \ml{test1term}, except that it deals with two variables rather than
one. By defining a suitable infix function, the pattern can be written in a
syntactically pleasant way.

\vfill
\begin{session}\begin{verbatim}
#let has_more_vars term1 term2 =
#   (length (frees (term1))) > (length (frees (term2)));;
has_more_vars = - : (term -> term -> bool)

#let has_more_vars_than = test2terms (has_more_vars);;
has_more_vars_than = - : (term -> term -> thmpattern)

#ml_curried_infix `has_more_vars_than`;;
() : void
\end{verbatim}\end{session}

\begin{session}\begin{verbatim}
#find_theorems (("conc:bool" has_body "(a:*) = b")
#                  Where ("a:*" has_more_vars_than "b:*")) s;;
Searching theory arithmetic
Step([((Theorem), `arithmetic`, `SUB_EQUAL_0`, |- !c. c - c = 0);
      ((Theorem), `arithmetic`, `ADD_SUB`, |- !a c. (a + c) - c = a);
      ((Theorem),
       `arithmetic`,
       `MULT_MONO_EQ`,
       |- !m i n. ((SUC n) * m = (SUC n) * i) = (m = i));
      ((Theorem),
       `arithmetic`,
       `LESS_MULT_MONO`,
       |- !m i n. ((SUC n) * m) < ((SUC n) * i) = m < i);
      ((Theorem), `arithmetic`, `MOD_ONE`, |- !k. k MOD (SUC 0) = 0);
      ((Theorem),
       `arithmetic`,
       `MULT_EXP_MONO`,
       |- !p q n m.
           (n * ((SUC q) EXP p) = m * ((SUC q) EXP p)) = (n = m));
      ((Theorem),
       `arithmetic`,
       `MULT_SUC_EQ`,
       |- !p m n. (n * (SUC p) = m * (SUC p)) = (n = m));
      ((Theorem),
       `arithmetic`,
       `LESS_EQ_MONO_ADD_EQ`,
       |- !m n p. (m + p) <= (n + p) = m <= n);
      ((Theorem),
       `arithmetic`,
       `EQ_MONO_ADD_EQ`,
       |- !m n p. (m + p = n + p) = (m = n));
      ((Theorem),
       `arithmetic`,
       `LESS_MONO_ADD_EQ`,
       |- !m n p. (m + p) < (n + p) = m < n);
      ((Theorem),
       `arithmetic`,
       `ADD_INV_0_EQ`,
       |- !m n. (m + n = m) = (n = 0));
      ((Theorem), `arithmetic`, `MULT_0`, |- !m. m * 0 = 0)],
     -)
: searchstep
\end{verbatim}\end{session}
\vfill

\index{patterns!type information}
Note that one does not have to know complete type information to do a search
because polymorphic types are treated as pattern variables. The types must
however be consistent, e.g., {\small\verb%y%} below must have a
{\small\verb%sum%} type.

\begin{session}\begin{verbatim}
#find_theorems (conc "!(x:*). (y:**+***) = (ABS_sum z)")
#              (Paths [Theory `sum`]);;
Searching theory sum
Step([((Definition),
       `sum`,
       `INR_DEF`,
       |- !e. INR e = ABS_sum(\b x y. (y = e) /\ ~b));
      ((Definition),
       `sum`,
       `INL_DEF`,
       |- !e. INL e = ABS_sum(\b x y. (x = e) /\ b))],
     -)
: searchstep
\end{verbatim}\end{session}

\index{errors!at run-time}
Unfortunately, inconsistent types may lead to run-time errors. In the next
example, the {\small\verb%z%} appearing in the side-condition has a different
type to the {\small\verb%z%} appearing in the main clause, and so they are
considered to be different variables
({\em Variables\index{patterns!distinct variables} are identified by both
name and type}.). The error only shows up if the side-condition is tested. It
is not possible to test for unknown variables in side-conditions prior to the
search, because side-conditions can be arbitrary functions. An informative
failure message is the best that can be achieved.

\begin{session}\begin{verbatim}
#find_theorems ((conc "!(x:*). (y:**+***) = (ABS_sum z)")
#                  Where (Not ("z:****" contains "$~")))
#              (Paths [Theory `sum`]);;
Searching theory sum
evaluation failed     match_of_var -- unknown wildvar (variable)
\end{verbatim}\end{session}

\noindent
The types are given correctly in the following example.

\begin{session}\begin{verbatim}
#find_theorems
#   ((conc "!(x:*). (y:**+***) = (ABS_sum z)")
#       Where (Not ("z:bool->(**->(***->bool))" contains "$~")))
#   (Paths [Theory `sum`]);;
Searching theory sum
Step([((Definition),
       `sum`,
       `INL_DEF`,
       |- !e. INL e = ABS_sum(\b x y. (x = e) /\ b))],
     -)
: searchstep

#quit();;
$
\end{verbatim}\end{session}


\section{Error messages}

\index{errors!illegal pattern construction}
This session illustrates some of the top-level error messages caused by using
an illegal pattern constructor within a side-condition.

\setcounter{sessioncount}{1}

\begin{session}\begin{verbatim}

          _  _    __    _      __    __
   |___   |__|   |  |   |     |__|  |__|
   |      |  |   |__|   |__   |__|  |__|
   
          Version 2

#load_library `trs`;;
Loading library `trs` ...
Updating help search path
.............................................................................
......................
Library `trs` loaded.
() : void

#find_theorems ((conc "x:bool") Where (kind Axiom))
#              (Paths [Theory `arithmetic`]);;
evaluation failed     Where -- `Kind' used in side-condition

#find_theorems ((conc "x:bool") Where (conc "a * b"))
#              (Paths [Theory `arithmetic`]);;
evaluation failed     Where -- `Conc' used in side-condition

#find_theorems ((conc "x:bool")
#                 Where (("x:bool" contains "a * b")
#                          Where ("a:num" contains "l - m")))
#              (Paths [Theory `arithmetic`]);;
evaluation failed     Where -- `Where' used in side-condition

#find_theorems (((conc "x:bool") Where ("x:bool" contains "a * b"))
#                 Where ("a:num" contains "l - m"))
#              (Paths [Theory `arithmetic`]);;
Searching theory arithmetic
Step([((Theorem),
       `arithmetic`,
       `RIGHT_SUB_DISTRIB`,
       |- !m n p. (m - n) * p = (m * p) - (n * p))],
     -)
: searchstep

#quit();;
$
\end{verbatim}\end{session}




\chapter{New ML Types in the trs Library}

{\def\_{{\char'137}}                     % \tt style `_' character

\begin{center}
\begin{tabular}{|l|l|}
\hline
{\it ML type}                 & {\it Description}\\
\hline
{\small\tt wildvar}           & Pattern variables for terms \\
{\small\tt wildtype}          & Pattern variables for types \\
{\small\tt termpattern}       & Patterns for terms \\
{\small\tt matching}          & Binding resulting from a term match \\
{\small\tt result\_of\_match} & Result of a match: either no match or a
binding \\
{\small\tt side\_condition}   & Side-condition tests \\
{\small\tt wildchar}          & Pattern variables for names \\
{\small\tt namepattern}       & Patterns for theorem and theory names \\
{\small\tt thmkind}           & The `kind' of a theorem: axiom, definition or
derived theorem \\
{\small\tt foundthm}          & A theorem, its name, its theory segment and its
`kind' \\
{\small\tt thmpattern\_rep}   & Representation type for theorem patterns \\
{\small\tt thmpattern}        & Patterns for theorems \\
{\small\tt searchpath}        & Paths through theory hierarchies \\
{\small\tt source}            & Sources for a search: either a list of theorems
or a search path \\
{\small\tt searchstep}        & Steps within a search: one step for each theory
segment \\
\hline
\end{tabular}
\end{center}

}
