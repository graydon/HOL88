\chapter{ML Functions in the pair Library}
This chapter provides documentation on all the \ML\ functions that are made
available in \HOL\ when the \ml{more\_arithmetic} library is loaded.  This
documentation is also available online via the \ml{help} facility.



\DOC{AND\_PEXISTS\_CONV}

\TYPE {\small\verb%AND_PEXISTS_CONV : conv%}\egroup

\SYNOPSIS
Moves a paired existential quantification outwards through a conjunction.

\DESCRIBE
When applied to a term of the form {\small\verb%(?p. t) /\ (?p. u)%}, where no
variables in {\small\verb%p%} are free in either {\small\verb%t%} or {\small\verb%u%}, {\small\verb%AND_PEXISTS_CONV%} returns
the theorem:
{\par\samepage\setseps\small
\begin{verbatim}
   |- (?p. t) /\ (?p. u) = (?p. t /\ u)
\end{verbatim}
}
\FAILURE
{\small\verb%AND_PEXISTS_CONV%} fails if it is applied to a term not of the form
{\small\verb%(?p. t) /\ (?p. u)%}, or if it is applied to a term {\small\verb%(?p. t) /\ (?p. u)%}
in which variables from {\small\verb%p%} are free in either {\small\verb%t%} or {\small\verb%u%}.

\SEEALSO
AND_EXISTS_CONV, PEXISTS_AND_CONV, LEFT_AND_PEXISTS_CONV,
RIGHT_AND_PEXISTS_CONV.

\ENDDOC
\DOC{AND\_PFORALL\_CONV}

\TYPE {\small\verb%AND_PFORALL_CONV : conv%}\egroup

\SYNOPSIS
Moves a paired universal quantification outwards through a conjunction.

\DESCRIBE
When applied to a term of the form {\small\verb%(!p. t) /\ (!p. t)%}, the conversion
{\small\verb%AND_PFORALL_CONV%} returns the theorem:
{\par\samepage\setseps\small
\begin{verbatim}
   |- (!p. t) /\ (!p. u) = (!p. t /\ u)
\end{verbatim}
}
\FAILURE
Fails if applied to a term not of the form {\small\verb%(!p. t) /\ (!p. t)%}.

\SEEALSO
AND_FORALL_CONV, PFORALL_AND_CONV, LEFT_AND_PFORALL_CONV,
RIGHT_AND_PFORALL_CONV.

\ENDDOC
\DOC{bndpair}

\TYPE {\small\verb%bndpair : (term -> term)%}\egroup

\SYNOPSIS
Returns the bound pair of a paired abstraction.

\DESCRIBE
{\small\verb%bndpair "\pair. t"%} returns {\small\verb%"pair"%}.

\FAILURE
Fails unless the term is a paired abstraction.

\SEEALSO
bndvar, pbody, dest_pabs.

\ENDDOC
\DOC{CURRY\_CONV}

\TYPE {\small\verb%CURRY_CONV : conv%}\egroup

\SYNOPSIS
Currys an application of a paired abstraction.

\EXAMPLE
{\par\samepage\setseps\small
\begin{verbatim}
#CURRY_CONV "(\(x,y). x + y) (1,2)";;
|- (\(x,y). x + y)(1,2) = (\x y. x + y)1 2

#CURRY_CONV "(\(x,y). x + y) z";;
|- (\(x,y). x + y)z = (\x y. x + y)(FST z)(SND z)
\end{verbatim}
}

\FAILURE
{\small\verb%CURRY_CONV tm%} fails if {\small\verb%tm%} is not an application of a paired abstraction.

\SEEALSO
UNCURRY_CONV.

\ENDDOC
\DOC{CURRY\_EXISTS\_CONV}

\TYPE {\small\verb%CURRY_EXISTS_CONV : conv%}\egroup

\SYNOPSIS
Currys paired existential quantifications into
consecutive existential quantifications.

\EXAMPLE
{\par\samepage\setseps\small
\begin{verbatim}
#CURRY_EXISTS_CONV "?(x,y). x + y = y + x";;
|- (?(x,y). x + y = y + x) = (?x y. x + y = y + x)

#CURRY_EXISTS_CONV "?((w,x),(y,z)). w+x+y+z = z+y+x+w";;
|- (?((w,x),y,z). w + (x + (y + z)) = z + (y + (x + w))) =
   (?(w,x) (y,z). w + (x + (y + z)) = z + (y + (x + w)))
\end{verbatim}
}

\FAILURE
{\small\verb%CURRY_EXISTS_CONV tm%} fails if {\small\verb%tm%} is not a paired existential quantification.

\SEEALSO
CURRY_CONV, UNCURRY_CONV, UNCURRY_EXISTS_CONV, CURRY_FORALL_CONV, 
UNCURRY_FORALL_CONV.

\ENDDOC
\DOC{CURRY\_FORALL\_CONV}

\TYPE {\small\verb%CURRY_FORALL_CONV : conv%}\egroup

\SYNOPSIS
Currys paired universal quantifications into
consecutive universal quantifications.

\EXAMPLE
{\par\samepage\setseps\small
\begin{verbatim}
#CURRY_FORALL_CONV "!(x,y). x + y = y + x";;
|- (!(x,y). x + y = y + x) = (!x y. x + y = y + x)

#CURRY_FORALL_CONV "!((w,x),(y,z)). w+x+y+z = z+y+x+w";;
|- (!((w,x),y,z). w + (x + (y + z)) = z + (y + (x + w))) =
   (!(w,x) (y,z). w + (x + (y + z)) = z + (y + (x + w)))
\end{verbatim}
}

\FAILURE
{\small\verb%CURRY_FORALL_CONV tm%} fails if {\small\verb%tm%} is not a paired universal quantification.

\SEEALSO
CURRY_CONV, UNCURRY_CONV, UNCURRY_FORALL_CONV, CURRY_EXISTS_CONV, 
UNCURRY_EXISTS_CONV.

\ENDDOC
\DOC{dest\_pabs}

\TYPE {\small\verb%dest_pabs : (term -> (term # term))%}\egroup

\SYNOPSIS
Breaks apart a paired abstraction into abstracted pair and body.

\DESCRIBE
{\small\verb%dest_pabs%} is a term destructor for paired abstractions:
{\small\verb%dest_abs "\pair. t"%} returns {\small\verb%("pair","t")%}.

\FAILURE
Fails with {\small\verb%dest_pabs%} if term is not a paired abstraction.

\SEEALSO
dest_abs, mk_pabs, is_pabs, strip_pabs.

\ENDDOC
\DOC{dest\_pexists}

\TYPE {\small\verb%dest_pexists : (term -> (term # term))%}\egroup

\SYNOPSIS
Breaks apart paired existential quantifiers into the bound pair and the body.

\DESCRIBE
{\small\verb%dest_pexists%} is a term destructor for paired existential quantification.
The application of {\small\verb%dest_pexists%} to {\small\verb%"?pair. t"%} returns {\small\verb%("pair","t")%}.

\FAILURE
Fails with {\small\verb%dest_pexists%} if term is not a paired existential quantification.

\SEEALSO
dest_exists, mk_pexists, is_pexists, strip_pexists.

\ENDDOC
\DOC{dest\_pforall}

\TYPE {\small\verb%dest_pforall : (term -> (term # term))%}\egroup

\SYNOPSIS
Breaks apart paired universal quantifiers into the bound pair and the body.

\DESCRIBE
{\small\verb%dest_pforall%} is a term destructor for paired universal quantification.
The application of {\small\verb%dest_pforall%} to {\small\verb%"!pair. t"%} returns {\small\verb%("pair","t")%}.

\FAILURE
Fails with {\small\verb%dest_pforall%} if term is not a paired universal quantification.

\SEEALSO
dest_forall, mk_pforall, is_pforall, strip_pforall.

\ENDDOC
\DOC{dest\_prod}

\TYPE {\small\verb%dest_prod : (type -> (type # type))%}\egroup

\SYNOPSIS
Breaks apart a product type into its component types.

\DESCRIBE
{\small\verb%dest_prod%} is a type destructor for products:
{\small\verb%dest_pair ":t1#t2"%} returns {\small\verb%(":t1",":t2")%}.

\FAILURE
Fails with {\small\verb%dest_prod%} if the argument is not a product type.

\SEEALSO
is_prod, mk_prod.

\ENDDOC
\DOC{dest\_pselect}

\TYPE {\small\verb%dest_pselect : (term -> (term # term))%}\egroup

\SYNOPSIS
Breaks apart a paired choice-term into the selected pair and the body.

\DESCRIBE
{\small\verb%dest_pselect%} is a term destructor for paired choice terms.
The application of {\small\verb%dest_select%} to {\small\verb%"@pair. t"%} returns {\small\verb%("pair","t")%}.

\FAILURE
Fails with {\small\verb%dest_pselect%} if term is not a paired choice-term.

\SEEALSO
dest_select, mk_pselect, is_pselect.

\ENDDOC
\DOC{FILTER\_PGEN\_TAC}

\TYPE {\small\verb%FILTER_PGEN_TAC : (term -> tactic)%}\egroup

\SYNOPSIS
Strips off a paired universal quantifier,
but fails for a given quantified pair.

\DESCRIBE
When applied to a term {\small\verb%q%} and a goal {\small\verb%A ?- !p. t%},
the tactic {\small\verb%FILTER_PGEN_TAC%} fails if the quantified pair {\small\verb%p%} is the same as
{\small\verb%p%}, but otherwise advances the goal in the same way as {\small\verb%PGEN_TAC%},
i.e. returns the goal {\small\verb%A ?- t[p'/p]%} where {\small\verb%p'%} is a variant of {\small\verb%p%} chosen to
avoid clashing with any variables free in the goal's assumption list.
Normally {\small\verb%p'%} is just {\small\verb%p%}.
{\par\samepage\setseps\small
\begin{verbatim}
     A ?- !p. t
   ==============  FILTER_PGEN_TAC "q"
    A ?- t[p'/p]
\end{verbatim}
}
\FAILURE
Fails if the goal's conclusion is not a paired universal quantifier or the
quantified pair is equal to the given term.

\SEEALSO
FILTER_GEN_TAC, PGEN, PGEN_TAC, PGENL, PGEN_ALL, PSPEC, PSPECL, PSPEC_ALL,
PSPEC_TAC, PSTRIP_TAC.

\ENDDOC
\DOC{FILTER\_PSTRIP\_TAC}

\TYPE {\small\verb%FILTER_PSTRIP_TAC : (term -> tactic)%}\egroup

\SYNOPSIS
Conditionally strips apart a goal by eliminating the outermost connective.

\DESCRIBE
Stripping apart a goal in a more careful way than is done by {\small\verb%PSTRIP_TAC%} may
be necessary when dealing with quantified terms and implications.
{\small\verb%FILTER_PSTRIP_TAC%} behaves like {\small\verb%PSTRIP_TAC%}, but it does not strip apart a
goal if it contains a given term.

If {\small\verb%u%} is a term, then {\small\verb%FILTER_PSTRIP_TAC u%} is a tactic that removes one
outermost occurrence of one of the connectives {\small\verb%!%}, {\small\verb%==>%}, {\small\verb%~%} or {\small\verb%/\%} from the
conclusion of the goal {\small\verb%t%}, provided the term being stripped does not contain
{\small\verb%u%}.  {\small\verb%FILTER_PSTRIP_TAC%} will strip paired universal quantifications.
A negation {\small\verb%~t%} is treated as the implication {\small\verb%t ==> F%}.
{\small\verb%FILTER_PSTRIP_TAC%} also breaks apart conjunctions without applying any
filtering.

If {\small\verb%t%} is a universally quantified term, {\small\verb%FILTER_PSTRIP_TAC u%}
strips off the quantifier:
{\par\samepage\setseps\small
\begin{verbatim}
      A ?- !p. v
   ================  FILTER_PSTRIP_TAC "u"       [where p is not u]
     A ?- v[p'/p]
\end{verbatim}
}
\noindent where {\small\verb%p'%} is a primed variant of the pair {\small\verb%p%} that does not contain
any variables that appear free in the assumptions {\small\verb%A%}.  
If {\small\verb%t%} is a conjunction, no filtering is done and {\small\verb%FILTER_PSTRIP_TAC%} simply
splits the conjunction:
{\par\samepage\setseps\small
\begin{verbatim}
      A ?- v /\ w
   =================  FILTER_PSTRIP_TAC "u"
    A ?- v   A ?- w
\end{verbatim}
}
\noindent If {\small\verb%t%} is an implication and the antecedent does not contain
a free instance of {\small\verb%u%}, then {\small\verb%FILTER_PSTRIP_TAC u%} moves the antecedent into the
assumptions and recursively splits the antecedent according to the following
rules (see {\small\verb%PSTRIP_ASSUME_TAC%}):
{\par\samepage\setseps\small
\begin{verbatim}
    A ?- v1 /\ ... /\ vn ==> v            A ?- v1 \/ ... \/ vn ==> v
   ============================        =================================
       A u {v1,...,vn} ?- v             A u {v1} ?- v ... A u {vn} ?- v

     A ?- (?p. w) ==> v
   ====================
    A u {w[p'/p]} ?- v
\end{verbatim}
}
\noindent where {\small\verb%p'%} is a variant of the pair {\small\verb%p%}.

\FAILURE
{\small\verb%FILTER_PSTRIP_TAC u (A,t)%} fails if {\small\verb%t%} is not a universally quantified term,
an implication, a negation or a conjunction; or if the term being
stripped contains {\small\verb%u%} in the sense described above (conjunction excluded).

\USES
{\small\verb%FILTER_PSTRIP_TAC%} is used when stripping outer connectives from a goal in a
more delicate way than {\small\verb%PSTRIP_TAC%}. A typical application is to keep
stripping by using the tactic {\small\verb%REPEAT (FILTER_PSTRIP_TAC u)%}
until one hits the term {\small\verb%u%} at which stripping is to stop.

\SEEALSO
PGEN_TAC, PSTRIP_GOAL_THEN, FILTER_PSTRIP_THEN, PSTRIP_TAC, FILTER_STRIP_TAC.

\ENDDOC
\DOC{FILTER\_PSTRIP\_THEN}

\TYPE {\small\verb%FILTER_PSTRIP_THEN : (thm_tactic -> term -> tactic)%}\egroup

\SYNOPSIS
Conditionally strips a goal, handing an antecedent to the theorem-tactic.

\DESCRIBE
Given a theorem-tactic {\small\verb%ttac%}, a term {\small\verb%u%} and a goal {\small\verb%(A,t)%},
{\small\verb%FILTER_STRIP_THEN ttac u%} removes one outer connective ({\small\verb%!%}, {\small\verb%==>%}, or {\small\verb%~%})
from {\small\verb%t%}, if the term being stripped does not contain a free instance of {\small\verb%u%}.
Note that {\small\verb%FILTER_PSTRIP_THEN%} will strip paired universal quantifiers.
A negation {\small\verb%~t%} is treated as the implication {\small\verb%t ==> F%}. The theorem-tactic
{\small\verb%ttac%} is applied only when stripping an implication, by using the antecedent
stripped off. {\small\verb%FILTER_PSTRIP_THEN%} also breaks conjunctions.

{\small\verb%FILTER_PSTRIP_THEN%} behaves like {\small\verb%PSTRIP_GOAL_THEN%}, if the term being stripped
does not contain a free instance of {\small\verb%u%}. In particular, {\small\verb%FILTER_PSTRIP_THEN
PSTRIP_ASSUME_TAC%} behaves like {\small\verb%FILTER_PSTRIP_TAC%}.

\FAILURE
{\small\verb%FILTER_PSTRIP_THEN ttac u (A,t)%} fails if {\small\verb%t%} is not a paired universally
quantified term, an implication, a negation or a conjunction;
or if the term being stripped contains the term {\small\verb%u%} (conjunction excluded);
or if the application of {\small\verb%ttac%} fails, after stripping the goal.

\USES
{\small\verb%FILTER_PSTRIP_THEN%} is used to manipulate intermediate results
using theorem-tactics, after stripping outer connectives
from a goal in a more delicate way than {\small\verb%PSTRIP_GOAL_THEN%}.

\SEEALSO
PGEN_TAC, PSTRIP_GOAL_THEN, FILTER_STRIP_THEN, PSTRIP_TAC, FILTER_PSTRIP_TAC.

\ENDDOC
\DOC{genlike}

\TYPE {\small\verb%genlike : (term -> term)%}\egroup

\SYNOPSIS
Returns a pair structure of variables whose names have not been previously used.

\DESCRIBE
When given a pair structure,
{\small\verb%genlike%} returns a paired structure of variables whose names have not been 
used for variables or constants in the HOL session so far.
The structure of the term returned will be identical to the structure of the
argument.

\FAILURE
Never fails.

\EXAMPLE
The following example illustrates the behaviour of {\small\verb%genlike%}:
{\par\samepage\setseps\small
\begin{verbatim}
   #genlike "((1,2),(x:*,x:*))";;
   "(GEN%VAR%487,GEN%VAR%488),GEN%VAR%489,GEN%VAR%490" : term
\end{verbatim}
}

\USES
Unique variables are useful in writing derived rules, for specializing
terms without having to worry about such things as free variable capture.
It is often important in such rules to keep the same structure.
If not, {\small\verb%genvar%} will be adequate.
If the names are to be visible to a typical user, the function {\small\verb%pvariant%} can
provide rather more meaningful names.

\SEEALSO
genvar, GPSPEC, pvariant.

\ENDDOC
\DOC{GEN\_PALPHA\_CONV}

\TYPE {\small\verb%GEN_PALPHA_CONV : (term -> conv)%}\egroup

\SYNOPSIS
Renames the bound pair of a paired abstraction, quantified term,
or other binder.

\DESCRIBE
The conversion {\small\verb%GEN_PALPHA_CONV%} provides alpha conversion for lambda
abstractions of the form {\small\verb%"\p.t"%}, quantified terms of the forms {\small\verb%"!p.t"%},
{\small\verb%"?p.t"%} or {\small\verb%"?!p.t"%}, and epsilon terms of the form {\small\verb%"@p.t"%}.  In general,
if {\small\verb%B%} is a binder constant, then {\small\verb%GEN_ALPHA_CONV%} implements alpha
conversion for applications of the form {\small\verb%"B p.t"%}.  
The function {\small\verb%is_binder%} determines what is regarded as a binder in this 
context.

The renaming of pairs is as described for {\small\verb%PALPHA_CONV%}.

\FAILURE
{\small\verb%GEN_PALPHA_CONV q tm%} fails if {\small\verb%q%} is not a variable,
or if {\small\verb%tm%} does not have one of the forms {\small\verb%"\p.t"%} or {\small\verb%"B p.t"%},
where {\small\verb%B%} is a binder (that is, {\small\verb%is_binder `B`%} returns {\small\verb%true%}).
{\small\verb%GEN_ALPHA_CONV q tm%} also fails if {\small\verb%tm%} does have one of these forms,
but types of the variables {\small\verb%p%} and {\small\verb%q%} differ.

\SEEALSO
GEN_ALPHA_CONV, PALPHA, PALPHA_CONV, is_binder.

\ENDDOC
\DOC{GPSPEC}

\TYPE {\small\verb%GPSPEC : (thm -> thm)%}\egroup

\SYNOPSIS
Specializes the conclusion of a theorem with unique pairs.

\DESCRIBE
When applied to a theorem {\small\verb%A |- !p1...pn. t%}, where the number of universally
quantified variables may be zero, {\small\verb%GPSPEC%} returns {\small\verb%A |- t[g1/p1]...[gn/pn]%},
where the {\small\verb%gi%} is paired structures of the same structure as {\small\verb%pi%} and
made up of distinct variables , chosen by {\small\verb%genvar%}.
{\par\samepage\setseps\small
\begin{verbatim}
        A |- !p1...pn. t
   -------------------------  GPSPEC
    A |- t[g1/p1]...[gn/pn]
\end{verbatim}
}
\FAILURE
Never fails.

\USES
{\small\verb%GPSPEC%} is useful in writing derived inference rules which need to specialize
theorems while avoiding using any variables that may be present elsewhere.

\SEEALSO
GSPEC, PGEN, PGENL, genvar, PGEN_ALL, PGEN_TAC, PSPEC, PSPECL, PSPEC_ALL,
PSPEC_TAC, PSPEC_PAIR.

\ENDDOC
\DOC{HALF\_MK\_PABS}

\TYPE {\small\verb%HALF_MK_PABS : (thm -> thm)%}\egroup

\SYNOPSIS
Converts a function definition to lambda-form.

\DESCRIBE
When applied to a theorem {\small\verb%A |- !p. t1 p = t2%}, whose conclusion is a
universally quantified equation, {\small\verb%HALF_MK_PABS%} returns the theorem
{\small\verb%A |- t1 = (\p. t2)%}.
{\par\samepage\setseps\small
\begin{verbatim}
    A |- !p. t1 p = t2
   --------------------  HALF_MK_PABS            [where p is not free in t1]
    A |- t1 = (\p. t2)
\end{verbatim}
}
\FAILURE
Fails unless the theorem is a singly paired universally quantified equation 
whose left-hand side is a function applied to the quantified pair,
or if any of the the variables in the quantified pair is free in that function.

\SEEALSO
HALF_MK_ABS, PETA_CONV, MK_PABS, MK_PEXISTS.

\ENDDOC
\DOC{IPSPEC}

\TYPE {\small\verb%IPSPEC : (term -> thm -> thm)%}\egroup

\SYNOPSIS
Specializes a theorem, with type instantiation if necessary.

\DESCRIBE
This rule specializes a paired quantification as does {\small\verb%PSPEC%};
it differs from it in also instantiating the type if needed:
{\par\samepage\setseps\small
\begin{verbatim}
     A |- !p:ty.tm
  -----------------------  IPSPEC "q:ty'"
      A |- tm[q/p]
\end{verbatim}
}
\noindent (where {\small\verb%q%} is free for {\small\verb%p%} in {\small\verb%tm%}, and {\small\verb%ty'%} is an instance
of {\small\verb%ty%}).

\FAILURE
{\small\verb%IPSPEC%} fails if the input theorem is not universally quantified, if
the type of the given term is not an instance of the type of the
quantified variable, or if the type variable is free in the
assumptions.

\SEEALSO
ISPEC, INST_TY_TERM, INST_TYPE, IPSPECL, PSPEC, match.

\ENDDOC
\DOC{IPSPECL}

\TYPE {\small\verb%IPSPECL : (term list -> thm -> thm)%}\egroup

\SYNOPSIS
Specializes a theorem zero or more times, with type instantiation if necessary.

\DESCRIBE
{\small\verb%IPSPECL%} is an iterative version of {\small\verb%IPSPEC%}
{\par\samepage\setseps\small
\begin{verbatim}
         A |- !p1...pn.tm
   ----------------------------  IPSPECL ["q1",...,"qn"]
    A |- t[q1,...qn/p1,...,pn]
\end{verbatim}
}
\noindent (where {\small\verb%qi%} is free for {\small\verb%pi%} in {\small\verb%tm%}).

\FAILURE
{\small\verb%IPSPECL%} fails if the list of terms is longer than the number of
quantified variables in the term, if the type instantiation fails, or
if the type variable being instantiated is free in the assumptions.

\SEEALSO
ISPECL, INST_TYPE, INST_TY_TERM, IPSPEC, MATCH, SPEC, PSPECL.

\ENDDOC
\DOC{is\_pabs}

\TYPE {\small\verb%is_pabs : (term -> bool)%}\egroup

\SYNOPSIS
Tests a term to see if it is a paired abstraction.

\DESCRIBE
{\small\verb%is_pabs "\pair. t"%} returns {\small\verb%true%}.
If the term is not a paired abstraction the result is {\small\verb%false%}.

\FAILURE
Never fails.

\SEEALSO
is_abs, mk_pabs, dest_pabs.

\ENDDOC
\DOC{is\_pexists}

\TYPE {\small\verb%is_pexists : (term -> bool)%}\egroup

\SYNOPSIS
Tests a term to see if it as a paired existential quantification.

\DESCRIBE
{\small\verb%is_pexists "?pair. t"%} returns {\small\verb%true%}.
If the term is not a paired existential quantification the result is {\small\verb%false%}.

\FAILURE
Never fails.

\SEEALSO
is_exists, mk_pexists, dest_pexists.

\ENDDOC
\DOC{is\_pforall}

\TYPE {\small\verb%is_pforall : (term -> bool)%}\egroup

\SYNOPSIS
Tests a term to see if it is a paired universal quantification.

\DESCRIBE
{\small\verb%is_pforall "!pair. t"%} returns {\small\verb%true%}.
If the term is not a a paired universal quantification the result is {\small\verb%false%}.

\FAILURE
Never fails.

\SEEALSO
is_forall, mk_pforall, dest_pforall.

\ENDDOC
\DOC{is\_prod}

\TYPE {\small\verb%is_prod : (type -> bool)%}\egroup

\SYNOPSIS
Tests a type to see if it is a product type.

\DESCRIBE
{\small\verb%is_prod ":t1#t2"%} returns {\small\verb%true%}.

\FAILURE
Never fails.

\SEEALSO
dest_prod, mk_prod.

\ENDDOC
\DOC{is\_pselect}

\TYPE {\small\verb%is_pselect : (term -> bool)%}\egroup

\SYNOPSIS
Tests a term to see if it is a paired choice-term.

\DESCRIBE
{\small\verb%is_select "@pair. t"%} returns {\small\verb%true%}.
If the term is not a paired choice-term the result is {\small\verb%false%}.

\FAILURE
Never fails.

\SEEALSO
is_select, mk_pselect, dest_pselect.

\ENDDOC
\DOC{is\_pvar}

\TYPE {\small\verb%is_pvar : (term -> bool)%}\egroup

\SYNOPSIS
Tests a term to see if it is a paired structure of variables.

\DESCRIBE
{\small\verb%is_pvar "pvar"%} returns {\small\verb%true%} iff {\small\verb%pvar%} is a paired structure of variables.
For example, {\small\verb%((a:*,b:*),(d:*,e:*))%} is a paired structure of variables,
{\small\verb%(1,2)%} is not.

\FAILURE
Never fails.

\SEEALSO
is_var.

\ENDDOC
\DOC{LEFT\_AND\_PEXISTS\_CONV}

\TYPE {\small\verb%LEFT_AND_PEXISTS_CONV : conv%}\egroup

\SYNOPSIS
Moves a paired existential quantification of the left conjunct outwards
through a conjunction.

\DESCRIBE
When applied to a term of the form {\small\verb%(?p. t) /\ u%}, the conversion
{\small\verb%LEFT_AND_PEXISTS_CONV%} returns the theorem:
{\par\samepage\setseps\small
\begin{verbatim}
   |- (?p. t) /\ u = (?p'. t[p'/p] /\ u)
\end{verbatim}
}
\noindent where {\small\verb%p'%} is a primed variant of the pair {\small\verb%p%} that does not 
contains variables free in the input term.

\FAILURE
Fails if applied to a term not of the form {\small\verb%(?p. t) /\ u%}.

\SEEALSO
LEFT_AND_EXISTS_CONV, AND_PEXISTS_CONV, PEXISTS_AND_CONV,
RIGHT_AND_PEXISTS_CONV.

\ENDDOC
\DOC{LEFT\_AND\_PFORALL\_CONV}

\TYPE {\small\verb%LEFT_AND_PFORALL_CONV : conv%}\egroup

\SYNOPSIS
Moves a paired universal quantification of the left conjunct outwards through
a conjunction.

\DESCRIBE
When applied to a term of the form {\small\verb%(!p. t) /\ u%}, the conversion
{\small\verb%LEFT_AND_PFORALL_CONV%} returns the theorem:
{\par\samepage\setseps\small
\begin{verbatim}
   |- (!p. t) /\ u = (!p'. t[p'/p] /\ u)
\end{verbatim}
}
\noindent where {\small\verb%p'%} is a primed variant of {\small\verb%p%} that does not appear free in
the input term.

\FAILURE
Fails if applied to a term not of the form {\small\verb%(!p. t) /\ u%}.

\SEEALSO
LEFT_AND_FORALL_CONV, AND_PFORALL_CONV, PFORALL_AND_CONV,
RIGHT_AND_PFORALL_CONV.

\ENDDOC
\DOC{LEFT\_IMP\_PEXISTS\_CONV}

\TYPE {\small\verb%LEFT_IMP_PEXISTS_CONV : conv%}\egroup

\SYNOPSIS
Moves a paired existential quantification of the antecedent outwards through
an implication.

\DESCRIBE
When applied to a term of the form {\small\verb%(?p. t) ==> u%}, the conversion
{\small\verb%LEFT_IMP_PEXISTS_CONV%} returns the theorem:
{\par\samepage\setseps\small
\begin{verbatim}
   |- (?p. t) ==> u = (!p'. t[p'/p] ==> u)
\end{verbatim}
}
\noindent where {\small\verb%p'%} is a primed variant of the pair {\small\verb%p%} that does not
contain any variables that appear free in the input term.

\FAILURE
Fails if applied to a term not of the form {\small\verb%(?p. t) ==> u%}.

\SEEALSO
LEFT_IMP_EXISTS_CONV, PFORALL_IMP_CONV, RIGHT_IMP_PFORALL_CONV.

\ENDDOC
\DOC{LEFT\_IMP\_PFORALL\_CONV}

\TYPE {\small\verb%LEFT_IMP_PFORALL_CONV : conv%}\egroup

\SYNOPSIS
Moves a paired universal quantification of the antecedent outwards through an
implication.

\DESCRIBE
When applied to a term of the form {\small\verb%(!p. t) ==> u%}, the conversion
{\small\verb%LEFT_IMP_PFORALL_CONV%} returns the theorem:
{\par\samepage\setseps\small
\begin{verbatim}
   |- (!p. t) ==> u = (?p'. t[p'/p] ==> u)
\end{verbatim}
}
\noindent where {\small\verb%p'%} is a primed variant of the pair {\small\verb%p%} that does not
contain any variables that appear free in the input term.

\FAILURE
Fails if applied to a term not of the form {\small\verb%(!p. t) ==> u%}.

\SEEALSO
LEFT_IMP_FORALL_CONV, PEXISTS_IMP_CONV, RIGHT_IMP_PFORALL_CONV.

\ENDDOC
\DOC{LEFT\_LIST\_PBETA}

\TYPE {\small\verb%LEFT_LIST_PBETA : (thm -> thm)%}\egroup

\SYNOPSIS
Iteratively beta-reduces a top-level paired beta-redex on the left-hand side
of an equation.

\DESCRIBE
When applied to an equational theorem, {\small\verb%LEFT_LIST_PBETA%} applies paired 
beta-reduction over a top-level chain of beta-redexes to the left-hand side
(only).
Variables are renamed if necessary to avoid free variable capture.
{\par\samepage\setseps\small
\begin{verbatim}
    A |- (\p1...pn. t) q1 ... qn = s
   ----------------------------------  LEFT_LIST_BETA
       A |- t[q1/p1]...[qn/pn] = s
\end{verbatim}
}
\FAILURE
Fails unless the theorem is equational, with its left-hand side being
a top-level paired beta-redex.

\SEEALSO
RIGHT_LIST_BETA, PBETA_CONV, PBETA_RULE, PBETA_TAC, LIST_PBETA_CONV,
LEFT_PBETA, RIGHT_PBETA, RIGHT_LIST_PBETA.

\ENDDOC
\DOC{LEFT\_OR\_PEXISTS\_CONV}

\TYPE {\small\verb%LEFT_OR_PEXISTS_CONV : conv%}\egroup

\SYNOPSIS
Moves a paired existential quantification of the left disjunct outwards
through a disjunction.

\DESCRIBE
When applied to a term of the form {\small\verb%(?p. t) \/ u%}, the conversion
{\small\verb%LEFT_OR_PEXISTS_CONV%} returns the theorem:
{\par\samepage\setseps\small
\begin{verbatim}
   |- (?p. t) \/ u = (?p'. t[p'/p] \/ u)
\end{verbatim}
}
\noindent where {\small\verb%p'%} is a primed variant of the pair {\small\verb%p%} that does not 
contain any variables free in the input term.

\FAILURE
Fails if applied to a term not of the form {\small\verb%(?p. t) \/ u%}.

\SEEALSO
LEFT_OR_EXISTS_CONV, PEXISTS_OR_CONV, OR_PEXISTS_CONV, RIGHT_OR_PEXISTS_CONV.

\ENDDOC
\DOC{LEFT\_OR\_PFORALL\_CONV}

\TYPE {\small\verb%LEFT_OR_PFORALL_CONV : conv%}\egroup

\SYNOPSIS
Moves a paired universal quantification of the left disjunct outwards through
a disjunction.

\DESCRIBE
When applied to a term of the form {\small\verb%(!p. t) \/ u%}, the conversion
{\small\verb%LEFT_OR_FORALL_CONV%} returns the theorem:
{\par\samepage\setseps\small
\begin{verbatim}
   |- (!p. t) \/ u = (!p'. t[p'/p] \/ u)
\end{verbatim}
}
\noindent where {\small\verb%p'%} is a primed variant of the pair {\small\verb%p%} that does not
contain any variables that appear free in the input term.

\FAILURE
Fails if applied to a term not of the form {\small\verb%(!p. t) \/ u%}.

\SEEALSO
LEFT_OR_FORALL_CONV, OR_PFORALL_CONV, PFORALL_OR_CONV, RIGHT_OR_PFORALL_CONV.

\ENDDOC
\DOC{LEFT\_PBETA}

\TYPE {\small\verb%LEFT_PBETA : (thm -> thm)%}\egroup

\SYNOPSIS
Beta-reduces a top-level paired beta-redex on the left-hand side of an equation.

\DESCRIBE
When applied to an equational theorem, {\small\verb%LEFT_PBETA%} applies paired
beta-reduction at top level to the left-hand side (only).
Variables are renamed if necessary to avoid free variable capture.
{\par\samepage\setseps\small
\begin{verbatim}
    A |- (\x. t1) t2 = s
   ----------------------  LEFT_PBETA
     A |- t1[t2/x] = s
\end{verbatim}
}
\FAILURE
Fails unless the theorem is equational, with its left-hand side being
a top-level paired beta-redex.

\SEEALSO
RIGHT_BETA, PBETA_CONV, PBETA_RULE, PBETA_TAC, RIGHT_PBETA, RIGHT_LIST_PBETA,
LEFT_LIST_PBETA.

\ENDDOC
\DOC{list\_mk\_pabs}

\TYPE {\small\verb%list_mk_pabs : ((term list # term) -> term)%}\egroup

\SYNOPSIS
Iteratively constructs paired abstractions.

\DESCRIBE
{\small\verb%list_mk_pabs(["p1";...;"pn"],"t")%} returns {\small\verb%"\p1 ... pn. t"%}.

\FAILURE
Fails with {\small\verb%list_mk_pabs%} if the terms in the list are not paired 
structures of variables.

\COMMENTS
The system shows the type as {\small\verb%goal -> term%}.

\SEEALSO
list_mk_abs, strip_pabs, mk_pabs.

\ENDDOC
\DOC{LIST\_MK\_PEXISTS}

\TYPE {\small\verb%LIST_MK_PEXISTS : (term list -> thm -> thm)%}\egroup

\SYNOPSIS
Multiply existentially quantifies both sides of an equation using the given
pairs.

\DESCRIBE
When applied to a list of terms {\small\verb%[p1;...;pn]%}, where the {\small\verb%pi%} are all
paired structures of variables, and a theorem {\small\verb%A |- t1 = t2%},
the inference rule {\small\verb%LIST_MK_PEXISTS%} existentially quantifies both sides
of the equation using the pairs given, none of the variables in the pairs
should be free in the assumption list.
{\par\samepage\setseps\small
\begin{verbatim}
                A |- t1 = t2
   --------------------------------------  LIST_MK_PEXISTS ["x1";...;"xn"]
    A |- (?x1...xn. t1) = (?x1...xn. t2)
\end{verbatim}
}
\FAILURE
Fails if any term in the list is not a paired structure of variables,
or if any variable is free in the assumption list,
or if the theorem is not equational.

\SEEALSO
LIST_MK_EXISTS, PEXISTS_EQ, MK_PEXISTS.

\ENDDOC
\DOC{list\_mk\_pexists}

\TYPE {\small\verb%list_mk_pexists : ((term list # term) -> term)%}\egroup

\SYNOPSIS
Iteratively constructs paired existential quantifications.

\DESCRIBE
{\small\verb%list_mk_pexists(["p1";...;"pn"],"t")%} returns {\small\verb%"?p1 ... pn. t"%}.

\FAILURE
Fails with {\small\verb%list_mk_pexists%} if the terms in the list are not paired structures
of variables or if {\small\verb%t%} is not of type {\small\verb%":bool"%} and the list of terms is 
nonempty.
If the list of terms is empty the type of {\small\verb%t%} can be anything.

\COMMENTS
The system shows the type as {\small\verb%(goal -> term)%}.

\SEEALSO
list_mk_exists, strip_pexists, mk_pexists.

\ENDDOC
\DOC{LIST\_MK\_PFORALL}

\TYPE {\small\verb%LIST_MK_PFORALL : (term list -> thm -> thm)%}\egroup

\SYNOPSIS
Multiply universally quantifies both sides of an equation using the given
pairs.

\DESCRIBE
When applied to a list of terms {\small\verb%[p1;...;pn]%}, where the {\small\verb%pi%} are all
paired structures of variables, and a theorem {\small\verb%A |- t1 = t2%},
the inference rule {\small\verb%LIST_MK_PFORALL%} universally quantifies both sides
of the equation using the pairs given, none of the variables in the pairs
should be free in the assumption list.
{\par\samepage\setseps\small
\begin{verbatim}
                A |- t1 = t2
   --------------------------------------  LIST_MK_PFORALL ["x1";...;"xn"]
    A |- (!x1...xn. t1) = (!x1...xn. t2)
\end{verbatim}
}
\FAILURE
Fails if any term in the list is not a paired structure of variables,
or if any variable is free in the assumption list,
or if the theorem is not equational.

\SEEALSO
LIST_MK_EXISTS, PFORALL_EQ, MK_PFORALL.

\ENDDOC
\DOC{list\_mk\_pforall}

\TYPE {\small\verb%list_mk_pforall : ((term list # term) -> term)%}\egroup

\SYNOPSIS
Iteratively constructs a paired universal quantification.

\DESCRIBE
{\small\verb%list_mk_pforall(["p1";...;"pn"],"t")%} returns {\small\verb%"!p1 ... pn. t"%}.

\FAILURE
Fails with {\small\verb%list_mk_pforall%} if the terms in the list are not paired structures
of variables or if {\small\verb%t%} is not of type {\small\verb%":bool"%} and the list of terms is 
nonempty.
If the list of terms is empty the type of {\small\verb%t%} can be anything.

\COMMENTS
The system shows the type as {\small\verb%(goal -> term)%}.

\SEEALSO
list_mk_forall, strip_pforall, mk_pforall.

\ENDDOC
\DOC{LIST\_PBETA\_CONV}

\TYPE {\small\verb%LIST_PBETA_CONV : conv%}\egroup

\SYNOPSIS
Performs an iterated paired beta-conversion.

\DESCRIBE
The conversion {\small\verb%LIST_PBETA_CONV%} maps terms of the form
{\par\samepage\setseps\small
\begin{verbatim}
   "(\p1 p2 ... pn. t) q1 q2 ... qn"
\end{verbatim}
}
\noindent to the theorems of the form
{\par\samepage\setseps\small
\begin{verbatim}
   |- (\p1 p2 ... pn. t) q1 q2 ... qn = t[q1/p1][q2/p2] ... [qn/pn]
\end{verbatim}
}
\noindent where {\small\verb%t[qi/pi]%} denotes the result of substituting {\small\verb%qi%} for all free
occurrences of {\small\verb%pi%} in {\small\verb%t%}, after renaming sufficient bound variables to avoid
variable capture.

\FAILURE
{\small\verb%LIST_PBETA_CONV tm%} fails if {\small\verb%tm%} does not have the form
{\small\verb%"(\p1 ... pn. t) q1 ... qn"%} for {\small\verb%n%} greater than 0.

\EXAMPLE
{\par\samepage\setseps\small
\begin{verbatim}
#LIST_PBETA_CONV "(\(a,b) (c,d) . a + b + c + d) (1,2) (3,4)";;
|- (\(a,b) (c,d). a + (b + (c + d)))(1,2)(3,4) = 1 + (2 + (3 + 4))
\end{verbatim}
}

\SEEALSO
LIST_BETA_CONV, PBETA_CONV, BETA_RULE, BETA_TAC, RIGHT_PBETA, RIGHT_LIST_PBETA,
LEFT_PBETA, LEFT_LIST_PBETA.

\ENDDOC
\DOC{MK\_PABS}

\TYPE {\small\verb%MK_PABS : (thm -> thm)%}\egroup

\SYNOPSIS
Abstracts both sides of an equation.

\DESCRIBE
When applied to a theorem {\small\verb%A |- !p. t1 = t2%}, whose conclusion is a
paired universally quantified equation, {\small\verb%MK_PABS%} returns the theorem
{\small\verb%A |- (\p. t1) = (\p. t2)%}.
{\par\samepage\setseps\small
\begin{verbatim}
        A |- !p. t1 = t2
   --------------------------  MK_PABS
    A |- (\p. t1) = (\p. t2)
\end{verbatim}
}
\FAILURE
Fails unless the theorem is a (singly) paired universally quantified equation.

\SEEALSO
MK_ABS, PABS, HALF_MK_PABS, MK_PEXISTS.

\ENDDOC
\DOC{mk\_pabs}

\TYPE {\small\verb%mk_pabs : ((term # term) -> term)%}\egroup

\SYNOPSIS
Constructs a paired abstraction.

\DESCRIBE
{\small\verb%mk_pabs "pair","t"%} returns the abstraction {\small\verb%"\pair. t"%}.

\FAILURE
Fails with {\small\verb%mk_pabs%} if first term is not a pair structure of variables.

\SEEALSO
mk_abs, dest_pabs, is_pabs, list_mk_pabs.

\ENDDOC
\DOC{MK\_PAIR}

\TYPE {\small\verb%MK_PAIR : (thm -> thm -> thm)%}\egroup

\SYNOPSIS
Proves equality of pairs constructed from equal components.

\DESCRIBE
When applied to theorems {\small\verb%A1 |- a = x%} and {\small\verb%A2 |- b = y%}, the inference
rule {\small\verb%MK_PAIR%} returns the theorem {\small\verb%A1 u A2 |- (a,b) = (x,y)%}.
{\par\samepage\setseps\small
\begin{verbatim}
    A1 |- a = x   A2 |- b = y
   ---------------------------  MK_PAIR
    A1 u A2 |- (a,b) = (x,y)
\end{verbatim}
}
\FAILURE
Fails unless both theorems are equational.

\SEEALSO

\ENDDOC
\DOC{MK\_PEXISTS}

\TYPE {\small\verb%MK_PEXISTS : (thm -> thm)%}\egroup

\SYNOPSIS
Existentially quantifies both sides of a universally quantified
equational theorem.

\DESCRIBE
When applied to a theorem {\small\verb%A |- !p. t1 = t2%}, the inference rule {\small\verb%MK_PEXISTS%}
returns the theorem {\small\verb%A |- (?x. t1) = (?x. t2)%}.
{\par\samepage\setseps\small
\begin{verbatim}
       A |- !p. t1 = t2
   --------------------------  MK_PEXISTS
    A |- (?p. t1) = (?p. t2)
\end{verbatim}
}
\FAILURE
Fails unless the theorem is a singly paired universally quantified equation.

\SEEALSO
PEXISTS_EQ, PGEN, LIST_MK_PEXISTS, MK_PABS.

\ENDDOC
\DOC{mk\_pexists}

\TYPE {\small\verb%mk_pexists : ((term # term) -> term)%}\egroup

\SYNOPSIS
Constructs a paired existential quantification.

\DESCRIBE
{\small\verb%mk_pexists("pair","t")%} returns {\small\verb%"?pair. t"%}.

\FAILURE
Fails with {\small\verb%mk_exists%} if first term is not a paired structure of variables or
if {\small\verb%t%} is not of type {\small\verb%":bool"%}.

\SEEALSO
mk_exists, dest_pexists, is_pexists, list_mk_pexists.

\ENDDOC
\DOC{MK\_PFORALL}

\TYPE {\small\verb%MK_PFORALL : (thm -> thm)%}\egroup

\SYNOPSIS
Universally quantifies both sides of a universally quantified
equational theorem.

\DESCRIBE
When applied to a theorem {\small\verb%A |- !p. t1 = t2%}, the inference rule {\small\verb%MK_PFORALL%}
returns the theorem {\small\verb%A |- (!x. t1) = (!x. t2)%}.
{\par\samepage\setseps\small
\begin{verbatim}
       A |- !p. t1 = t2
   --------------------------  MK_PFORALL
    A |- (!p. t1) = (!p. t2)
\end{verbatim}
}
\FAILURE
Fails unless the theorem is a singly paired universally quantified equation.

\SEEALSO
PFORALL_EQ, LIST_MK_PFORALL, MK_PABS.

\ENDDOC
\DOC{mk\_pforall}

\TYPE {\small\verb%mk_pforall : ((term # term) -> term)%}\egroup

\SYNOPSIS
Constructs a paired universal quantification.

\DESCRIBE
{\small\verb%mk_pforall("pair","t")%} returns {\small\verb%"!pair. t"%}.

\FAILURE
Fails with {\small\verb%mk_pforall%} if first term is not a a paired structure of variables
or if {\small\verb%t%} is not of type {\small\verb%":bool"%}.

\SEEALSO
mk_forall, dest_pforall, is_pforall, list_mk_pforall.

\ENDDOC
\DOC{mk\_prod}

\TYPE {\small\verb%mk_prod : ((type # type) -> type)%}\egroup

\SYNOPSIS
Constructs a product type from two constituent types.

\DESCRIBE
{\small\verb%mk_prod(":t1",":t2")%} returns {\small\verb%":t1#t2"%}.

\FAILURE
Never fails.

\SEEALSO
is_prod, dest_prod.

\ENDDOC
\DOC{MK\_PSELECT}

\TYPE {\small\verb%MK_PSELECT : (thm -> thm)%}\egroup

\SYNOPSIS
Quantifies both sides of a universally quantified equational theorem with
the choice quantifier.

\DESCRIBE
When applied to a theorem {\small\verb%A |- !p. t1 = t2%}, the inference rule {\small\verb%MK_PSELECT%}
returns the theorem {\small\verb%A |- (@x. t1) = (@x. t2)%}.
{\par\samepage\setseps\small
\begin{verbatim}
       A |- !p. t1 = t2
   --------------------------  MK_PSELECT
    A |- (@p. t1) = (@p. t2)
\end{verbatim}
}
\FAILURE
Fails unless the theorem is a singly paired universally quantified equation.

\SEEALSO
PSELECT_EQ, MK_PABS.

\ENDDOC
\DOC{mk\_pselect}

\TYPE {\small\verb%mk_pselect : ((term # term) -> term)%}\egroup

\SYNOPSIS
Constructs a paired choice-term.

\DESCRIBE
{\small\verb%mk_pselect("pair","t")%} returns {\small\verb%"@pair. t"%}.

\FAILURE
Fails with {\small\verb%mk_select%} if first term is not a paired structure of variables or
if {\small\verb%t%} is not of type {\small\verb%":bool"%}.

\SEEALSO
mk_select, dest_pselect, is_pselect.

\ENDDOC
\DOC{NOT\_PEXISTS\_CONV}

\TYPE {\small\verb%NOT_PEXISTS_CONV : conv%}\egroup

\SYNOPSIS
Moves negation inwards through a paired existential quantification.

\DESCRIBE
When applied to a term of the form {\small\verb%~(?p. t)%}, the conversion
{\small\verb%NOT_PEXISTS_CONV%} returns the theorem:
{\par\samepage\setseps\small
\begin{verbatim}
   |- ~(?p. t) = (!p. ~t)
\end{verbatim}
}
\FAILURE
Fails if applied to a term not of the form {\small\verb%~(?p. t)%}.

\SEEALSO
NOT_EXISTS_CONV, PEXISTS_NOT_CONV, PFORALL_NOT_CONV, NOT_PFORALL_CONV.

\ENDDOC
\DOC{NOT\_PFORALL\_CONV}

\TYPE {\small\verb%NOT_PFORALL_CONV : conv%}\egroup

\SYNOPSIS
Moves negation inwards through a paired universal quantification.

\DESCRIBE
When applied to a term of the form {\small\verb%~(!p. t)%}, the conversion
{\small\verb%NOT_PFORALL_CONV%} returns the theorem:
{\par\samepage\setseps\small
\begin{verbatim}
   |- ~(!p. t) = (?p. ~t)
\end{verbatim}
}
\noindent It is irrelevant whether any variables in {\small\verb%p%} occur free in {\small\verb%t%}.

\FAILURE
Fails if applied to a term not of the form {\small\verb%~(!p. t)%}.

\SEEALSO
NOT_FORALL_CONV, PEXISTS_NOT_CONV, PFORALL_NOT_CONV, NOT_PEXISTS_CONV.

\ENDDOC
\DOC{occs\_in}

\TYPE {\small\verb%occs_in : (term -> term -> bool)%}\egroup

\SYNOPSIS
Occurrence check for bound variables.

\DESCRIBE
When applied to two terms {\small\verb%p%} and {\small\verb%t%},
where {\small\verb%p%} is a paired structure of variables,
the function {\small\verb%occs_in%} returns {\small\verb%true%} if and of the constituent variables of
{\small\verb%p%} occurs free in {\small\verb%t%},
and {\small\verb%false%} otherwise.

\FAILURE
Fails of {\small\verb%p%} is not a paired structure of variables.

\SEEALSO
free_in, frees, freesl, thm_frees.

\ENDDOC
\DOC{OR\_PEXISTS\_CONV}

\TYPE {\small\verb%OR_PEXISTS_CONV : conv%}\egroup

\SYNOPSIS
Moves a paired existential quantification outwards through a disjunction.

\DESCRIBE
When applied to a term of the form {\small\verb%(?p. t) \/ (?p. u)%}, the conversion
{\small\verb%OR_PEXISTS_CONV%} returns the theorem:
{\par\samepage\setseps\small
\begin{verbatim}
   |- (?p. t) \/ (?p. u) = (?p. t \/ u)
\end{verbatim}
}
\FAILURE
Fails if applied to a term not of the form {\small\verb%(?p. t) \/ (?p. u)%}.

\SEEALSO
OR_EXISTS_CONV, PEXISTS_OR_CONV, LEFT_OR_PEXISTS_CONV, RIGHT_OR_PEXISTS_CONV.

\ENDDOC
\DOC{OR\_PFORALL\_CONV}

\TYPE {\small\verb%OR_PFORALL_CONV : conv%}\egroup

\SYNOPSIS
Moves a paired universal quantification outwards through a disjunction.

\DESCRIBE
When applied to a term of the form {\small\verb%(!p. t) \/ (!p. u)%}, where no variables
from {\small\verb%p%} are free in either {\small\verb%t%} nor {\small\verb%u%}, {\small\verb%OR_PFORALL_CONV%} returns the theorem:
{\par\samepage\setseps\small
\begin{verbatim}
   |- (!p. t) \/ (!p. u) = (!p. t \/ u)
\end{verbatim}
}
\FAILURE
{\small\verb%OR_PFORALL_CONV%} fails if it is applied to a term not of the form
{\small\verb%(!p. t) \/ (!p. u)%}, or if it is applied to a term {\small\verb%(!p. t) \/ (!p. u)%}
in which the variables from {\small\verb%p%} are free in either {\small\verb%t%} or {\small\verb%u%}.

\SEEALSO
OR_FORALL_CONV, PFORALL_OR_CONV, LEFT_OR_PFORALL_CONV, RIGHT_OR_PFORALL_CONV.

\ENDDOC
\DOC{PABS}

\TYPE {\small\verb%PABS : (term -> thm -> thm)%}\egroup

\SYNOPSIS
Paired abstraction of both sides of an equation.

\DESCRIBE
{\par\samepage\setseps\small
\begin{verbatim}
         A |- t1 = t2
   ------------------------  ABS "p"            [Where p is not free in A]
    A |- (\p.t1) = (\p.t2)
\end{verbatim}
}
\FAILURE
If the theorem is not an equation, or if any variable in the paired structure
of variables {\small\verb%p%} occurs free in the assumptions {\small\verb%A%}.

EXAMPLE
{\par\samepage\setseps\small
\begin{verbatim}
#PABS "(x:*,y:**)" (REFL "(x:*,y:**)");;
|- (\(x,y). (x,y)) = (\(x,y). (x,y))
\end{verbatim}
}
\SEEALSO
ABS, PABS_CONV, PETA_CONV, PEXT, MK_PABS.

\ENDDOC
\DOC{PABS\_CONV}

\TYPE {\small\verb%PABS_CONV : (conv -> conv)%}\egroup

\SYNOPSIS
Applies a conversion to the body of a paired abstraction.

\DESCRIBE
If {\small\verb%c%} is a conversion that maps a term {\small\verb%"t"%} to the theorem {\small\verb%|- t = t'%}, then
the conversion {\small\verb%PABS_CONV c%} maps abstractions of the form {\small\verb%"\p.t"%} to theorems
of the form:
{\par\samepage\setseps\small
\begin{verbatim}
   |- (\p.t) = (\p.t')
\end{verbatim}
}
\noindent That is, {\small\verb%ABS_CONV c "\p.t"%} applies {\small\verb%p%} to the body of the paired
abstraction {\small\verb%"\p.t"%}.

\FAILURE
{\small\verb%PABS_CONV c tm%} fails if {\small\verb%tm%} is not a paired abstraction or if {\small\verb%tm%} has the
form {\small\verb%"\p.t"%} but the conversion {\small\verb%c%} fails when applied to the term {\small\verb%t%}.
The function returned by {\small\verb%ABS_CONV p%} may also fail if the ML function
{\small\verb%c:term->thm%} is not, in fact, a conversion (i.e. a function that maps a term
{\small\verb%t%} to a theorem {\small\verb%|- t = t'%}).

\EXAMPLE
{\par\samepage\setseps\small
\begin{verbatim}
#PABS_CONV SYM_CONV "\(x,y). (1,2) = (x,y)";;
|- (\(x,y). 1,2 = x,y) = (\(x,y). x,y = 1,2)
\end{verbatim}
}
\SEEALSO
ABS_CONV, PSUB_CONV.

\ENDDOC
\DOC{paconv}

\TYPE {\small\verb%paconv : (term -> term -> bool)%}\egroup

\SYNOPSIS
Tests for alpha-equivalence of terms.

\DESCRIBE
When applied to a pair of terms {\small\verb%t1%} and {\small\verb%t2%}, {\small\verb%paconv%} returns true
if the terms are alpha-equivalent.

\FAILURE
Never fails.

\COMMENTS
{\small\verb%paconv%} is implemented as {\small\verb%curry (can (uncurry PALPHA))%}.

\SEEALSO
PALPHA, aconv.

\ENDDOC
\DOC{PAIR\_CONV}

\TYPE {\small\verb%PAIR_CONV : (conv -> conv)%}\egroup

\SYNOPSIS
Applies a conversion to all the components of a pair structure.

\DESCRIBE
For any conversion {\small\verb%c%}, the function returned by {\small\verb%PAIR_CONV c%} is a conversion
that applies {\small\verb%c%} to all the components of a pair.
If the term {\small\verb%t%} is not a pair, them {\small\verb%PAIR_CONV c t%} applies {\small\verb%c%} to {\small\verb%t%}.
If the term {\small\verb%t%} is the pair {\small\verb%(t1,t2)%} then {\small\verb%PAIR c t%} recursively applies
{\small\verb%PAIR_CONV c%} to {\small\verb%t1%} and {\small\verb%t2%}.

\FAILURE
The conversion returned by {\small\verb%PAIR_CONV c%} will fail for the pair structure {\small\verb%t%}
if the conversion {\small\verb%c%} would fail for any of the components of {\small\verb%t%}.

\SEEALSO
RAND_CONV, RATOR_CONV.

\ENDDOC
\DOC{PALPHA}

\TYPE {\small\verb%PALPHA : (term -> term -> thm)%}\egroup

\SYNOPSIS
Proves equality of paired alpha-equivalent terms.

\DESCRIBE
When applied to a pair of terms {\small\verb%t1%} and {\small\verb%t1'%} which are
alpha-equivalent, {\small\verb%ALPHA%} returns the theorem {\small\verb%|- t1 = t1'%}.
{\par\samepage\setseps\small
\begin{verbatim}

   -------------  PALPHA "t1" "t1'"
    |- t1 = t1'
\end{verbatim}
}
The difference between {\small\verb%PALPHA%} and {\small\verb%ALPHA%} is that 
{\small\verb%PALPHA%} is prepared to consider pair structures of different
structure to be alpha-equivalent.
In its most trivial case this means that {\small\verb%PALPHA%} can consider
a variable and a pair to alpha-equivalent.

\FAILURE
Fails unless the terms provided are alpha-equivalent.

\EXAMPLE
{\par\samepage\setseps\small
\begin{verbatim}
#PALPHA "\(x:*,y:*). (x,y)" "\xy:*#*.xy";;
|- (\(x,y). (x,y)) = (\xy. xy)
\end{verbatim}
}
\COMMENTS
The system shows the type of {\small\verb%PALPHA%} as {\small\verb%term -> conv%}.

Alpha-converting a paired abstraction to a nonpaired abstraction 
can introduce instances of the terms {\small\verb%"FST"%} and {\small\verb%"SND"%}.
A paired abstraction and a nonpaired abstraction will be considered
equivalent by {\small\verb%PALPHA%} if the nonpaired abstraction contains all
those instances of {\small\verb%"FST"%} and {\small\verb%"SND"%} present in the paired
abstraction, plus the minimum additional instances of {\small\verb%"FST"%} and {\small\verb%"SND"%}.
For example:
{\par\samepage\setseps\small
\begin{verbatim}
   #PALPHA
      "\(x:*,y:**). (f x y (x,y)):***"
      "\xy:*#**. (f (FST xy) (SND xy) xy):***";;
   |- (\(x,y). f x y(x,y)) = (\xy. f(FST xy)(SND xy)xy)

   #PALPHA
      "\(x:*,y:**). (f x y (x,y)):***"
      "\xy:*#**. (f (FST xy) (SND xy) (FST xy, SND xy)):***";;
   evaluation failed     PALPHA
\end{verbatim}
}

\SEEALSO
ALPHA, aconv, PALPHA_CONV, GEN_PALPHA_CONV.

\ENDDOC
\DOC{PALPHA\_CONV}

\TYPE {\small\verb%PALPHA_CONV : (term -> conv)%}\egroup

\SYNOPSIS
Renames the bound variables of a paired lambda-abstraction.

\DESCRIBE
If {\small\verb%"q"%} is a variable of type {\small\verb%ty%} and {\small\verb%"\p.t"%} is a paired abstraction in
which the bound pair {\small\verb%p%} also has type {\small\verb%ty%}, then {\small\verb%ALPHA_CONV "q" "\p.t"%}
returns the theorem:
{\par\samepage\setseps\small
\begin{verbatim}
   |- (\p.t) = (\q'. t[q'/p])
\end{verbatim}
}
\noindent where the pair {\small\verb%q':ty%} is a primed variant of {\small\verb%q%} chosen so
that none of its components are free in {\small\verb%"\p.t"%}.
The pairs {\small\verb%p%} and {\small\verb%q%} need not have the same structure, but they must
be of the same type.

\EXAMPLE
{\small\verb%PALPHA_CONV%} renames the variables in a bound pair:
{\par\samepage\setseps\small
\begin{verbatim}
   #PALPHA_CONV
     "((w:*,x:*),(y:*,z:*))"
     "\((a:*,b:*),(c:*,d:*)). (f a b c d):*";;
   |- (\((a,b),c,d). f a b c d) = (\((w,x),y,z). f w x y z)
\end{verbatim}
}
The new bound pair and the old bound pair need not have the same structure.
{\par\samepage\setseps\small
\begin{verbatim}
   #PALPHA_CONV
     "((wx:*#*),(y:*,z:*))"
     "\((a:*,b:*),(c:*,d:*)). (f a b c d):*";;
   |- (\((a,b),c,d). f a b c d) = (\(wx,y,z). f(FST wx)(SND wx)y z)
\end{verbatim}
}
{\small\verb%PALPHA_CONV%} recognises subpairs of a pair as variables and preserves
structure accordingly.
{\par\samepage\setseps\small
\begin{verbatim}
   #PALPHA_CONV
     "((wx:*#*),(y:*,z:*))"
     "\((a:*,b:*),(c:*,d:*)). (f (a,b) c d):*";;
   |- (\((a,b),c,d). f(a,b)c d) = (\(wx,y,z). f wx y z)
\end{verbatim}
}

\COMMENTS
{\small\verb%PALPHA_CONV%} will only ever add the terms {\small\verb%"FST"%} and {\small\verb%"SND"%}.
(i.e. it will never remove them).
This means that while {\small\verb%"\(x,y). x + y"%} can be converted to
{\small\verb%"\xy. (FST xy) + (SND xy)"%}, it can not be converted back again.

\FAILURE
{\small\verb%PALPHA_CONV "q" "tm"%} fails if {\small\verb%q%} is not a variable,
if {\small\verb%tm%} is not an abstraction,
or if {\small\verb%q%} is a variable and {\small\verb%tm%} is the lambda abstraction
{\small\verb%\p.t%} but the types of {\small\verb%p%} and {\small\verb%q%} differ.

\SEEALSO
ALPHA_CONV, PALPHA, GEN_PALPHA_CONV.

\ENDDOC
\DOC{PART\_PMATCH}

\TYPE {\small\verb%PART_PMATCH : ((term -> term) -> thm -> term -> thm)%}\egroup

\SYNOPSIS
Instantiates a theorem by matching part of it to a term.

\DESCRIBE
When applied to a `selector' function of type {\small\verb%term -> term%}, a theorem and a
term:
{\par\samepage\setseps\small
\begin{verbatim}
   PART_MATCH fn (A |- !p1...pn. t) tm
\end{verbatim}
}
\noindent the function {\small\verb%PART_PMATCH%} applies {\small\verb%fn%} to {\small\verb%t'%} (the result
of specializing universally quantified pairs in the conclusion of
the theorem), and attempts to match the resulting term to the argument term
{\small\verb%tm%}. If it succeeds, the appropriately instantiated version of the theorem is
returned.

\FAILURE
Fails if the selector function {\small\verb%fn%} fails when applied to the instantiated
theorem, or if the match fails with the term it has provided.

\SEEALSO
PART_MATCH.

\ENDDOC
\DOC{PBETA\_CONV}

\TYPE {\small\verb%PBETA_CONV : conv%}\egroup

\SYNOPSIS
Performs a general beta-conversion.

\DESCRIBE
The conversion {\small\verb%PBETA_CONV%} maps a paired beta-redex {\small\verb%"(\p.t)q"%} to the theorem
{\par\samepage\setseps\small
\begin{verbatim}
   |- (\p.t)q = t[q/p]
\end{verbatim}
}
\noindent where {\small\verb%u[q/p]%} denotes the result of substituting {\small\verb%q%} for all free
occurrences of {\small\verb%p%} in {\small\verb%t%}, after renaming sufficient bound variables to avoid
variable capture.
Unlike {\small\verb%PAIRED_BETA_CONV%}, {\small\verb%PBETA_CONV%} does not require that the structure
of the argument match the structure of the pair bound by the abstraction.
However, if the structure of the argument does match the structure of the
pair bound by the abstraction, then {\small\verb%PAIRED_BETA_CONV%} will do the job
much faster.

\FAILURE
{\small\verb%PBETA_CONV tm%} fails if {\small\verb%tm%} is not a paired beta-redex.

\EXAMPLE
{\small\verb%PBETA_CONV%} will reduce applications with arbitrary structure.
{\par\samepage\setseps\small
\begin{verbatim}
    #PBETA_CONV "((\((a:*,b:*),(c:*,d:*)). f a b c d) ((w,x),(y,z))):*";;
    |- (\((a,b),c,d). f a b c d)((w,x),y,z) = f w x y z
\end{verbatim}
}
{\small\verb%PBETA_CONV%} does not require the structure of the argument and the bound
pair to match.
{\par\samepage\setseps\small
\begin{verbatim}
	#PBETA_CONV "((\((a:*,b:*),(c:*,d:*)). f a b c d) ((w,x),yz)):*";;
	|- (\((a,b),c,d). f a b c d)((w,x),yz) = f w x(FST yz)(SND yz)
\end{verbatim}
}
{\small\verb%PBETA_CONV%} regards component pairs of the bound pair as variables in their
own right and preserves structure accordingly:
{\par\samepage\setseps\small
\begin{verbatim}
	#PBETA_CONV "((\((a:*,b:*),(c:*,d:*)). f (a,b) (c,d)) (wx,(y,z))):*";;
	|- (\((a,b),c,d). f(a,b)(c,d))(wx,y,z) = f wx(y,z)
\end{verbatim}
}

\SEEALSO
BETA_CONV, PAIRED_BETA_CONV, PBETA_RULE, PBETA_TAC, LIST_PBETA_CONV, 
RIGHT_PBETA, RIGHT_LIST_PBETA, LEFT_PBETA, LEFT_LIST_PBETA.

\ENDDOC
\DOC{PBETA\_RULE}

\TYPE {\small\verb%PBETA_RULE : (thm -> thm)%}\egroup

\SYNOPSIS
Beta-reduces all the paired beta-redexes in the conclusion of a theorem.

\DESCRIBE
When applied to a theorem {\small\verb%A |- t%}, the inference rule {\small\verb%PBETA_RULE%} beta-reduces
all beta-redexes, at any depth, in the conclusion {\small\verb%t%}. Variables are renamed
where necessary to avoid free variable capture.
{\par\samepage\setseps\small
\begin{verbatim}
    A |- ....((\p. s1) s2)....
   ----------------------------  BETA_RULE
      A |- ....(s1[s2/p])....
\end{verbatim}
}
\FAILURE
Never fails, but will have no effect if there are no paired beta-redexes.

\SEEALSO
BETA_RULE, PBETA_CONV, PBETA_TAC, RIGHT_PBETA, LEFT_PBETA.

\ENDDOC
\DOC{PBETA\_TAC}

\TYPE {\small\verb%PBETA_TAC : tactic%}\egroup

\SYNOPSIS
Beta-reduces all the paired beta-redexes in the conclusion of a goal.

\DESCRIBE
When applied to a goal {\small\verb%A ?- t%}, the tactic {\small\verb%PBETA_TAC%} produces a new goal
which results from beta-reducing all paired beta-redexes, at any depth, in {\small\verb%t%}.
Variables are renamed where necessary to avoid free variable capture.
{\par\samepage\setseps\small
\begin{verbatim}
    A ?- ...((\p. s1) s2)...
   ==========================  PBETA_TAC
     A ?- ...(s1[s2/p])...
\end{verbatim}
}
\FAILURE
Never fails, but will have no effect if there are no paired beta-redexes.

\SEEALSO
BETA_TAC, PBETA_CONV, PBETA_RULE.

\ENDDOC
\DOC{pbody}

\TYPE {\small\verb%pbody : (term -> term)%}\egroup

\SYNOPSIS
Returns the body of a paired abstraction.

\DESCRIBE
{\small\verb%pbody "\pair. t"%} returns {\small\verb%"t"%}.

\FAILURE
Fails unless the term is a paired abstraction.

\SEEALSO
body, bndpair, dest_pabs.

\ENDDOC
\DOC{PCHOOSE}

\TYPE {\small\verb%PCHOOSE : ((term # thm) -> thm -> thm)%}\egroup

\SYNOPSIS
Eliminates paired existential quantification using deduction from a
particular witness.

\DESCRIBE
When applied to a term-theorem pair {\small\verb%(q,A1 |- ?p. s)%} and a second
theorem of the form {\small\verb%A2 u {s[q/p]} |- t%}, the inference rule {\small\verb%PCHOOSE%}
produces the theorem {\small\verb%A1 u A2 |- t%}.
{\par\samepage\setseps\small
\begin{verbatim}
    A1 |- ?p. s   A2 u {s[q/p]} |- t
   ------------------------------------  PCHOOSE ("q",(A1 |- ?q. s))
               A1 u A2 |- t
\end{verbatim}
}
\noindent Where no variable in the paired variable structure {\small\verb%q%} is free in
{\small\verb%A1%}, {\small\verb%A2%} or {\small\verb%t%}.

\FAILURE
Fails unless the terms and theorems correspond as indicated above; in
particular {\small\verb%q%} must have the same type as the pair existentially
quantified over, and must not contain any variable free in {\small\verb%A1%}, {\small\verb%A2%} or {\small\verb%t%}.

\SEEALSO
CHOOSE, PCHOOSE_TAC, PEXISTS, PEXISTS_TAC, PSELECT_ELIM.

\ENDDOC
\DOC{PCHOOSE\_TAC}

\TYPE {\small\verb%PCHOOSE_TAC : thm_tactic%}\egroup

\SYNOPSIS
Adds the body of a paired existentially quantified theorem to the assumptions
of a goal.

\DESCRIBE
When applied to a theorem {\small\verb%A' |- ?p. t%} and a goal, {\small\verb%CHOOSE_TAC%} adds
{\small\verb%t[p'/p]%} to the assumptions of the goal, where {\small\verb%p'%} is a variant of 
the pair {\small\verb%p%} which has no components free in the assumption list;
normally {\small\verb%p'%} is just {\small\verb%p%}.
{\par\samepage\setseps\small
\begin{verbatim}
         A ?- u
   ====================  CHOOSE_TAC (A' |- ?q. t)
    A u {t[p'/p]} ?- u
\end{verbatim}
}
\noindent Unless {\small\verb%A'%} is a subset of {\small\verb%A%}, this is not a valid tactic.

\FAILURE
Fails unless the given theorem is a paired existential quantification.

\SEEALSO
CHOOSE_TAC, PCHOOSE_THEN, P_PCHOOSE_TAC.

\ENDDOC
\DOC{PCHOOSE\_THEN}

\TYPE {\small\verb%PCHOOSE_THEN : thm_tactical%}\egroup

\SYNOPSIS
Applies a tactic generated from the body of paired existentially quantified
theorem.

\DESCRIBE
When applied to a theorem-tactic {\small\verb%ttac%}, a paired existentially quantified
theorem:
{\par\samepage\setseps\small
\begin{verbatim}
    A' |- ?p. t
\end{verbatim}
}
and a goal, {\small\verb%CHOOSE_THEN%} applies the tactic {\small\verb%ttac (t[p'/p] |- t[p'/p])%} to
the goal, where {\small\verb%p'%} is a variant of the pair {\small\verb%p%} chosen to have no
components free in the assumption list of the goal. Thus if:
{\par\samepage\setseps\small
\begin{verbatim}
    A ?- s1
   =========  ttac (t[q'/q] |- t[q'/q])
    B ?- s2
\end{verbatim}
}
\noindent then
{\par\samepage\setseps\small
\begin{verbatim}
    A ?- s1
   ==========  CHOOSE_THEN ttac (A' |- ?q. t)
    B ?- s2
\end{verbatim}
}
\noindent This is invalid unless {\small\verb%A'%} is a subset of {\small\verb%A%}.

\FAILURE
Fails unless the given theorem is a paired existential quantification,
or if the resulting tactic fails when applied to the goal.

\SEEALSO
CHOOSE_THEN, PCHOOSE_TAC, P_PCHOOSE_THEN.

\ENDDOC
\DOC{PETA\_CONV}

\TYPE {\small\verb%PETA_CONV : conv%}\egroup

\SYNOPSIS
Performs a top-level paired eta-conversion.

\DESCRIBE
{\small\verb%PETA_CONV%} maps an eta-redex {\small\verb%"\p. t p"%}, where none of variables in the
paired structure of variables {\small\verb%p%} occurs free in {\small\verb%t%},
to the theorem {\small\verb%|- (\p. t p) = t%}.

\FAILURE
Fails if the input term is not a paired eta-redex.

\ENDDOC
\DOC{PEXISTENCE}

\TYPE {\small\verb%PEXISTENCE : (thm -> thm)%}\egroup

\SYNOPSIS
Deduces paired existence from paired unique existence.

\DESCRIBE
When applied to a theorem with a paired unique-existentially quantified
conclusion, {\small\verb%EXISTENCE%} returns the same theorem with normal paired 
existential quantification over the same pair.
{\par\samepage\setseps\small
\begin{verbatim}
    A |- ?!p. t
   -------------  PEXISTENCE
    A |- ?p. t
\end{verbatim}
}
\FAILURE
Fails unless the conclusion of the theorem is a paired unique-existential
quantification.

\SEEALSO
EXISTENCE, PEXISTS_UNIQUE_CONV.

\ENDDOC
\DOC{PEXISTS}

\TYPE {\small\verb%PEXISTS : ((term # term) -> thm -> thm)%}\egroup

\SYNOPSIS
Introduces paired existential quantification given a particular witness.

\DESCRIBE
When applied to a pair of terms and a theorem, where the first term a paired
existentially quantified pattern indicating the desired form of the result,
and the second a witness whose substitution for the quantified pair gives a
term which is the same as the conclusion of the theorem,
{\small\verb%PEXISTS%} gives the desired theorem.
{\par\samepage\setseps\small
\begin{verbatim}
    A |- t[q/p]
   -------------  EXISTS ("?p. t","q")
    A |- ?p. t
\end{verbatim}
}
\FAILURE
Fails unless the substituted pattern is the same as the conclusion of the
theorem.

\EXAMPLE
The following examples illustrate the various uses of {\small\verb%PEXISTS%}:
{\par\samepage\setseps\small
\begin{verbatim}
   #PEXISTS ("?x. x + 2 = x + 2", "1") (REFL "1 + 2");;
   |- ?x. x + 2 = x + 2

   #PEXISTS ("?y. 1 + y = 1 + y", "2") (REFL "1 + 2");;
   |- ?y. 1 + y = 1 + y

   #PEXISTS ("?(x,y). x + y = x + y", "(1,2)") (REFL "1 + 2");;
   |- ?(x,y). x + y = x + y

   #PEXISTS ("?(a:*,b:*). (a,b) = (a,b)", "ab:*#*") (REFL "ab:*#*");;
   |- ?(a,b). a,b = a,b
\end{verbatim}
}
\SEEALSO
EXISTS, PCHOOSE, PEXISTS_TAC.

\ENDDOC
\DOC{PEXISTS\_AND\_CONV}

\TYPE {\small\verb%PEXISTS_AND_CONV : conv%}\egroup

\SYNOPSIS
Moves a paired existential quantification inwards through a conjunction.

\DESCRIBE
When applied to a term of the form {\small\verb%?p. t /\ u%}, where variables in {\small\verb%p%} are
not free in both {\small\verb%t%} and {\small\verb%u%}, {\small\verb%PEXISTS_AND_CONV%} returns a theorem of one of
three forms, depending on occurrences of variables from {\small\verb%p%} in {\small\verb%t%} and {\small\verb%u%}.  
If {\small\verb%p%} contains variables free in {\small\verb%t%} but none in {\small\verb%u%}, then the theorem:
{\par\samepage\setseps\small
\begin{verbatim}
   |- (?p. t /\ u) = (?p. t) /\ u
\end{verbatim}
}
\noindent is returned.  If {\small\verb%p%} contains variables free in {\small\verb%u%} but none in {\small\verb%t%},
then the result is:
{\par\samepage\setseps\small
\begin{verbatim}
   |- (?p. t /\ u) = t /\ (?x. u)
\end{verbatim}
}
\noindent And if {\small\verb%p%} does not contain any variable free in either {\small\verb%t%} nor {\small\verb%u%},
then the result is:
{\par\samepage\setseps\small
\begin{verbatim}
   |- (?p. t /\ u) = (?x. t) /\ (?x. u)
\end{verbatim}
}
\FAILURE
{\small\verb%PEXISTS_AND_CONV%} fails if it is applied to a term not of the form
{\small\verb%?p. t /\ u%}, or if it is applied to a term {\small\verb%?p. t /\ u%} in which 
variables in {\small\verb%p%} are free in both {\small\verb%t%} and {\small\verb%u%}.

\SEEALSO
EXISTS_AND_CONV, AND_PEXISTS_CONV, LEFT_AND_PEXISTS_CONV,
RIGHT_AND_PEXISTS_CONV.

\ENDDOC
\DOC{PEXISTS\_CONV}

\TYPE {\small\verb%PEXISTS_CONV : conv%}\egroup

\SYNOPSIS
Eliminates paired existential quantifier by introducing a paired choice-term.

\DESCRIBE
The conversion {\small\verb%PEXISTS_CONV%} expects a boolean term of the form
{\small\verb%(?p. t[p])%}, where {\small\verb%p%} may be a paired structure or variables,
and converts it to the form {\small\verb%(t [@p. t[p]])%}.
{\par\samepage\setseps\small
\begin{verbatim}
   ---------------------------------  PEXISTS_CONV "(?p. t[p])"
    (|- (?p. t[p]) = (t [@p. t[p]])
\end{verbatim}
}

\FAILURE
Fails if applied to a term that is not a paired existential quantification.

\SEEALSO
PSELECT_RULE, PSELECT_CONV, PEXISTS_RULE, PSELECT_INTRO, PSELECT_ELIM.

\ENDDOC
\DOC{PEXISTS\_EQ}

\TYPE {\small\verb%PEXISTS_EQ : (term -> thm -> thm)%}\egroup

\SYNOPSIS
Existentially quantifies both sides of an equational theorem.

\DESCRIBE
When applied to a paired structure of variables {\small\verb%p%} and a theorem
whose conclusion is equational:
{\par\samepage\setseps\small
\begin{verbatim}
    A |- t1 = t2
\end{verbatim}
}
the inference rule {\small\verb%PEXISTS_EQ%} returns the theorem:
{\par\samepage\setseps\small
\begin{verbatim}
    A |- (?p. t1) = (?p. t2)
\end{verbatim}
}
provided the none of the variables in {\small\verb%p%} is not free in any of the
assumptions.
{\par\samepage\setseps\small
\begin{verbatim}
          A |- t1 = t2
   --------------------------  PEXISTS_EQ "p"      [where p is not free in A]
    A |- (?p. t1) = (?p. t2)
\end{verbatim}
}
\FAILURE
Fails unless the theorem is equational with both sides having type {\small\verb%bool%},
or if the term is not a paired structure of variables, or if any variable in
the pair to be quantified over is free in any of the assumptions.

\SEEALSO
EXISTS_EQ, PEXISTS_IMP, PFORALL_EQ, MK_PEXISTS, PSELECT_EQ.

\ENDDOC
\DOC{PEXISTS\_IMP}

\TYPE {\small\verb%PEXISTS_IMP : (term -> thm -> thm)%}\egroup

\SYNOPSIS
Existentially quantifies both the antecedent and consequent of an implication.

\DESCRIBE
When applied to a paired structure of variables {\small\verb%p%} and a
theorem {\small\verb%A |- t1 ==> t2%}, the inference rule {\small\verb%PEXISTS_IMP%} returns the
theorem {\small\verb%A |- (?p. t1) ==> (?p. t2)%},
provided no variable in {\small\verb%p%} is free in the assumptions.
{\par\samepage\setseps\small
\begin{verbatim}
         A |- t1 ==> t2
   --------------------------  EXISTS_IMP "x"   [where x is not free in A]
    A |- (?x.t1) ==> (?x.t2)
\end{verbatim}
}
\FAILURE
Fails if the theorem is not implicative, or if the term is not a paired 
structure of variables, of if any variable in the pair is free in the
assumption list.

\SEEALSO
EXISTS_IMP, PEXISTS_EQ.

\ENDDOC
\DOC{PEXISTS\_IMP\_CONV}

\TYPE {\small\verb%PEXISTS_IMP_CONV : conv%}\egroup

\SYNOPSIS
Moves a paired existential quantification inwards through an implication.

\DESCRIBE
When applied to a term of the form {\small\verb%?p. t ==> u%}, where variables from {\small\verb%p%}
are not free in both {\small\verb%t%} and {\small\verb%u%}, {\small\verb%PEXISTS_IMP_CONV%} returns a theorem of one
of three forms, depending on occurrences of variable from {\small\verb%p%} in {\small\verb%t%} and {\small\verb%u%}.
If variables from {\small\verb%p%} are free in {\small\verb%t%} but none are in {\small\verb%u%}, then the theorem:
{\par\samepage\setseps\small
\begin{verbatim}
   |- (?p. t ==> u) = (!p. t) ==> u
\end{verbatim}
}
\noindent is returned.  If variables from {\small\verb%p%} are free in {\small\verb%u%} but none are in
{\small\verb%t%}, then the result is:
{\par\samepage\setseps\small
\begin{verbatim}
   |- (?p. t ==> u) = t ==> (?p. u)
\end{verbatim}
}
\noindent And if no variable from {\small\verb%p%} is free in either {\small\verb%t%} nor {\small\verb%u%},
then the result is:
{\par\samepage\setseps\small
\begin{verbatim}
   |- (?p. t ==> u) = (!p. t) ==> (?p. u)
\end{verbatim}
}
\FAILURE
{\small\verb%PEXISTS_IMP_CONV%} fails if it is applied to a term not of the form
{\small\verb%?p. t ==> u%}, or if it is applied to a term {\small\verb%?p. t ==> u%} in which the
variables from {\small\verb%p%} are free in both {\small\verb%t%} and {\small\verb%u%}.

\SEEALSO
EXISTS_IMP_CONV, LEFT_IMP_PFORALL_CONV, RIGHT_IMP_PEXISTS_CONV.

\ENDDOC
\DOC{PEXISTS\_NOT\_CONV}

\TYPE {\small\verb%PEXISTS_NOT_CONV : conv%}\egroup

\SYNOPSIS
Moves a paired existential quantification inwards through a negation.

\DESCRIBE
When applied to a term of the form {\small\verb%?p. ~t%}, the conversion {\small\verb%PEXISTS_NOT_CONV%}
returns the theorem:
{\par\samepage\setseps\small
\begin{verbatim}
   |- (?p. ~t) = ~(!p. t)
\end{verbatim}
}
\FAILURE
Fails if applied to a term not of the form {\small\verb%?p. ~t%}.

\SEEALSO
EXISTS_NOT_CONV, PFORALL_NOT_CONV, NOT_PEXISTS_CONV, NOT_PFORALL_CONV.

\ENDDOC
\DOC{PEXISTS\_OR\_CONV}

\TYPE {\small\verb%PEXISTS_OR_CONV : conv%}\egroup

\SYNOPSIS
Moves a paired existential quantification inwards through a disjunction.

\DESCRIBE
When applied to a term of the form {\small\verb%?p. t \/ u%}, the conversion
{\small\verb%PEXISTS_OR_CONV%} returns the theorem:
{\par\samepage\setseps\small
\begin{verbatim}
   |- (?p. t \/ u) = (?p. t) \/ (?p. u)
\end{verbatim}
}
\FAILURE
Fails if applied to a term not of the form {\small\verb%?p. t \/ u%}.

\SEEALSO
EXISTS_OR_CONV, OR_PEXISTS_CONV, LEFT_OR_PEXISTS_CONV, RIGHT_OR_PEXISTS_CONV.

\ENDDOC
\DOC{PEXISTS\_RULE}

\TYPE {\small\verb%PEXISTS_RULE : (thm -> thm)%}\egroup

\SYNOPSIS
Introduces a paired existential quantification in place of a paired choice.

\DESCRIBE
The inference rule {\small\verb%PEXISTS_RULE%} expects a theorem asserting that
{\small\verb%(@p. t)%} denotes a pair for which {\small\verb%t%} holds.
The equivalent assertion that there exists a {\small\verb%p%} for which {\small\verb%t%} holds is
returned.
{\par\samepage\setseps\small
\begin{verbatim}
    A |- t[(@p. t)/p]
   ------------------  PEXISTS_RULE
       A |- ?p. t
\end{verbatim}
}
\FAILURE
Fails if applied to a theorem the conclusion of which is not
of the form {\small\verb%(t[(@p.t)/p])%}.

\SEEALSO
PEXISTS_CONV, PSELECT_RULE, PSELECT_CONV, PSELECT_INTRO, PSELECT_ELIM.

\ENDDOC
\DOC{PEXISTS\_TAC}

\TYPE {\small\verb%PEXISTS_TAC : (term -> tactic)%}\egroup

\SYNOPSIS
Reduces paired existentially quantified goal to one involving a 
specific witness.

\DESCRIBE
When applied to a term {\small\verb%q%} and a goal {\small\verb%?p. t%}, the tactic
{\small\verb%PEXISTS_TAC%} reduces the goal to {\small\verb%t[q/p]%}.
{\par\samepage\setseps\small
\begin{verbatim}
    A ?- ?p. t
   =============  EXISTS_TAC "q"
    A ?- t[q/p]
\end{verbatim}
}
\FAILURE
Fails unless the goal's conclusion is a paired existential quantification
and the term supplied has the same type as the quantified pair in the goal.

\EXAMPLE
The goal:
{\par\samepage\setseps\small
\begin{verbatim}
   ?- ?(x,y). (x,y)=(1,2)
\end{verbatim}
}
\noindent can be solved by:
{\par\samepage\setseps\small
\begin{verbatim}
   PEXISTS_TAC "(1,2)" THEN REFL_TAC
\end{verbatim}
}
\SEEALSO
EXISTS_TAC, PEXISTS.

\ENDDOC
\DOC{PEXISTS\_UNIQUE\_CONV}

\TYPE {\small\verb%PEXISTS_UNIQUE_CONV : conv%}\egroup

\SYNOPSIS
Expands with the definition of paired unique existence.

\DESCRIBE
Given a term of the form {\small\verb%"?!p. t[p]"%}, the conversion {\small\verb%PEXISTS_UNIQUE_CONV%}
proves that this assertion is equivalent to the conjunction of two statements,
namely that there exists at least one pair {\small\verb%p%} such that {\small\verb%t[p]%}, and that
there is at most one value {\small\verb%p%} for which {\small\verb%t[p]%} holds. The theorem returned is:
{\par\samepage\setseps\small
\begin{verbatim}
   |- (?!p. t[p]) = (?p. t[p]) /\ (!p p'. t[p] /\ t[p'] ==> (p = p'))
\end{verbatim}
}
\noindent where {\small\verb%p'%} is a primed variant of the pair {\small\verb%p%} none of the components
of which appear free in the input term.  Note that the quantified pair {\small\verb%p%}
need not in fact appear free in the body of the input term.  
For example, {\small\verb%PEXISTS_UNIQUE_CONV "?!(x,y). T"%}
returns the theorem:
{\par\samepage\setseps\small
\begin{verbatim}
   |- (?!(x,y). T) =
      (?(x,y). T) /\ (!(x,y) (x',y'). T /\ T ==> ((x,y) = (x',y')))
\end{verbatim}
}
\FAILURE
{\small\verb%PEXISTS_UNIQUE_CONV tm%} fails if {\small\verb%tm%} does not have the form {\small\verb%"?!p.t"%}.

\SEEALSO
EXISTS_UNIQUE_CONV, PEXISTENCE.

\ENDDOC
\DOC{PEXT}

\TYPE {\small\verb%PEXT : (thm -> thm)%}\egroup

\SYNOPSIS
Derives equality of functions from extensional equivalence.

\DESCRIBE
When applied to a theorem {\small\verb%A |- !p. t1 p = t2 p%},
the inference rule {\small\verb%PEXT%} returns the theorem {\small\verb%A |- t1 = t2%}.
{\par\samepage\setseps\small
\begin{verbatim}
    A |- !p. t1 p = t2 p
   ----------------------  PEXT          [where p is not free in t1 or t2]
        A |- t1 = t2
\end{verbatim}
}

\FAILURE
Fails if the theorem does not have the form indicated above, or
if any of the component variables in the paired variable structure {\small\verb%p%}
is free either of the functions {\small\verb%t1%} or {\small\verb%t2%}.

\EXAMPLE
{\par\samepage\setseps\small
\begin{verbatim}
#PEXT (ASSUME "!(x,y). ((f:(*#*)->*) (x,y)) = (g (x,y))");;
. |- f = g
\end{verbatim}
}

\SEEALSO
EXT, AP_THM, PETA_CONV, FUN_EQ_CONV, P_FUN_EQ_CONV.

\ENDDOC
\DOC{PFORALL\_AND\_CONV}

\TYPE {\small\verb%PFORALL_AND_CONV : conv%}\egroup

\SYNOPSIS
Moves a paired universal quantification inwards through a conjunction.

\DESCRIBE
When applied to a term of the form {\small\verb%!p. t /\ u%}, the conversion
{\small\verb%PFORALL_AND_CONV%} returns the theorem:
{\par\samepage\setseps\small
\begin{verbatim}
   |- (!p. t /\ u) = (!p. t) /\ (!p. u)
\end{verbatim}
}
\FAILURE
Fails if applied to a term not of the form {\small\verb%!p. t /\ u%}.

\SEEALSO
FORALL_AND_CONV, AND_PFORALL_CONV, LEFT_AND_PFORALL_CONV,
RIGHT_AND_PFORALL_CONV.

\ENDDOC
\DOC{PFORALL\_EQ}

\TYPE {\small\verb%PFORALL_EQ : (term -> thm -> thm)%}\egroup

\SYNOPSIS
Universally quantifies both sides of an equational theorem.

\DESCRIBE
When applied to a paired structure of variables {\small\verb%p%}
and a theorem
{\par\samepage\setseps\small
\begin{verbatim}
    A |- t1 = t2
\end{verbatim}
}
whose conclusion is an equation between boolean terms:
{\par\samepage\setseps\small
\begin{verbatim}
    PFORALL_EQ
\end{verbatim}
}
returns the theorem:
{\par\samepage\setseps\small
\begin{verbatim}
    A |- (!p. t1) = (!p. t2)
\end{verbatim}
}
unless any of the variables in {\small\verb%p%} is free in any of the assumptions.
{\par\samepage\setseps\small
\begin{verbatim}
          A |- t1 = t2
   --------------------------  PFORALL_EQ "p"      [where p is not free in A]
    A |- (!p. t1) = (!p. t2)
\end{verbatim}
}
\FAILURE
Fails if the theorem is not an equation between boolean terms, or if the
supplied term is not a paired structure of variables, or if any of the
variables in the supplied pair is free in any of the assumptions.

\SEEALSO
FORALL_EQ, PEXISTS_EQ, PSELECT_EQ.

\ENDDOC
\DOC{PFORALL\_IMP\_CONV}

\TYPE {\small\verb%PFORALL_IMP_CONV : conv%}\egroup

\SYNOPSIS
Moves a paired universal quantification inwards through an implication.

\DESCRIBE
When applied to a term of the form {\small\verb%!p. t ==> u%}, where variables from {\small\verb%p%}
are not free in both {\small\verb%t%} and {\small\verb%u%}, {\small\verb%PFORALL_IMP_CONV%} returns a theorem of
one of three forms, depending on occurrences of the variables from {\small\verb%p%} in {\small\verb%t%}
and {\small\verb%u%}.  If variables from {\small\verb%p%} are free in {\small\verb%t%} but none are in {\small\verb%u%},
then the theorem:
{\par\samepage\setseps\small
\begin{verbatim}
   |- (!p. t ==> u) = (?p. t) ==> u
\end{verbatim}
}
\noindent is returned.  If variables from {\small\verb%p%} are free in {\small\verb%u%} but none are
in {\small\verb%t%}, then the result is:
{\par\samepage\setseps\small
\begin{verbatim}
   |- (!p. t ==> u) = t ==> (!p. u)
\end{verbatim}
}
\noindent And if no variable from {\small\verb%p%} is free in either {\small\verb%t%} nor {\small\verb%u%},
then the result is:
{\par\samepage\setseps\small
\begin{verbatim}
   |- (!p. t ==> u) = (?p. t) ==> (!p. u)
\end{verbatim}
}
\FAILURE
{\small\verb%PFORALL_IMP_CONV%} fails if it is applied to a term not of the form
{\small\verb%!p. t ==> u%}, or if it is applied to a term {\small\verb%!p. t ==> u%} in which
variables from {\small\verb%p%} are free in both {\small\verb%t%} and {\small\verb%u%}.

\SEEALSO
FORALL_IMP_CONV, LEFT_IMP_PEXISTS_CONV, RIGHT_IMP_PFORALL_CONV.

\ENDDOC
\DOC{PFORALL\_NOT\_CONV}

\TYPE {\small\verb%PFORALL_NOT_CONV : conv%}\egroup

\SYNOPSIS
Moves a paired universal quantification inwards through a negation.

\DESCRIBE
When applied to a term of the form {\small\verb%!p. ~t%}, the conversion {\small\verb%PFORALL_NOT_CONV%}
returns the theorem:
{\par\samepage\setseps\small
\begin{verbatim}
   |- (!p. ~t) = ~(?p. t)
\end{verbatim}
}
\FAILURE
Fails if applied to a term not of the form {\small\verb%!p. ~t%}.

\SEEALSO
FORALL_NOT_CONV, PEXISTS_NOT_CONV, NOT_PEXISTS_CONV, NOT_PFORALL_CONV.

\ENDDOC
\DOC{PFORALL\_OR\_CONV}

\TYPE {\small\verb%PFORALL_OR_CONV : conv%}\egroup

\SYNOPSIS
Moves a paired universal quantification inwards through a disjunction.

\DESCRIBE
When applied to a term of the form {\small\verb%!p. t \/ u%}, where no variable in {\small\verb%p%} is
free in both {\small\verb%t%} and {\small\verb%u%}, {\small\verb%PFORALL_OR_CONV%} returns a theorem of one of three
forms, depending on occurrences of the variables from {\small\verb%p%} in {\small\verb%t%} and {\small\verb%u%}.
If variables from {\small\verb%p%} are free in {\small\verb%t%} but not in {\small\verb%u%}, then the theorem:
{\par\samepage\setseps\small
\begin{verbatim}
   |- (!p. t \/ u) = (!p. t) \/ u
\end{verbatim}
}
\noindent is returned.  If variables from {\small\verb%p%} are free in {\small\verb%u%} but none are
free in {\small\verb%t%}, then the result is:
{\par\samepage\setseps\small
\begin{verbatim}
   |- (!p. t \/ u) = t \/ (!t. u)
\end{verbatim}
}
\noindent And if no variable from {\small\verb%p%} is free in either {\small\verb%t%} nor {\small\verb%u%},
then the result is:
{\par\samepage\setseps\small
\begin{verbatim}
   |- (!p. t \/ u) = (!p. t) \/ (!p. u)
\end{verbatim}
}
\FAILURE
{\small\verb%PFORALL_OR_CONV%} fails if it is applied to a term not of the form
{\small\verb%!p. t \/ u%}, or if it is applied to a term {\small\verb%!p. t \/ u%} in which 
variables from {\small\verb%p%} are free in both {\small\verb%t%} and {\small\verb%u%}.

\SEEALSO
FORALL_OR_CONV, OR_PFORALL_CONV, LEFT_OR_PFORALL_CONV, RIGHT_OR_PFORALL_CONV.

\ENDDOC
\DOC{PGEN}

\TYPE {\small\verb%PGEN : (term -> thm -> thm)%}\egroup

\SYNOPSIS
Generalizes the conclusion of a theorem.

\DESCRIBE
When applied to a paired structure of variables {\small\verb%p%} and a theorem {\small\verb%A |- t%},
the inference rule {\small\verb%PGEN%} returns the theorem {\small\verb%A |- !p. t%}, provided that
no variable in {\small\verb%p%} occurs free in the assumptions {\small\verb%A%}.
There is no compulsion that the variables of {\small\verb%p%} should be free in {\small\verb%t%}.
{\par\samepage\setseps\small
\begin{verbatim}
      A |- t
   ------------  PGEN "p"               [where p does not occur free in A]
    A |- !p. t
\end{verbatim}
}
\FAILURE
Fails if {\small\verb%p%} is not a paired structure of variables,
of if any variable in {\small\verb%p%} is free in the assumptions.

\SEEALSO
GEN, PGENL, PGEN_ALL, PGEN_TAC, PSPEC, PSPECL, PSPEC_ALL, PSPEC_TAC.

\ENDDOC
\DOC{PGENL}

\TYPE {\small\verb%PGENL : (term list -> thm -> thm)%}\egroup

\SYNOPSIS
Generalizes zero or more pairs in the conclusion of a theorem.

\DESCRIBE
When applied to a list of paired variable structures {\small\verb%[p1;...;pn]%}
and a theorem {\small\verb%A |- t%}, the inference rule {\small\verb%PGENL%} returns
the theorem {\small\verb%A |- !p1...pn. t%}, provided none of the constituent variables
from any of the pairs {\small\verb%pi%} occur free in the assumptions.
{\par\samepage\setseps\small
\begin{verbatim}
         A |- t
   ------------------  PGENL "[p1;...;pn]"       [where no pi is free in A]
    A |- !p1...pn. t
\end{verbatim}
}
\FAILURE
Fails unless all the terms in the list are paired structures of variables,
none of the variables from which are free in the assumption list.

\SEEALSO
GENL, PGEN, PGEN_ALL, PGEN_TAC, PSPEC, PSPECL, PSPEC_ALL, PSPEC_TAC.

\ENDDOC
\DOC{PGEN\_TAC}

\TYPE {\small\verb%PGEN_TAC : tactic%}\egroup

\SYNOPSIS
Strips the outermost paired universal quantifier from the conclusion of a goal.

\DESCRIBE
When applied to a goal {\small\verb%A ?- !p. t%}, the tactic {\small\verb%PGEN_TAC%} reduces it to
{\small\verb%A ?- t[p'/p]%} where {\small\verb%p'%} is a variant of the paired variable structure {\small\verb%p%}
chosen to avoid clashing with any variables free in the goal's assumption list.
Normally {\small\verb%p'%} is just {\small\verb%p%}.
{\par\samepage\setseps\small
\begin{verbatim}
     A ?- !p. t
   ==============  PGEN_TAC
    A ?- t[p'/p]
\end{verbatim}
}
\FAILURE
Fails unless the goal's conclusion is a paired universally quantification.

\SEEALSO
GEN_TAC, FILTER_PGEN_TAC, PGEN, PGENL, PGEN_ALL, PSPEC, PSPECL, PSPEC_ALL,
PSPEC_TAC, PSTRIP_TAC, P_PGEN_TAC.

\ENDDOC
\DOC{PMATCH\_MP}

\TYPE {\small\verb%PMATCH_MP : (thm -> thm -> thm)%}\egroup

\SYNOPSIS
Modus Ponens inference rule with automatic matching.

\DESCRIBE
When applied to theorems {\small\verb%A1 |- !p1...pn. t1 ==> t2%} and {\small\verb%A2 |- t1'%}, the
inference rule {\small\verb%PMATCH_MP%} matches {\small\verb%t1%} to {\small\verb%t1'%} by instantiating free or
paired universally quantified variables in the first theorem (only),
and returns a theorem {\small\verb%A1 u A2 |- !pa..pk. t2'%}, where {\small\verb%t2'%} is a
correspondingly instantiated version of {\small\verb%t2%}. Polymorphic types are also
instantiated if necessary.

Variables free in the consequent but not the antecedent of the first argument
theorem will be replaced by variants if this is necessary to maintain the full
generality of the theorem, and any pairs which were universally quantified
over in the first argument theorem will be universally quantified over in the
result, and in the same order.
{\par\samepage\setseps\small
\begin{verbatim}
    A1 |- !p1..pn. t1 ==> t2   A2 |- t1'
   --------------------------------------  MATCH_MP
          A1 u A2 |- !pa..pk. t2'
\end{verbatim}
}
\FAILURE
Fails unless the first theorem is a (possibly repeatedly paired universally
quantified) implication whose antecedent can be instantiated to match
the conclusion of the second theorem, without instantiating any variables
which are free in {\small\verb%A1%}, the first theorem's assumption list.

\SEEALSO
MATCH_MP.

\ENDDOC
\DOC{PMATCH\_MP\_TAC}

\TYPE {\small\verb%PMATCH_MP_TAC : thm_tactic%}\egroup

\SYNOPSIS
Reduces the goal using a supplied implication, with matching.

\DESCRIBE
When applied to a theorem of the form
{\par\samepage\setseps\small
\begin{verbatim}
   A' |- !p1...pn. s ==> !q1...qm. t
\end{verbatim}
}
\noindent {\small\verb%PMATCH_MP_TAC%} produces a tactic that reduces a goal whose
conclusion {\small\verb%t'%} is a substitution and/or type instance of {\small\verb%t%} to the
corresponding instance of {\small\verb%s%}. Any variables free in {\small\verb%s%} but not in {\small\verb%t%} will
be existentially quantified in the resulting subgoal:
{\par\samepage\setseps\small
\begin{verbatim}
     A ?- !u1...ui. t'
  ======================  PMATCH_MP_TAC (A' |- !p1...pn. s ==> !q1...qm. t)
     A ?- ?w1...wp. s'
\end{verbatim}
}
\noindent where {\small\verb%w1%}, ..., {\small\verb%wp%} are (type instances of) those pairs among
{\small\verb%p1%}, ..., {\small\verb%pn%} having variables that do not occur free in {\small\verb%t%}.
Note that this is not a valid tactic unless {\small\verb%A'%} is a subset of {\small\verb%A%}.

\FAILURE
Fails unless the theorem is an (optionally paired universally quantified)
implication whose consequent can be instantiated to match the goal.
The generalized pairs {\small\verb%u1%}, ..., {\small\verb%ui%} must occur in {\small\verb%s'%} in order for the
conclusion {\small\verb%t%} of the supplied theorem to match {\small\verb%t'%}.

\SEEALSO
MATCH_MP_TAC.

\ENDDOC
\DOC{PSELECT\_CONV}

\TYPE {\small\verb%PSELECT_CONV : conv%}\egroup

\SYNOPSIS
Eliminates a paired epsilon term by introducing a existential quantifier.

\DESCRIBE
The conversion {\small\verb%PSELECT_CONV%} expects a boolean term of the form
{\small\verb%"t[@p.t[p]/p]"%}, which asserts that the epsilon term {\small\verb%@p.t[p]%} denotes
a pair, {\small\verb%p%} say, for which {\small\verb%t[p]%} holds.  This assertion is equivalent
to saying that there exists such a pair, and {\small\verb%PSELECT_CONV%} applied to a
term of this form returns the theorem {\small\verb%|- t[@p.t[p]/p] = ?p. t[p]%}.

\FAILURE
Fails if applied to a term that is not of the form {\small\verb%"p[@p.t[p]/p]"%}.

\SEEALSO
SELECT_CONV, PSELECT_ELIM, PSELECT_INTRO, PSELECT_RULE.

\ENDDOC
\DOC{PSELECT\_ELIM}

\TYPE {\small\verb%PSELECT_ELIM : (thm -> (term # thm) -> thm)%}\egroup

\SYNOPSIS
Eliminates a paired epsilon term, using deduction from a particular instance.

\DESCRIBE
{\small\verb%PSELECT_ELIM%} expects two arguments, a theorem {\small\verb%th1%}, and a pair
{\small\verb%(p,th2):(term # thm)%}.  The conclusion of {\small\verb%th1%} must have the form {\small\verb%P($@ P)%},
which asserts that the epsilon term {\small\verb%$@ P%} denotes some value at which
{\small\verb%P%} holds.  The paired variable structure {\small\verb%p%} appears only in the assumption
{\small\verb%P p%} of the theorem {\small\verb%th2%}.  The conclusion of the resulting theorem matches
that of {\small\verb%th2%}, and the hypotheses include the union of all hypotheses
of the premises excepting {\small\verb%P p%}.
{\par\samepage\setseps\small
\begin{verbatim}
    A1 |- P($@ P)     A2 u {P p} |- t
   -------------------------------------  PSELECT_ELIM th1 (p ,th2)
              A1 u A2 |- t
\end{verbatim}
}
\noindent where {\small\verb%p%} is not free in {\small\verb%A2%}. If {\small\verb%p%} appears in the conclusion of
{\small\verb%th2%}, the epsilon term will NOT be eliminated, and the conclusion will be
{\small\verb%t[$@ P/p]%}.

\FAILURE
Fails if the first theorem is not of the form {\small\verb%A1 |- P($@ P)%}, or if
any of the variables from the variable structure {\small\verb%p%} occur free in any
other assumption of {\small\verb%th2%}.

\SEEALSO
SELECT_ELIM, PCHOOSE, SELECT_AX, PSELECT_CONV, PSELECT_INTRO, PSELECT_RULE.

\ENDDOC
\DOC{PSELECT\_EQ}

\TYPE {\small\verb%PSELECT_EQ : (term -> thm -> thm)%}\egroup

\SYNOPSIS
Applies epsilon abstraction to both terms of an equation.

\DESCRIBE
When applied to a paired structure of variables {\small\verb%p%} and a theorem
whose conclusion is equational:
{\par\samepage\setseps\small
\begin{verbatim}
    A |- t1 = t2
\end{verbatim}
}
the inference rule {\small\verb%PSELECT_EQ%} returns the theorem:
{\par\samepage\setseps\small
\begin{verbatim}
    A |- (@p. t1) = (@p. t2)
\end{verbatim}
}
provided no variable in {\small\verb%p%} is free in the assumptions.
{\par\samepage\setseps\small
\begin{verbatim}
         A |- t1 = t2
   --------------------------  SELECT_EQ "p"      [where p is not free in A]
    A |- (@p. t1) = (@p. t2)
\end{verbatim}
}
\FAILURE
Fails if the conclusion of the theorem is not an equation, 
of if {\small\verb%p%} is not a paired structure of variables, 
or if any variable in {\small\verb%p%} is free in {\small\verb%A%}.

\SEEALSO
SELECT_EQ, PFORALL_EQ, PEXISTS_EQ.

\ENDDOC
\DOC{PSELECT\_INTRO}

\TYPE {\small\verb%PSELECT_INTRO : (thm -> thm)%}\egroup

\SYNOPSIS
Introduces an epsilon term.

\DESCRIBE
{\small\verb%PSELECT_INTRO%} takes a theorem with an applicative conclusion, say
{\small\verb%P x%}, and returns a theorem with the epsilon term {\small\verb%$@ P%} in place
of the original operand {\small\verb%x%}.
{\par\samepage\setseps\small
\begin{verbatim}
     A |- P x
   --------------  PSELECT_INTRO
    A |- P($@ P)
\end{verbatim}
}
\noindent The returned theorem asserts that {\small\verb%$@ P%} denotes some value
at which {\small\verb%P%} holds.

\FAILURE
Fails if the conclusion of the theorem is not an application.

\COMMENTS
This function is exactly the same as {\small\verb%SELECT_INTRO%}, it is duplicated in
the pair library for completeness.

\SEEALSO
SELECT_INTRO, PEXISTS, SELECT_AX, PSELECT_CONV, PSELECT_ELIM, PSELECT_RULE.

\ENDDOC
\DOC{PSELECT\_RULE}

\TYPE {\small\verb%PSELECT_RULE : (thm -> thm)%}\egroup

\SYNOPSIS
Introduces a paired epsilon term in place of a paired existential quantifier.

\DESCRIBE
The inference rule {\small\verb%PSELECT_RULE%} expects a theorem asserting the
existence of a pair {\small\verb%p%} such that {\small\verb%t%} holds.  The equivalent assertion
that the epsilon term {\small\verb%@p.t%} denotes a pair {\small\verb%p%} for
which {\small\verb%t%} holds is returned as a theorem.
{\par\samepage\setseps\small
\begin{verbatim}
       A |- ?p. t
   ------------------  PSELECT_RULE
    A |- t[(@p.t)/p]
\end{verbatim}
}
\FAILURE
Fails if applied to a theorem the conclusion of which is not
a paired existential quantifier.

\SEEALSO
SELECT_RULE, PCHOOSE, SELECT_AX, PSELECT_CONV, PEXISTS_CONV, PSELECT_ELIM, 
PSELECT_INTRO.

\ENDDOC
\DOC{PSKOLEM\_CONV}

\TYPE {\small\verb%PSKOLEM_CONV : conv%}\egroup

\SYNOPSIS
Proves the existence of a pair of Skolem functions.

\DESCRIBE
When applied to an argument of the form {\small\verb%!p1...pn. ?q. tm%}, the conversion
{\small\verb%PSKOLEM_CONV%} returns the theorem:
{\par\samepage\setseps\small
\begin{verbatim}
   |- (!p1...pn. ?q. tm) = (?q'. !p1...pn. tm[q' p1 ... pn/yq)
\end{verbatim}
}
\noindent where {\small\verb%q'%} is a primed variant of the pair {\small\verb%q%} not free in the
input term.

\FAILURE
{\small\verb%PSKOLEM_CONV tm%} fails if {\small\verb%tm%} is not a term of the form {\small\verb%!p1...pn. ?q. tm%}.

\EXAMPLE Both {\small\verb%q%} and any {\small\verb%pi%} may be a paired structure of variables:
{\par\samepage\setseps\small
\begin{verbatim}
   #PSKOLEM_CONV
     "!(x11:*,x12:*) (x21:*,x22:*). ?(y1:*,y2:*). tm x11 x12 x21 x21 y1 y2";;
   |- (!(x11,x12) (x21,x22). ?(y1,y2). tm x11 x12 x21 x21 y1 y2) =
      (?(y1,y2).
        !(x11,x12) (x21,x22).
         tm x11 x12 x21 x21(y1(x11,x12)(x21,x22))(y2(x11,x12)(x21,x22)))
\end{verbatim}
}

\SEEALSO
SKOLEM_CONV, P_PSKOLEM_CONV.

\ENDDOC
\DOC{PSPEC}

\TYPE {\small\verb%PSPEC : (term -> thm -> thm)%}\egroup

\SYNOPSIS
Specializes the conclusion of a theorem.

\DESCRIBE
When applied to a term {\small\verb%q%} and a theorem {\small\verb%A |- !p. t%}, then {\small\verb%PSPEC%} returns
the theorem {\small\verb%A |- t[q/p]%}.
If necessary, variables will be renamed prior to the specialization to ensure
that {\small\verb%q%} is free for {\small\verb%p%} in {\small\verb%t%}, that is,
no variables free in {\small\verb%q%} become bound after substitution.
{\par\samepage\setseps\small
\begin{verbatim}
     A |- !p. t
   --------------  PSPEC "q"
    A |- t[q/p]
\end{verbatim}
}
\FAILURE
Fails if the theorem's conclusion is not a paired universal quantification,
or if {\small\verb%p%} and {\small\verb%q%} have different types.

\EXAMPLE
{\small\verb%PSPEC%} specialised paired quantifications.
{\par\samepage\setseps\small
\begin{verbatim}
   #PSPEC "(1,2)" (ASSUME "!(x,y). (x + y) = (y + x)");;
   . |- 1 + 2 = 2 + 1
\end{verbatim}
}
{\small\verb%PSPEC%} treats paired structures of variables as variables and 
preserves structure accordingly.
{\par\samepage\setseps\small
\begin{verbatim}
   #PSPEC "x:*#*" (ASSUME "!(x:*,y:*). (x,y) = (x,y)");;
   . |- x = x
\end{verbatim}
}

\SEEALSO
SPEC, IPSPEC, PSPECL, PSPEC_ALL, PSPEC_VAR, PGEN, PGENL, PGEN_ALL.

\ENDDOC
\DOC{PSPECL}

\TYPE {\small\verb%PSPECL : (term list -> thm -> thm)%}\egroup

\SYNOPSIS
Specializes zero or more pairs in the conclusion of a theorem.

\DESCRIBE
When applied to a term list {\small\verb%[q1;...;qn]%} and a theorem
{\small\verb%A |- !p1...pn. t%}, the inference rule {\small\verb%SPECL%} returns the theorem
{\small\verb%A |- t[q1/p1]...[qn/pn]%}, where the substitutions are made
sequentially left-to-right in the same way as for {\small\verb%PSPEC%}.
{\par\samepage\setseps\small
\begin{verbatim}
       A |- !p1...pn. t
   --------------------------  SPECL "[q1;...;qn]"
     A |- t[q1/p1]...[qn/pn]
\end{verbatim}
}
\noindent It is permissible for the term-list to be empty, in which case
the application of {\small\verb%PSPECL%} has no effect.

\FAILURE
Fails unless each of the terms is of the same type
as that of the appropriate quantified variable in the original
theorem.
Fails if the list of terms is longer than the number of quantified
pairs in the theorem.

\SEEALSO
SPECL, PGEN, PGENL, PGEN_ALL, PGEN_TAC, PSPEC, PSPEC_ALL, PSPEC_TAC.

\ENDDOC
\DOC{PSPEC\_ALL}

\TYPE {\small\verb%PSPEC_ALL : (thm -> thm)%}\egroup

\SYNOPSIS
Specializes the conclusion of a theorem with its own quantified pairs.

\DESCRIBE
When applied to a theorem {\small\verb%A |- !p1...pn. t%}, the inference rule {\small\verb%PSPEC_ALL%}
returns the theorem {\small\verb%A |- t[p1'/p1]...[pn'/pn]%} where the {\small\verb%pi'%} are distinct
variants of the corresponding {\small\verb%pi%}, chosen to avoid clashes with any variables
free in the assumption list and with the names of constants. Normally {\small\verb%pi'%} is
just {\small\verb%pi%}, in which case {\small\verb%PSPEC_ALL%} simply removes all universal quantifiers.
{\par\samepage\setseps\small
\begin{verbatim}
       A |- !p1...pn. t
   ---------------------------  PSPEC_ALL
    A |- t[p1'/x1]...[pn'/xn]
\end{verbatim}
}
\FAILURE
Never fails.

\SEEALSO
SPEC_ALL, PGEN, PGENL, PGEN_ALL, PGEN_TAC, PSPEC, PSPECL, PSPEC_TAC.

\ENDDOC
\DOC{PSPEC\_PAIR}

\TYPE {\small\verb%PSPEC_PAIR : (thm -> (term # thm))%}\egroup

\SYNOPSIS
Specializes the conclusion of a theorem, returning the chosen variant.

\DESCRIBE
When applied to a theorem {\small\verb%A |- !p. t%}, the inference rule {\small\verb%PSPEC_PAIR%}
returns the term {\small\verb%q'%} and the theorem {\small\verb%A |- t[q'/p]%}, where {\small\verb%q'%} is a variant
of {\small\verb%p%} chosen to avoid free variable capture.
{\par\samepage\setseps\small
\begin{verbatim}
     A |- !p. t
   --------------  PSPEC_PAIR
    A |- t[q'/q]
\end{verbatim}
}
\FAILURE
Fails unless the theorem's conclusion is a paired universal quantification.

\COMMENTS
This rule is very similar to plain {\small\verb%PSPEC%}, except that it returns the
variant chosen, which may be useful information under some circumstances.

\SEEALSO
SPEC_VAR, PGEN, PGENL, PGEN_ALL, PGEN_TAC, PSPEC, PSPECL, PSPEC_ALL.

\ENDDOC
\DOC{PSPEC\_TAC}

\TYPE {\small\verb%PSPEC_TAC : ((term # term) -> tactic)%}\egroup

\SYNOPSIS
Generalizes a goal.

\DESCRIBE
When applied to a pair of terms {\small\verb%(q,p)%}, where {\small\verb%p%} is a paired structure of
variables and a goal {\small\verb%A ?- t%}, the tactic {\small\verb%PSPEC_TAC%} generalizes the
goal to {\small\verb%A ?- !p. t[p/q]%}, that is, all components of {\small\verb%q%} are turned into 
the corresponding components of {\small\verb%p%}.
{\par\samepage\setseps\small
\begin{verbatim}
        A ?- t
   =================  PSPEC_TAC ("q","p")
    A ?- !x. t[p/q]
\end{verbatim}
}
\FAILURE
Fails unless {\small\verb%p%} is a paired structure of variables with the same type as {\small\verb%q%}.

\EXAMPLE
{\par\samepage\setseps\small
\begin{verbatim}
g "1 + 2 = 2 + 1";;
"1 + 2 = 2 + 1"

() : void

#e (PSPEC_TAC ("(1,2)","(x:num,y:num)"));;
OK..
"!(x,y). x + y = y + x"

() : void
\end{verbatim}
}

\USES
Removing unnecessary speciality in a goal, particularly as a prelude to
an inductive proof.

\SEEALSO
PGEN, PGENL, PGEN_ALL, PGEN_TAC, PSPEC, PSPECL, PSPEC_ALL, PSTRIP_TAC.

\ENDDOC
\DOC{PSTRIP\_ASSUME\_TAC}

\TYPE {\small\verb%PSTRIP_ASSUME_TAC : thm_tactic%}\egroup

\SYNOPSIS
Splits a theorem into a list of theorems and then adds them to the assumptions.

\DESCRIBE
Given a theorem {\small\verb%th%} and a goal {\small\verb%(A,t)%}, {\small\verb%PSTRIP_ASSUME_TAC th%} splits {\small\verb%th%} into
a list of theorems. This is done by recursively breaking conjunctions into
separate conjuncts, cases-splitting disjunctions, and eliminating paired
existential quantifiers by choosing arbitrary variables.  Schematically,
the following rules are applied:
{\par\samepage\setseps\small
\begin{verbatim}
           A ?- t
   ======================  PSTRIP_ASSUME_TAC (A' |- v1 /\ ... /\ vn)
    A u {v1,...,vn} ?- t

                A ?- t
   =================================  PSTRIP_ASSUME_TAC (A' |- v1 \/ ... \/ vn)
    A u {v1} ?- t ... A u {vn} ?- t

          A ?- t
   ====================  PSTRIP_ASSUME_TAC (A' |- ?p. v)
    A u {v[p'/p]} ?- t
\end{verbatim}
}
\noindent where {\small\verb%p'%} is a variant of the pair {\small\verb%p%}.

If the conclusion of {\small\verb%th%} is not a conjunction, a disjunction or a paired
existentially quantified term, the whole theorem {\small\verb%th%} is added to the
assumptions.

As assumptions are generated, they are examined to see if they solve the goal
(either by being alpha-equivalent to the conclusion of the goal or by deriving
a contradiction).

The assumptions of the theorem being split are not added to the assumptions of
the goal(s), but they are recorded in the proof.  This means that if {\small\verb%A'%} is
not a subset of the assumptions {\small\verb%A%} of the goal (up to alpha-conversion),
{\small\verb%PSTRIP_ASSUME_TAC (A'|-v)%} results in an invalid tactic.

\FAILURE
Never fails.

\USES
{\small\verb%PSTRIP_ASSUME_TAC%} is used when applying a previously proved theorem to solve
a goal, or when enriching its assumptions so that resolution,
rewriting with assumptions and other operations involving assumptions have
more to work with.

\SEEALSO
PSTRIP_THM_THEN, ,PSTRIP_ASSUME_TAC, PSTRIP_GOAL_THEN, PSTRIP_TAC.

\ENDDOC
\DOC{PSTRIP\_GOAL\_THEN}

\TYPE {\small\verb%PSTRIP_GOAL_THEN : (thm_tactic -> tactic)%}\egroup

\SYNOPSIS
Splits a goal by eliminating one outermost connective, applying the
given theorem-tactic to the antecedents of implications.

\DESCRIBE
Given a theorem-tactic {\small\verb%ttac%} and a goal {\small\verb%(A,t)%}, {\small\verb%PSTRIP_GOAL_THEN%} removes one
outermost occurrence of one of the connectives {\small\verb%!%}, {\small\verb%==>%}, {\small\verb%~%} or {\small\verb%/\%} from the
conclusion of the goal {\small\verb%t%}.  If {\small\verb%t%} is a universally quantified term, then
{\small\verb%STRIP_GOAL_THEN%} strips off the quantifier.   Note that {\small\verb%PSTRIP_GOAL_THEN%}
will strip off paired universal quantifications.
{\par\samepage\setseps\small
\begin{verbatim}
      A ?- !p. u
   ==============  PSTRIP_GOAL_THEN ttac
    A ?- u[p'/p]
\end{verbatim}
}
\noindent where {\small\verb%p'%} is a primed variant that contains no variables that
appear free in the assumptions {\small\verb%A%}.  If {\small\verb%t%} is a conjunction,
then {\small\verb%PSTRIP_GOAL_THEN%} simply splits
the conjunction into two subgoals:
{\par\samepage\setseps\small
\begin{verbatim}
      A ?- v /\ w
   =================  PSTRIP_GOAL_THEN ttac
    A ?- v   A ?- w
\end{verbatim}
}
\noindent If {\small\verb%t%} is an implication {\small\verb%"u ==> v"%} and if:
{\par\samepage\setseps\small
\begin{verbatim}
      A ?- v
  ===============  ttac (u |- u)
     A' ?- v'
\end{verbatim}
}
\noindent then:
{\par\samepage\setseps\small
\begin{verbatim}
      A ?- u ==> v
  ====================  PSTRIP_GOAL_THEN ttac
        A' ?- v'
\end{verbatim}
}
\noindent Finally, a negation {\small\verb%~t%} is treated as the implication {\small\verb%t ==> F%}.

\FAILURE
{\small\verb%PSTRIP_GOAL_THEN ttac (A,t)%} fails if {\small\verb%t%} is not a paired universally
quantified term, an implication, a negation or a conjunction.  
Failure also occurs if the application of {\small\verb%ttac%} fails,
after stripping the goal.

\USES
{\small\verb%PSTRIP_GOAL_THEN%} is used when manipulating intermediate results (obtained by
stripping outer connectives from a goal) directly, rather than as assumptions.

\SEEALSO
PGEN_TAC, STRIP_GOAL_THEN, FILTER_PSTRIP_THEN, PSTRIP_TAC, FILTER_PSTRIP_TAC.

\ENDDOC
\DOC{PSTRIP\_TAC}

\TYPE {\small\verb%PSTRIP_TAC : tactic%}\egroup

\SYNOPSIS
Splits a goal by eliminating one outermost connective.

\DESCRIBE
Given a goal {\small\verb%(A,t)%}, {\small\verb%PSTRIP_TAC%} removes one outermost occurrence of one of 
the connectives {\small\verb%!%}, {\small\verb%==>%}, {\small\verb%~%} or {\small\verb%/\%} from the conclusion of the goal {\small\verb%t%}.
If {\small\verb%t%} is a universally quantified term, then {\small\verb%STRIP_TAC%} strips off the
quantifier. Note that {\small\verb%PSTRIP_TAC%} will strip off paired quantifications.
{\par\samepage\setseps\small
\begin{verbatim}
     A ?- !p. u
   ==============  PSTRIP_TAC
    A ?- u[p'/p]
\end{verbatim}
}
\noindent where {\small\verb%p'%} is a primed variant of the pair {\small\verb%p%} that does not contain
any variables that appear free in the assumptions {\small\verb%A%}.  If {\small\verb%t%} is a
conjunction, then {\small\verb%PSTRIP_TAC%} simply splits the conjunction into two subgoals:
{\par\samepage\setseps\small
\begin{verbatim}
      A ?- v /\ w
   =================  PSTRIP_TAC
    A ?- v   A ?- w
\end{verbatim}
}
\noindent If {\small\verb%t%} is an implication, {\small\verb%PSTRIP_TAC%} moves the antecedent into the
assumptions, stripping conjunctions, disjunctions and existential
quantifiers according to the following rules:
{\par\samepage\setseps\small
\begin{verbatim}
    A ?- v1 /\ ... /\ vn ==> v            A ?- v1 \/ ... \/ vn ==> v
   ============================        =================================
       A u {v1,...,vn} ?- v             A u {v1} ?- v ... A u {vn} ?- v

     A ?- (?p. w) ==> v
   =====================
    A u {w[p'/p]} ?- v
\end{verbatim}
}
\noindent where {\small\verb%p'%} is a primed variant of the pair {\small\verb%p%} that does not appear
free in {\small\verb%A%}. Finally, a negation {\small\verb%~t%} is treated as the implication {\small\verb%t ==> F%}.

\FAILURE
{\small\verb%PSTRIP_TAC (A,t)%} fails if {\small\verb%t%} is not a paired universally quantified term,
an implication, a negation or a conjunction.

\USES
When trying to solve a goal, often the best thing to do first
is {\small\verb%REPEAT PSTRIP_TAC%} to split the goal up into manageable pieces.

\SEEALSO
PGEN_TAC, PSTRIP_GOAL_THEN, FILTER_PSTRIP_THEN, STRIP_TAC, FILTER_PSTRIP_TAC.

\ENDDOC
\DOC{PSTRIP\_THM\_THEN}

\TYPE {\small\verb%PSTRIP_THM_THEN : thm_tactical%}\egroup

\SYNOPSIS
{\small\verb%PSTRIP_THM_THEN%} applies the given theorem-tactic using the result of
stripping off one outer connective from the given theorem.

\DESCRIBE
Given a theorem-tactic {\small\verb%ttac%}, a theorem {\small\verb%th%} whose conclusion is a
conjunction, a disjunction or a paired existentially quantified term,
and a goal {\small\verb%(A,t)%}, {\small\verb%STRIP_THM_THEN ttac th%} first strips apart the 
conclusion of {\small\verb%th%}, next applies {\small\verb%ttac%} to the theorem(s) resulting from the
stripping and then applies the resulting tactic to the goal.

In particular, when stripping a conjunctive theorem {\small\verb%A'|- u /\ v%}, the tactic
{\par\samepage\setseps\small
\begin{verbatim}
   ttac(u|-u) THEN ttac(v|-v)
\end{verbatim}
}
\noindent resulting from applying {\small\verb%ttac%} to the conjuncts, is applied to the
goal.  When stripping a disjunctive theorem {\small\verb%A'|- u \/ v%}, the tactics
resulting from applying {\small\verb%ttac%} to the disjuncts, are applied to split the goal
into two cases. That is, if
{\par\samepage\setseps\small
\begin{verbatim}
    A ?- t                           A ?- t
   =========  ttac (u|-u)    and    =========  ttac (v|-v)
    A ?- t1                          A ?- t2
\end{verbatim}
}
\noindent then:
{\par\samepage\setseps\small
\begin{verbatim}
         A ?- t
   ================== PSTRIP_THM_THEN ttac (A'|- u \/ v)
    A ?- t1  A ?- t2
\end{verbatim}
}
\noindent When stripping a paired existentially quantified theorem
{\small\verb%A'|- ?p. u%}, the tactic resulting from applying {\small\verb%ttac%} to the
body of the paired existential quantification, {\small\verb%ttac(u|-u)%},
is applied to the goal.  
That is, if:
{\par\samepage\setseps\small
\begin{verbatim}
    A ?- t
   =========  ttac (u|-u)
    A ?- t1
\end{verbatim}
}
\noindent then:
{\par\samepage\setseps\small
\begin{verbatim}
      A ?- t
   =============  PSTRIP_THM_THEN ttac (A'|- ?p. u)
      A ?- t1
\end{verbatim}
}
The assumptions of the theorem being split are not added to the assumptions of
the goal(s) but are recorded in the proof.  If {\small\verb%A'%} is not a subset of the
assumptions {\small\verb%A%} of the goal (up to alpha-conversion), {\small\verb%PSTRIP_THM_THEN ttac th%}
results in an invalid tactic.

\FAILURE
{\small\verb%PSTRIP_THM_THEN ttac th%} fails if the conclusion of {\small\verb%th%} is not a conjunction,
a disjunction or a paired  existentially quantification.  Failure also occurs
if the application of {\small\verb%ttac%} fails, after stripping the outer connective from
the conclusion of {\small\verb%th%}.

\USES
{\small\verb%PSTRIP_THM_THEN%} is used enrich the assumptions of a goal with a stripped
version of a previously-proved theorem.

\SEEALSO
STRIP_THM_THEN, , PSTRIP_ASSUME_TAC, PSTRIP_GOAL_THEN, PSTRIP_TAC.

\ENDDOC
\DOC{PSTRUCT\_CASES\_TAC}

\TYPE {\small\verb%PSTRUCT_CASES_TAC : thm_tactic%}\egroup

\SYNOPSIS
Performs very general structural case analysis.

\DESCRIBE
When it is applied to a theorem of the form:
{\par\samepage\setseps\small
\begin{verbatim}
   th = A' |- ?p11...p1m. (x=t1) /\ (B11 /\ ... /\ B1k) \/ ... \/
                ?pn1...pnp. (x=tn) /\ (Bn1 /\ ... /\ Bnp)
\end{verbatim}
}
\noindent in which there may be no paired existential quantifiers where a
`vector' of them is shown above, {\small\verb%PSTRUCT_CASES_TAC th%} splits a goal
{\small\verb%A ?- s%} into {\small\verb%n%} subgoals as follows:
{\par\samepage\setseps\small
\begin{verbatim}
                             A ?- s
   ===============================================================
    A u {B11,...,B1k} ?- s[t1/x] ... A u {Bn1,...,Bnp} ?- s[tn/x]
\end{verbatim}
}
\noindent that is, performs a case split over the possible constructions (the
{\small\verb%ti%}) of a term, providing as assumptions the given constraints, having
split conjoined constraints into separate assumptions. Note that unless {\small\verb%A'%}
is a subset of {\small\verb%A%}, this is an invalid tactic.

\FAILURE
Fails unless the theorem has the above form, namely a conjunction of
(possibly multiply paired existentially quantified) terms which assert the
equality of the same variable {\small\verb%x%} and the given terms.

\USES
Generating a case split from the axioms specifying a structure.

\SEEALSO
STRUCT_CASES_TAC.

\ENDDOC
\DOC{PSUB\_CONV}

\TYPE {\small\verb%PSUB_CONV : (conv -> conv)%}\egroup

\SYNOPSIS
Applies a conversion to the top-level subterms of a term.

\DESCRIBE
For any conversion {\small\verb%c%}, the function returned by {\small\verb%PSUB_CONV c%} is a conversion
that applies {\small\verb%c%} to all the top-level subterms of a term.  If the conversion
{\small\verb%c%} maps {\small\verb%t%} to {\small\verb%|- t = t'%}, then {\small\verb%SUB_CONV c%} maps a paired
abstraction {\small\verb%"\p.t"%} to the theorem:
{\par\samepage\setseps\small
\begin{verbatim}
   |- (\p.t) = (\p.t')
\end{verbatim}
}
\noindent That is, {\small\verb%PSUB_CONV c "\p.t"%} applies {\small\verb%c%} to the body of the
paired abstraction {\small\verb%"\p.t"%}.
If {\small\verb%c%} is a conversion that maps {\small\verb%"t1"%} to the theorem
{\small\verb%|- t1 = t1'%} and {\small\verb%"t2"%} to the theorem {\small\verb%|- t2 = t2'%}, then the conversion
{\small\verb%PSUB_CONV c%} maps an application {\small\verb%"t1 t2"%} to the theorem:
{\par\samepage\setseps\small
\begin{verbatim}
   |- (t1 t2) = (t1' t2')
\end{verbatim}
}
\noindent That is, {\small\verb%PSUB_CONV c "t1 t2"%} applies {\small\verb%c%} to the both the operator
{\small\verb%t1%} and the operand {\small\verb%t2%} of the application {\small\verb%"t1 t2"%}.  Finally, for any
conversion {\small\verb%c%}, the function returned by {\small\verb%PSUB_CONV c%} acts as the identity
conversion on variables and constants.  That is, if {\small\verb%"t"%} is a variable or
constant, then {\small\verb%PSUB_CONV c "t"%} returns {\small\verb%|- t = t%}.

\FAILURE
{\small\verb%PSUB_CONV c tm%} fails if {\small\verb%tm%} is a paired abstraction {\small\verb%"\p.t"%} and the 
conversion {\small\verb%c%} fails when applied to {\small\verb%t%},
or if {\small\verb%tm%} is an application {\small\verb%"t1 t2"%} and the
conversion {\small\verb%c%} fails when applied to either {\small\verb%t1%} or {\small\verb%t2%}.  The function
returned by {\small\verb%PSUB_CONV c%} may also fail if the ML function {\small\verb%c:term->thm%} is not,
in fact, a conversion (i.e. a function that maps a term {\small\verb%t%} to a theorem
{\small\verb%|- t = t'%}).

\SEEALSO
SUB_CONV, PABS_CONV, RAND_CONV, RATOR_CONV.

\ENDDOC
\DOC{pvariant}

\TYPE {\small\verb%pvariant : (term list -> term -> term)%}\egroup

\SYNOPSIS
Modifies variable and constant names in a paired structure to avoid clashes.

\DESCRIBE
When applied to a list of (possibly paired structures of) variables to avoid
clashing with, and a pair to modify, {\small\verb%pvariant%} returns a variant of the pair.
That is, it changes the names of variables and constants in the pair as
intuitively as possible to make them distinct from any variables in the list,
or any (non-hidden) constants.
This is normally done by adding primes to the names.

The exact form of the altered names should not be relied on,
except that the original variables will be unmodified unless they
are in the list to avoid clashing with.
Also note that if the same variable occurs more that one in the pair,
then each instance of the variable will be modified in the same way.

\FAILURE
{\small\verb%pvariant l p%} fails if any term in the list {\small\verb%l%} is not a paired structure of
variables, or if {\small\verb%p%} is not a paired structure of variables and constants.

\EXAMPLE
The following shows a case that exhibits most possible behaviours:
{\par\samepage\setseps\small
\begin{verbatim}
   #pvariant ["b:*"; "(c:*,c':*)"] "((a:*,b:*),(c:*,b':*,T,b:*))";;
   "(a,b''),c'',b',T',b''" : term
\end{verbatim}
}

\USES
The function {\small\verb%pvariant%} is extremely useful for complicated derived rules which
need to rename pairs variable to avoid free variable capture while still
making the role of the pair obvious to the user.

\SEEALSO
variant, genvar, hide_constant, genlike.

\ENDDOC
\DOC{P\_FUN\_EQ\_CONV}

\TYPE {\small\verb%P_FUN_EQ_CONV : (term -> conv)%}\egroup

\SYNOPSIS
Performs extensionality conversion for functions (function equality).

\DESCRIBE
The conversion {\small\verb%P_FUN_EQ_CONV%} embodies the fact that two functions are equal
precisely when they give the same results for all values to which they can be
applied. For any paired variable structure {\small\verb%"p"%} and equation {\small\verb%"f = g"%},
where {\small\verb%p%} is of type {\small\verb%ty1%} and {\small\verb%f%} and {\small\verb%g%} are functions of type {\small\verb%ty1->ty2%},
a call to {\small\verb%P_FUN_EQ_CONV "p" "f = g"%} returns the theorem:
{\par\samepage\setseps\small
\begin{verbatim}
   |- (f = g) = (!p. f p = g p)
\end{verbatim}
}
\FAILURE
{\small\verb%P_FUN_EQ_CONV p tm%} fails if {\small\verb%p%} is not a paired structure of variables
or if {\small\verb%tm%} is not an equation {\small\verb%f = g%} where {\small\verb%f%} and {\small\verb%g%} are functions.  
Furthermore, if {\small\verb%f%} and {\small\verb%g%} are functions of type {\small\verb%ty1->ty2%},
then the pair {\small\verb%x%} must have type {\small\verb%ty1%}; otherwise the conversion fails.  
Finally, failure also occurs if any of the variables in {\small\verb%p%} is free in either
{\small\verb%f%} or {\small\verb%g%}.

\SEEALSO
FUN_EQ_CONV, PEXT.

\ENDDOC
\DOC{P\_PCHOOSE\_TAC}

\TYPE {\small\verb%P_PCHOOSE_TAC : (term -> thm_tactic)%}\egroup

\SYNOPSIS
Assumes a theorem, with existentially quantified pair replaced by a given
witness.

\DESCRIBE
{\small\verb%P_PCHOOSE_TAC%} expects a pair {\small\verb%q%} and theorem with a paired existentially
quantified conclusion.  When applied to a goal, it adds a new
assumption obtained by introducing the pair {\small\verb%q%} as a witness for
the pair {\small\verb%p%} whose existence is asserted in the theorem.
{\par\samepage\setseps\small
\begin{verbatim}
           A ?- t
   ===================  P_CHOOSE_TAC "q" (A1 |- ?p. u)
    A u {u[q/p]} ?- t         ("y" not free anywhere)
\end{verbatim}
}
\FAILURE
Fails if the theorem's conclusion is not a paired existential quantification,
or if the first argument is not a paired structure of variables.  
Failures may arise in the tactic-generating function.  
An invalid tactic is produced if the introduced variable is free in {\small\verb%u%} or {\small\verb%t%},
or if the theorem has any hypothesis which is not alpha-convertible to an 
assumption of the goal.

\SEEALSO
X_CHOOSE_TAC, PCHOOSE, PCHOOSE_THEN, P_PCHOOSE_THEN.

\ENDDOC
\DOC{P\_PCHOOSE\_THEN}

\TYPE {\small\verb%P_PCHOOSE_THEN : (term -> thm_tactical)%}\egroup

\SYNOPSIS
Replaces existentially quantified pair with given witness,
and passes it to a theorem-tactic.

\DESCRIBE
{\small\verb%P_PCHOOSE_THEN%} expects a pair {\small\verb%q%}, a tactic-generating function
{\small\verb%f:thm->tactic%}, and a theorem of the form {\small\verb%(A1 |- ?p. u)%} as
arguments.  A new theorem is created by introducing the given pair
{\small\verb%q%} as a witness for the pair {\small\verb%p%} whose existence is asserted in the original
theorem, {\small\verb%(u[q/p] |- u[q/p])%}.  If the tactic-generating function {\small\verb%f%}
applied to this theorem produces results as follows when applied to a
goal {\small\verb%(A ?- u)%}:
{\par\samepage\setseps\small
\begin{verbatim}
    A ?- t
   =========  f ({u[q/p]} |- u[q/p])
    A ?- t1
\end{verbatim}
}
\noindent then applying {\small\verb%(P_PCHOOSE_THEN "q" f (A1 |- ?p. u))%} to the
goal {\small\verb%(A ?- t)%} produces the subgoal:
{\par\samepage\setseps\small
\begin{verbatim}
    A ?- t
   =========  P_PCHOOSE_THEN "q" f (A1 |- ?p. u)
    A ?- t1         ("q" not free anywhere)
\end{verbatim}
}
\FAILURE
Fails if the theorem's conclusion is not existentially quantified, or if
the first argument is not a paired structure of variables.  
Failures may arise in the tactic-generating function.  
An invalid tactic is produced if the introduced variable is free in {\small\verb%u%} or {\small\verb%t%},
or if the theorem has any hypothesis which is not alpha-convertible to an
assumption of the goal.

\SEEALSO
X_CHOOSE_THEN, PCHOOSE, PCHOOSE_THEN, P_PCHOOSE_TAC.

\ENDDOC
\DOC{P\_PGEN\_TAC}

\TYPE {\small\verb%P_PGEN_TAC : (term -> tactic)%}\egroup

\SYNOPSIS
Specializes a goal with the given paired structure of variables.

\DESCRIBE
When applied to a paired structure of variables {\small\verb%p'%}, and a goal
{\small\verb%A ?- !p. t%}, the tactic {\small\verb%P_PGEN_TAC%} returns the goal {\small\verb%A ?- t[p'/p]%}.
{\par\samepage\setseps\small
\begin{verbatim}
     A ?- !p. t
   ==============  P_PGEN_TAC "p'"
    A ?- t[p'/x]
\end{verbatim}
}
\FAILURE
Fails unless the goal's conclusion is a paired universal quantification
and the term a paired structure of variables of the appropriate type.
It also fails if any of the variables of the supplied structure occurs free
in either the assumptions or (initial) conclusion of the goal.

\SEEALSO
X_GEN_TAC, FILTER_PGEN_TAC, PGEN, PGENL, PGEN_ALL, PSPEC, PSPECL, PSPEC_ALL,
PSPEC_TAC.

\ENDDOC
\DOC{P\_PSKOLEM\_CONV}

\TYPE {\small\verb%P_PSKOLEM_CONV : (term -> conv)%}\egroup

\SYNOPSIS
Introduces a user-supplied Skolem function.

\DESCRIBE
{\small\verb%P_PSKOLEM_CONV%} takes two arguments.  The first is a variable {\small\verb%f%}, which
must range over functions of the appropriate type, and the second is a term of
the form {\small\verb%!p1...pn. ?q. t%} (where {\small\verb%pi%} and {\small\verb%q%} may be pairs).  
Given these arguments, {\small\verb%P_PSKOLEM_CONV%} returns the theorem:
{\par\samepage\setseps\small
\begin{verbatim}
   |- (!p1...pn. ?q. t) = (?f. !p1...pn. tm[f p1 ... pn/q])
\end{verbatim}
}
\noindent which expresses the fact that a skolem function {\small\verb%f%} of the
universally quantified variables {\small\verb%p1...pn%} may be introduced in place of the
the existentially quantified pair {\small\verb%p%}.

\FAILURE
{\small\verb%P_PSKOLEM_CONV f tm%} fails if {\small\verb%f%} is not a variable, or if the input term {\small\verb%tm%}
is not a term of the form {\small\verb%!p1...pn. ?q. t%}, or if the variable {\small\verb%f%} is free in
{\small\verb%tm%}, or if the type of {\small\verb%f%} does not match its intended use as an {\small\verb%n%}-place
curried function from the pairs {\small\verb%p1...pn%} to a value having the same type
as {\small\verb%p%}.

\SEEALSO
X_SKOLEM_CONV, PSKOLEM_CONV.

\ENDDOC
\DOC{RIGHT\_AND\_PEXISTS\_CONV}

\TYPE {\small\verb%RIGHT_AND_PEXISTS_CONV : conv%}\egroup

\SYNOPSIS
Moves a paired existential quantification of the right conjunct outwards
through a conjunction.

\DESCRIBE
When applied to a term of the form {\small\verb%t /\ (?p. t)%}, the conversion
{\small\verb%RIGHT_AND_PEXISTS_CONV%} returns the theorem:
{\par\samepage\setseps\small
\begin{verbatim}
   |- t /\ (?p. u) = (?p'. t /\ (u[p'/p]))
\end{verbatim}
}
\noindent where {\small\verb%p'%} is a primed variant of the pair {\small\verb%p%} that does not
contain any variables free in the input term.

\FAILURE
Fails if applied to a term not of the form {\small\verb%t /\ (?p. u)%}.

\SEEALSO
RIGHT_AND_EXISTS_CONV, AND_PEXISTS_CONV, PEXISTS_AND_CONV,
LEFT_AND_PEXISTS_CONV.

\ENDDOC
\DOC{RIGHT\_AND\_PFORALL\_CONV}

\TYPE {\small\verb%RIGHT_AND_PFORALL_CONV : conv%}\egroup

\SYNOPSIS
Moves a paired universal quantification of the right conjunct outwards through a
conjunction.

\DESCRIBE
When applied to a term of the form {\small\verb%t /\ (!p. u)%}, the conversion
{\small\verb%RIGHT_AND_PFORALL_CONV%} returns the theorem:
{\par\samepage\setseps\small
\begin{verbatim}
   |- t /\ (!p. u) = (!p'. t /\ (u[p'/p]))
\end{verbatim}
}
\noindent where {\small\verb%p'%} is a primed variant of the pair {\small\verb%p%} that does not 
contain any variables free in the input term.

\FAILURE
Fails if applied to a term not of the form {\small\verb%t /\ (!p. u)%}.

\SEEALSO
RIGHT_AND_FORALL_CONV, AND_PFORALL_CONV, PFORALL_AND_CONV,
LEFT_AND_PFORALL_CONV.

\ENDDOC
\DOC{RIGHT\_IMP\_PEXISTS\_CONV}

\TYPE {\small\verb%RIGHT_IMP_PEXISTS_CONV : conv%}\egroup

\SYNOPSIS
Moves a paired existential quantification of the consequent outwards through
an implication.

\DESCRIBE
When applied to a term of the form {\small\verb%t ==> (?p. u)%}, 
{\small\verb%RIGHT_IMP_PEXISTS_CONV%} returns the theorem:
{\par\samepage\setseps\small
\begin{verbatim}
   |- t ==> (?p. u) = (?p'. t ==> (u[p'/p]))
\end{verbatim}
}
\noindent where {\small\verb%p'%} is a primed variant of the pair {\small\verb%p%} that does not
contain any variables that appear free in the input term.

\FAILURE
Fails if applied to a term not of the form {\small\verb%t ==> (?p. u)%}.

\SEEALSO
RIGHT_IMP_EXISTS_CONV, PEXISTS_IMP_CONV, LEFT_IMP_PFORALL_CONV.

\ENDDOC
\DOC{RIGHT\_IMP\_PFORALL\_CONV}

\TYPE {\small\verb%RIGHT_IMP_PFORALL_CONV : conv%}\egroup

\SYNOPSIS
Moves a paired universal quantification of the consequent outwards through an
implication.

\DESCRIBE
When applied to a term of the form {\small\verb%t ==> (!p. u)%}, the conversion
{\small\verb%RIGHT_IMP_FORALL_CONV%} returns the theorem:
{\par\samepage\setseps\small
\begin{verbatim}
   |- t ==> (!p. u) = (!p'. t ==> (u[p'/p]))
\end{verbatim}
}
\noindent where {\small\verb%p'%} is a primed variant of the pair {\small\verb%p%} that does not
contain any variables that appear free in the input term.

\FAILURE
Fails if applied to a term not of the form {\small\verb%t ==> (!p. u)%}.

\SEEALSO
RIGHT_IMP_FORALL_CONV, PFORALL_IMP_CONV, LEFT_IMP_PEXISTS_CONV.

\ENDDOC
\DOC{RIGHT\_LIST\_PBETA}

\TYPE {\small\verb%RIGHT_LIST_PBETA : (thm -> thm)%}\egroup

\SYNOPSIS
Iteratively beta-reduces a top-level paired beta-redex on the right-hand side
of an equation.

\DESCRIBE
When applied to an equational theorem, {\small\verb%RIGHT_LIST_PBETA%} applies paired 
beta-reduction over a top-level chain of beta-redexes to the right-hand side
(only).
Variables are renamed if necessary to avoid free variable capture.
{\par\samepage\setseps\small
\begin{verbatim}
    A |- s = (\p1...pn. t) q1 ... qn
   ----------------------------------  RIGHT_LIST_BETA
       A |- s = t[q1/p1]...[qn/pn]
\end{verbatim}
}
\FAILURE
Fails unless the theorem is equational, with its right-hand side being
a top-level paired beta-redex.

\SEEALSO
RIGHT_LIST_BETA, PBETA_CONV, PBETA_RULE, PBETA_TAC, LIST_PBETA_CONV,
RIGHT_PBETA, LEFT_PBETA, LEFT_LIST_PBETA.

\ENDDOC
\DOC{RIGHT\_OR\_PEXISTS\_CONV}

\TYPE {\small\verb%RIGHT_OR_PEXISTS_CONV : conv%}\egroup

\SYNOPSIS
Moves a paired existential quantification of the right disjunct outwards
through a disjunction.

\DESCRIBE
When applied to a term of the form {\small\verb%t \/ (?p. u)%}, the conversion
{\small\verb%RIGHT_OR_PEXISTS_CONV%} returns the theorem:
{\par\samepage\setseps\small
\begin{verbatim}
   |- t \/ (?p. u) = (?p'. t \/ (u[p'/p]))
\end{verbatim}
}
\noindent where {\small\verb%p'%} is a primed variant of the pair {\small\verb%p%} that does not 
contain any variables free in the input term.

\FAILURE
Fails if applied to a term not of the form {\small\verb%t \/ (?p. u)%}.

\SEEALSO
RIGHT_OR_EXISTS_CONV, OR_PEXISTS_CONV, PEXISTS_OR_CONV, LEFT_OR_PEXISTS_CONV.

\ENDDOC
\DOC{RIGHT\_OR\_PFORALL\_CONV}

\TYPE {\small\verb%RIGHT_OR_PFORALL_CONV : conv%}\egroup

\SYNOPSIS
Moves a paired universal quantification of the right disjunct outwards through a
disjunction.

\DESCRIBE
When applied to a term of the form {\small\verb%t \/ (!p. u)%}, the conversion
{\small\verb%RIGHT_OR_FORALL_CONV%} returns the theorem:
{\par\samepage\setseps\small
\begin{verbatim}
   |- t \/ (!p. u) = (!p'. t \/ (u[p'/p]))
\end{verbatim}
}
\noindent where {\small\verb%p'%} is a primed variant of the pair {\small\verb%p%} that does not
contain any variables that appear free in the input term.

\FAILURE
Fails if applied to a term not of the form {\small\verb%t \/ (!p. u)%}.

\SEEALSO
RIGHT_OR_FORALL_CONV, OR_PFORALL_CONV, PFORALL_OR_CONV, LEFT_OR_PFORALL_CONV.

\ENDDOC
\DOC{RIGHT\_PBETA}

\TYPE {\small\verb%RIGHT_PBETA : (thm -> thm)%}\egroup

\SYNOPSIS
Beta-reduces a top-level paired beta-redex on the right-hand side of an 
equation.

\DESCRIBE
When applied to an equational theorem, {\small\verb%RIGHT_PBETA%} applies paired
beta-reduction at top level to the right-hand side (only).
Variables are renamed if necessary to avoid free variable capture.
{\par\samepage\setseps\small
\begin{verbatim}
    A |- s = (\p. t1) t2
   ----------------------  RIGHT_PBETA
     A |- s = t1[t2/p]
\end{verbatim}
}
\FAILURE
Fails unless the theorem is equational, with its right-hand side being
a top-level paired beta-redex.

\SEEALSO
RIGHT_BETA, PBETA_CONV, PBETA_RULE, PBETA_TAC, RIGHT_LIST_PBETA, LEFT_PBETA,
LEFT_LIST_PBETA.

\ENDDOC
\DOC{rip\_pair}

\TYPE {\small\verb%rip_pair : (term -> term list)%}\egroup

\SYNOPSIS
Recursively breaks a paired structure into its constituent pieces.

\EXAMPLE
{\par\samepage\setseps\small
\begin{verbatim}
#rip_pair "((1,2),(3,4))";;
["1"; "2"; "3"; "4"] : term list
\end{verbatim}
}

\COMMENTS
Note that {\small\verb%rip_pair%} is similar, but not identical, to {\small\verb%strip_pair%}
which iteratively breaks apart tuples (flat paired structures).

\FAILURE
Never fails.

\SEEALSO
strip_pair.

\ENDDOC
\DOC{strip\_pabs}

\TYPE {\small\verb%strip_pabs : (term -> goal)%}\egroup

\SYNOPSIS
Iteratively breaks apart paired abstractions.

\DESCRIBE
{\small\verb%strip_pabs "\p1 ... pn. t"%} returns {\small\verb%(["p1";...;"pn"],"t")%}. Note that
{\par\samepage\setseps\small
\begin{verbatim}
   strip_pabs(list_mk_abs(["p1";...;"pn"],"t"))
\end{verbatim}
}
\noindent will not return {\small\verb%(["p1";...;"pn"],"t")%} if {\small\verb%t%} is
a paired abstraction.

\FAILURE
Never fails.

\SEEALSO
strip_abs, list_mk_pabs, dest_pabs.

\ENDDOC
\DOC{strip\_pexists}

\TYPE {\small\verb%strip_pexists : (term -> goal)%}\egroup

\SYNOPSIS
Iteratively breaks apart paired existential quantifications.

\DESCRIBE
{\small\verb%strip_pexists "?p1 ... pn. t"%} returns {\small\verb%(["p1";...;"pn"],"t")%}. Note that
{\par\samepage\setseps\small
\begin{verbatim}
   strip_pexists(list_mk_pexists(["[p1";...;"pn"],"t"))
\end{verbatim}
}
\noindent will not return {\small\verb%(["p1";...;"pn"],"t")%} if {\small\verb%t%} is a paired
existential quantification.

\FAILURE
Never fails.

\SEEALSO
strip_exists, list_mk_pexists, dest_pexists.

\ENDDOC
\DOC{strip\_pforall}

\TYPE {\small\verb%strip_pforall : (term -> goal)%}\egroup

\SYNOPSIS
Iteratively breaks apart paired universal quantifications.

\DESCRIBE
{\small\verb%strip_pforall "!p1 ... pn. t"%} returns {\small\verb%(["p1";...;"pn"],"t")%}. Note that
{\par\samepage\setseps\small
\begin{verbatim}
   strip_pforall(list_mk_pforall(["p1";...;"pn"],"t"))
\end{verbatim}
}
\noindent will not return {\small\verb%(["p1";...;"pn"],"t")%} if {\small\verb%t%} is a paired universal
quantification.

\FAILURE
Never fails.

\SEEALSO
strip_forall, list_mk_pforall, dest_pforall.

\ENDDOC
\DOC{SWAP\_PEXISTS\_CONV}

\TYPE {\small\verb%SWAP_PEXISTS_CONV : conv%}\egroup

\SYNOPSIS
Interchanges the order of two existentially quantified pairs.

\DESCRIBE
When applied to a term argument of the form {\small\verb%?p q. t%}, the conversion
{\small\verb%SWAP_PEXISTS_CONV%} returns the theorem:
{\par\samepage\setseps\small
\begin{verbatim}
   |- (?p q. t) = (?q t. t)
\end{verbatim}
}

\FAILURE
{\small\verb%SWAP_PEXISTS_CONV%} fails if applied to a term that is not of the form
{\small\verb%?p q. t%}.

\SEEALSO
SWAP_EXISTS_CONV, SWAP_PFORALL_CONV.

\ENDDOC
\DOC{SWAP\_PFORALL\_CONV}

\TYPE {\small\verb%SWAP_PFORALL_CONV : conv%}\egroup

\SYNOPSIS
Interchanges the order of two universally quantified pairs.

\DESCRIBE
When applied to a term argument of the form {\small\verb%!p q. t%}, the conversion
{\small\verb%SWAP_PFORALL_CONV%} returns the theorem:
{\par\samepage\setseps\small
\begin{verbatim}
   |- (!p q. t) = (!q t. t)
\end{verbatim}
}

\FAILURE
{\small\verb%SWAP_PFORALL_CONV%} fails if applied to a term that is not of the form
{\small\verb%!p q. t%}.

\SEEALSO
SWAP_PEXISTS_CONV.

\ENDDOC
\DOC{UNCURRY\_CONV}

\TYPE {\small\verb%UNCURRY_CONV : conv%}\egroup

\SYNOPSIS
Uncurrys an application of an abstraction.

\EXAMPLE
{\par\samepage\setseps\small
\begin{verbatim}
#UNCURRY_CONV "(\x y. x + y) 1 2";;
|- (\x y. x + y)1 2 = (\(x,y). x + y)(1,2)
\end{verbatim}
}

\FAILURE
{\small\verb%UNCURRY_CONV tm%} fails if {\small\verb%tm%} is not double abstraction applied to two 
arguments

\SEEALSO
CURRY_CONV.

\ENDDOC
\DOC{UNCURRY\_EXISTS\_CONV}

\TYPE {\small\verb%UNCURRY_EXISTS_CONV : conv%}\egroup

\SYNOPSIS
Uncurrys consecutive existential quantifications into
a paired existential quantification.

\EXAMPLE
{\par\samepage\setseps\small
\begin{verbatim}
#UNCURRY_EXISTS_CONV "?x y. x + y = y + x";;
|- (?x y. x + y = y + x) = (?(x,y). x + y = y + x)

#UNCURRY_EXISTS_CONV "?(w,x) (y,z). w+x+y+z = z+y+x+w";;
|- (?(w,x) (y,z). w + (x + (y + z)) = z + (y + (x + w))) =
   (?((w,x),y,z). w + (x + (y + z)) = z + (y + (x + w)))
\end{verbatim}
}

\FAILURE
{\small\verb%UNCURRY_EXISTS_CONV tm%} fails if {\small\verb%tm%} is not a
consecutive existential quantification.

\SEEALSO
CURRY_CONV, UNCURRY_CONV, CURRY_EXISTS_CONV, CURRY_FORALL_CONV, 
UNCURRY_FORALL_CONV.

\ENDDOC
\DOC{UNCURRY\_FORALL\_CONV}

\TYPE {\small\verb%UNCURRY_FORALL_CONV : conv%}\egroup

\SYNOPSIS
Uncurrys consecutive universal quantifications into a paired
universal quantification.

\EXAMPLE
{\par\samepage\setseps\small
\begin{verbatim}
#UNCURRY_FORALL_CONV "!x y. x + y = y + x";;
|- (!x y. x + y = y + x) = (!(x,y). x + y = y + x)

#UNCURRY_FORALL_CONV "!(w,x) (y,z). w+x+y+z = z+y+x+w";;
|- (!(w,x) (y,z). w + (x + (y + z)) = z + (y + (x + w))) =
   (!((w,x),y,z). w + (x + (y + z)) = z + (y + (x + w)))
\end{verbatim}
}

\FAILURE
{\small\verb%UNCURRY_FORALL_CONV tm%} fails if {\small\verb%tm%} is not a
consecutive universal quantification.

\SEEALSO
CURRY_CONV, UNCURRY_CONV, CURRY_FORALL_CONV, CURRY_EXISTS_CONV, 
UNCURRY_EXISTS_CONV.

\ENDDOC
\DOC{UNPBETA\_CONV}

\TYPE {\small\verb%UNPBETA_CONV : (term -> conv)%}\egroup

\SYNOPSIS
Creates an application of a paired abstraction from a term.

\DESCRIBE
The user nominates some pair structure of variables {\small\verb%p%} and a term {\small\verb%t%},
and {\small\verb%UNPBETA_CONV%} turns {\small\verb%t%} into an abstraction on {\small\verb%p%} applied to {\small\verb%p%}.
{\par\samepage\setseps\small
\begin{verbatim}
   ------------------  UNPBETA_CONV "p" "t"
    |- t = (\p. t) p
\end{verbatim}
}
\FAILURE
Fails if {\small\verb%p%} is not a paired structure of variables.

\SEEALSO
PBETA_CONV, PAIRED_BETA_CONV.

\ENDDOC
