% title.tex %
% Author, prerequisites and abstract %

% ---------------------------------------------------------------------
% Parameters customising the document; set by whoever \input's it
% ---------------------------------------------------------------------
% \self{} is one word denoting this document, such as "article" or "chapter"
% \path{} is the path denoting the case-study directory
%
% Typical examples:
% \def\self{article}
% \def\path{\verb%Training/studies/binomial%}
% ---------------------------------------------------------------------

\begin{titlepage}

\begin{center}
{\Huge\bf HOL CASE STUDY}\\
\vskip .3in
{\Large\bf The Binomial Theorem proven in HOL.}\\
\end{center}
\vskip .4in

\begin{inset}{Author}
Andrew Gordon\newline
Computer Laboratory\newline
New Museums Site\newline
Pembroke Street\newline
Cambridge CB2 3QG, UK.\newline
{\bf Telephone:} +44 223 334650\newline
{\bf Email:} {\verb%Andrew.Gordon@cl.cam.ac.uk%}
\end{inset}

\begin{inset}{Concepts illustrated}
A simple way to define algebraic structures in \HOL{}.
Translation of mathematical definitions such as sums of series
and binomial coefficients into \HOL{}.
Division of a medium sized proof into manageable lemmas.
\end{inset}

\begin{inset}{Prerequisites}
At least as much knowledge of \HOL\  as is
given in {\sl Getting started with HOL}.
Basic understanding of how to use the goal stack to do forward proof,
but not necessarily possession of a large vocabulary of tactics.
\end{inset}

\begin{inset}{Supporting files}
In the directory \path{}.
\end{inset}

\begin{inset}{Abstract}
The Binomial Theorem in \HOL{} is a medium sized worked example whose subject 
matter is more widely known than any specific piece of hardware 
or software. This \self{} introduces the small amount of algebra and 
mathematical notation needed to state and prove the Binomial Theorem, shows 
how this is rendered in \HOL{}, and outlines the structure of the proof.
\end{inset}

\setcounter{page}{1}
\end{titlepage}

\newpage
\tableofcontents 
\addtocontents{toc}{\protect\thispagestyle{empty}}
\newpage
