%File:elsa/papers/L-domains/Jim-L-abst.tex
%This is the root file for the paper about L-domains

\documentstyle[12pt]{article}

%Spacing between rows
\def\baselinestretch{1.2}

%Width of text
\setlength{\textwidth}{6in}
%\setlength{\oddsidemargin}{.6in}

%Height of text and page numbers
\setlength{\topmargin}{0in}
\setlength{\textheight}{8.5in}

%To give empty headers
\pagestyle{myheadings}                       %for page numbers in top
\markright{\ }                                 %right-hand corner
%\pagestyle{headings}

\reversemarginpar                            %makes marginal notes on left

\begin{document}

\sloppy
\thicklines
\setlength{\unitlength}{6em}

%Macros file

%******************************DATE**********************************



%***********************EQALIGN & EQALIGNNO**************************

\def\mxxth{\mathsurround=0pt}
\dimendef\dimenxx=0
\def\openup{\afterassignment\xxpenup\dimenxx=}
\def\xxpenup{\advance\lineskip\dimenxx
  \advance\baselineskip\dimenxx \advance\lineskiplimit\dimenxx}
\def\eqalign#1{\,\vcenter{\openup1\jot \mxxth
  \ialign{\strut\hfil$\displaystyle{##}$&$\displaystyle{{}##}$\hfil
     \crcr#1\crcr}}\,}

\newif\ifdtxxp
\def\displxxy{\global\dtxxptrue \openup1\jot \mxxth
  \everycr{\noalign{\ifdtxxp \global\dtxxpfalse
      \vskip-\lineskiplimit \vskip\normallineskiplimit
      \else \penalty\interdisplaylinepenalty \fi}}}
\def\displaylines#1{\displxxy
  \halign{\hbox to\displaywidth{$\hfil\displaystyle##\hfil$}\crcr
      #1\crcr}}

\newskip\mycntring \mycntring=0pt plus 1000pt minus 1000pt
\def\eqalignno#1{\displxxy \tabskip=\mycntring
  \halign to\displaywidth{\hfil$\displaystyle{##}$\tabskip=0pt
      &$\displaystyle{{}##}$\hfil\tabskip=\mycntring
      &\llap{$##$}\tabskip=0pt\crcr
      #1\crcr}}

%***********************THEOREM TYPE JUNK*****************************

\newtheorem{theorem}{Theorem}[section]		    %for article style
\newtheorem{cor}[theorem]{Corollary}
\newtheorem{prop}[theorem]{Proposition}
\newtheorem{lemma}[theorem]{Lemma}
\newtheorem{example}[theorem]{Example}
\newtheorem{df}[theorem]{Definition}
\newenvironment{definition}{\begin{df}\rm}{\end{df}}

% Some environments for displaying things %

\newenvironment{display}[1]{\begin{trivlist}\item[\hspace{\labelsep}{\bf
            #1}]}{\end{trivlist}}

%\newenvironment{example}{\begin{display}{Example:}}{\end{display}}%

\renewcommand{\proof}{\noindent{\it Proof. }}
\newcommand{\qed}{\bigskip\rule[-.4ex]{.4em}{2ex}}
\newcommand{\cons}{\noindent{\it Construction. }}
\newenvironment{claim}{\begin{display}{\it Claim:}}{\end{display}}
\newenvironment{note}{\begin{display}{Note:}}{\end{display}}

%*\newcommand{\imp}{\Rightarrow}
\renewcommand{\l}{\char'134}
\renewcommand{\lnot}{\char'176}
\renewcommand{\land}{/{\l}}
\renewcommand{\lor}{{\l}/}




%for use on "rough draft" copies
%\newcommand{\mylabel}[1]{\label{#1}\marginpar{{\rm\scriptsize #1}}}
%\newcommand{\myref}[1]{\ref{#1}\marginpar{{\rm\scriptsize used #1}}}

%for use on "final" copies
\newcommand{\mylabel}[1]{\label{#1}}
\newcommand{\myref}[1]{\ref{#1}}

\begin{center}
\LARGE 
Doing Algebra in Simple Type Theory\\
\normalsize
Elsa L.~Gunter\\
University of Pennsylvania\\
\date\\
\end{center}
\begin{quotation}
\begin{center} \bf Abstract \end{center}

To fully utilize the power of higher-order logic in interactive theorem
proving,  it is desirable to be able to develop abstract areas of Mathematics
such as algebra and topology in an automated setting.  Theorems provers
capable of higher order reasoning have generally had some form of type
theory as their object language.  But mathematicians have tended to use
the language of set theory to give definitions and prove theorems in
algebra and topology.  In this paper, we give an incremental description
of how to express various basic algebraic concepts in terms of simple type
theory.  We present a method for representing algebras, subalgebras,
quotient algebras, homomorphisms and isomorphisms in simple type theory, using
group theory as an example in each case.  Following this, we discuss how
to automatically apply such an abstract theory to concrete examples.
Finally, we conclude with some observations about a
potential inconvenience associated with this method of representation,
and discuss a difficulty inherent in any attempt to remove this
inconvenience.

\end{quotation}

\end{document}
