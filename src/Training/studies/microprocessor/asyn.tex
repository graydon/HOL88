\section{Using temporal logic to deal with asynchrony}
\label{sec-asyn}

This section gives a very brief outline of the formal verification
of \Tamarack\ when operating in fully asynchronous mode.
The main specification problem is to define a behavioural model
for fully asynchronous memory which is completely general and
independent from the microprocessor specification.
The main verification problem is to derive an abstract view of
the instruction cycle where handshaking interactions are collapsed
into fixed sequences of steps.

Our solution to these problems is to embed a form of
temporal logic in higher-order logic by regarding a set of
temporal logic operators as abbreviations for higher-order functions.
Temporal logic is especially useful for specifying and reasoning
about relative order in a set of events.
This approach results in succinct specifications and simplifies
the verification task by hiding some
low-level aspects of proof, in particular, the use of mathematical
induction, in derived inference rules for temporal logic operators.

In this section, we only present details on how temporal logic
operators are used to specify constraints on the handshaking signals
themselves, i.e., \verb"dreq" and \verb"dack".
A complete model of external memory also involves a formal specification
of constraints on the address and data signals which are synchronized by the
handshaking signals.  For completeness,
this latter part of the memory specification
is given in Appendix~\ref{apx-asynmem} but
these details are not discussed here.

\subsection{Specification using temporal logic operators}

We use the temporal logic operators,

\begin{quote}
\begin{tabular}{ll}
$\Box$ & - ``henceforth''\\
$\Diamond$ & -  ``eventually''\\
$\TLNext$ & - ``next''\\
$\cup$ & - ``until''
\end{tabular}
\end{quote}

\noindent
and a number of connectives
({\TLNot}, $\longrightarrow$, {\TLAnd})
corresponding to logical connectives in propositional logic.
Later on, we give precise definitions for these operators and
connectives as abbreviations for higher-order functions.
For now, the intuitive meanings suggested above are enough to
illustrate the use of these operators in the formal
specification of a handshaking protocol.

Figure~\ref{fig-style} shows a stylized timing diagram for a four-phase
handshaking sequence.
A request at time \verb"t1"
by the sender is acknowledged at time \verb"t2" by the receiver.
A request to end the interaction is signaled at time \verb"t3"
and eventually acknowledged at time \verb"t4".

\begin{figure}
\begin{center}
% ===== Handshaking Timing =============================================
% ======================================================================
\setlength{\unitlength}{1mm}
\begin{picture}(116,48)(3,-3)

\put(40,0){\makebox(0,0)[t]{\verb"t1"}}
\put(60,0){\makebox(0,0)[t]{\verb"t2"}}
\put(80,0){\makebox(0,0)[t]{\verb"t3"}}
\put(100,0){\makebox(0,0)[t]{\verb"t4"}}
\put(34,3){\line(0,1){37}}
\put(40,3){\line(0,1){37}}
\put(54,3){\line(0,1){37}}
\put(60,3){\line(0,1){37}}
\put(74,3){\line(0,1){37}}
\put(80,3){\line(0,1){37}}
\put(94,3){\line(0,1){37}}
\put(100,3){\line(0,1){37}}

\put(18,14){\makebox(0,0)[r]{\verb"dack"}}
\put(20,10){\thicklines {\line(1,0){36}}}
\put(56,10){\thicklines {\line(1,4){2}}}
\put(58,18){\thicklines {\line(1,0){38}}}
\put(96,18){\thicklines {\line(1,-4){2}}}
\put(98,10){\thicklines {\line(1,0){22}}}

\put(18,29){\makebox(0,0)[r]{\verb"dreq"}}
\put(20,25){\thicklines {\line(1,0){16}}}
\put(36,25){\thicklines {\line(1,4){2}}}
\put(38,33){\thicklines {\line(1,0){38}}}
\put(76,33){\thicklines {\line(1,-4){2}}}
\put(78,25){\thicklines {\line(1,0){42}}}

\end{picture}


\caption{Four-Phase Handshaking Timing Diagram.}
\label{fig-style}
\end{center}
\end{figure}

The handshaking protocol is formally specified by a number of
constraints imposed on both the sender and receiver.
For now, these can be understood as constraints imposed on the
sender and receiver viewed together as a single system;
later on, we say which constraints are imposed on which process.

A handshaking interaction is initiated whenever
\verb"dreq" becomes true.
\verb"dreq" must continue to be true until this is acknowledged.
This constraint is expressed by:

\begin{quote}
\verb"(dreq "$\longrightarrow$\verb" (dreq "$\cup$\verb" dack))"
\end{quote}

A request must be eventually be acknowledged. Furthermore,
once \verb"dack" becomes true, it must remain true until
\verb"dreq" becomes false.

\begin{quote}
\verb"(dreq "$\longrightarrow$\verb" ("$\Diamond$\verb" dack))"

\verb"(dack "$\longrightarrow$\verb" (dack "$\cup$\verb" ("{\TLNot}\verb"  dreq)))"
\end{quote}

Once a request has been acknowledged,
\verb"dreq" must eventually become false again to signal
the end of the interaction.
The next request cannot be signaled until \verb"dack" also returns
to false.

\begin{quote}
\verb"(dack "$\longrightarrow$\verb" ("$\Diamond$\verb" ("{\TLNot}\verb"  dreq)))"

\verb"(("{\TLNot}\verb"  dreq) "$\longrightarrow$\verb" (("{\TLNot}\verb"  dreq) "$\cup$\verb" ("{\TLNot}\verb"  dack)))"
\end{quote}

When \verb"dreq" becomes false, \verb"dack" must
also becomes false and stay false until the next request.

\begin{quote}
\verb"(("{\TLNot}\verb"  dreq) "$\longrightarrow$\verb" ("$\Diamond$\verb" ("{\TLNot}\verb"  dack)))"

\verb"(("{\TLNot}\verb"  dack)  "$\longrightarrow$\verb" (("{\TLNot}\verb"  dack) "$\cup$\verb" dreq))"
\end{quote}

Upon completion of the handshaking interaction,
the protocol requires another request to be initiated
sometime in the future.
This final constraint is usually included in the
handshaking protocol specification
to obtain the property that system activity never ceases.

\begin{quote}
\verb"(("{\TLNot}\verb"  dack) "$\longrightarrow$\verb" ("$\Diamond$\verb" dreq))"
\end{quote}

The use of temporal logic operators to specify the relative order
of events in a handshaking protocol
has been described by others including
Bochman \cite{Bochman},
Dill and Clarke \cite{Clarke},
and Fujita et al.\ \cite{Fujita}.

\subsection{Embedding temporal logic in higher-order logic}

Temporal logics are often treated as primitive systems
based on semantic definitions for temporal logic operators.
However, our higher-order logic framework allows us to regard
temporal logic operators
as simply abbreviations for higher-order functions.
The following definitions yield temporal logic operators
for a form of temporal logic known as {\it Linear Temporal Logic}.

\begintt
let H = new_definition (`H`,"\(\Box\) P = \(\lambda\)t. \(\forall\)n. P (t+n)");;

let E = new_definition (`E`,"\(\Diamond\) P = \(\lambda\)t. \(\exists\)n. P (t+n)");;

let N = new_definition (`N`,"\(\TLNext\) P = \(\lambda\)t. ((P (t+1)):bool)");;

let Until = new_infix_definition (
  `Until`,
  "$\(\cup\) P Q = \(\lambda\)t. \(\forall\)n. (\(\forall\)m. m < n ==> \(\neg\)(Q (t+m))) ==> P (t+n)");;

let Not = new_definition (`Not`,"\(\TLNot\) P = \(\lambda\)t. \(\neg\)(P t)");;

let Imp = new_infix_definition (`Imp`,"$\(\TLImp\) P Q = \(\lambda\)t. P t ==> Q t");;

let And = new_infix_definition (`And`,"$\(\TLAnd\) P Q = \(\lambda\)t. P t \(\wedge\) Q t");;
\endtt

Each operator is defined in terms of a function
which maps discrete points in time,
modelled by the natural numbers, to Boolean values.
These operators can
can be combined with variables such as \verb"P" and \verb"Q"
to form assertions in temporal logic.
In both first-order logic and higher-order logic,
every assertion, for instance,

\begin{quote}
$\forall$\verb"b. b "$\vee$\verb" "$\neg$\verb"b"
\end{quote}

\noindent
is a Boolean expression which is either true or false.
However, an assertion in temporal logic such as,

\begin{quote}
\verb" (P "$\TLImp$\verb" "$\Diamond$\verb" (Q "$\cup$\verb" R))"
\end{quote}

\noindent
is only true or false relative to a particular instant of time.
When stated as an assertion,
this is informally understood to mean
that the assertion is true at every instant of time.
However, to formally represent temporal logic assertions
as assertions in
our higher-order logic framework,
we introduce an notion of validity
where a temporal logic assertion is valid if and only if it
is true at all times.

\begintt
let VALID = new_definition (`VALID`,"VALID P = \(\forall\)t. P t");;
\endtt

Powerful inferences rules for direct manipulation of temporal
logic assertions can be derived in higher-order logic from
the above definitions.
Many such rules effectively `package up'
what would otherwise be tedious and repetitive patterns of
inference.
For instance, the following theorem provides a particularly
useful rule;
this rule achieves, in a single step, an inference which
would otherwise involve a proof by mathematical induction.

\begintt
|- \(\forall\)P Q. VALID((P \(\TLAnd\) (\(\TLNot\) Q)) \(\TLImp\) (\(\TLNext\) P)) ==> VALID(P \(\TLImp\) (P U Q))
\endtt

A recent survey article
\cite{Camurati} describes `traditional logic' (which includes
higher-order logic) and temporal logic as two alternative kinds
of pure formalisms for reasoning about hardware.
But when
temporal logic operators are
simply abbreviations for higher-order functions,
anything which can done with the temporal logic operators
can also be done without them using explicit time variables.
In fact, we have taken a mixed-mode approach of using both
temporal operators and explicit time variables.
The right mixture of temporal operators and explicit time variables
yields relatively succinct specifications and much easier proofs.

Previous work by Hale \cite{Hale:thesis} demonstrated that
the idea of embedding of a temporal logic
in higher-order logic was of practical use.
Leeser \cite{Leeser:thesis} has also embedded temporal logic in
a theorem proving environment to reason about hardware.
Both Hale and Leeser used another form of temporal logic called
{\it Interval Temporal Logic} developed by Moszkowski
\cite{Moszkowski:thesis,Moszkowski:journal}
to reason about digital systems in general.
However, the more ordinary form of temporal logic
captured in our definitions for $\Box$, $\Diamond$, $\TLNext$, etc.,
is adequate for the very specific purpose
of reasoning about
handshaking interactions between \Tamarack\ and
external memory.

\subsection{Sender and receiver specifications}

Earlier we gave a formal specification of the handshaking protocol
in terms of constraints
expressed by a set of temporal logic assertions.
Seitz \cite{Seitz} distinguishes between {\it functional constraints}
on outputs which must
be satisfied by a process and {\it domain constraints}
which are allowable assumptions about inputs.
In the handshaking protocol,
constraints imposed on \verb"dreq" are functional constraints
for the sender process and domain constraints for the
receiver process.
Constraints imposed on \verb"dack" are functional constraints
for the receiver process and domain constraints for the
sender process.

It turns out that only some of the domain constraints are
actually needed in each case.
Furthermore, 
these domain constraints are only needed to establish specific
functional constraints.
For example, the domain constraint,

\begin{quote}
\verb"(dreq "$\longrightarrow$\verb" (dreq "$\cup$\verb" dack))"
\end{quote}

\noindent
that \verb"dreq", once true,
must remain true can be assumed in showing that the receiver
satisfies the functional constraint,

\begin{quote}
\verb"(dreq "$\longrightarrow$\verb" ("$\Diamond$\verb" dack))"
\end{quote}

\noindent
that it will eventually detect and acknowledge the request.

The following definitions
give the functional constraints for
the sender and receiver respectively and, where required,
the domain constraints which can be assumed in showing that
a process satisfies a particular functional constraint.

\begintt
let Sender = new_definition (
  `Sender`,
  "Sender (dreq,dack) =
    VALID (dreq \(\TLImp\) (dreq \(\cup\) dack)) \(\wedge\)
    VALID ((\(\TLNot\) dreq) \(\TLImp\) ((\(\TLNot\) dreq) \(\cup\) (\(\TLNot\) dack))) \(\wedge\)
    (VALID (dack \(\TLImp\) (dack \(\cup\) (\(\TLNot\) dreq))) ==>
     VALID (dack \(\TLImp\) (\(\Diamond\) (\(\TLNot\) dreq)))) \(\wedge\)
    (VALID ((\(\TLNot\) dack) \(\TLImp\) ((\(\TLNot\) dack) \(\cup\) dreq)) ==>
     VALID ((\(\TLNot\) dack) \(\TLImp\) (\(\Diamond\) dreq)))");;

let Receiver = new_definition (
  `Receiver`,
  "Receiver (dreq,dack) =
    VALID (dack \(\TLImp\) (dack \(\cup\) (\(\TLNot\) dreq))) \(\wedge\)
    VALID ((\(\TLNot\) dack)  \(\TLImp\) ((\(\TLNot\) dack) \(\cup\) dreq)) \(\wedge\)
    (VALID (dreq \(\TLImp\) (dreq \(\cup\) dack)) ==>
     VALID (dreq \(\TLImp\) (\(\Diamond\) dack))) \(\wedge\)
    (VALID ((\(\TLNot\) dreq) \(\TLImp\) ((\(\TLNot\) dreq) \(\cup\) (\(\TLNot\) dack))) ==>
     VALID ((\(\TLNot\) dreq) \(\TLImp\) (\(\Diamond\) (\(\TLNot\) dack))))");;
\endtt

When the sender and receiver parts of the handshaking protocol
specification are both satisfied, this results in the set of
constraint mentioned at the beginning of this section
for the system as a whole.
This is shown by the following theorem.

\begintt
|- Sender(dreq,dack) \(\wedge\)
   Receiver(dreq,dack)
   ==>
   VALID (dreq \(\TLImp\) (dreq \(\cup\) dack)) \(\wedge\)
   VALID (dreq \(\TLImp\) (\(\Diamond\) dack)) \(\wedge\)
   VALID (dack \(\TLImp\) (dack \(\cup\) (\(\TLNot\) dreq))) \(\wedge\)
   VALID (dack \(\TLImp\) (\(\Diamond\) (\(\TLNot\) dreq))) \(\wedge\)
   VALID ((\(\TLNot\) dreq) \(\TLImp\) ((\(\TLNot\) dreq) \(\cup\) (\(\TLNot\) dack))) \(\wedge\)
   VALID ((\(\TLNot\) dreq) \(\TLImp\) (\(\Diamond\) (\(\TLNot\) dack))) \(\wedge\)
   VALID ((\(\TLNot\) dack) \(\TLImp\) ((\(\TLNot\) dack) \(\cup\) dreq)) \(\wedge\)
   VALID ((\(\TLNot\) dack) \(\TLImp\) (\(\Diamond\) dreq))
\endtt

\subsection{Memory specification}

The predicate \verb"Receiver" is used to partially specify a
behavioural model of fully asynchronous memory.
The rest of the memory specification
(given in Appendix~\ref{apx-asynmem})
specifies constraints on the address and data signals which
are synchronized by the handshaking signals.
These two parts of the specification are combined together
in the definition of \verb"AsynMemory"
(also given in Appendix~\ref{apx-asynmem}).

The memory specification is a formal rendering of the
description given earlier in Section~\ref{sec-memory}.
In particular,
it is precise statement about details such as the constraint
that:

\begin{quotation}
After signaling a memory request,
\verb"dreq" must remain equal to \verb"T"
and the \verb"wmem", address bus \verb"addr" and dataout bus \verb"dataout"
must remain at stable values until \verb"dack" becomes equal to \verb"T"
signaling that the request has been satisfied.
\end{quotation}

As with the specification of fully synchronous memory, \verb"SynMemory",
the operations performed by memory are selected from the representation
variable by \verb"fetch" and \verb"store".
For example, when \verb"dack" becomes equal to \verb"T"
at time \verb"t+n"
in response to a read request initiated at time \verb"t",
the value of the \verb"datain" bus is described by the following equation:

\begin{quote}
\verb"datain (t+n) = (fetch rep) (mem t,(address rep) (mar t))"
\end{quote}

One of the main differences from the FM8501
specification \cite{Hunt:thesis} is that our relational specification style
allows the memory model to have an entirely independent specification.
Thus, it is possible to reason independently about
the internal operation of
\Tamarack\ at the phase level without need of a memory model.
It is only necessary to compose specifications for \Tamarack\
and external memory
in order to reason about their interaction at the microprogramming level.
This contrasts with the functional specification style used for FM8501
where ``the characterization of external devices are wrapped up
in the same function which specifies the microprocessor" \cite{Hunt:thesis}.

\subsection{Verification}

Verifying the operation of \Tamarack\ in fully asynchronous mode
is much more difficult than fully synchronous mode because of
the unknown length of microcode repeat-loops during
an instruction cycle.
The overall strategy is to first show that every handshaking sequence
runs to completion,
then collapse repeat-loops to single steps in an abstract view
of behaviour.
Symbolic execution is then used to reason about the fixed sequences
of operations in this abstract view of behaviour.
Finally, we use the function \verb"TimeOf" described earlier
in Section~\ref{sec-temp} to specify a synchronization scheme
for asynchronous mode.

\subsubsection{Phase level}

Earlier correctness results described in Section~\ref{sec-verfphase}
(and given in Appendix~\ref{apx-phase})
for the phase level operation of \Tamarack\
are re-used in this version of the correctness proof without
change or addition.
We recall that phase level correctness results do not depend
on any particular model of external memory.
The interaction of the microprocessor with external memory
only concerns the operation of its internal architecture at
the microprogramming level.

\subsubsection{Implementation of the sender specification}

One of the first steps in verifying the operation of \Tamarack\
in fully asynchronous mode is to show that the microprogramming
level correctly implements the sender part of the handshaking specification.
The derivation of the following theorem is tedious but
very routine.
This proof step does not involve a behavioural model for external memory.

\begintt
|- Val3_CASES_ASM rep \(\wedge\)
   Val4Word4_ASM rep \(\wedge\)
   TamarackImp rep (
     datain,dack,gnd,idreq,mpc,mar,pc,
     acc,ir,rtn,arg,buf,idack,dataout,wmem,dreq,addr) \(\wedge\)
   GND gnd \(\wedge\)
   (((val4 rep) o mpc) 0 = 0)
    ==>
   Sender (dreq,dack)
\endtt

The proof procedure used to establish this result
is illustrated in the following \HOL\ source file
by proving that a very simple state machine also satisfies the
sender part of the handshaking protocol specification.

\begin{quote}
\verb"hol/Training/studies/microprocessor/proof/example4.ml"
\end{quote}

Once it has been established that the microprogramming level
satisfies the constraints expressed by \verb"Sender",
this result can be combined with the specification of external
memory, in particular, the constraints expressed by \verb"Receiver"
to obtain some very useful facts for reasoning about
handshaking interactions.
Among other things, one of the useful facts implied by
the combined constraints of
\verb"Sender" and \verb"Receiver" is that every handshaking sequence
runs to completion, that is,
every repeat-loop in the microcode eventually terminates.

\subsubsection{Collapsing repeat-loops to single steps}

The microprocessor specification and behavioural model of external
memory are combined together in the definition of \verb"AsynSystem"
to reason about interactions of the \Tamarack\ microchip
with fully asynchronous memory.
As mentioned earlier,
the memory specification \verb"AsynSystem" includes
the constraints
expressed by \verb"Receiver".

\begintt
let AsynSystem = new_definition (
  `AsynSystem`,
  "AsynSystem (rep:^rep_ty)
    (ireq,mpc,mar,pc,acc,ir,rtn,arg,buf,iack,dack,mem) =
    \(\exists\)datain gnd dataout wmem dreq addr.
      TamarackImp rep (
        datain,dack,gnd,ireq,mpc,mar,pc,
        acc,ir,rtn,arg,buf,iack,dataout,wmem,dreq,addr) \(\wedge\)
      AsynMemory rep (dreq,dack,mem,wmem,addr,dataout,datain) \(\wedge\)
      GND gnd");;
\endtt

The fact that every repeat-loop in the microcode terminates
makes it possible to derive theorems
which, in effect, collapse repeat-loops to single steps
in an abstract view of behaviour at the microprogramming level.

For instance, the following theorem
provides a collapsed view of the repeat-loop in state~6
of the FSM flow graph.
This theorem states that the repeat-loop will terminate
after zero or more clock cycles
with no changes to the internal state of the microprocessor.

\begintt
|- Val3_CASES_ASM rep \(\wedge\)
   Val4Word4_ASM rep \(\wedge\)
   AsynSystem rep (idreq,mpc,mar,pc,acc,ir,rtn,arg,buf,idack,dack,mem) \(\wedge\)
   (((val4 rep) o mpc) 0 = 0)
   ==>
   \(\forall\)t.
     (((val4 rep) o mpc) t = 6)
     ==>
     \(\exists\)n.
       (dack (t+n) = F) \(\wedge\)
       (\(\forall\)m. m < n ==> (dack (t+m) = T)) \(\wedge\)
       (((val4 rep) o mpc) (t+n) = 6) \(\wedge\)
       (mem (t+n) = mem t) \(\wedge\)
       (mar (t+n) = mar t) \(\wedge\)
       (pc (t+n) = pc t) \(\wedge\)
       (acc (t+n) = acc t) \(\wedge\)
       (rtn (t+n) = rtn t) \(\wedge\)
       (idack (t+n) = idack t)
\endtt

Another example is the theorem shown below which provides a collapsed
view of the repeat-loop in state~13.
It is convenient to state this result as a transition which begins
in state~6 and terminates in state~15.
During this transition, a value is fetched from memory \verb"mem",
added to the current contents of the accumulator \verb"acc", and the result
temporarily placed in the ALU buffer \verb"buf".

\begintt
|- Val3_CASES_ASM rep \(\wedge\)
   Val4Word4_ASM rep \(\wedge\)
   AsynSystem rep (idreq,mpc,mar,pc,acc,ir,rtn,arg,buf,idack,dack,mem) \(\wedge\)
   (((val4 rep) o mpc) 0 = 0)
   ==>
   \(\forall\)t.
     (((val4 rep) o mpc) t = 6) \(\wedge\)
     (dack t = F)
     ==>
     \(\exists\)n.
       (\(\forall\)m. m < n ==> \(\neg\)(((val4 rep) o mpc) (t+m) = 0)) \(\wedge\)
       (((val4 rep) o mpc) (t+n) = 15) \(\wedge\)
       (mem (t+n) = mem t) \(\wedge\)
       (pc (t+n) = pc t) \(\wedge\)
       (buf (t+n) =
         add rep (acc t,fetch rep (mem t,(address rep (mar t))))) \(\wedge\)
       (rtn (t+n) = rtn t) \(\wedge\)
       (idack (t+n) = idack t)
\endtt

The above pair of theorems describe two successive steps in the
symbolic execution of the microprogramming level.
In the course of reasoning about the microinstruction sequence
for an ADD instruction,
the first theorem advances the state of the symbolic execution
to end of the repeat-loop in state~6 and the second theorem
advances the symbolic execution from this point to state~15.

\subsubsection{Symbolic execution}

A complete set of theorems for collapsing
repeat-loops into single steps represent
an abstract view of behaviour where programming level
operations are implemented by fixed sequences of actions.
Hence, the symbolic execution proof technique described earlier
in Section~\ref{sec-verf}
for fully synchronous mode can also be used here
to show that programming level operations are correctly implemented
by microinstruction sequences which involve handshaking interactions
with external memory.

Correctness results for each programming level operation
are combined by case analysis to obtain the following theorem.

\begintt
|- Val3_CASES_ASM rep \(\wedge\)
   Val4Word4_ASM rep \(\wedge\)
   AsynSystem rep (idreq,mpc,mar,pc,acc,ir,rtn,arg,buf,idack,dack,mem) \(\wedge\)
   (((val4 rep) o mpc) 0 = 0)
   ==>
   \(\forall\)t.
     (((val4 rep) o mpc) t = 0) \(\wedge\)
     (dack t = F)
     ==>
     \(\exists\)n.
       Next ((((val4 rep) o mpc) Eq 0) \(\TLAnd\) (\(\TLNot\) dack)) (t,t+n) \(\wedge\)
       ((mem (t+n),pc (t+n),acc (t+n),rtn (t+n),idack (t+n)) =
        NextState rep (idreq t,mem t,pc t,acc t,rtn t,idack t))
\endtt

The above theorem is almost identical to the top-level correctness theorem
for fully synchronous mode
given in Section~\ref{sec-temp} except for the term used to denote the
next-instruction signal.  Instead of,

\begin{quote}
\verb"(((val4 rep) o mpc) Eq 0)"
\end{quote}

\noindent
the next-instruction signal is denoted by the term,

\begin{quote}
\verb"(((val4 rep) o mpc) Eq 0) "\TLAnd\verb" ("\TLNot\verb" dack)"
\end{quote}

\noindent
which states that the next instruction cycle does
not begin until the control unit FSM is about to leave state 0.
The use of this condition for the next-instruction signal
is not necessarily an obvious choice.
We leave it an exercise for the reader to
determine why other conditions,
e.g., when the FSM first enters state~0,
are not suitable.

\subsubsection{Top level correctness statement}

This version of the proof is completed by repeating the proof
procedure outlined earlier in Section~\ref{sec-temp}
for using the function \verb"TimeOf" to construct a mapping
from abstract time to concrete time.
The top-level correctness statement for this version of the
proof is shown below.

\begintt
|- Val3_CASES_ASM rep \(\wedge\)
   Val4Word4_ASM rep \(\wedge\)
   AsynSystem rep (idreq,mpc,mar,pc,acc,ir,rtn,arg,buf,idack,dack,mem) \(\wedge\)
   (((val4 rep) o mpc) 0 = 0)
   ==>
   let f = TimeOf ((((val4 rep) o mpc) Eq 0) \(\TLAnd\) (\(\TLNot\) dack)) in
   TamarackBeh rep (idreq o f,mem o f,pc o f,acc o f,rtn o f,idack o f)
\endtt
