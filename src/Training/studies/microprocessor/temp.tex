\section{Synchronizing multiple levels of timing}
\label{sec-temp}

The simple approach just described in Section~\ref{sec-verfconcl}
for relating
different granularities of time is not very general because
it depends on the implementation of programming level operation
by fixed microinstruction sequences whose pre-determined lengths
are specified by the definition of \verb"MicroCycles".
In general, programming level operations may not
necessarily be implemented at lower levels by fixed sequences.

An example of when
it is not possible to define a simple function like \verb"MicroCycles"
to compute the exact length of each instruction cycle
is the operation of \Tamarack\ in fully asynchronous mode
when delayed handshaking signals
may cause microinstructions to be repeated.
Although there are ways
to partially circumvent this problem
using an oracle function
to `guess' the length of each instruction cycle,
the result may not be completely
general\footnote{
If all functions must be total,
then use of an oracle function to guess the length of each
instruction cycle implies that every instruction cycle
is finite, in particular,
that handshaking sequences always run to completion.
A completely general approach would
not imply that every handshaking sequence terminates,
but instead, require (as we do) that this fact be established by
formal proof.}
or correspond very well to an intuitive view of this
timing relationship.

The term {\it temporal abstraction}
(or {\it temporal projection})
has been used
in verification circles
\cite{Eveking,Dhingra:thesis,Hale:thesis,Herbert:thesis,Melham:calgary86,Moszkowski:thesis,Subra:calgary86}
to denote the idea of relating
different rates of computation in a formal specification.
This idea naturally arises from the more general concept
of viewing computer hardware as a hierarchy of interpretation levels
where a timing relationship between two levels of interpretation
is sometimes called a {\it synchronization scheme}
(or {\it synchronization mechanism}).
The control part of
each level in the interpretation hierarchy can be
viewed as a virtual machine with the generalized form shown
in Figure~\ref{fig-next}.
To paraphrase the description by Anceau in~\cite{Anceau}:

\begin{quotation}
A given level is only allowed to generate a new command
once it has received an end-of-execution signal for the present
command from the levels below.
In turn, it sends a similar signal to the control part
immediately above requesting the next instruction.
The end-of-instruction signal received by the lowest level
of synchronous logic is simply the clock signal of the machine.
The next-instruction signal issued by the microprogramming
level implies the fetching of the next instruction from memory.
\end{quotation}

\begin{figure}
\begin{center}
% ===== Next Instruction Signal ========================================
% ======================================================================
\setlength{\unitlength}{1mm}
\begin{picture}(70,77)(0,0)

  {\thicklines {\put(0,20){\framebox(70,30){
\begin{minipage}[c]{25mm}
\begin{center}
interpreter\\
control part
\end{center}
\end{minipage}
}}}}

  \put(10,9){\vector(0,1){11}}
  \put(35,20){\vector(0,-1){11}}
  \put(60,9){\vector(0,1){11}}

  \put(10,8){\makebox(0,0)[t]{
    \begin{minipage}[c]{25mm}
    \begin{center}
    end of\\
    execution
    \end{center}
    \end{minipage}
    }}

  \put(35,8){\makebox(0,0)[t]{commands}}
  \put(60,8){\makebox(0,0)[t]{status}}

  \put(10,50){\vector(0,1){11}}
  \put(35,61){\vector(0,-1){11}}
  \put(60,50){\vector(0,1){11}}

  \put(10,62){\makebox(0,0)[b]{
    \begin{minipage}[c]{25mm}
    \begin{center}
    next
    instruction
    request
    \end{center}
    \end{minipage}
    }}

  \put(35,62){\makebox(0,0)[b]{instructions}}
  \put(60,62){\makebox(0,0)[b]{status}}

\end{picture}


\caption{Control Part of an Interpreter.}
\label{fig-next}
\end{center}
\end{figure}

In this section,
we describe a formalization which corresponds directly
to the imagery of
Figure~\ref{fig-next}
where a next-instruction signal at the microprogramming
level implies the fetching of the next programming level instruction
from memory.
At this level,
the next-instruction signal is not a physical signal,
but rather, a time-dependent condition on the internal state of
the machine.

In fully synchronous mode,
the end of the instruction cycle occurs when the
value of the microprocessor program counter returns to zero.
Hence, the next-instruction signal generated by the microprogramming
level is the time-dependent condition expressed by the term,

\begin{quote}
\verb"((val4 rep) o mpc) Eq 0"
\end{quote}

\noindent
where \verb"Eq" is an infix temporal operator for relating
a signal to a particular value in terms of equality.

\begintt
let Eq = new_infix_definition (`Eq`,"$Eq (P:time->*) c = \(\lambda\)t. P t = c");;
\endtt

If an instruction cycle begins at time \verb"t",
then it terminates at time \verb"t+m" if and only if
\verb"t+m" is the first time that the microcode program
counter \verb"mpc" becomes equal to zero since time \verb"t".
This condition is expressed by the term,

\begin{quote}
\verb"Next (((val4 rep) o mpc) Eq 0) (t,t+m)"
\end{quote}

\noindent
where the predicate \verb"Next" is defined as:

\begintt
let Next = new_definition (
  `Next`,
  "Next g (t1,t2) = t1 < t2 \(\wedge\) (\(\forall\)t. t1 < t \(\wedge\) t < t2 ==> \(\neg\)(g t)) \(\wedge\) (g t2)");;
\endtt

It is possible to show that this condition holds for
each of the fixed microinstruction sequences
that implement programming level operations in fully synchronous mode.
In place of the predicate \verb"SynCorrectness1",
a slightly different form of parameterized correctness condition
is used to verify
the implementation of each programming level operation.

\begintt
let SynCorrectness2 = new_definition (
  `SynCorrectness2`,
  "SynCorrectness2 (rep:^rep_ty) (ircond,opcval,iszerocond) =
    \(\forall\)ireq mpc mar pc acc ir rtn arg buf iack mem.
      Val3_CASES_ASM (rep:^rep_ty) \(\wedge\)
      Val4Word4_ASM rep \(\wedge\)
      SynSystem rep (ireq,mpc,mar,pc,acc,ir,rtn,arg,buf,iack,mem)
      ==>
      \(\forall\)t:time.
        (((val4 rep) o mpc) t = 0) \(\wedge\)
        ((ireq t \(\wedge\) \(\neg\)(iack t)) = ircond) \(\wedge\)
        (OpcVal rep (mem t,pc t) = opcval) \(\wedge\)
        ((iszero rep) (acc t) = iszerocond)
        ==>
        \(\exists\)m.
          Next (((val4 rep) o mpc) Eq 0) (t,t+m) \(\wedge\)
          ((mem (t+m),pc (t+m),acc (t+m),rtn (t+m),iack (t+m)) =
           NextState rep (ireq t,mem t,pc t,acc t,rtn t,iack t))");;
\endtt

In this version of the correctness condition,
the length of the instruction cycle
is an existentially quantified
variable instead of a value computed by \verb"MicroCycles".
The proofs of correctness for each programming level operation
are almost the same as before
except that we must keep track of the
microcode program counter \verb"mpc" throughout the instruction cycle
and show that it
is never equal to zero before the end of the instruction cycle.
Combining individual correctness results for each programming level
operation and eliminating, as before, the overhead of a parameterized
correctness condition yields the following theorem.

\begintt
|- Val3_CASES_ASM (rep:^rep_ty) \(\wedge\)
   Val4Word4_ASM rep \(\wedge\)
   SynSystem rep (ireq,mpc,mar,pc,acc,ir,rtn,arg,buf,iack,mem)
   ==>
   \(\forall\)t.
     (((val4 rep) o mpc) t = 0)
     ==>
     \(\exists\)m.
       Next (((val4 rep) o mpc) Eq 0) (t,t+m) \(\wedge\)
       ((mem (t+m),pc (t+m),acc (t+m),rtn (t+m),iack (t+m)) =
        NextState rep (ireq t,mem t,pc t,acc t,rtn t,iack t))
\endtt

\begin{figure}
\begin{center}
% ===== Temporal Abstraction ===========================================
% ======================================================================
\setlength{\unitlength}{1mm}
\begin{picture}(125,21)(0,0)
  \put(25,19){\makebox(0,0)[r]{abstract time:}}
  \put(30,19){\line(1,0){100}}
  \put(55,19){\makebox(0,0)[c]{$\bullet$}}
  \put(55,17){\thicklines {\vector(0,-1){10}}}
  \put(95,19){\makebox(0,0)[c]{$\bullet$}}
  \put(95,17){\thicklines {\vector(0,-1){10}}}
  \put(115,19){\makebox(0,0)[c]{$\bullet$}}
  \put(115,17){\thicklines {\vector(0,-1){10}}}
  \put(25,12){\makebox(0,0)[r]{\verb"f":}}
  \put(25,5){\makebox(0,0)[r]{concrete time:}}
  \put(30,5){\line(1,0){100}}
  \multiput(35,5)(10,0){10}{\makebox(0,0)[c]{$\bullet$}}
  \put(25,0){\makebox(0,0)[r]{\verb"g":}}
  \put(35,0){\makebox(0,0)[c]{\verb"F"}}
  \put(45,0){\makebox(0,0)[c]{\verb"F"}}
  \put(55,0){\makebox(0,0)[c]{\verb"T"}}
  \put(65,0){\makebox(0,0)[c]{\verb"F"}}
  \put(75,0){\makebox(0,0)[c]{\verb"F"}}
  \put(85,0){\makebox(0,0)[c]{\verb"F"}}
  \put(95,0){\makebox(0,0)[c]{\verb"T"}}
  \put(105,0){\makebox(0,0)[c]{\verb"F"}}
  \put(115,0){\makebox(0,0)[c]{\verb"T"}}
  \put(125,0){\makebox(0,0)[c]{\verb"F"}}
\end{picture}


\caption{Abstract and Concrete Time Scales.}
\label{fig-temp}
\end{center}
\end{figure}

The next step is to define a general purpose function \verb"TimeOf"
which uses an next-instruction signal \verb"g" to construct
a mapping \verb"f" from an abstract time scale to a concrete
time scale.
The relationship between abstract and concrete time is shown
in Figure~\ref{fig-temp}.
The following definition of \verb"TimeOf" is
similar to definitions given in previous work by
Dhingra \cite{Dhingra:thesis},
Herbert \cite{Herbert:thesis}
and Melham \cite{Melham:calgary86}
(the name `\verb"TimeOf"'
is used here for continuity with previous work).
It is defined
in terms of the predicates \verb"First"
and \verb"Next" using primitive recursion and Hilbert's
$\varepsilon$-operator.
\footnote{
Our definition of \verb"TimeOf"
is simplified by avoiding the existential quantification found
in corresponding definitions given by
\cite{Dhingra:thesis,Herbert:thesis,Melham:calgary86}.
When the constraint that \verb"g" is `true infinitely often' holds,
this function yields exactly the same mapping from abstract time
to concrete time.  When this constraint does not hold,
both versions (with and without existential quantification)
yield an incomplete mapping.}

\begintt
let First = new_definition (
  `First`,
  "First g t = (\(\forall\)p. p < t ==> \(\neg\)(g p)) \(\wedge\) (g t)");;

let TimeOf = new_prim_rec_definition (
  `TimeOf`,
  "(TimeOf g 0 = \(\varepsilon\)t. First g t) \(\wedge\)
   (TimeOf g (SUC u) = \(\varepsilon\)t. Next g (TimeOf g u,t))");;
\endtt

The first point on the abstract time scale corresponds to the first time
that the next-instruction signal \verb"g"
is true with respect to the concrete time scale.
Subsequent points on the abstract time scale are defined recursively.
The next point after \verb"u" on the abstract time scale,
i.e., \verb"u+1",
corresponds to the next time that the next-instruction signal \verb"g"
becomes true with respect to the concrete time scale.

Applying \verb"TimeOf" to the
next-instruction signal generated at microprogramming level of \Tamarack,

\begin{quote}
\verb"TimeOf (((val4 rep) o mpc) Eq 0)"
\end{quote}

\noindent
yields a mapping from
points on the abstract time scale to points on the
concrete time scale.
This mapping is the synchronization scheme which
relates behaviour at the abstract programming level time scale
to behaviour at the concrete microprogramming level time scale.

As with most definitions that use Hilbert's
$\varepsilon$-operator,
some additional proof infra-structure is needed to
reason about specifications involving \verb"TimeOf".
For instance,
it is necessary to establish the fact that the
next-instruction signal is true
infinitely often \cite{Joyce:tech100,Melham:calgary86}.
Fortunately,
the main result of
this proof infrastructure can be summed up in a compact theorem:

\begintt
|- \(\forall\)g r.
     (\(\exists\)t. g t) \(\wedge\) (\(\forall\)t. g t ==> \(\exists\)m. Next g (t,t+m) \(\wedge\) r (t,t+m))
     ==>
     \(\forall\)u. r (TimeOf g u,TimeOf g (u+1))
\endtt

For any next-instruction signal \verb"g" and next state relation \verb"r",
this theorem can be used to reduce the problem of establishing,

\begin{quote}
$\forall$\verb"u. r (TimeOf g u,TimeOf g (u+1))"
\end{quote}

\noindent
to a pair of simpler problems:

\begin{quote}
$\exists$\verb"t. g t"

$\forall$\verb"t. g t ==> "$\exists$\verb"m. Next g (t,t+m) "$\wedge$\verb" r (t,t+m)"
\end{quote}

The last step in verifying \Tamarack\
is essentially a question about the synchronization scheme
which relates behaviour at the programming level time scale
to behaviour at the microprogramming level time scale.
The above theorem provides a way to
translate this question
into a pair of simpler problems.
\footnote{
This is analogous to how
axioms and rules of Hoare logic
can be used to translate questions about program correctness
into verification conditions \cite{Gordon:banff87}.}
In particular, it is used mid-way through a
backward proof of the top-level correctness statement
to reduce the goal,

\begintt
OK..
"\(\forall\)u.
  mem(TimeOf g(u + 1)),pc(TimeOf g(u + 1)),acc(TimeOf g(u + 1)),
  rtn(TimeOf g(u + 1)),iack(TimeOf g(u + 1)) =
  NextState 
  rep
  (ireq(TimeOf g u),mem(TimeOf g u),pc(TimeOf g u),acc(TimeOf g u),
   rtn(TimeOf g u),iack(TimeOf g u))"
    [ "((val4 rep) o mpc) Eq 0 = g" ]
    [ "Val3_CASES_ASM rep" ]
    [ "Val4Word4_ASM rep" ]
    [ "SynSystem rep(ireq,mpc,mar,pc,acc,ir,rtn,arg,buf,iack,mem)" ]
    [ "\(\exists\)t. ((val4 rep) o mpc)t = 0" ]

() : void
\endtt

\noindent
to the following pair of sub-goals.
\footnote{
A generalized form of this backward proof step could be implemented
as a \HOL\ tactic in the same way that axioms and rules of Hoare Logic
can be `inverted' into a tactic \cite{Gordon:banff87}.}

\begintt
"\(\exists\)t. g t"
    [ "((val4 rep) o mpc) Eq 0 = g" ]
    [ "Val3_CASES_ASM rep" ]
    [ "Val4Word4_ASM rep" ]
    [ "SynSystem rep(ireq,mpc,mar,pc,acc,ir,rtn,arg,buf,iack,mem)" ]
    [ "\(\exists\)t. ((val4 rep) o mpc)t = 0" ]

() : void
\endtt

\begintt
"\(\forall\)t.
  g t ==>
  (\(\exists\)m.
    Next g(t,t+m) \(\wedge\)
    (mem(t + m),pc(t + m),acc(t + m),rtn(t + m),iack(t + m) =
     NextState rep(ireq t,mem t,pc t,acc t,rtn t,iack t)))"
    [ "((val4 rep) o mpc) Eq 0 = g" ]
    [ "Val3_CASES_ASM rep" ]
    [ "Val4Word4_ASM rep" ]
    [ "SynSystem rep(ireq,mpc,mar,pc,acc,ir,rtn,arg,buf,iack,mem)" ]
    [ "\(\exists\)t. ((val4 rep) o mpc)t = 0" ]

() : void
\endtt

The first sub-goal is satisfied by assumptions in the assumption list
combined with the definition of \verb"Eq".
The second sub-goal is satisfied by
previously mentioned correctness results for
the microprogramming level.
The satisfaction of these two sub-goals yields
the following correctness theorem for \Tamarack.

\begintt
|- Val3_CASES_ASM (rep:^rep_ty) \(\wedge\)
   Val4Word4_ASM rep \(\wedge\)
   SynSystem rep (ireq,mpc,mar,pc,acc,ir,rtn,arg,buf,iack,mem) \(\wedge\)
   (\(\exists\)t. ((val4 rep) o mpc) t = 0)
   ==>
   let f = TimeOf (((val4 rep) o mpc) Eq 0) in
   TamarackBeh rep (ireq o f,mem o f,pc o f,acc o f,rtn o f,iack o f)
\endtt

This is nearly
identical to the previous version of the top level correctness
statement in Section~\ref{sec-verfconcl}
except for the sub-term which specifies the relationship
between the two time scales.
