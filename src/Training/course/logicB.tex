% =====================================================================
% HOL Course Slides: overview of higher order logic   (c) T melham 1990
% =====================================================================

\documentstyle[12pt,layout]{article}

% ---------------------------------------------------------------------
% Preliminary settings.
% ---------------------------------------------------------------------

\renewcommand{\textfraction}{0.01}	  % 0.01 of the page must contain text
\setcounter{totalnumber}{10}	 	  % max of 10 figures per page
\flushbottom				  % text extends right to the bottom
\pagestyle{slides}			  % slides page style
\setlength{\unitlength}{1mm}		  % unit = 1 mm

% ---------------------------------------------------------------------
% load macros
% ---------------------------------------------------------------------
% =====================================================================
% Macros for HOL course slides 	                      (c) T Melham 1990
% =====================================================================

% ---------------------------------------------------------------------
% Macro for title page of a section:
%
%     \bsectitle
%       ...<title>...
%     \esectitle
%
% starts a new page, and typesets <title> in a large box at the top
% of the page.
% ---------------------------------------------------------------------

\newbox\stitle

\long\def\Frame#1{\leavevmode
    \hbox{\vbox{\hrule height2pt
              \hbox{\vrule width2pt #1\vrule width2pt}
  	      \hrule height2pt}}}

\long\def\bsectitle{\setbox\stitle=\hbox\bgroup\begin{minipage}{140mm}
		   \begin{center}\Huge\bf\vskip10mm}

\long\def\esectitle{\vskip10mm\end{center}\end{minipage}\egroup
		    \clearpage\vspace*{30mm}%
		   \begin{center}\Frame{\box\stitle}\end{center}}

% ---------------------------------------------------------------------
% Macro for title of slide:
%
%     \slide{<title>} 
%
% starts a new page, and typesets <title> in a large box at the top
% of the page.
% ---------------------------------------------------------------------

\long\def\slide#1{\newpage\leavevmode
 \hbox{\vbox{\hrule height2pt%
 \hbox to \textwidth{\vrule width2pt\hskip5mm%
 \vbox to 16mm{\vfill\vskip1mm\hbox{\Huge\bf #1}\vfill}\hfil\vrule width2pt}%
 \hrule height2pt}}\vskip7mm}

% ---------------------------------------------------------------------
% round circles: \blob and \subblob 
% ---------------------------------------------------------------------

\newcommand{\blob}{\begin{picture}(8.5,4)(0,-0.1)
		    \put(2.5,2){\circle*{3}} \end{picture}}

\newcommand{\subblob}{\begin{picture}(4.5,4)(0,0)
		    \put(1,1.5){\circle*{2}} \end{picture}}

% ---------------------------------------------------------------------
% Various lengths
%
% \blobwidth     = the width of a blob
% \subblobwidth  = the width of a sub-blob
% \pointwidth    = the text width of a major point
% \sind          = the standard indentation
% ---------------------------------------------------------------------

\newlength{\blobwidth}
\settowidth{\blobwidth}{\blob}

\newlength{\pointwidth}
\setlength{\pointwidth}{\textwidth}
\addtolength{\pointwidth}{-1\blobwidth}

\newlength{\subblobwidth}
\settowidth{\subblobwidth}{\subblob}

\newlength{\sind}
\setlength{\sind}{15mm}

% ---------------------------------------------------------------------
% \point : for making points
% ---------------------------------------------------------------------

\def\point#1{\vskip10mm\blob\parbox[t]{\pointwidth}{\LARGE\bf#1}\par}

% ---------------------------------------------------------------------
% \subpoint : for making sub-points
% ---------------------------------------------------------------------

\def\subpoint#1{{\addtolength{\pointwidth}{-\sind}
	      \addtolength{\pointwidth}{-1\subblobwidth}\vskip5mm
	      \hspace*{\sind}\subblob
	      \parbox[t]{\pointwidth}{\Large\bf#1}\par}}

% ---------------------------------------------------------------------
% \pindent : indent to the level of a point
% ---------------------------------------------------------------------

\def\bpindent{\bgroup\leftskip=\blobwidth}
\def\epindent{\par\egroup}

% ---------------------------------------------------------------------
% \pindent : indent to the level of a subpoint
% ---------------------------------------------------------------------

\def\bspindent{\bgroup\leftskip=\subblobwidth\addtolength{\leftskip}{\sind}}
\def\espindent{\par\egroup}

% ---------------------------------------------------------------------
% Macros for little HOL sessions displayed in boxes.
%
% Usage:     \begin{session}\begin{verbatim}
%	      .
%	       < lines from hol session >
%	      .
%	     \end{verbatim}\end{session}   
%
%            typesets the session in a box.
%
% Lengths: \hsbw = width of box, \lwid = width of lines
% ---------------------------------------------------------------------

\newlength{\lwid}
\setlength{\lwid}{0.8pt}

\newlength{\hsbw}
\setlength{\hsbw}{\textwidth}
\addtolength{\hsbw}{-\lwid}
\addtolength{\hsbw}{-\tabcolsep}
\addtolength{\hsbw}{-\blobwidth}
\addtolength{\hsbw}{-1mm}

\newenvironment{session}{\setlength{\arrayrulewidth}{\lwid}
 \begin{flushleft}\hskip\blobwidth\hskip1mm
 \begin{tabular}{@{}|c@{}|@{}}\hline 
 \begin{minipage}[b]{\hsbw}
 \vspace*{5.5pt}
 \begingroup\Large\baselineskip17.5pt}{\endgroup\end{minipage}\\[1.5pt]\hline 
 \end{tabular}
 \end{flushleft}}

\def\Rule#1#2{\mbox{${\displaystyle\raise 6pt\hbox{$\;\;\;#1\;\;\;$}} \over 
                     {\displaystyle\lower7pt\hbox{$\;\;\;#2\;\;\;$}}$}}


% ---------------------------------------------------------------------
% set caption at the foot of pages for this series of slides
% ---------------------------------------------------------------------
\ftext{Primitive basis of the logic}{3}

% ---------------------------------------------------------------------
% Slides
% ---------------------------------------------------------------------
\begin{document}

% ---------------------------------------------------------------------
% Title page for this series of slides
% ---------------------------------------------------------------------

\bsectitle
Overview of\\
Higher Order Logic\\
{\vrule width20mm height2.5mm depth-2mm}\\
Primitive Basis
\esectitle

% =====================================================================
\slide{The Theory {\rm bool}}

\point{The most basic HOL theory is {\rm bool}.}

\point{The theory {\rm bool} contains:}

\vskip 7mm
\bspindent{\Large{\bf Type constants:~~~$bool$ }}
\espindent
\vskip 5mm
\bspindent{\Large{\bf Type operators:~~~$\rightarrow$ }}
\espindent

\vskip 5mm
\bspindent{\Large{\bf Constants:}
\vskip 4mm
\begin{tabular}{@{\qquad} l}
 \( \supset \;\; = \;\; \epsilon \) \\
 \( T \;\; F \;\; \neg \) \\
 \$$\forall$~~~\$$\exists$~~~\$$\exists${\em !} \\
 \$$\vee$~~~\$$\wedge$
\end{tabular} }
\espindent
\vskip5mm
\bpindent\LARGE\bf
plus a few other constants (discussed later).
\epindent
\vskip7mm

\point{The constants $\supset$, $=$, and $\epsilon$ are primitive.}
\point{All the other constants are defined.}

% =====================================================================
\slide{Definitions in the Theory {\rm bool}}

\point{The constant definitions are:}

\bspindent{\LARGE
\vskip 6mm
\( \vdash T = ((\lambda x.\: x) \; = \; (\lambda x.\:x)) \) 
\vskip 6mm
\( \vdash\) \$$\forall$ \( \;=\; \lambda P.\:\; P = (\lambda x.\:T) \) 
\vskip 6mm
\( \vdash\) \$$\exists$ \( \;=\; \lambda P.\:\; P(\) \$$\epsilon$ \( P) \) 
\vskip 6mm
\( \vdash F \;=\; \forall b. \; b \) 
\vskip 6mm
\( \vdash \neg \;=\; \forall b. \; b \supset F \) 
\vskip 6mm
\( \vdash\) \$$\wedge$ \( \;=\; \lambda b_1\;b_2. \; \forall b. \;
             (b_1 \supset (b_2 \supset b)) \supset b \)
\vskip 6mm
\( \vdash\) \$$\vee$ \( \;=\; \lambda b_1\;b_2. \; \forall b. \;
             (b_1 \supset b) \supset ((b_2 \supset b) \supset b) \)
\vskip 6mm
\( \vdash\) \$$\exists${\em !} \( \;=\; \lambda P.\: \exists\;P \wedge
\forall x \; y.\: P\; x \wedge P\; y \supset x{=}y \)
}
\espindent

\vskip7mm
\point{All the usual logical properties follow from\\
these definitions.}


% =====================================================================
\slide{Axioms in the Theory {\rm bool}}

\point{The theory {\rm bool} has only 4 axioms:}

\vskip7mm

\bspindent\LARGE

$\begin{array}[t]{@{}l}
\vdash \forall \; b. \;\; (b = T) \; \vee \; (b = F) \\[3mm]

\vdash \forall \; b_1\;b_2. \;\;
         (b_1 \supset b_2) \supset (b_2 \supset b_1) \supset (b_1 = b_2)\\[3mm]

\vdash \forall \; f. \;\; (\lambda x. \; f x) = f \\[3mm]

\vdash \forall \; P \; x. \;\; P\; x \supset P(\epsilon\; P)

\end{array}$
\espindent

\vskip7mm

% =====================================================================
\slide{The Theory {\rm ind}}

\point{There is only one other primitive theory: {\rm ind}}
\vskip7mm
\point{The theory {\rm bool} is a parent of {\rm ind}.}
\vskip7mm
\point{The theory {\rm ind} contains:}

\vskip 7mm
\bspindent{\Large{\bf Type constants:~~~$ind$}}
\espindent
\vskip 5mm
\bspindent\Large{\bf Axioms:}
\vskip 4mm
\qquad$\exists f\begin{array}[t]{@{}l}{:}ind{\rightarrow}ind.\:\\
       (\forall x\;y.\:f x = f y \supset x = y)  \wedge
       (\neg \forall x.\:\exists y.\: x = f\; y) \end{array}$
\espindent

% =====================================================================
\slide{Primitive Inference Rules}

\point{Inference rules enable us to deduce new\\
theorems from existing ones.}

\point{Notation used to describe the rules:}

\vskip10mm

\bspindent\LARGE\bf
\Rule
{\Gamma_1 \vdash t_1,\; \dots,\; \Gamma_n \vdash t_n}
{\Gamma \vdash t}
\espindent

\vskip7mm
\point{
this means that
if $\Gamma_1 \vdash t_1$, \dots, $\Gamma_n \vdash t_n$
are all theorems 
then so is $\Gamma \vdash t$.}

% =====================================================================
\slide{Primitive Inference Rules}

\point{There are only 8 primitive inference rules:}
\vskip18mm

\bspindent\LARGE\bf
{\Large\bf ASSUME:\quad}
$\Rule{\mbox{-}}{\{ t \} \vdash t}$
\espindent

\vskip18mm

\bspindent\LARGE\bf
{\Large\bf REFL:\quad}
$\Rule{\mbox{-}}{\vdash t=t}$
\espindent

\vskip18mm

\bspindent\LARGE\bf
{\Large\bf MP:\quad} 
$\Rule{\Gamma_1 \vdash t_1 \supset t_2 \qquad \Gamma_2 \vdash t_1}
{\Gamma_1 \cup \Gamma_2 \vdash t_2}$
\espindent

\vskip18mm

\bspindent\LARGE\bf
{\Large\bf DISCH:\quad}
$\Rule{\Gamma \vdash t_2}
{ \Gamma - \{ t_1 \} \vdash t_1 \supset t_2 }$
\espindent

\vskip18mm

\bspindent\LARGE\bf
{\Large\bf ABS:\quad}
$\Rule{\Gamma \vdash t_1 = t_2}
{\Gamma \vdash (\lambda x. \; t_1) = (\lambda x.\; t_2) }$
\vskip4mm
{\Large\bf $x$ not free in $\Gamma$}
\espindent


% =====================================================================
\slide{Beta Conversion}

\point{The rule for beta-conversion:}
\vskip15mm
\bspindent\LARGE\bf
{\Large\bf BETA\_CONV:\quad}
$\Rule{\mbox{-}}
{ \vdash (\lambda x. \; t_1) t_2 = t_1 [ t_2 / x ] }$
\espindent


\vskip 15mm

\point{\( t_1 [ t_2 / x ]  \) denotes the result of substituting $t_2$ for all free occurrences of $x$ in $t_1$.}

\vskip 7mm
\point{Bound variables renamed if necessary, so that
no free variable in $t_2$ becomes bound.}

\vskip 7mm
\point{Examples:}
\vskip5mm
\bspindent\LARGE
$\begin{array}[t]{@{}l}
\vdash (\lambda x.\; x+3)\;7 \;=\; 7+3 \\[3mm]
\vdash (\lambda x.\; (\forall x.\: x=T) \supset x)\;F \;=\; 
(\forall x.\: x=T) \supset F) \\[3mm]
\vdash (\lambda y. \; \forall x.\: x=y)\;x \; = \; (\forall x'.\: x'=x)
\end{array}$
\espindent


% =====================================================================
\slide{Substitution}


\point{The rule for substitution:}
\vskip15mm
\bspindent\LARGE\bf
{\Large\bf SUBST:\quad}
$\Rule{\Gamma_1 \vdash t_1 = t_2 \qquad \Gamma_2 \vdash t [ t_1 ] }
{\Gamma_1 \cup \Gamma_2 \vdash t [ t_2 ]}$
\espindent

\vskip 7mm
\bpindent\LARGE\bf
where
\epindent
\vskip5mm
\bspindent\Large\bf
\( t [ t_1 ] \) is a term with 
selected free occurences of $t_1$\\ `singled out' for 
replacement
\vskip 5mm
\( t [ t_2 ] \) is the result of replacing those chosen $t_1$ by $t_2$
\vskip 5mm
bound variables are renamed so that variables free in $t_2$
do not become bound in $t [t_2]$
\espindent


% =====================================================================
\slide{Examples of Substitution}

\point{Examples:}
\vskip 15mm

\bspindent\LARGE\bf
$\Rule
{\Gamma \vdash x=17 \qquad \Gamma \vdash x<20 \supset x<21}
{\Gamma \vdash 17<20 \supset x<21}$
\vskip 20mm

$\Rule{\Gamma \vdash x=17 \qquad \Gamma \vdash x<20 \supset x<21}
{\Gamma \vdash x<20 \supset 17<21}$
\vskip20mm

$\Rule{\{x=y\} \vdash x=y \quad \Gamma \vdash x=0 \supset
\forall y.\; y+1 > x }
{\{x=y\},\Gamma \vdash x=0 \supset \forall y'.\; y'+1 > y}$
\espindent


% =====================================================================
\slide{Type Instantiation}

\point{The rule for type instantiation}
\vskip15mm
\bspindent\LARGE\bf
{\Large\bf INST\_TYPE:\quad}
$\Rule{\Gamma \vdash t}
{ \Gamma \vdash t 
[ \sigma_1 \ldots \sigma_n / \alpha_1 \ldots \alpha_n]}$
\espindent
\vskip15mm
\bpindent\LARGE\bf
which effects the parallel substitution of types \( \sigma_1 \ldots \sigma_n\)
for type variables \( \alpha_1 \ldots \alpha_n\) in $t$.
\epindent
\vskip7mm
\point{
Restriction: none of  \( \alpha_1 \ldots \alpha_n\) occur in $\Gamma$.}
\vskip7mm
\point{Example:}
\vskip 15mm
\bspindent\LARGE
$\Rule{\vdash I(x:\alpha)=x}
{ \vdash I(x:num)=x}$
\espindent

% =====================================================================
\slide{Higher Order Logic}

\point{The primitive basis of the logic:}

\subpoint{Two types: $bool$ and $ind$}
\subpoint{One type operator: $\rightarrow$}
\subpoint{Five axioms}
\subpoint{Eight inference rules}
\subpoint{Three rules of definition}

\vskip7mm

\point{From these, the following are defined:}
\subpoint{the type of pairs: $\alpha \times \beta$}
\subpoint{the type of natural numbers: $num$}
\subpoint{the type of lists: $(\alpha)list$}
\subpoint{the theory of primitive recursive definitions}
\subpoint{the theory of arithmetic}
\subpoint{\dots and much more}


\vskip7mm
\point{We shall see how new types are defined later.}


% =====================================================================
\slide{An Example Proof}

\point{Here is a proof of the theorem \( \;\; \vdash \neg F \)}

\vskip 10mm
\bspindent{\Large{\bf
\begin{tabular}{@{}l@{$\;\;\;\;$}l@{$\;\;\;\;$}l@{}}
1) & \( \vdash (\lambda b. \; b \supset F)F = (F \supset F) \) &
    {\Large\bf BETA\_CONV}\\[3mm]
2) & \( \vdash (\lambda b. \; b \supset F)F =
         (\lambda b. \; b \supset F)F \) &
{\Large\bf REFL} \\[3mm]
3) & \( \vdash (F \supset F) = (\lambda b. \; b \supset F)F \) &
SUBST 1,2 \\[3mm]
4) & \( \{F\} \vdash F \) & ASSUME \\[3mm]
5) & \( \vdash F \supset F \) & DISCH 4 \\[3mm]
6) & \( \vdash (\lambda b. \; b \supset F)F \) & SUBST 3,5\\[3mm]
7) & \( \vdash \neg = (\lambda b. \; b \supset F)  \) & DEFN of $\neg$\\[3mm]
8) & \( \vdash \neg = \neg  \) & REFL\\[3mm]
9) & \( \vdash (\lambda b. \; b \supset F) = \neg \) & SUBST 7,8\\[3mm]
10) & \( \vdash \neg F \) & SUBST 6,9 \\
\end{tabular}}}
\espindent

\vskip 7mm
\point{Such proofs are tedious; we need to be able to take larger steps.}


% =====================================================================
\slide{Derived Inference Rules}

\point{Notice that the sequences 1--3 and 7--9 both reverse
the order of an equality:}

\vskip 7mm
\bspindent{\Large{\bf
1) \( \vdash (\lambda b. \; b \supset F)F = (F \supset F) \)
\vskip 4mm
2) \( \vdash (\lambda b. \; b \supset F)F = (\lambda b. \; b \supset F)F \)
\vskip 4mm
3) \( \vdash (F \supset F) = (\lambda b. \; b \supset F)F \) 
\vskip 10mm
7) \( \vdash\neg = (\lambda b. \; b \supset F) \) 
\vskip 4mm 
8) \( \vdash \neg = \neg  \)
\vskip 4mm 
9) \( \vdash (\lambda b. \; b \supset F) = \neg \) 
}}
\espindent
\vskip 7mm

\point{The same sequence could be used to reverse the
direction of any equation.}
\vskip 7mm
\point{This suggests the derived inference rule:}
\vskip 7mm
\bspindent\LARGE\bf
{\Large\bf SYM:\quad}
\Rule{\Gamma \vdash t_1  =  t_2}
{\Gamma \vdash t_2  =  t_1}
\espindent


% =====================================================================
\slide{Derived Inference Rules}


\point{Every derived inference rule must have a proof 
of soundness in the form of a derivation.}

\point{Example:}
\vskip10mm
\bspindent\LARGE\bf
{\Large\bf SYM:\quad}
\Rule{\Gamma \vdash t_1 \; = \; t_2}
{\Gamma \vdash t_2 \; = \; t_1}
\vskip 10mm
\begin{tabular}{l l @{\qquad} l}
1) & \( \Gamma \vdash t_1 \; = \; t_2 \) & {\Large\bf HYPOTHESIS} \\
2) & \( \vdash t_1 \; = \; t_1  \)       & {\Large\bf REFL}\\
3) & \( \Gamma \vdash t_2 \; = \; t_1 \) & {\Large\bf SUBST 1,2}
\end{tabular}
\espindent
\vskip 7mm

\point{Note:}

\subpoint{$t_1$ and $t_2$ are arbitrary terms of the same type}

\subpoint{$\Gamma$ is an arbitrary set of assumptions}

\vskip 7mm
\point{Once derived, {\Large\bf SYM} can be used just like\\ a primitive
inference rule.}


% =====================================================================
\slide{Another Derived Inference Rule}

\point{Analysis of boolean equality:}

\vskip15mm
\bspindent\LARGE\bf
{\Large\bf EQ\_IMP\_RULE:\quad}
\Rule{\Gamma \vdash t_1 \; = \; t_2}
{\Gamma \vdash t_1 \supset t_2 \qquad 
   \Gamma \vdash t_2 \supset t_1}
\espindent

\point{Derivation:}

\vskip 7mm
\bspindent{\Large{\bf
\begin{tabular}{@{}l@{$\;\;\;\;$}l@{$\;\;\;\;$}l@{}}
1) & \( \Gamma \vdash t_1 = t_2 \) & HYPOTHESIS \\
2) & \( t_1 \vdash t_2 \)          & ASSUME \\
3) & \( \Gamma,t_1 \vdash t_2 \)   & SUBST 1,2\\
4) & \( \Gamma \vdash t_1 \supset t_2 \) & DISCH 3\\
5) & \( \Gamma \vdash t_2 = t_1 \) & SYM 1\\
6) & \( t_2 \vdash t_2 \)          & ASSUME\\
7) & \( \Gamma,t_2 \vdash t_1\)      & SUBST 5,6\\
8) & \( \Gamma \vdash t_2 \supset t_1\) & DISCH 7\\
9) & \( \Gamma \vdash t_1 \supset t_2\), 
     \( \;\; \Gamma \vdash t_1 \supset t_2 \;\; \)
  & LINES 4,8
\end{tabular}}}
\espindent

% =====================================================================
\slide{Summary of Higher Order Logic}

\point{Syntax}
\subpoint{Variables, Constants, Applications, Abstractions}

\point{Types}
\subpoint{Simple, Compound, Polymorphic}

\point{Syntactic Sugar}
\subpoint{Infixes, Binders, Pairs, Lists, Conditionals}

\point{Formulas}
\subpoint{Boolean terms}

\point{Sequents}
\subpoint{\( (\Gamma, t) \) -- assumptions and conclusion}

\point{Theorems}
\subpoint{\( \Gamma \vdash t \) -- sequents derived by proof}

\point{Constant definitions}
\subpoint{ \( \vdash c = t\) -- constants abbreviating terms}


% =====================================================================
\slide{Summary of Higher Order Logic}

\point{Theories}
\subpoint{Sets of types, type operators, constants, \\
definitions, axioms, theorems}

\point{The basic theories}
\subpoint{{\rm bool} and {\rm ind}}

\point{Five axioms}

\point{Inference Rules}
\subpoint{Eight primitive. Many derived.}

\point{Still to be discussed:}
\subpoint{Type definitions}
\subpoint{Rule of constant specification}
\subpoint{Semantics}

\point{Reference:} 

\subpoint{Hvg, {\it The HOL System: DESCRIPTION} (DSTO, University of Cambridge
and SRI International, 1991).}


% =====================================================================
\slide{Type Definitions}

\point{Theories can contain type definitions.}

\point{The primitive types are:}

\subpoint{$bool$ and $ind$}
\subpoint{type variables}
\subpoint{types formed using the type operator $\rightarrow$}

\point{All other types are defined by taking `subsets' of existing
types.}


% =====================================================================
\slide{Type Definitions}

\point{Suppose $ty$ is a type denoting some set\\
 of values:}

\vskip 7mm
\begin{center}
\setlength{\unitlength}{1mm}
\begin{picture}(100,40)
\thicklines
\put(25,20){\oval(30,30)}
\put(24,28){\large $ty$}
\end{picture}
\end{center}

\point{A type definition introduces a new type that denotes a subset of the
existing type $ty$.}


\point{We specify this subset with a predicate}
\vskip5mm
\bspindent\LARGE
$P: ty \rightarrow bool$\espindent\vskip5mm
\bpindent\LARGE\bf
that is true of just those values of type $ty$ that the subset contains:
\epindent

\vskip 7mm
\begin{center}
\setlength{\unitlength}{1mm}
\begin{picture}(100,40)
\thicklines
\put(25,20){\oval(30,30)}
\put(25,15){\oval(12,12)}
\put(24,28){\large $ty$}
\put(24,15){\large $P$}
\end{picture}
\end{center}



% =====================================================================
\slide{Type Definitions}

\point{The subset that is to form the new type must be non-empty}
\vskip 7mm
\begin{center}
\setlength{\unitlength}{1mm}
\begin{picture}(100,40)
\thicklines
\put(25,20){\oval(30,30)}
\put(25,15){\oval(12,12)}
\put(24,28){\large $ty$}
\put(24,15){\large $P$}
\put(28,12){\circle*{2}}
\end{picture}
\end{center}
\bpindent\LARGE\bf
since all types must denote non-empty sets. \epindent

\vskip7mm
\point{We must therefore prove:}
\vskip 7mm
\bspindent\LARGE\bf
$\vdash$ $\exists$ $x{:}ty.$ $P$($x$) 
\espindent
\vskip7mm

\point{This is because $\epsilon$-terms involving variables of the new type
must denote something.}

% =====================================================================
\slide{Type Definitions}

\point{Given a subset $P$ of an existing type $ty$}
\vskip 7mm
\begin{center}
\setlength{\unitlength}{1mm}
\begin{picture}(100,40)
\thicklines
\put(25,20){\oval(30,30)}
\put(25,15){\oval(12,12)}
\put(24,28){\large $ty$}
\put(24,15){\large $P$}
\put(28,12){\circle*{2}}
\end{picture}
\end{center}
\bpindent\LARGE\bf
we can introduce a new type symbol, say $ty'$\epindent


\point{A type definition then asserts that $ty'$ denotes a set of values in
one-to-one corespondence with values in the subset $P$:}
\vskip 7mm
\begin{center}
\setlength{\unitlength}{1mm}
\begin{picture}(100,40)
\thicklines
\put(25,20){\oval(30,30)}
\put(25,15){\oval(12,12)}
\put(65,15){\oval(12,12)}
\thinlines
\put(25,21){\line(1,0){40}}
\put(25,9){\line(1,0){40}}
\put(24,28){\large $ty$}
\put(64,15){\large $ty'$}
\put(24,15){\large $P$}
\put(28,12){\circle*{2}}
\end{picture}
\end{center}

\point{That is, there is a bijection between $ty'$ and $P$.}

% =====================================================================
\slide{Type Definitions}

\point{A type definition states that there is a\\
bijection from the new type $ty'$ to the subset\\
of the existing type $ty$ given by $P$:}

\vskip 7mm
\bspindent\LARGE
$\vdash \exists f\begin{array}[t]{@{}l}: ty' \rightarrow ty.\\
   \forall a_1\; a_2.\:f\;a_1 = f\;a_2 \supset a_1 = a_2\; \wedge \\
   \forall r.\: P\;r = \exists a.\:r = f\; r\end{array}$
\espindent
\vskip 7mm
\bpindent\LARGE\bf
That is:\epindent

\vskip 7mm
\begin{center}
\setlength{\unitlength}{1mm}
\begin{picture}(100,40)
\thicklines
\put(25,20){\oval(30,30)}
\put(25,15){\oval(12,12)}
\put(65,15){\oval(12,12)}
\thinlines
\put(25,21){\line(1,0){40}}
\put(25,9){\line(1,0){40}}

\put(45,15){\large $\longleftarrow$ $f$}

\put(24,28){\large $ty$}
\put(64,15){\large $ty'$}
\put(24,15){\large $P$}
\put(28,12){\circle*{2}}
\end{picture}
\end{center}

\point{Any such definition is a {\it conservative extension\/}\\
 of the logic.}

% =====================================================================
\slide{Type Operators}

\point{Type operators can be defined by having type variables in the
representing type.}

\vskip7mm
\point{For example $\alpha \times \beta$ can be defined as an\\
appropriate subset of the type:}
\vskip5mm
\bspindent\LARGE\bf
$\alpha \rightarrow \beta \rightarrow {\rm bool}$
\espindent

\vskip7mm

\point{Examples will be given later \ldots}


% =====================================================================
\slide{Summary: Type Definitions}

\point{To define a new type:}

\subpoint{Introduce a new type symbol $\sigma$}

\subpoint{Specify a representing subset of an existing type $\tau$ by a
predicate $P : \tau{\rightarrow} bool$}

\subpoint{Prove that $\exists x.\: P(x)$}

\subpoint{Add the definitional axiom:}
\vskip 7mm
\bspindent\Large
\quad$\vdash \exists f\begin{array}[t]{@{}l}: \sigma \rightarrow \tau.\\
   \forall a_1\; a_2.\:f\;a_1 = f\;a_2 \supset a_1 = a_2\; \wedge \\
   \forall r. P\;r = \exists a.\:r = f\; r\end{array}$
\espindent
\vskip 7mm

\point{All non-primitive types in HOL are defined this way.}


\end{document}
