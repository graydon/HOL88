% =====================================================================
% HOL Course Slides: introductory slides              (c) T Melham 1990
% =====================================================================

\documentstyle[12pt,layout]{article}

% ---------------------------------------------------------------------
% Preliminary settings.
% ---------------------------------------------------------------------

\renewcommand{\textfraction}{0.01}	  % 0.01 of the page must contain text
\setcounter{totalnumber}{10}	 	  % max of 10 figures per page
\flushbottom				  % text extends right to the bottom
\pagestyle{slides}			  % slides page style
\setlength{\unitlength}{1mm}		  % unit = 1 mm

% ---------------------------------------------------------------------
% load macros
% ---------------------------------------------------------------------
% =====================================================================
% Macros for HOL course slides 	                      (c) T Melham 1990
% =====================================================================

% ---------------------------------------------------------------------
% Macro for title page of a section:
%
%     \bsectitle
%       ...<title>...
%     \esectitle
%
% starts a new page, and typesets <title> in a large box at the top
% of the page.
% ---------------------------------------------------------------------

\newbox\stitle

\long\def\Frame#1{\leavevmode
    \hbox{\vbox{\hrule height2pt
              \hbox{\vrule width2pt #1\vrule width2pt}
  	      \hrule height2pt}}}

\long\def\bsectitle{\setbox\stitle=\hbox\bgroup\begin{minipage}{140mm}
		   \begin{center}\Huge\bf\vskip10mm}

\long\def\esectitle{\vskip10mm\end{center}\end{minipage}\egroup
		    \clearpage\vspace*{30mm}%
		   \begin{center}\Frame{\box\stitle}\end{center}}

% ---------------------------------------------------------------------
% Macro for title of slide:
%
%     \slide{<title>} 
%
% starts a new page, and typesets <title> in a large box at the top
% of the page.
% ---------------------------------------------------------------------

\long\def\slide#1{\newpage\leavevmode
 \hbox{\vbox{\hrule height2pt%
 \hbox to \textwidth{\vrule width2pt\hskip5mm%
 \vbox to 16mm{\vfill\vskip1mm\hbox{\Huge\bf #1}\vfill}\hfil\vrule width2pt}%
 \hrule height2pt}}\vskip7mm}

% ---------------------------------------------------------------------
% round circles: \blob and \subblob 
% ---------------------------------------------------------------------

\newcommand{\blob}{\begin{picture}(8.5,4)(0,-0.1)
		    \put(2.5,2){\circle*{3}} \end{picture}}

\newcommand{\subblob}{\begin{picture}(4.5,4)(0,0)
		    \put(1,1.5){\circle*{2}} \end{picture}}

% ---------------------------------------------------------------------
% Various lengths
%
% \blobwidth     = the width of a blob
% \subblobwidth  = the width of a sub-blob
% \pointwidth    = the text width of a major point
% \sind          = the standard indentation
% ---------------------------------------------------------------------

\newlength{\blobwidth}
\settowidth{\blobwidth}{\blob}

\newlength{\pointwidth}
\setlength{\pointwidth}{\textwidth}
\addtolength{\pointwidth}{-1\blobwidth}

\newlength{\subblobwidth}
\settowidth{\subblobwidth}{\subblob}

\newlength{\sind}
\setlength{\sind}{15mm}

% ---------------------------------------------------------------------
% \point : for making points
% ---------------------------------------------------------------------

\def\point#1{\vskip10mm\blob\parbox[t]{\pointwidth}{\LARGE\bf#1}\par}

% ---------------------------------------------------------------------
% \subpoint : for making sub-points
% ---------------------------------------------------------------------

\def\subpoint#1{{\addtolength{\pointwidth}{-\sind}
	      \addtolength{\pointwidth}{-1\subblobwidth}\vskip5mm
	      \hspace*{\sind}\subblob
	      \parbox[t]{\pointwidth}{\Large\bf#1}\par}}

% ---------------------------------------------------------------------
% \pindent : indent to the level of a point
% ---------------------------------------------------------------------

\def\bpindent{\bgroup\leftskip=\blobwidth}
\def\epindent{\par\egroup}

% ---------------------------------------------------------------------
% \pindent : indent to the level of a subpoint
% ---------------------------------------------------------------------

\def\bspindent{\bgroup\leftskip=\subblobwidth\addtolength{\leftskip}{\sind}}
\def\espindent{\par\egroup}

% ---------------------------------------------------------------------
% Macros for little HOL sessions displayed in boxes.
%
% Usage:     \begin{session}\begin{verbatim}
%	      .
%	       < lines from hol session >
%	      .
%	     \end{verbatim}\end{session}   
%
%            typesets the session in a box.
%
% Lengths: \hsbw = width of box, \lwid = width of lines
% ---------------------------------------------------------------------

\newlength{\lwid}
\setlength{\lwid}{0.8pt}

\newlength{\hsbw}
\setlength{\hsbw}{\textwidth}
\addtolength{\hsbw}{-\lwid}
\addtolength{\hsbw}{-\tabcolsep}
\addtolength{\hsbw}{-\blobwidth}
\addtolength{\hsbw}{-1mm}

\newenvironment{session}{\setlength{\arrayrulewidth}{\lwid}
 \begin{flushleft}\hskip\blobwidth\hskip1mm
 \begin{tabular}{@{}|c@{}|@{}}\hline 
 \begin{minipage}[b]{\hsbw}
 \vspace*{5.5pt}
 \begingroup\Large\baselineskip17.5pt}{\endgroup\end{minipage}\\[1.5pt]\hline 
 \end{tabular}
 \end{flushleft}}


% ---------------------------------------------------------------------
% set caption at the foot of pages for this series of slides
% ---------------------------------------------------------------------
\ftext{Introduction}{1}

% ---------------------------------------------------------------------
% Slides
% ---------------------------------------------------------------------
\begin{document}

% ---------------------------------------------------------------------
% Title page for this series of slides
% ---------------------------------------------------------------------

\bsectitle
Introduction\\
to the\\
HOL Theorem Prover\\
\esectitle

\vskip20mm

\begin{center}
\large\bf
T F Melham
\end{center}
\vskip10mm
\begin{center}
\bf
University of Cambridge\\
Computer Laboratory\\
New Museums Site\\
Pembroke Street\\
Cambridge, CB2 3QG\\
England
\end{center}

% =====================================================================
\slide{What is HOL?}

\point{A system for proving theorems in 
Higher\\ Order Logic.}

\point{Features of HOL:}
\subpoint{has a rigorous formal basis}
\subpoint{supports both `forward' and goal-directed proof}
\subpoint{secure: can't prove false theorems}
\subpoint{embedded in a general purpose functional\\ programming language (ML)}
\subpoint{user-extendable, without compromising security}
\subpoint{NOT an `automatic' theorem prover}

% =====================================================================
\slide{History of HOL}

\vskip10mm

\begin{center}
\def\_{\leavevmode \kern-0.5mm \vbox{\hrule height0.2mm width0.3em}}
\setlength{\unitlength}{1mm}
\begin{picture}(100,120)
{
\thicklines
\put(45,55){\line(-3,-1){30}}
\put(55,55){\line( 3,-1){30}}
\put(85,19){\line( 0,-1){10}}
\put(40,95){\line(-3,-1){30}}
\put(50,95){\line( 0,-1){30}}
\put(50,115){\line( 0,-1){10}}

\put(1,16){\bf {\Large{\bf 
  {\shortstack{LCF{\kern-.3mm}\_{\kern.7mm}LSM\\[2mm]
 (logic for\\ sequential\\ machines)}}}}}
\put(85,5){\makebox(0,0)[c]{\Large{\bf HOL88}}}
\put(65,21){\Large{\bf {\shortstack{ HOL\\[2mm] (Higher Order\\Logic) }}}}
\put(50,60){\makebox(0,0)[c]{\Large{\bf Cambridge LCF (PP\kern-1mm$\lambda$)}}}
\put(10,80){\makebox(0,0)[c]{\Large{\bf Standard ML}}}
\put(50,100){\makebox(0,0)[c]{\Large{\bf Edinburgh LCF (ML, PP\kern-1mm$\lambda$)}}}
\put(50,120){\makebox(0,0)[c]{\Large{\bf Stanford LCF}}} 
}
\end{picture}
\end{center}
\vskip 7mm

% =====================================================================
\slide{HOL and ML}

\point{HOL is built on top of ML.}


\point{Roughly speaking, HOL = ML plus:}

\vskip7mm
\bspindent\LARGE
{\bf some predefined ML programs (functions),} 
\vskip7mm
{\bf and some data type declarations.}
\espindent

\point{There are also a few enhancements to the ML parser and pretty-printer.}

% =====================================================================
\slide{Tools and System Support}

\point{HOL tools include:}

\subpoint{support for goal-directed proof}
\subpoint{many built-in inference rules}
\subpoint{automatic recursive type definitions}
\subpoint{structural induction tools}
\subpoint{rewriting tools (from LCF)}
\subpoint{automatic primitive recursive definitions}
\subpoint{built-in theories of arithmetic, lists, sets,\dots}
\subpoint{tautology checker}
\subpoint{automatic inductive definitions}
\subpoint{parser and pretty-printer generator}
\subpoint{online help facility}
\subpoint{full documentation}
\subpoint{user-loadable libraries}


% =====================================================================
\slide{Applications}

\point{HOL applications include:}

\subpoint{hardware design and verification}
\subpoint{reasoning about security}
\subpoint{verification of fault-tolerant computers}
\subpoint{reasoning about real-time system}
\subpoint{semantics of HDLs (e.g.\ VHDL, ELLA)}
\subpoint{compiler verification}
\subpoint{program refinement calculus}
\subpoint{software verification (e.g.\ Hoare logic)}
\subpoint{modelling concurrency (e.g.\ CCS, CSP)}
\subpoint{automata theory}
\subpoint{\dots\ et cetera}


% =====================================================================
\slide{Course Outline}

\point{Overview of higher order logic}

\subpoint{Syntax of the logic}

\subpoint{Primitive basis of the logic}

\point{Brief introduction to ML}

\point{Introduction to the HOL system}

\subpoint{Embedding the logic in ML}

\subpoint{Inference rules, theorems and proof}

\subpoint{The core system}

\point{Forward proof in HOL}

\point{Goal-directed proof in HOL}

\point{Primitive recursion and induction}

\point{The recursive types package}

\end{document}
