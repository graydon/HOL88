% =====================================================================
% HOL Course Slides: overview of higher order logic   (c) T melham 1990
% =====================================================================

\documentstyle[12pt,layout]{article}

% ---------------------------------------------------------------------
% Preliminary settings.
% ---------------------------------------------------------------------

\renewcommand{\textfraction}{0.01}	  % 0.01 of the page must contain text
\setcounter{totalnumber}{10}	 	  % max of 10 figures per page
\flushbottom				  % text extends right to the bottom
\pagestyle{slides}			  % slides page style
\setlength{\unitlength}{1mm}		  % unit = 1 mm

% ---------------------------------------------------------------------
% load macros
% ---------------------------------------------------------------------
% =====================================================================
% Macros for HOL course slides 	                      (c) T Melham 1990
% =====================================================================

% ---------------------------------------------------------------------
% Macro for title page of a section:
%
%     \bsectitle
%       ...<title>...
%     \esectitle
%
% starts a new page, and typesets <title> in a large box at the top
% of the page.
% ---------------------------------------------------------------------

\newbox\stitle

\long\def\Frame#1{\leavevmode
    \hbox{\vbox{\hrule height2pt
              \hbox{\vrule width2pt #1\vrule width2pt}
  	      \hrule height2pt}}}

\long\def\bsectitle{\setbox\stitle=\hbox\bgroup\begin{minipage}{140mm}
		   \begin{center}\Huge\bf\vskip10mm}

\long\def\esectitle{\vskip10mm\end{center}\end{minipage}\egroup
		    \clearpage\vspace*{30mm}%
		   \begin{center}\Frame{\box\stitle}\end{center}}

% ---------------------------------------------------------------------
% Macro for title of slide:
%
%     \slide{<title>} 
%
% starts a new page, and typesets <title> in a large box at the top
% of the page.
% ---------------------------------------------------------------------

\long\def\slide#1{\newpage\leavevmode
 \hbox{\vbox{\hrule height2pt%
 \hbox to \textwidth{\vrule width2pt\hskip5mm%
 \vbox to 16mm{\vfill\vskip1mm\hbox{\Huge\bf #1}\vfill}\hfil\vrule width2pt}%
 \hrule height2pt}}\vskip7mm}

% ---------------------------------------------------------------------
% round circles: \blob and \subblob 
% ---------------------------------------------------------------------

\newcommand{\blob}{\begin{picture}(8.5,4)(0,-0.1)
		    \put(2.5,2){\circle*{3}} \end{picture}}

\newcommand{\subblob}{\begin{picture}(4.5,4)(0,0)
		    \put(1,1.5){\circle*{2}} \end{picture}}

% ---------------------------------------------------------------------
% Various lengths
%
% \blobwidth     = the width of a blob
% \subblobwidth  = the width of a sub-blob
% \pointwidth    = the text width of a major point
% \sind          = the standard indentation
% ---------------------------------------------------------------------

\newlength{\blobwidth}
\settowidth{\blobwidth}{\blob}

\newlength{\pointwidth}
\setlength{\pointwidth}{\textwidth}
\addtolength{\pointwidth}{-1\blobwidth}

\newlength{\subblobwidth}
\settowidth{\subblobwidth}{\subblob}

\newlength{\sind}
\setlength{\sind}{15mm}

% ---------------------------------------------------------------------
% \point : for making points
% ---------------------------------------------------------------------

\def\point#1{\vskip10mm\blob\parbox[t]{\pointwidth}{\LARGE\bf#1}\par}

% ---------------------------------------------------------------------
% \subpoint : for making sub-points
% ---------------------------------------------------------------------

\def\subpoint#1{{\addtolength{\pointwidth}{-\sind}
	      \addtolength{\pointwidth}{-1\subblobwidth}\vskip5mm
	      \hspace*{\sind}\subblob
	      \parbox[t]{\pointwidth}{\Large\bf#1}\par}}

% ---------------------------------------------------------------------
% \pindent : indent to the level of a point
% ---------------------------------------------------------------------

\def\bpindent{\bgroup\leftskip=\blobwidth}
\def\epindent{\par\egroup}

% ---------------------------------------------------------------------
% \pindent : indent to the level of a subpoint
% ---------------------------------------------------------------------

\def\bspindent{\bgroup\leftskip=\subblobwidth\addtolength{\leftskip}{\sind}}
\def\espindent{\par\egroup}

% ---------------------------------------------------------------------
% Macros for little HOL sessions displayed in boxes.
%
% Usage:     \begin{session}\begin{verbatim}
%	      .
%	       < lines from hol session >
%	      .
%	     \end{verbatim}\end{session}   
%
%            typesets the session in a box.
%
% Lengths: \hsbw = width of box, \lwid = width of lines
% ---------------------------------------------------------------------

\newlength{\lwid}
\setlength{\lwid}{0.8pt}

\newlength{\hsbw}
\setlength{\hsbw}{\textwidth}
\addtolength{\hsbw}{-\lwid}
\addtolength{\hsbw}{-\tabcolsep}
\addtolength{\hsbw}{-\blobwidth}
\addtolength{\hsbw}{-1mm}

\newenvironment{session}{\setlength{\arrayrulewidth}{\lwid}
 \begin{flushleft}\hskip\blobwidth\hskip1mm
 \begin{tabular}{@{}|c@{}|@{}}\hline 
 \begin{minipage}[b]{\hsbw}
 \vspace*{5.5pt}
 \begingroup\Large\baselineskip17.5pt}{\endgroup\end{minipage}\\[1.5pt]\hline 
 \end{tabular}
 \end{flushleft}}

\def\meta#1{{$\langle\hbox{\it #1\/}\rangle$}}
\def\Rule#1#2{\mbox{${\displaystyle\raise 6pt\hbox{$\;\;\;#1\;\;\;$}} \over 
                     {\displaystyle\lower7pt\hbox{$\;\;\;#2\;\;\;$}}$}}


% ---------------------------------------------------------------------
% set caption at the foot of pages for this series of slides
% ---------------------------------------------------------------------
\ftext{Forward Proof}{7}

% ---------------------------------------------------------------------
% Slides
% ---------------------------------------------------------------------
\begin{document}

% ---------------------------------------------------------------------
% Title page for this series of slides
% ---------------------------------------------------------------------

\bsectitle
Basic Theorem Proving\\
{\vrule width20mm height2.5mm depth-2mm}\\
Forward Proof
\esectitle

% =====================================================================
\slide{Forward Proof in HOL}

\point{Forward inference:}
\vskip5mm
\subpoint{That is, apply the inference rules in sequence to already proved
theorems in order to prove the desired theorem.} 
\vskip5mm
\subpoint{All theorems in HOL ultimately proved this way.}
\vskip5mm
\subpoint{Not natural for `one-off' proofs (for this, use {\it goal directed
proof\/} instead).}
\vskip5mm
\subpoint{But it is essential to know the inference rules for tool building.}


% =====================================================================
\slide{Example of Forward Proof}


\point{Example: the proof of $\vdash \neg F$ we did before:}


\vskip 10mm
\bspindent{\Large{\bf
\begin{tabular}{@{}l@{$\;\;\;\;$}l@{$\;\;\;\;$}l@{}}
1) & \( \vdash (\lambda b. \; b \supset F)F = (F \supset F) \) &
    {\Large\bf BETA\_CONV}\\[3mm]
2) & \( \vdash (\lambda b. \; b \supset F)F =
         (\lambda b. \; b \supset F)F \) &
{\Large\bf REFL} \\[3mm]
3) & \( \vdash (F \supset F) = (\lambda b. \; b \supset F)F \) &
SUBST 1,2 \\[3mm]
4) & \( \{F\} \vdash F \) & ASSUME \\[3mm]
5) & \( \vdash F \supset F \) & DISCH 4 \\[3mm]
6) & \( \vdash (\lambda b. \; b \supset F)F \) & SUBST 3,5\\[3mm]
7) & \( \vdash \neg = (\lambda b. \; b \supset F)  \) & DEFN of $\neg$\\[3mm]
8) & \( \vdash \neg = \neg  \) & REFL\\[3mm]
9) & \( \vdash (\lambda b. \; b \supset F) = \neg \) & SUBST 7,8\\[3mm]
10) & \( \vdash \neg F \) & SUBST 6,9 \\
\end{tabular}}}
\espindent
\vskip 10mm
\point{The proof uses only primitive inference rules.}

\def\_{\char'137}

% =====================================================================
\slide{The Proof in HOL}

\point{Apply beta-conversion:}
\vskip4mm
\begin{session}\begin{verbatim}
#let thm1 = BETA_CONV "(\b. b ==> F) F";;
thm1 = |- (\b. b ==> F)F = F ==> F
\end{verbatim}\end{session}

\point{Reverse the equation just derived:}\vskip4mm
\begin{session}\begin{verbatim}
#let thm2 = REFL "(\b. b==> F) F";;
thm2 = |- (\b. b ==> F)F = (\b. b ==> F)F

#let thm3 = SUBST [(thm1,"m:bool")]
                  "m = (\b. b==>F) F" thm2;;
thm3 = |- F ==> F = (\b. b ==> F)F
\end{verbatim}\end{session}

\point{Assume falsity, and discharge the assumption:}\vskip4mm
\begin{session}\begin{verbatim}
#let thm4 = ASSUME "F";;
thm4 = . |- F

#let thm5 = DISCH "F" thm4;;
thm5 = |- F ==> F
\end{verbatim}\end{session}

\point{Conclude as follows:}\vskip4mm
\begin{session}\begin{verbatim}
#let thm6 = SUBST [(thm3,"m:bool")] "m:bool" thm5;;
thm6 = |- (\b. b ==> F)F
\end{verbatim}\end{session}

\slide{The Proof Continued}

\point{The theorem already derived:}\vskip4mm
\begin{session}\begin{verbatim}
#thm6;;
|- (\b. b ==> F)F
\end{verbatim}\end{session}

\point{Reverse the defining equation for negation:}\vskip4mm

\begin{session}\begin{verbatim}
#let thm7 = REFL "$~";;
thm7 = |- $~ = $~

#let thm8 = SUBST [(NOT_DEF,"m:bool->bool")] 
                   "m = $~" thm7;;
thm8 = |- (\t. t ==> F) = $~
\end{verbatim}\end{session}

\point{Substitute with this equation:}\vskip4mm
\begin{session}\begin{verbatim}
#let thm = SUBST [(thm8, "m:bool->bool")] 
                  "(m:bool->bool) F" thm6;;
thm = |- ~F
\end{verbatim}\end{session}


% =====================================================================
\slide{Derived Inference Rules}

\point{In HOL, derived inference rules are just ML programs that call the
correct sequence of primitive (or derived) inference rules.}

\vskip7mm
\point{Consider the example we looked at before:}
\vskip 10mm
\bspindent\LARGE\bf
{\Large\bf SYM:\quad}
\Rule{\Gamma \vdash t_1  =  t_2}
{\Gamma \vdash t_2  =  t_1}
\espindent

\vskip7mm
\point{Derivation:}
\bspindent\LARGE\bf\vskip7mm
\begin{tabular}{l l @{\qquad} l}
1) & \( \Gamma \vdash t_1 \; = \; t_2 \) & {\Large\bf HYPOTHESIS} \\
2) & \( \vdash t_1 \; = \; t_1  \)       & {\Large\bf REFL}\\
3) & \( \Gamma \vdash t_2 \; = \; t_1 \) & {\Large\bf SUBST 1,2}
\end{tabular}
\espindent
\vskip7mm

\point{We will now look at the derivation in HOL.}


% =====================================================================
\slide{Deriving the Rule in HOL}

\point{Start by assuming the premise:}
\vskip 4mm
\begin{session}\begin{verbatim}
#let th = ASSUME "t1:* = t2:*";;
th = . |- t1 = t2
\end{verbatim}\end{session}

\point{We need to know what {\tt t1} and {\tt t2} are:}
\vskip 4mm
\begin{session}\begin{verbatim}
#let t1,t2 = dest_eq(concl th);;
t1 = "t1" : term
t2 = "t2" : term
\end{verbatim}\end{session}

\point{We need a marker variable:}
\vskip 4mm
\begin{session}\begin{verbatim}
#let v = genvar(type_of t1);;
v = "GEN%VAR%464" : term
\end{verbatim}\end{session}

\point{The result is then given by:}
\vskip 4mm
\begin{session}\begin{verbatim}
#SUBST [(th,v)] "^v=^t1" (REFL t1);;
. |- t2 = t1
\end{verbatim}\end{session}


% =====================================================================
\slide{The Derived Rule}

\point{Since {\tt t1} and {\tt t2} were arbitrary, this derivation suggests
the following code:}

\vskip 4mm
\begin{session}\begin{verbatim}
#let SYM th =
   (let t1,t2 = dest_eq(concl th) in
    let v = genvar(type_of t1) in
    SUBST [(th,v)] (mk_eq(v,t1)) (REFL t1)
   )? failwith `SYM: not an equation`;;
SYM = - : (thm -> thm)
\end{verbatim}\end{session}

\point{Test the rule:}

\vskip 4mm
\begin{session}\begin{verbatim}
#F_DEF;;
|- F = (!t. t)

#SYM (F_DEF);;
|- (!t. t) = F

#SYM(ASSUME "!t. t=F");;
evaluation failed     SYM: not an equation
\end{verbatim}\end{session}

% =====================================================================
\slide{Derived Inference Rules}

\point{There are {\it many\/} derived inference rules built into the system.}

\vskip7mm

\point{We shall introduce the rules only as required, but a full list is given
in the manual.}

\vskip7mm

\point{To become an `expert' HOL user, you should try to become familiar with
and use as many of the rules as possible.}

\vskip7mm

\point{The more rules you know, the better your chances of finding the one you
need.}

\vskip7mm

\point{There are also many pre-proved {\it theorems\/} in the system (see the
manual).}

% =====================================================================
\slide{Example of Forward Proof}

\point{We will use forward proof to derive}
\vskip6mm
\bspindent\LARGE\bf
$((\exists x. \: P\;x) \supset Q) = (\forall x. \: P\;x \supset Q) $
\espindent
\vskip6mm
\bpindent\LARGE\bf
by proving implication in both directions.\epindent

\vskip10mm

\point{Show hypotheses during the proof:}
\vskip4mm
\begin{session}\begin{verbatim}
#top_print print_all_thm;;
- : (thm -> void)
\end{verbatim}\end{session}

\point{To prove the left-to-right implication, first\\
 assume the antecedant:}
\vskip 4mm
\begin{session}\begin{verbatim}
#let th1 = ASSUME "(?x:*. P(x)) ==> Q";;
th1 = (?x. P x) ==> Q |- (?x. P x) ==> Q
\end{verbatim}\end{session}

\point{Assume {\tt P x}, then deduce {\tt ?x.\ P x}:}
\vskip 4mm
\begin{session}\begin{verbatim}
#let th2 = ASSUME "P(x:*):bool";;
th2 = P x |- P x

#let th3 = EXISTS ("?x:*. P(x)", "x:*") th2;;
th3 = P x |- ?x. P x
\end{verbatim}\end{session}


% =====================================================================
\slide{Example Continued}

\point{Now, deduce {\tt Q}:}

\vskip 4mm
\begin{session}\begin{verbatim}
#th1;;
(?x. P x) ==> Q |- (?x. P x) ==> Q

#th3;;
P x |- ?x. P x

#let th4 = MP th1 th3;;
th4 = (?x. P x) ==> Q, P x |- Q
\end{verbatim}\end{session}

\point{Discharge the assumption {\tt P x}:}
\vskip 4mm
\begin{session}\begin{verbatim}
#let th5 = DISCH "P(x:*):bool" th4;;
th5 = (?x. P x) ==> Q |- P x ==> Q
\end{verbatim}\end{session}

\point{Generalize {\tt x} and discharge the assumption:}
\vskip 4mm
\begin{session}\begin{verbatim}
#let th6 = DISCH_ALL (GEN "x:*" th5);;
th6 = |- ((?x. P x) ==> Q) ==> (!x. P x ==> Q)
\end{verbatim}\end{session}


\point{To prove the reverse implication, again start\\ by
assuming the antecedant:}
\vskip 4mm
\begin{session}\begin{verbatim}
#let th7 = ASSUME "!(x:*).P(x) ==> Q";;
th7 = !x. P x ==> Q |- !x. P x ==> Q
\end{verbatim}\end{session}



% =====================================================================
\slide{Example Continued}

\point{Specialize {\tt x} to {\tt x}:}
\vskip 4mm
\begin{session}\begin{verbatim}
#th7;;
!x. P x ==> Q |- !x. P x ==> Q

#let th8 = SPEC "x:*" th7;;
th8 = !x. P x ==> Q |- P x ==> Q
\end{verbatim}\end{session}

\point{Move {\tt P x} into the assumptions:}
\vskip 4mm
\begin{session}\begin{verbatim}
#let th9 = UNDISCH th8;;
th9 = !x. P x ==> Q, P x |- Q
\end{verbatim}\end{session}

\point{Now, assume {\tt ?x.\ P x}:}
\vskip 4mm
\begin{session}\begin{verbatim}
#let th10 = ASSUME "?x:*. P(x)";;
th10 = ?x. P x |- ?x. P x
\end{verbatim}\end{session}

\point{Use {\tt CHOOSE} to conclude {\tt Q}:}
\vskip 4mm
\begin{session}\begin{verbatim}
#let th11 = CHOOSE ("x:*",th10) th9;;
th11 = ?x. P x, !x. P x ==> Q |- Q
\end{verbatim}\end{session}

\point{Discharge the assumption {\tt ?x.\ P x}:}
\vskip 4mm
\begin{session}\begin{verbatim}
#let th12 = DISCH "?x:*.P(x)" th11;;
th12 = !x. P x ==> Q |- (?x. P x) ==> Q
\end{verbatim}\end{session}

% =====================================================================
\slide{The Example Concluded}

\point{The results so far:}
\vskip 4mm
\begin{session}\begin{verbatim}
#th6;;
|- ((?x. P x) ==> Q) ==> (!x. P x ==> Q)

#th12;;
!x. P x ==> Q |- (?x. P x) ==> Q
\end{verbatim}\end{session}

\point{Discharge the remaining assumption:}
\vskip 4mm
\begin{session}\begin{verbatim}
#let th13 = DISCH_ALL th12;;
th13 = |- (!x. P x ==> Q) ==> (?x. P x) ==> Q
\end{verbatim}\end{session}

\point{Put the two implications together:}
\vskip 4mm
\begin{session}\begin{verbatim}
#let thm = IMP_ANTISYM_RULE th6 th13;;
thm = |- (?x. P x) ==> Q = (!x. P x ==> Q)
\end{verbatim}\end{session}

\point{Q.E.D.}

% ===================================================================== 
\slide{Summary}

\point{Forward proof in HOL}
\vskip5mm
\subpoint{Primitive Inference Rules:}
\vskip5mm
\bspindent\Large\bf\obeylines
\qquad{\tt ASSUME}, {\tt REFL}, {\tt BETA\_CONV}, {\tt SUBST}, 
\qquad{\tt ABS}, {\tt INST\_TYPE}, {\tt DISCH}, {\tt MP}.
\espindent

\subpoint{Derived Inference Rules:}
\vskip5mm
\bspindent\Large\bf\obeylines
\qquad{\tt SYM}, {\tt EXISTS}, {\tt DISCH\_ALL}, {\tt SPEC}, {\tt GEN}
\qquad{\tt UNDISCH}, {\tt CHOOSE}, {\tt IMP\_ANTISYM\_RULE}, 
\qquad\dots etc.
\espindent



% =====================================================================
\slide{Rewriting Inference Rules}

\point{A powerful set of built-in inference rules is provided in HOL for 
rewriting.}

\vskip7mm
\point{Rewriting is different from substitution:}
\subpoint{it deals with pattern matching}
\subpoint{it deals with bound variables}
\vskip7mm


\point{Example:}
\vskip 4mm
\begin{session}\begin{verbatim}
#num_CASES;;
|- !m. (m = 0) \/ (?n. m = SUC n)

#ADD1;;
|- !m. SUC m = m + 1

#REWRITE_RULE [ADD1] num_CASES;;
|- !m. (m = 0) \/ (?n. m = n + 1)
\end{verbatim}\end{session}

\point{We didn't have to:}
\subpoint{specialize {\tt m} in {\tt ADD1}}
\subpoint{deal explicitly with the bound variable {\tt n}}


% =====================================================================
\slide{Basic Rewriting Rules}

\point{There are two basic rules:}

\vskip5mm
\bspindent\Large\bf
\verb!REWRITE_RULE : thm list -> thm -> thm!
\vskip3mm
\verb!PURE_REWRITE_RULE : thm list -> thm -> thm!
\espindent
\vskip5mm
\subpoint{first argument: a list of equational theorems to be used as
left-to-right rewrite rules.}
\subpoint{second argument: the theorem to be rewritten.}

\vskip4mm
\point{Rewriting is done:}
\subpoint{with all the supplied equations}
\subpoint{on all subterms of the theorem to be rewritten}
\subpoint{repeatedly, until no rewrite rule applies}
\vskip4mm

\point{In addition to equations, rewrite rules can be:}

\subpoint{Quantified: {\tt |- !x1}\dots{\tt xn}. \meta{lhs} = \meta{rhs}}
\subpoint{Negations, which become {\tt |- \meta{lhs} = F}}
\subpoint{Non-equations, which become {\tt |- \meta{lhs} = T}}
\subpoint{Conjunctions of the above}

% =====================================================================
\slide{Basic Rewrite Rules}

\point{The difference between
{\tt REWRITE\_RULE} and\\
{\tt PURE\_REWRITE\_RULE} is:}
\vskip5mm
\subpoint{{\tt PURE\_REWRITE\_RULE} rewrites with only the\\
supplied list of theorems.}

\subpoint{{\tt REWRITE\_RULE} also rewrites with some standard
simplifications.}

\vskip7mm

\point{Example:}
\vskip4mm
\begin{session}\begin{verbatim}
#let th = ASSUME "!x:num.(P /\ (SUC 0=1)) ==> F";;
th = . |- !x. P /\ (SUC 0 = 1) ==> F

#let one = num_CONV "1";;
one = |- 1 = SUC 0

#PURE_REWRITE_RULE [one] th;;
. |- !x. P /\ (SUC 0 = SUC 0) ==> F

#REWRITE_RULE [one] th;;
. |- ~P
\end{verbatim}\end{session}

% =====================================================================
\slide{Standard Simplifications}

\point{{\tt REWRITE\_RULE} uses the simplifications:}

\vskip 4mm
\begin{session}\begin{verbatim}
#basic_rewrites;;
[|- !x. (x = x) = T;
 |- !t.
     ((T = t) = t) /\ ((t = T) = t) /\ 
     ((F = t) = ~t) /\ ((t = F) = ~t);
 |- (!t. ~~t = t) /\ (~T = F) /\ (~F = T);
 |- !t. (T /\ t = t) /\ (t /\ T = t) /\ 
        (F /\ t = F) /\ (t /\ F = F) /\ 
        (t /\ t = t);
 |- !t. (T \/ t = T) /\ (t \/ T = T) /\ 
        (F \/ t = t) /\ (t \/ F = t) /\ 
        (t \/ t = t);
 |- !t. (T ==> t = t) /\ (t ==> T = T) /\
     (F ==> t = T) /\ (t ==> t = T) /\
     (t ==> F = ~t);
 |- !t1 t2. ((T => t1 | t2) = t1) /\ 
            ((F => t1 | t2) = t2);
 |- !t. (!x. t) = t;
 |- !t. (?x. t) = t;
 |- !t1 t2. (\x. t1)t2 = t1;
 |- !x. FST x,SND x = x;
 |- !x y. FST(x,y) = x;
 |- !x y. SND(x,y) = y]
: thm list
\end{verbatim}\end{session}

\point{If you do not need these, {\tt PURE\_REWRITE\_RULE} is often much
faster.}

% =====================================================================
\slide{Other Rewriting Rules}

\point{Rewriting using the assumptions:}

\vskip7mm
\bspindent\Large\bf
\verb!ASM_REWRITE_RULE!
\vskip3mm
\verb!ASM_PURE_REWRITE_RULE!
\espindent
\vskip7mm
\bpindent \LARGE\bf
These add the assumptions of the theorem to be rewritten 
to the list of rewrites.
\epindent


\vskip7mm

\point{Rewriting only once:}

\vskip7mm
\bspindent\Large\bf
\verb!ONCE_REWRITE_RULE!
\vskip3mm
\verb!ONCE_ASM_REWRITE_RULE!
\vskip3mm
\verb!PURE_ONCE_REWRITE_RULE!
\vskip3mm
\verb!PURE_ONCE_ASM_REWRITE_RULE!
\espindent
\vskip7mm
\bpindent \LARGE\bf
These apply at most one rewrite rule to any subterm.
\epindent


% =====================================================================
\slide{Summary of Rewriting}

\point{Rewrite using standard simpifications:}

\subpoint{\tt REWRITE\_RULE}
\subpoint{\tt ASM\_REWRITE\_RULE}
\subpoint{\tt ONCE\_REWRITE\_RULE}
\subpoint{\tt ONCE\_ASM\_REWRITE\_RULE}

\point{Rewrite without standard simpifications:}

\subpoint{\tt PURE\_REWRITE\_RULE}
\subpoint{\tt PURE\_ASM\_REWRITE\_RULE}
\subpoint{\tt PURE\_ONCE\_REWRITE\_RULE}
\subpoint{\tt PURE\_ONCE\_ASM\_REWRITE\_RULE}

\point{Rewrite using assumptions:}

\subpoint{\tt ASM\_REWRITE\_RULE}
\subpoint{\tt PURE\_ASM\_REWRITE\_RULE}
\subpoint{\tt ONCE\_ASM\_REWRITE\_RULE}
\subpoint{\tt PURE\_ONCE\_ASM\_REWRITE\_RULE}

\end{document}



