% =====================================================================
% HOL Course Slides: overview of higher order logic   (c) T melham 1990
% =====================================================================

\documentstyle[12pt,layout]{article}

% ---------------------------------------------------------------------
% Preliminary settings.
% ---------------------------------------------------------------------

\renewcommand{\textfraction}{0.01}	  % 0.01 of the page must contain text
\setcounter{totalnumber}{10}	 	  % max of 10 figures per page
\flushbottom				  % text extends right to the bottom
\pagestyle{slides}			  % slides page style
\setlength{\unitlength}{1mm}		  % unit = 1 mm

% ---------------------------------------------------------------------
% load macros
% ---------------------------------------------------------------------
% =====================================================================
% Macros for HOL course slides 	                      (c) T Melham 1990
% =====================================================================

% ---------------------------------------------------------------------
% Macro for title page of a section:
%
%     \bsectitle
%       ...<title>...
%     \esectitle
%
% starts a new page, and typesets <title> in a large box at the top
% of the page.
% ---------------------------------------------------------------------

\newbox\stitle

\long\def\Frame#1{\leavevmode
    \hbox{\vbox{\hrule height2pt
              \hbox{\vrule width2pt #1\vrule width2pt}
  	      \hrule height2pt}}}

\long\def\bsectitle{\setbox\stitle=\hbox\bgroup\begin{minipage}{140mm}
		   \begin{center}\Huge\bf\vskip10mm}

\long\def\esectitle{\vskip10mm\end{center}\end{minipage}\egroup
		    \clearpage\vspace*{30mm}%
		   \begin{center}\Frame{\box\stitle}\end{center}}

% ---------------------------------------------------------------------
% Macro for title of slide:
%
%     \slide{<title>} 
%
% starts a new page, and typesets <title> in a large box at the top
% of the page.
% ---------------------------------------------------------------------

\long\def\slide#1{\newpage\leavevmode
 \hbox{\vbox{\hrule height2pt%
 \hbox to \textwidth{\vrule width2pt\hskip5mm%
 \vbox to 16mm{\vfill\vskip1mm\hbox{\Huge\bf #1}\vfill}\hfil\vrule width2pt}%
 \hrule height2pt}}\vskip7mm}

% ---------------------------------------------------------------------
% round circles: \blob and \subblob 
% ---------------------------------------------------------------------

\newcommand{\blob}{\begin{picture}(8.5,4)(0,-0.1)
		    \put(2.5,2){\circle*{3}} \end{picture}}

\newcommand{\subblob}{\begin{picture}(4.5,4)(0,0)
		    \put(1,1.5){\circle*{2}} \end{picture}}

% ---------------------------------------------------------------------
% Various lengths
%
% \blobwidth     = the width of a blob
% \subblobwidth  = the width of a sub-blob
% \pointwidth    = the text width of a major point
% \sind          = the standard indentation
% ---------------------------------------------------------------------

\newlength{\blobwidth}
\settowidth{\blobwidth}{\blob}

\newlength{\pointwidth}
\setlength{\pointwidth}{\textwidth}
\addtolength{\pointwidth}{-1\blobwidth}

\newlength{\subblobwidth}
\settowidth{\subblobwidth}{\subblob}

\newlength{\sind}
\setlength{\sind}{15mm}

% ---------------------------------------------------------------------
% \point : for making points
% ---------------------------------------------------------------------

\def\point#1{\vskip10mm\blob\parbox[t]{\pointwidth}{\LARGE\bf#1}\par}

% ---------------------------------------------------------------------
% \subpoint : for making sub-points
% ---------------------------------------------------------------------

\def\subpoint#1{{\addtolength{\pointwidth}{-\sind}
	      \addtolength{\pointwidth}{-1\subblobwidth}\vskip5mm
	      \hspace*{\sind}\subblob
	      \parbox[t]{\pointwidth}{\Large\bf#1}\par}}

% ---------------------------------------------------------------------
% \pindent : indent to the level of a point
% ---------------------------------------------------------------------

\def\bpindent{\bgroup\leftskip=\blobwidth}
\def\epindent{\par\egroup}

% ---------------------------------------------------------------------
% \pindent : indent to the level of a subpoint
% ---------------------------------------------------------------------

\def\bspindent{\bgroup\leftskip=\subblobwidth\addtolength{\leftskip}{\sind}}
\def\espindent{\par\egroup}

% ---------------------------------------------------------------------
% Macros for little HOL sessions displayed in boxes.
%
% Usage:     \begin{session}\begin{verbatim}
%	      .
%	       < lines from hol session >
%	      .
%	     \end{verbatim}\end{session}   
%
%            typesets the session in a box.
%
% Lengths: \hsbw = width of box, \lwid = width of lines
% ---------------------------------------------------------------------

\newlength{\lwid}
\setlength{\lwid}{0.8pt}

\newlength{\hsbw}
\setlength{\hsbw}{\textwidth}
\addtolength{\hsbw}{-\lwid}
\addtolength{\hsbw}{-\tabcolsep}
\addtolength{\hsbw}{-\blobwidth}
\addtolength{\hsbw}{-1mm}

\newenvironment{session}{\setlength{\arrayrulewidth}{\lwid}
 \begin{flushleft}\hskip\blobwidth\hskip1mm
 \begin{tabular}{@{}|c@{}|@{}}\hline 
 \begin{minipage}[b]{\hsbw}
 \vspace*{5.5pt}
 \begingroup\Large\baselineskip17.5pt}{\endgroup\end{minipage}\\[1.5pt]\hline 
 \end{tabular}
 \end{flushleft}}



% ---------------------------------------------------------------------
% Macro for a tricky underbrace
% ---------------------------------------------------------------------
\def\rul#1{\vrule width0.25mm height#1mm depth-1mm}
\newbox\ubox
\def\und#1#2{%
    \setbox\ubox=\hbox{${\stackrel{\rul{#2}}{\scriptstyle\hbox{\rm #1}}}$}%
    {\wd\ubox=0mm\box\ubox}}

% ---------------------------------------------------------------------
% set caption at the foot of pages for this series of slides
% ---------------------------------------------------------------------
\ftext{Syntax of the logic}{2}

% ---------------------------------------------------------------------
% Slides
% ---------------------------------------------------------------------
\begin{document}

% ---------------------------------------------------------------------
% Title page for this series of slides
% ---------------------------------------------------------------------

\bsectitle
Overview of\\
Higher Order Logic\\
{\vrule width20mm height2.5mm depth-2mm}\\
Syntax
\esectitle

% =====================================================================
\slide{Syntax of Terms}

\point{There are four kinds of terms:}
   \subpoint{constants}
   \subpoint{variables}
   \subpoint{function applications}
   \subpoint{lambda abstractions}

\vskip5mm

\point{Syntax of raw terms:}

\vskip7mm

\bspindent\LARGE
$T\quad {:}{:}{=}\quad
\underbrace{\mathstrut c}_{\und{constant}{10}} \quad | \quad
\underbrace{\mathstrut v}_{\und{variable}{20}} \quad | \quad
\underbrace{\mathstrut T_1 \;\;\; T_2}_{\und{application}{10}} \quad | \quad
\underbrace{\mathstrut \lambda v.\:T}_{\und{abstraction}{20}} $
\espindent

% =====================================================================
\slide{Constants and Variables}

\point{These are just identifiers:}

\vskip7mm

\bspindent\LARGE
$x$, $y$, $f\mkern-1.4mu oo$, $t'$, $k_2$, $c$\_$val$, \dots
\espindent

\vskip7mm

\bpindent
\LARGE\bf or certain special symbols:
\epindent

\vskip7mm

\bspindent\LARGE
$\exists$, $\forall$, $\supset$, $\wedge$, $\vee$, $\neg$, 
$1$, $2$, $3$, \dots, $+$, $\times$, $=$, \dots
\espindent

\vskip5mm

\point{The distinction between a constant and a \\variable depends on 
       the context.}

% =====================================================================
\slide{Function Applications}

\point{An application looks like:}

\vskip7mm

\bspindent\LARGE
$\langle {\hbox{\LARGE\it term}}_1 \rangle\;\;\;
 \langle {\hbox{\LARGE\it term}}_2\rangle$
\espindent

\vskip7mm

\bpindent\LARGE\bf 
and denotes the result of applying the function $\langle {\hbox{\LARGE\it
term}}_1 \rangle$ to the value $\langle {\hbox{\LARGE\it term}}_2\rangle$.
\epindent

\vskip4mm

\point{Parentheses can be used for grouping:}

\vskip7mm

\bspindent\LARGE
$f(x)$, $\;f\;(g \; y)$, $\;(f\; x) \; y$, \dots
\espindent

\vskip4mm

\point{Convention:}

\vskip7mm

\bspindent\LARGE
$f\;\:x_1\;\:x_2\;\:\cdots\;\:x_n\; = \;
(((f \;\: x_1) \;\: x_2 ) \;\: \cdots \;\: x_n)$
\espindent


% =====================================================================
\slide{Abstractions}

\point{An abstraction looks like:}

\vskip7mm

\bspindent\LARGE\bf
$\lambda \langle {\hbox{\LARGE\it var}} \rangle$ . 
$\langle {\hbox{\LARGE\it term}}\rangle$
\espindent

\vskip7mm

\bpindent\LARGE\bf 
and denotes the function: 
$x \mapsto \hbox{\LARGE\it term}[x / \hbox{\LARGE\it var}]$.
\epindent

\vskip4mm

\point{Examples:}

\vskip7mm

\bspindent\Large\bf
\begin{tabular}{@{}l@{\qquad}l@{}}
$ \lambda x. \; x$ & the identity function \\[2mm]
$ \lambda x. \; f(f\;x)$ & function that applies $f$ twice\\[2mm]
$ \lambda f. \lambda g. \lambda x. \; f(g\;x)$ & function composition
\end{tabular}
\espindent

\vskip4mm

\point{Convention:}

\vskip7mm

\bspindent\LARGE
$\lambda x_1\;x_2\;\cdots\;x_n\:.\:t \; = \;
\lambda x_1\:.\:\lambda x_2\:.\:\; \cdots \;\:
\lambda x_n\:.\:t$
\espindent


% =====================================================================
\slide{Free and Bound Variables}

\point{A variable $x$ is {\it free} in a term if it does not
occur inside the body of an abstraction:}

\vskip7mm

\bspindent\LARGE\(
\lambda x. \langle{\it body}\rangle
\)
\espindent

\vskip7mm

\point{If an instance of a varible is not free, it is {\it bound}.}

\vskip7mm


\point{Thus, in the term:}

\begin{center}\LARGE
\begin{picture}(20,30)\thicklines
 \put(10,19){\makebox(0,0)[t]{$(\lambda x.\: f\; x)(\lambda y.\: x)$}}
 \put(8.5,0){\vector(0,1){10}}
 \put(8.5,-1){\makebox(0,0)[t]{\Large bound}}
 \put(30,0){\vector(0,1){10}}
 \put(30,-1){\makebox(0,0)[t]{\Large free}}
\end{picture}\end{center}


% =====================================================================
\slide{Summary of Terms}

\point{Terms may be :}

\vskip 7mm 
\subpoint{Variables:
$x$,~~$y$,~~$a'$,~~$a\_var$,~~$phi_1$, \dots}


\subpoint{Constants:
$T$,~~$F$,~~$phi$,~~$\exists$,~~$+$, \dots}


\subpoint{Applications:
$t_1\;t_2$,~~$t_1\;t_2\;t_3\;\ldots \;t_n$}


\subpoint{Abstractions:
$\lambda x.\:t$,~~$\lambda\;x_1\;x_2\;\ldots\;x_n.\: t$}

\vskip7mm

\point{Free and bound variables:}

\begin{center}\LARGE
\begin{picture}(20,30)(30,0)\thicklines
 \put(10,19){\makebox(0,0)[t]{$(\lambda x.\: f\; x)(\lambda y.\: x)$}}
 \put(8.5,0){\vector(0,1){10}}
 \put(8.5,-1){\makebox(0,0)[t]{\Large bound}}
 \put(30,0){\vector(0,1){10}}
 \put(30,-1){\makebox(0,0)[t]{\Large free}}
\end{picture}\end{center}


% =====================================================================
\slide{Higher Order Terms}

\point{Variables can range over functions or \\
predicates (i.e.\ boolean-valued functions).}

\vskip7mm

\subpoint{in $\lambda f. f \; 0$, the variable $f$ ranges over functions}

\subpoint{in $\forall P.\:P(n) \supset P(n{+}1)$, $P$ ranges over predicates}

\vskip5mm

\point{\bf A typical higher-order assertion:}

\vskip7mm
\bspindent\LARGE\bf
$ \forall x f.\,\exists g.\:(g\; 0 = x) \wedge 
   \forall n.\,g\:(n{+}1) = (f\;(g\;n))\;n$
\espindent



% =====================================================================
\slide{Problem: Russell's Paradox}

\point{Having variables that range over predicates\\
allows us write terms like:}

\vskip 7mm
\bspindent\LARGE
\( \Omega \stackrel{def}{=}\lambda P.\: \neg(P\; P) \)
\espindent
\vskip 7mm

\bpindent\LARGE
{\bf where $P$ is a variable.}
\espindent
\vskip 7mm

\point{But by $\beta$-reduction, we have:}

\vskip 7mm
\bspindent\LARGE
\( \Omega \;\Omega = (\lambda P.\: \neg(P\; P))\;\Omega = \neg(\Omega\;\Omega)
\)
\espindent
\vskip 7mm

\bpindent\LARGE
{\bf which is nonsense (Russell's paradox)}
\epindent

\vskip 7mm

\point{To avoid this kind of thing we use {\it types}.}


% =====================================================================
\slide{Types}

\point{Syntax of types:}

\vskip7mm

\bspindent\LARGE
$\sigma\quad {:}{:}{=}\quad
\underbrace{\mathstrut c}_{\und{type constant}{10}} \quad | \quad
\underbrace{\mathstrut \alpha}_{\und{type variable}{20}} \quad | \quad
\underbrace{\mathstrut (\sigma_1,\dots,\sigma_n)op}_{\und{compound type}{10}}$
\espindent

\vskip7mm

\point{Examples of type constants:}

\vskip 7mm
\bspindent{\Large{\bf
\begin{tabular}{l l}
$bool$ & booleans\\
$num$  & natural numbers\\
$ind$  & individuals
\end{tabular}}}
\espindent
\vskip 7mm

\point{Examples of compound types:}

\vskip 7mm
\bspindent\Large{\bf
\begin{tabular}{l @{\qquad} l}
\( (\sigma_1,\sigma_2)fun \)  & functions from $\sigma_1$ to $\sigma_2$\\[2mm]
\( (\sigma_1,\sigma_2)prod \) & pairs of values\\
\end{tabular}}
\espindent
\vskip 7mm


\point{Type variables: $\alpha$, $\beta$, $\gamma$, \dots}

% =====================================================================
\slide{Terminology and Notation}

\point{Type operators:}

\subpoint{the `$op$' in $(\sigma_1,\dots,\sigma_n)op$ is called a {\it type
operator}}

\vskip 7mm

\point{Notation:}

\subpoint{The type $(\sigma_1,\sigma_2)fun$ is usually written
 \( \sigma_1 \rightarrow \sigma_2\)}

\subpoint{The type $(\sigma_1,\sigma_2)prod$ is usually written
 \( \sigma_1 \times \sigma_2\)}

\vskip 7mm

\point{Conventions:}

\vskip5mm

\bspindent\LARGE
$ \sigma_1 \rightarrow \sigma_2 \rightarrow \dots
\rightarrow \sigma_n = 
(\sigma_1 \rightarrow (\sigma_2 \rightarrow
(\dots \rightarrow \sigma_n))) $
\espindent

\vskip5mm


\bspindent\LARGE
$\sigma_1 \times \sigma_2 \times \dots
\times \sigma_n = 
(\sigma_1 \times (\sigma_2 \times 
(\dots \times \sigma_n))) $
\espindent


% =====================================================================
\slide{Typing of Terms}

\point{All terms must be well-typed.}

\vskip5mm

\point{Notation: `$t{:}\sigma$' means the term
$t$ is well-typed and has type $\sigma$.}
\vskip7mm

\point{Variables and constants:}

\vskip4mm

\subpoint{Variables may have any type: $v{:}\sigma$}

\subpoint{Constants have a fixed {\it generic\/} type: $c{:}\sigma$}


\vskip7mm

\point{Rules for assigning types to other terms:}

\vskip 7mm
\bspindent\LARGE{
{\large\bf App:}%
\begin{tabular}{@{\quad}c@{}} 
\( t_1{:}\sigma_1 \rightarrow \sigma_2 \qquad t_2{:}\sigma_1 \)\\[-4mm]
\vbox{\hrule width70mm height0.3mm}\\
\( (t_1 \; t_2){:}\sigma_2 \) \\
\end{tabular} }
\espindent

\vskip 7mm

\vskip 4mm

\bspindent\LARGE{
{\large\bf Abs:}%
\begin{tabular}{ @{\quad} c @{}} 
\( x{:}\sigma_1 \qquad t{:}\sigma_2 \)\\[-4mm] 
\vbox{\hrule width70mm height0.3mm}\\
\( (\lambda x. t){:}\sigma_1 \rightarrow \sigma_2 \) 
\end{tabular} }
\espindent



% =====================================================================
\slide{Examples}

\point{Some well-typed terms:}

\vskip 7mm
\bspindent\LARGE
\begin{tabular}{l}
\( T:bool\)\\[2mm]
\( \lambda x.\: x{+}23:num {\rightarrow} num\)\\[2mm]
\( (\lambda x.\:x{+}23)\;7: num\)\\[2mm]
\( (\lambda x.\:\lambda y.\:x{+}y):
   num {\rightarrow} num {\rightarrow} num\) \\[2mm]
\( (\lambda x.\:x): \alpha {\rightarrow} \alpha \)
\end{tabular}
\espindent
\vskip 7mm

% =====================================================================
\slide{Typing Avoids Meaningless Terms}

\point{Suppose $(\lambda P.\: \neg(P\; P))$ is well-typed.}

\vskip7mm

\point{Then, by the typing rules:}

\vskip4mm

\bspindent\LARGE\bf
$ (P{:}\sigma{\rightarrow}bool\;\;P{:}\sigma){:}bool$
\espindent
\vskip4mm

\bpindent
\LARGE\bf
for some type $\sigma$.
\epindent

\vskip 7mm
\point{That is, there are two {\it different\/} variables:}

\vskip4mm

\bspindent\LARGE\bf
$ P{:}\sigma{\rightarrow}bool$ and $P{:}\sigma$
\espindent

\vskip 7mm

\point{Furthermore, if \(\Omega = \lambda P.\: \neg(P \; P)\) then}

\vskip 7mm
\bspindent\LARGE
$ (\Omega{:}\sigma_1{\rightarrow}\sigma_2) \;\;\Omega{:}\sigma_1 $
\espindent

\vskip 7mm
\bpindent\LARGE\bf
is impossible, since $\sigma_1{\rightarrow}\sigma_2 \neq \sigma_1$.
\epindent

% =====================================================================
\slide{Polymorphism}


\point{Consider the constant $I$, defined by:}

\vskip 7mm
\bspindent\LARGE 
$I\stackrel{def}{=}\lambda x.\:x  $
\espindent
\vskip 7mm

\point{We may want to apply the function $I$ to things of different types:}

\vskip 7mm
\bspindent\LARGE
$\begin{array}{l @{\quad} l  @{\quad} l}
I \; 7 = 7 & {\bf with} & I : num \rightarrow num \\
I \; T = T & {\bf with} & I : bool \rightarrow bool \\
\end{array} $
\espindent
\vskip 7mm

\point{It seems that $I$ must have two different types.}


% =====================================================================
\slide{Polymorphism and Generic Types}

\point{The types of polymorphic functions such as $I$ contain type
variables:}

\vskip5mm

\bspindent\LARGE
$I\; \stackrel{def}{=}\; (\lambda x.\: x){:} \alpha{\rightarrow} \alpha$
\espindent

\vskip 5mm
\bpindent\LARGE{\bf
where $\alpha$ stands for `any type'.}
\epindent
\vskip 7mm

\point{Here, $\alpha{\rightarrow}\alpha$ is the {\it generic\/} type of $I$.}

\vskip 7mm

\point{The constant $I$ then has every type obtainable by substituting
any type for the variable $\alpha$ in its generic type:}

\vskip5mm
\bspindent\LARGE
\[ \begin{array}{l}
I : bool \rightarrow bool\\[2mm]
I : num \rightarrow num\\[2mm]
I : (\alpha \rightarrow bool) \rightarrow (\alpha \rightarrow bool)\\[2mm]
I :\alpha  \rightarrow \alpha
\end{array} \]
\espindent

\vskip5mm

\bpindent\LARGE{\bf
and so `$I \; 7$' and `$I \; T$' are both well-typed.}
\epindent


% =====================================================================
\slide{Polymorphism: Examples}

\point{Definition of function composition:}

\vskip5mm

\bspindent\LARGE
\(o \;\stackrel{def}{=}\; \lambda f. \lambda g. \lambda x. f(g(x))\)
\espindent
\vskip5mm

\bpindent\LARGE\bf
where \( o:(\beta \rightarrow \gamma) \rightarrow (\alpha \rightarrow \beta)
\rightarrow (\alpha \rightarrow \gamma) \)
\epindent


\vskip 7mm

\point{Equality is also polymorphic:}

\vskip 7mm
\bspindent\LARGE
\( = \; : \; \alpha \rightarrow \alpha \rightarrow bool \)
\espindent

\vskip 7mm

\point{Apply a function and add:}


\vskip 7mm
\bspindent\LARGE
\( app\_add \; \stackrel{def}{=} \; \lambda f. (\lambda x. f(x) + f(x)) \)
\espindent


\vskip 7mm
\bspindent\LARGE
\( app\_add : (\alpha \rightarrow num) \rightarrow (\alpha \rightarrow num) \)
\espindent


% =====================================================================
\slide{Summary of Types}

\point{Types are used to avoid inconsistency.}

\vskip5mm

\point{Type constants: $bool$, $num$, \dots}

\vskip5mm

\point{Type variables: $\alpha$, $\beta$, $\gamma$, \dots}

\vskip5mm

\point{Compound Types: $(\sigma_1,\dots,\sigma_n)op$}

\subpoint{e.g. $\sigma_1 \rightarrow \sigma_2$, and 
$\sigma_1 \times \sigma_2$.}

\vskip5mm

\point{Well-typed terms:}

\subpoint{
\( (\lambda x. t):\sigma_1 \rightarrow \sigma_2\)
   when \( x:\sigma_1\) and \(t:\sigma_2 \)}
\subpoint{\( (t_1 \; t_2):\sigma\)
when \(t_1:\sigma_1 \rightarrow \sigma\) and \(t_2:\sigma_1\)}

\vskip5mm

\point{Polymorphism:}
\vskip5mm
\bspindent\LARGE 
\( twice \;\stackrel{def}{=}\; \lambda f. \lambda x. f(f(x)) \) \\
\( twice : (\alpha \rightarrow \alpha) \rightarrow (\alpha \rightarrow \alpha)\)
\espindent


% =====================================================================
\slide{Syntactic Sugar: Infix Applications}
				       
\point{Certain constants are written in infix position:}

\vskip7mm

\bspindent\LARGE{\bf
\begin{tabular}{@{}l@{\qquad}l@{\qquad}l}
\(t_1 + t_2\) & abbreviates & \( + \; t_1 \; t_2 \)\\
\(t_1 \times t_2\) & abbreviates &\( \times \; t_1 \; t_2 \) \\
\(t_1 \wedge t_2\) &  abbreviates &\( \wedge \; t_1 \; t_2 \) 
\end{tabular}}
\espindent

\vskip7mm

\point{HOL convention: A constant's infix status is supressed by
prefixing it with `$\$$'}

\vskip7mm
\bspindent\LARGE{\bf
\begin{tabular}{@{}l@{\qquad}l@{\qquad}l}
\( \$+ \; t_1 \; t_2 \) & means & \( t_1 + t_2 \) \\
\( \$\vee \; t_1 \; t_2\) & means & \( t_1 \vee t_2 \) \\
\end{tabular}}
\espindent

% =====================================================================
\slide{Syntactic Sugar: Binders}

\point{The quantifiers $\forall$ and $\exists$
are in fact polymorphic constants with types:}

\vskip 7mm
\bspindent\LARGE 
\( \forall : (\alpha \rightarrow bool) \rightarrow bool \)
\espindent
\vskip 5mm
\bspindent\LARGE 
\( \exists : (\alpha \rightarrow bool) \rightarrow bool \)
\espindent

\vskip5mm

\point{They are {\it defined\/} such that for $P:(\alpha \rightarrow bool)$:}

\vskip 7mm
\bspindent\LARGE 
{\bf $\forall P$ means $P(x)=T$ for all $x$}
\espindent

\vskip 7mm
\bspindent\LARGE 
{\bf $\exists P$ means $P(x)=T$ for some $x$}
\espindent

% =====================================================================
\slide{Binders}

\point{We can use $\forall$ and $\exists$ as follows}

\vskip 5mm
\bspindent\LARGE\bf\(\forall(\lambda x. \; 0 \leq x)\)~~~~and~~~~
\(\exists(\lambda x. \; x = 2+3) \)
\espindent

\vskip 5mm

\bpindent\LARGE{\bf but it is more conventional to write}
\epindent

\vskip 5mm
\bspindent\LARGE{\bf
\(\forall x. \; 0 \leq x\)~~~~and~~~~\(\exists x. \; x = 2+3\)}
\espindent

\vskip7mm

\point{We simply make:}

\vskip5mm

\bspindent\LARGE{\bf
\begin{tabular}{@{}l@{\qquad}l@{\qquad}l}
\(\forall x.\:t \) & abbreviate & \( \forall\;(\lambda x.\:t)\)\\
\(\exists x.\:t \) & abbreviate & \( \exists\;(\lambda x.\:t)\)
\end{tabular}}
\espindent

\vskip7mm

\point{Constants like $\forall$ and $\exists$ are called {\it binders}.}

\vskip5mm

\point{Any constant of type \( (\sigma_1 \rightarrow \sigma_2) \rightarrow
\sigma_3 \) can be treated as a binder.}

\vskip5mm

\point{HOL convention: binder status is supressed by the prefix `\(\$\)', as in
`\(\$\exists\)'}

% =====================================================================
\slide{Pairs}

\point{The symbol `$,$' is an infix constant of type:}
\vskip 5mm
\bspindent\LARGE
$\alpha \rightarrow \beta \rightarrow (\alpha \times \beta)$
\espindent
\vskip 5mm
\bpindent\LARGE\bf
where $\alpha \times \beta$ is the type of pairs $(a,b)$ for which
$a{:}\alpha$ and $b{:}\beta$.
\epindent

\vskip7mm

\point{Constants for selecting components:}
\vskip5mm
\bspindent\LARGE
$\begin{array}{@{}l}
\forall a\:b.\:FST(a,b) = a\\
\forall a\:b.\:SND(a,b) = a\\
\end{array} $
\espindent
\vskip5mm
\bpindent\LARGE\bf where\epindent
\vskip5mm
\bspindent\LARGE
$ \begin{array}{@{}l}
FST : (\alpha \times \beta) \rightarrow \alpha \\[2mm]
SND : (\alpha \times \beta) \rightarrow \beta \end{array}$
\espindent
\vskip7mm


\point{Convention:}

\vskip5mm

\bspindent\LARGE
$(x_1,x_2,\dots,x_n)\; = \;
(x_1,(x_2,(\dots,x_n)))$
\espindent
		


% =====================================================================
\slide{Lists}

\point{The type $\alpha \; list$ is the type of finite lists of values of 
type $\alpha$.}

\vskip7mm

\point{Constants:}
\vskip 5mm
\bspindent\LARGE
\begin{tabular}{l}
\( HD : \alpha \; list \rightarrow \alpha \)\\[2mm]
\( TL : \alpha \; list \rightarrow \alpha \; list \) \\[2mm]
\( CONS : \alpha \rightarrow \alpha \; list \rightarrow \alpha \; list\) \\[2mm]
\( NULL : \alpha \; list \rightarrow bool\)
\( NIL : \alpha \; list \)
\end{tabular}
\espindent

\point{Syntatic sugar:}
\vskip 5mm
\bspindent\LARGE
\begin{tabular}{l}
{\em [1;2;3] : num list}\\[2mm]
{\em [T;F;T;F] : bool list}\\[2mm]
{\em [] : $\alpha$ list}
\end{tabular}
\espindent\vskip5mm
\bpindent\LARGE\bf
where:
\epindent
\vskip5mm

\bspindent\LARGE
 $[x_1;\dots;x_n]\;=\; CONS\;x_1\;(\dots (CONS\;x_n\;NIL))$
\espindent


% =====================================================================
\slide{Conditionals}

\point{The expression:}
\vskip5mm
\bspindent\LARGE
{\( (t \Rightarrow t_1 \mid t_2) \)}
\espindent
\vskip5mm
\bpindent\LARGE\bf means\epindent
\vskip5mm
\bspindent\LARGE\bf
`if $t$ then $t_1$ else $t_2$'
\espindent
\vskip5mm
\bpindent\LARGE\bf and is an abbreviation for\epindent
\vskip5mm
\bspindent\LARGE\bf
$COND\;\;t\;\;t_1\;\;t_2$
\espindent
\vskip7mm

\point{The generic type of the constant $COND$ is}
\vskip 5mm
\bspindent\LARGE
{$COND:bool \rightarrow \alpha \rightarrow \alpha \rightarrow \alpha$}
\espindent


% =====================================================================
\slide{Summary of Special Syntax}

\point{Infixes:}
\vskip5mm
\bspindent\LARGE\bf
E.g.\ \( b_1 \vee b_2 \;=\; \$\vee b_1 b_2 \)
\espindent

\point{Binders:}
\vskip5mm
\bspindent\LARGE\bf
E.g.\ \( \exists x.t  \;=\;  \$\exists(\lambda x.t) \)
\espindent

\point{Pairs and tuples:}
\vskip5mm
\bspindent\LARGE\bf
$(t_1, t_2): \sigma_1 \times \sigma_2$, where $t_1{:}\sigma_1$ and
$t_2{:}\sigma_2$
\espindent

\vskip 5mm

\bspindent\LARGE
$(t_1, t_2, t_3,\ldots) \;=\; (t_1, (t_2, (t_3, \ldots )))$
\espindent

\point{Lists:}
\vskip 5mm
\bspindent\LARGE
\begin{tabular}{@{}l}
{\em [1;2;3] : num list}\\[2mm]
{\em [T;F;T;F] : bool list}\\[2mm]
{\em [] : $\alpha$ list}
\end{tabular}
\espindent

\point{Conditionals: \((t \Rightarrow t_1 \mid t_2)\) }

% =====================================================================
\slide{The $\epsilon$ Operator}

\point{The primitive constant $\epsilon$ is a binder of type}
\vskip 7mm
\bspindent\LARGE
\((\alpha \rightarrow bool) \rightarrow \alpha \)
\espindent

\point{If $x$ has type $\sigma$ and $t[x]$ is a term involving (free
occurrences of) $x$ then:}

\vskip 7mm
\bspindent\LARGE
\( \epsilon x. \; t[x] \)
\espindent
\vskip 7mm

\bpindent\LARGE{\bf denotes some value of type $\sigma$, $v{:}\sigma$ say, such
that $t[v]$ is true} 
\epindent \vskip 7mm

\point{If no such value exists, then \( \epsilon x. \; t[x] \) denotes an
unspecified but fixed value of type $\sigma$}


% =====================================================================
\slide{Examples of $\epsilon$-terms}

\point{This term denotes the number 1:}
\vskip5mm
\bspindent\LARGE
$\epsilon x.\: 0<x \wedge x<2 \;\;$
\espindent
\vskip7mm

\point{This term denotes an even number:}
\vskip5mm
\bspindent\LARGE
$\epsilon x.\: \exists y.\: x = 2 \times y \;\;$
\espindent
\vskip7mm

\point{An unspecified natural number:}
\vskip5mm
\bspindent\LARGE
$\epsilon x.\: x+1 = x \;\;$
\espindent
\vskip7mm

\point{The following proposition is true:}
\vskip5mm
\bspindent\LARGE
$(\epsilon x.\: x+3 = 9) = 6 \;\;$
\espindent


% =====================================================================
\slide{Formulas}

\point{Formulas in HOL are just terms of type $bool$:}
\vskip 5mm
\subpoint{\( \forall x.\: x=0 \vee \neg(x=0) \)}
\subpoint{\( T \)}
\subpoint{\( (\lambda x. \; \neg\; x) (\forall y. \; y=y) \)}
\subpoint{\( \forall x.\; x=T \)}
\subpoint{\( \$\forall(\lambda x. \; x=T) \)}


% =====================================================================
\slide{Sequents}

\point{A sequent has the form:}

\vskip 7mm
\bspindent\LARGE \( ( \Gamma, t) \)
\espindent
\vskip 5mm
\bpindent\LARGE\bf where\epindent\vskip5mm

\subpoint{$\Gamma$ is a set of formulas (assumptions) and}
\subpoint{$t$ is a formula (conclusion)}

\vskip7mm
\point{A sequent `\( ( \Gamma, t) \)' means:}
\vskip5mm
\bspindent\Large\bf
{`if every formula in $\Gamma$ is true then
so is $t$'}
\espindent

\point{Example:}
\vskip 5mm
\bspindent\LARGE{\bf
$($ \{ $x=3,\;  \forall n.\: n=n$ \}, $x=99$ $)$
}\espindent
\vskip 5mm
\bpindent\LARGE\bf is a sequent which means\epindent
\vskip 5mm
\bspindent\LARGE{\bf
\( \forall x.\: (x=3 \wedge \forall n.\:n=n) \; \supset \; x=99 \)}
\espindent


% =====================================================================
\slide{Theorems}

\point{A {\it theorem\/} is a sequent that is either}
\subpoint{an axiom, or}
\subpoint{follows from other theorems by an inference rule}

\vskip7mm
\point{For theorems we write}
\vskip 5mm
\bspindent\LARGE
\( \Gamma \vdash t \)
\espindent
\vskip 5mm
\bpindent\LARGE
{\bf or just \( \; \vdash t \; \) if $\; \Gamma \; $ is empty}
\epindent
\vskip7mm
\point{Axioms are just sequents that are postulated to be theorems.}


% =====================================================================
\slide{Constant Definitions}

\point{We introduce axioms to define new constants.}
\vskip7mm

\point{Rule of constant definition:}
\vskip5mm
\bspindent\Large\bf
{\Large\bf If}\\[2mm]
\mbox{}\hskip10mm $\bullet$ $c$ is not already a constant, and\\
\mbox{}\hskip10mm  $\bullet$ $t$ has no free variables, and\\
\mbox{}\hskip10mm  $\bullet$ $t$ does not contain $c$, and\\
\mbox{}\hskip10mm  $\bullet$ all the type variables in $t$ also appear in\\
\mbox{}\hskip10mm  $\phantom\bullet$\ the type of $c$ \\[2mm]
{\Large\bf then we can}\\[2mm]
\mbox{}\hskip10mm  $\bullet$ make $c$ a constant, and \\
\mbox{}\hskip10mm  $\bullet$ take as an axiom \(\vdash c = t\)
\espindent
\vskip5mm

\point{$c$ is just an abbreviation for the term $t$.}

\vskip5mm

\point{Examples:}

\vskip 5mm
\bspindent\LARGE{
\(\vdash o \; = \; \lambda f \; g\; x. \; f(g(x))\)
\vskip 5mm
\(\vdash Onto = \lambda f. \; (\forall y. \; \exists x. \;\; y = f(x))\)
\vskip 5mm
\(\vdash I \; = \; \lambda x. x\)}
\espindent

% =====================================================================
\slide{Derived Definitions}

\point{A definition of the form:}

\vskip 5mm

\bspindent\LARGE{\bf
\( \vdash c = \lambda x_1 \; \dots \; x_n.\:\; t \)}
\espindent
\vskip 5mm
\bpindent\LARGE\bf
is equivalent to
\epindent
\vskip 5mm
\bspindent\LARGE
\( \vdash c \;\; x_1 \; \dots \; x_3 \; = \; t\) 
\espindent
\vskip 5mm
\bpindent\LARGE\bf
which therefore  also counts as a definition.
\epindent


\vskip7mm
\point{Derived definitions also include, for example:}

\vskip 5mm

\bspindent\LARGE
\( \vdash c \;\; (v_1,v_2)\;= \; t[v_1,v_2]\) 
\espindent
\vskip 5mm
\bpindent\LARGE\bf
which is equivalent to
\epindent
\vskip 5mm
\bspindent\LARGE{\bf
\( \vdash c = \lambda p.\: t[FST\; p,SND\; p] \)}
\espindent




% =====================================================================
\slide{Theories}

\point{A theory is a collection of}

\subpoint{type constants}
\subpoint{type operators}
\subpoint{constants}
\subpoint{constant definitions}
\subpoint{type definitions (discussed later)}
\subpoint{axioms}
\subpoint{theorems}

\vskip 7mm 

\point{A theory is called {\it defined\/} if it contains no \\
axioms, other than constant definitions\\ and type definitions.}

\vskip 7mm 

\point{All non-primitive HOL theories are defined!}

% =====================================================================
\slide{Theory Hierarchies}

\point{Theories can have other theories as parents}

\vskip 7mm
\vskip 7mm

\begin{center}
\setlength{\unitlength}{0.0125in}%
\begin{picture}(309,262)(190,460)
\thicklines
\put(474,477){\vector(-2, 3){ 64.615}}
\put(330,475){\vector( 2, 3){ 64.615}}
\put(183,475){\vector( 2, 3){ 64.615}}
\put(327,477){\vector(-2, 3){ 64.615}}
\put(399,602){\vector(-2, 3){ 64.615}}
\put(255,600){\vector( 2, 3){ 64.615}}
\put(450,460){\makebox(0,0)[lb]{\raisebox{0pt}[0pt][0pt]{\elvrm{\Large theory6}}}}
\put(305,460){\makebox(0,0)[lb]{\raisebox{0pt}[0pt][0pt]{\elvrm{\Large theory5}}}}
\put(160,460){\makebox(0,0)[lb]{\raisebox{0pt}[0pt][0pt]{\elvrm{\Large theory4}}}}
\put(375,580){\makebox(0,0)[lb]{\raisebox{0pt}[0pt][0pt]{\elvrm{\Large theory3}}}}
\put(230,580){\makebox(0,0)[lb]{\raisebox{0pt}[0pt][0pt]{\elvrm{\Large theory2}}}}
\put(300,710){\makebox(0,0)[lb]{\raisebox{0pt}[0pt][0pt]{\elvrm{\Large theory1}}}}
\end{picture}
\end{center}
\vskip 7mm

\point{Everything in the parent theory is also in the child or descendent
theories.}

\point{Loops in the parent graph are not allowed.}

% =====================================================================
\slide{Summary}

\point{The $\epsilon$-operator: $\epsilon x.\:t[x]$}

\point{Formulas: are just boolean terms}

\point{Sequents: $(\Gamma, t)$}

\point{Theorems: $\Gamma \vdash t$ and $\vdash t$}

\point{Constant definitions:}
\vskip 5mm
\bspindent\LARGE{\bf
\( \vdash c = t \)}
\espindent
\vskip 4mm
\bspindent\LARGE{\bf
\( \vdash c = \lambda x_1 \; \dots \; x_n.\:\; t \)}
\espindent
\vskip 4mm
\bspindent\LARGE
\( \vdash c \;\; x_1 \; \dots \; x_3 \; = \; t\) 
\espindent
\vskip 4mm
\bspindent\LARGE
\( \vdash c \;\; (v_1,v_2)\;= \; t[v_1,v_2]\) 
\espindent

% =====================================================================
\slide{Summary Continued}


\point{Theories:}

\subpoint{type constants}
\subpoint{type operators}
\subpoint{constants}
\subpoint{constant definitions}
\subpoint{type definitions (discussed later)}
\subpoint{axioms}
\subpoint{theorems}




\end{document}
