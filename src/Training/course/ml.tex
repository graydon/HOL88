% =====================================================================
% HOL Course Slides: introduction to ML		      (c) T melham 1990
% =====================================================================

\documentstyle[alltt,12pt,layout]{article}

% ---------------------------------------------------------------------
% Preliminary settings.
% ---------------------------------------------------------------------

\renewcommand{\textfraction}{0.01}	  % 0.01 of the page must contain text
\setcounter{totalnumber}{10}	 	  % max of 10 figures per page
\flushbottom				  % text extends right to the bottom
\pagestyle{slides}			  % slides page style
\setlength{\unitlength}{1mm}		  % unit = 1 mm

% ---------------------------------------------------------------------
% load macros
% ---------------------------------------------------------------------
% =====================================================================
% Macros for HOL course slides 	                      (c) T Melham 1990
% =====================================================================

% ---------------------------------------------------------------------
% Macro for title page of a section:
%
%     \bsectitle
%       ...<title>...
%     \esectitle
%
% starts a new page, and typesets <title> in a large box at the top
% of the page.
% ---------------------------------------------------------------------

\newbox\stitle

\long\def\Frame#1{\leavevmode
    \hbox{\vbox{\hrule height2pt
              \hbox{\vrule width2pt #1\vrule width2pt}
  	      \hrule height2pt}}}

\long\def\bsectitle{\setbox\stitle=\hbox\bgroup\begin{minipage}{140mm}
		   \begin{center}\Huge\bf\vskip10mm}

\long\def\esectitle{\vskip10mm\end{center}\end{minipage}\egroup
		    \clearpage\vspace*{30mm}%
		   \begin{center}\Frame{\box\stitle}\end{center}}

% ---------------------------------------------------------------------
% Macro for title of slide:
%
%     \slide{<title>} 
%
% starts a new page, and typesets <title> in a large box at the top
% of the page.
% ---------------------------------------------------------------------

\long\def\slide#1{\newpage\leavevmode
 \hbox{\vbox{\hrule height2pt%
 \hbox to \textwidth{\vrule width2pt\hskip5mm%
 \vbox to 16mm{\vfill\vskip1mm\hbox{\Huge\bf #1}\vfill}\hfil\vrule width2pt}%
 \hrule height2pt}}\vskip7mm}

% ---------------------------------------------------------------------
% round circles: \blob and \subblob 
% ---------------------------------------------------------------------

\newcommand{\blob}{\begin{picture}(8.5,4)(0,-0.1)
		    \put(2.5,2){\circle*{3}} \end{picture}}

\newcommand{\subblob}{\begin{picture}(4.5,4)(0,0)
		    \put(1,1.5){\circle*{2}} \end{picture}}

% ---------------------------------------------------------------------
% Various lengths
%
% \blobwidth     = the width of a blob
% \subblobwidth  = the width of a sub-blob
% \pointwidth    = the text width of a major point
% \sind          = the standard indentation
% ---------------------------------------------------------------------

\newlength{\blobwidth}
\settowidth{\blobwidth}{\blob}

\newlength{\pointwidth}
\setlength{\pointwidth}{\textwidth}
\addtolength{\pointwidth}{-1\blobwidth}

\newlength{\subblobwidth}
\settowidth{\subblobwidth}{\subblob}

\newlength{\sind}
\setlength{\sind}{15mm}

% ---------------------------------------------------------------------
% \point : for making points
% ---------------------------------------------------------------------

\def\point#1{\vskip10mm\blob\parbox[t]{\pointwidth}{\LARGE\bf#1}\par}

% ---------------------------------------------------------------------
% \subpoint : for making sub-points
% ---------------------------------------------------------------------

\def\subpoint#1{{\addtolength{\pointwidth}{-\sind}
	      \addtolength{\pointwidth}{-1\subblobwidth}\vskip5mm
	      \hspace*{\sind}\subblob
	      \parbox[t]{\pointwidth}{\Large\bf#1}\par}}

% ---------------------------------------------------------------------
% \pindent : indent to the level of a point
% ---------------------------------------------------------------------

\def\bpindent{\bgroup\leftskip=\blobwidth}
\def\epindent{\par\egroup}

% ---------------------------------------------------------------------
% \pindent : indent to the level of a subpoint
% ---------------------------------------------------------------------

\def\bspindent{\bgroup\leftskip=\subblobwidth\addtolength{\leftskip}{\sind}}
\def\espindent{\par\egroup}

% ---------------------------------------------------------------------
% Macros for little HOL sessions displayed in boxes.
%
% Usage:     \begin{session}\begin{verbatim}
%	      .
%	       < lines from hol session >
%	      .
%	     \end{verbatim}\end{session}   
%
%            typesets the session in a box.
%
% Lengths: \hsbw = width of box, \lwid = width of lines
% ---------------------------------------------------------------------

\newlength{\lwid}
\setlength{\lwid}{0.8pt}

\newlength{\hsbw}
\setlength{\hsbw}{\textwidth}
\addtolength{\hsbw}{-\lwid}
\addtolength{\hsbw}{-\tabcolsep}
\addtolength{\hsbw}{-\blobwidth}
\addtolength{\hsbw}{-1mm}

\newenvironment{session}{\setlength{\arrayrulewidth}{\lwid}
 \begin{flushleft}\hskip\blobwidth\hskip1mm
 \begin{tabular}{@{}|c@{}|@{}}\hline 
 \begin{minipage}[b]{\hsbw}
 \vspace*{5.5pt}
 \begingroup\Large\baselineskip17.5pt}{\endgroup\end{minipage}\\[1.5pt]\hline 
 \end{tabular}
 \end{flushleft}}


% ---------------------------------------------------------------------
% set caption at the foot of pages for this series of slides
% ---------------------------------------------------------------------
\ftext{Introduction to ML}{4}

% ---------------------------------------------------------------------
% Slides
% ---------------------------------------------------------------------
\begin{document}

% ---------------------------------------------------------------------
% Title page for this series of slides
% ---------------------------------------------------------------------

\bsectitle
A Brief\\
Introduction to ML
\esectitle

% =====================================================================
\slide{The ML Programming Language}

\point{Functional programming language.}
\point{Normally used interactively.}
\point{Strongly typed, with:}
   \subpoint{type inference,}
   \subpoint{abstract data types, and}
   \subpoint{polymorphic types.}
\point{Has exception-handling mechanisms.}

\vskip5mm

\point{ML is the meta-language for HOL.}
\point{Note: the ML of HOL88 is \underline{not} SML.}


% =====================================================================
\slide{Evaluation and Bindings}

\point{Evaluation:}

\begin{session}\begin{verbatim}
#2 + 3;;   
5 : int

#rev [1;2;3;4;5];;
[5; 4; 3; 2; 1] : int list
\end{verbatim}\end{session}

\point{Simple bindings:}

\begin{session}\begin{verbatim}
#let n = 8 * 2 + 5;;
n = 21 : int

#n * 2;;
42 : int
\end{verbatim}\end{session}

\point{The special identifier `{\tt it}':}

\begin{session}\begin{verbatim}
#it;;
42 : int
\end{verbatim}\end{session}

% =====================================================================
\slide{Multiple and Local Bindings}

\point{Multiple bindings:}

\begin{session}\begin{verbatim}
#let one = 1 and two = 2;;
one = 1 : int
two = 2 : int
\end{verbatim}\end{session}

\point{Local bindings:}

\begin{session}\begin{verbatim}
#let n = 0;;
n = 0 : int

#let n = 12 in n/6;;
2 : int

#n;;
0 : int
\end{verbatim}\end{session}

\point{Multiple local bindings:}

\begin{session}\begin{verbatim}
#let n = 5 and m = 6 in n + m;;
11 : int
\end{verbatim}\end{session}


% =====================================================================
\slide{Integers and Booleans}

\point{Operations on integers:}

\begin{session}\begin{verbatim}
#let n = 2 + (3 * 4);;
n = 14 : int

#let n = (10 / 2) - 7;;
n = -2 : int
\end{verbatim}\end{session}


\point{Operations on booleans:}

\begin{session}\begin{verbatim}
#let b1 = true and b2 = false;;
b1 = true : bool
b2 = false : bool

#1 = (1 + 1);;
false : bool

#not (b1 or b2);;
false : bool

#(7 < 3) & (false or 2 > 0);;
true : bool
\end{verbatim}\end{session}


% =====================================================================
\slide{Defining Functions}

\point{One argument:}

\begin{session}\begin{verbatim}
#let f n = n + n;;
f = - : (int -> int)

#f 22;;
44 : int
\end{verbatim}\end{session}

\point{Two or more arguments:}

\begin{session}\begin{verbatim}
#let plus(n,m) = n + m;;
plus = - : ((int # int) -> int)

#plus (1,2);;
3 : int
\end{verbatim}\end{session}


% =====================================================================
\slide{Curried Functions}

\point{Curried addition:}

\begin{session}\begin{verbatim}
#let plus n m = n + m;;
plus = - : (int -> int -> int)

#plus 1 2;;
3 : int
\end{verbatim}\end{session}

\point{Partial application:}

\begin{session}\begin{verbatim}
#let inc = plus 1;;
inc = - : (int -> int)

#inc 7;;
8 : int
\end{verbatim}\end{session}

\point{Higher-order functions:}

\begin{session}\begin{verbatim}
#let foo f n = (f(n+1)) / 2 ;;
foo = - : ((int -> int) -> int -> int)

#foo inc 3;;
2 : int
\end{verbatim}\end{session}


% =====================================================================
\slide{Defining Recursive Functions}

\point{{\tt let} won't define recursive functions:}

\begin{session}\begin{verbatim}
#let f n = n + n;;
f = - : (int -> int)

#let f n = if (n=0) then 1 else n * f(n-1);;
f = - : (int -> int)

#f 3;;
12 : int
\end{verbatim}\end{session}

\point{Defining recursive functions, {\tt letrec}:}

\begin{session}\begin{verbatim}
#letrec f n = if (n=0) then 1 else n * f(n-1);;
f = - : (int -> int)

#f 3;;
6 : int
\end{verbatim}\end{session}

% =====================================================================
\slide{Lambda Abstractions}

\point{The increment function:}

\begin{session}\begin{verbatim}
#\x. x + 1;;
- : (int -> int)

#(\x. x + 1) 2;;
3 : int
\end{verbatim}\end{session}

\point{Curried multiplication:}

\begin{session}\begin{verbatim}
#\x. \y. x * y;;
- : (int -> int -> int)

#let double = (\x. \y. x * y) 2;;
double = - : (int -> int)

#double 22;;
44 : int
\end{verbatim}\end{session}

\point{Abbreviation:}
\begin{session}\begin{verbatim}
#\x y. x * y;;
- : (int -> int -> int)
\end{verbatim}\end{session}


% =====================================================================
\slide{Other Data Types}

\point{Strings:}
\begin{session}\begin{verbatim}
#`abc`;;
`abc` : string

#ascii;;
- : (int -> string)

#ascii 97;;
`a` : string
\end{verbatim}\end{session}

\point{Void:}

\begin{session}\begin{verbatim}
#();;
() : void

#quit;;
- : (void -> void)
\end{verbatim}\end{session}

% =====================================================================
\slide{Lists}

\point{Head and tail:}

\begin{session}\begin{verbatim}
#let l = [2;3;2+3];;
l = [2; 3; 5] : int list

#hd l;;
2 : int

#tl l;;
[3; 5] : int list
\end{verbatim}\end{session}

\point{Cons and concatenation:}
\begin{session}\begin{verbatim}
#9 . l;;
[9; 2; 3; 5] : int list

#[true;false] @ [false;true];;
[true; false; false; true] : bool list
\end{verbatim}\end{session}

\point{Empty lists:}
\begin{session}\begin{verbatim}
#null l;;
false : bool

#null [];;
true : bool
\end{verbatim}\end{session}

% =====================================================================
\slide{Pairs}

\point{Notation:}

\begin{session}\begin{verbatim}
#let p = (2,3);;
p = (2, 3) : (int # int)

#fst p;;
2 : int

#snd p;;
3 : int
\end{verbatim}\end{session}

\point{Right associativity:}

\begin{session}\begin{verbatim}
#fst (2,3,4);;
2 : int

#snd(2,3,4);;
(3, 4) : (int # int)
\end{verbatim}\end{session}

\point{Example:}
\begin{session}\begin{verbatim}
#let swap p = (snd p, fst p);;
rev = - : ((* # **) -> (** # *))

#rev p;;
(3, 2) : (int # int)
\end{verbatim}\end{session}


% =====================================================================
\slide{Polymorphism}

\point{Example: the function {\tt hd}}

\begin{session}\begin{verbatim}
#hd [2;3];;
2 : int

#hd [true;false];;
true : bool
\end{verbatim}\end{session}
\vskip1mm
\bpindent\LARGE\bf
What is the type of {\tt hd}:
\epindent
\vskip7mm
\bspindent\LARGE
\bf \verb!int list -> int! or  \verb!bool list -> bool!?
\espindent
\vskip5mm

\point{The function {\tt hd} has both these types:}

\begin{session}\begin{verbatim}
#hd;;
- : (* list -> *)
\end{verbatim}\end{session}
\vskip1mm
\bpindent\LARGE\bf
The `{\tt *}' here is a type variable.
\epindent

\vskip5mm


\point{That is, {\tt hd} can have any type of the form}
\vskip7mm
\bspindent\LARGE
 {\tt $\sigma$ list -> $\sigma$} 
\espindent 
\vskip7mm
\bpindent\LARGE\bf
where $\sigma$ is an ML type. 
\epindent


% =====================================================================
\slide{Type Inference}

\point{ML infers the most general type.}

\vskip4mm

\point{For example, the mapping function:}
\begin{session}\begin{verbatim}
#letrec map f l =
        if (null l)
           then []
           else f(hd l).(map f (tl l));;
map = - : ((* -> **) -> * list -> ** list)

#map (\x.0);;
- : (* list -> int list)
\end{verbatim}\end{session}

\point{Function composition:}
\begin{session}\begin{verbatim}
#let comp f g x = f(g x);;
comp = - : ((* -> **) -> (*** -> *) -> *** -> **)

#comp null (map \y.y+1);;
- : (int list -> bool)
\end{verbatim}\end{session}

% =====================================================================
\slide{Exception Handling}

\point{Failure:}

\begin{session}\begin{verbatim}
#hd [];;
evaluation failed     hd

#1 / 0;;
evaluation failed     div
\end{verbatim}\end{session}

\point{Explicitly generating failure}
\begin{session}\begin{verbatim}
#failwith `foo`;;
evaluation failed     foo

#letrec fact n =
    if (n<0)
       then failwith `negative argument to fact`
       else if (n=0) then 1 else n * fact(n-1);;
fact = - : (int -> int)

#fact (-1);;
evaluation failed     negative argument to fact
\end{verbatim}\end{session}

\point{Trapping failure:}
\begin{session}\begin{verbatim}
#hd [] ? 99;;
99 : int
\end{verbatim}\end{session}

% =====================================================================
\slide{Common Errors}

\point{Unbound variable:}
\begin{session}\begin{verbatim}
#foo;;

unbound or non-assignable variable foo
1 error in typing
typecheck failed     
\end{verbatim}\end{session}

\point{Type violation:}
\begin{session}\begin{verbatim}
#hd 2;;

ill-typed phrase: 2
has an instance of type  int
which should match type  * list
1 error in typing
typecheck failed     
\end{verbatim}\end{session}

\point{Mismatched parentheses}
\begin{session}\begin{verbatim}
#let f x = (x + 2;;
bad paren balance
skipping: 2 ;; parse failed     

#let f x = x + 2);;
non top level decln must have IN clause
skipping: 2 ) ;; parse failed     
\end{verbatim}\end{session}


% =====================================================================
\slide{Some System Functions}


\point{Load a file called {\tt foo.ml}:}

\begin{session}\begin{alltt}
#load;;
- : ((string # bool) -> void)

#load(`foo`, false);;
......() : void

#load(`foo`,true);;
      \(\vdots\)   
   {\it (output from evaluating contents of{\tt foo})}
File foo loaded
() : void
\end{alltt}\end{session}

\point{Get online help:}
\begin{session}\begin{alltt}
#help;;
- : (string -> void)

#help `item`;;
      \(\vdots\)
   {\it (prints information about{\tt item})}
() : void
\end{alltt}\end{session}

\point{Terminate the session:}

\begin{session}\begin{alltt}
#quit();;
\end{alltt}\end{session}


% =====================================================================
\slide{Summary}

\def\bk{{\tt\char`\\ }}
\def\_{\char'137}

\point{Bindings}

  \subpoint{{\tt let $x_1$ = $e_1$ and $\dots$ and $x_n$ = $e_n$;;}}
  \subpoint{{\tt let $x_1$ = $e_1$ and $\dots$ and $x_n$ = $e_n$ in $e$;;}}

\point{Defining functions}

  \subpoint{{\tt let f $v_1$ $\dots$ $v_n$ = $e$;;}}
  \subpoint{{\tt letrec f $v_1$ $\dots$ $v_n$ = $e$;;}}

\point{Lambda abstractions}

  \subpoint{{\tt \bk $v_1$ $\dots$ $v_n$ .\ $e$}}

\point{Data types}

  \subpoint{{\tt bool}, {\tt int}, {\tt string}, {\tt void}}
  \subpoint{{\tt * list}, {\tt * \# **}, {\tt * -> **}}

\point{Exception handling}

\subpoint{{\tt $e_1$ ?\ $e_2$}, {\tt failwith `{\it string}`}}

\point{Useful system functions}

  \subpoint{{\tt load}, {\tt help}, {\tt quit}}


\end{document}

